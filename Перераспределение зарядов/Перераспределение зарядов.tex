\documentclass{article}
\usepackage[12pt]{extsizes}
\usepackage[T2A]{fontenc}
\usepackage[utf8]{inputenc}
\usepackage[english, russian]{babel}

\usepackage{amssymb}
\usepackage{amsfonts}
\usepackage{amsmath}
\usepackage{enumitem}
\usepackage{graphics}

\usepackage{lipsum}



\usepackage{geometry} % Меняем поля страницы
\geometry{left=1cm}% левое поле
\geometry{right=1cm}% правое поле
\geometry{top=1.5cm}% верхнее поле
\geometry{bottom=1cm}% нижнее поле


\usepackage{fancyhdr} % Headers and footers
\pagestyle{fancy} % All pages have headers and footers
\fancyhead{} % Blank out the default header
\fancyfoot{} % Blank out the default footer
\fancyhead[L]{Математика}
\fancyhead[C]{\textit{Комбинаторика}}
\fancyhead[R]{18 августа 2023}% Custom header text


%----------------------------------------------------------------------------------------

%\begin{document}\normalsize
\begin{document}\large
	
	
\begin{center}
	\textbf{Перераспределение зарядов}
\end{center}

\textbf{Пример.} В некоторых клетках прямоугольной таблицы нарисованы звездочки. Известно, что для любой отмеченной клетки количество звездочек в её столбце совпадает с количеством звездочек в её строке. Докажите, что число строк в таблице, в которых есть хоть одна звездочка, равно числу столбцов таблицы, в которых есть хоть одна звездочка.

\begin{enumerate}[label*=\protect\fbox{\arabic{enumi}}]
\item Утром на 7 деревьев расселись бакланы. Вечером эти же бакланы сели на 8 деревьев (так, что на каждом дереве есть хотя бы 1 баклан). Докажите, что найдется баклан, у которого утром соседей по дереву было строго больше, чем вечером.
\item В прямоугольной таблице $ m $ строк и $ n $ столбцов, где $ m < n $. В некоторых клетках таблицы стоят звёздочки, так что в каждом столбце стоит хотя бы одна звёздочка. Докажите, что существует хотя бы одна такая звёздочка, что в одной строке с нею находится больше звёздочек, чем с нею в одном столбце.
\item В библиотеке на полках стоят книги, ровно $ k $ полок пусты. Книги переставили так, что теперь пустых полок нет. Докажите, что найдётся хотя бы $ k + 1 $ книга, которая теперь стоит на полке с меньшим числом книг, чем стояла раньше.
\item Каждая клетка доски $ 4 \times n $ покрашена в черный или белый цвет. Каждая белая клетка граничит по стороне хотя бы с одной черной. Докажите, что черных клеток хотя бы $ n $.
\item На плоскости дано $ n $ окружностей радиуса 1, причем известно, что каждая пересекается хотя бы с одной другой окружностью, и никакая пара не касается. Докажите, что все вместе окружности образуют не меньше $ n $ точек пересечения (в одной точке могут пересекаться более двух окружностей).
\item Некоторые клетки таблицы покрашены в зеленый. Известно, что для любой зелёной клетки число зелёных клеток в её столбце совпадает с числом зелёных клеток в строке. Докажите, что число столбцов, в которых есть хотя бы одна зелёная клетка, равно числу строк, в которых есть хотя бы одна зеленая клетка.

\item На бесконечной в обе стороны полосе из клеток, пронумерованных целыми числами, лежит несколько камешков (возможно, по нескольку в одной клетке). Расположение камешков называется неподвижным, если в этом состоянии невозможно выполнить операцию.
\begin{enumerate} 
	\item За одну операцию разрешается снять два камешка с клетки с номером $n$ и добавить один в клетку с номером $n + 1$. Докажите, что неподвижное состояние не зависит от порядка операций.
	\item За одну операцию разрешается снять по одному камешку с клеток с номерами $ n $ и $ n + 1 $ и добавить камешек в клетку с номером $ n + 2 $. Докажите, что все неподвижные состояния, в которых на каждой клетке лежит не более одного камешка, одинаковы.
\end{enumerate} 
\item На турнир по игре в мяч приехало $ B $ баскетболистов и $ V $ волейболистов. После турнира каждый волейболист сыграл в настольный теннис по крайней мере с одним баскетболистом, а каждый баскетболист — не более чем с десятью волейболистами. Также известно, что у каждого волейболиста соперников-баскетболистов было больше, чем у любого из них — соперников-волейболистов. Докажите, что $ V \leqslant \dfrac{10}{11} B $.

\item В классе учатся $ m $ мальчиков и $ d $ девочек. У каждого мальчика есть хотя бы одна подруга, при этом у него количество подруг хотя бы вдвое больше, чем количество друзей у любой из его подруг. Докажите, что $ d > 2m $.

\item Пусть есть выпуклый $ n $-угольник и выбрано $ m $ красных точек, отличных от вершин, таких, что любой отрезок между двумя вершинами многоугольника содержит по крайней мере одну красную точку. Тогда
$$ m\geqslant n \left(1+\dfrac{1}{2}+\dfrac{1}{3}+\dotsc+\dfrac{1}{\lfloor (n-1)/2\rfloor}\right). $$

\item В некоторых узлах целочисленной решётки с неотрицательными координатами лежат фишки. За одну операцию разрешается снять фишку с узла с координатами $ (i, j) $ и добавить по фишке в узлы $ (i + 1, j), (i, j + 1); $ при этом запрещено попадание двух и более фишек в один узел.
\begin{enumerate} 
	\item Докажите, что если изначально в трёх узлах с наименьшей суммой координат стоит по фишке, то такими операциями нельзя добиться того, чтобы они все стали пустыми.
 	\item Докажите,что если изначально в узле $ (0,0) $ стоит фишка, то такими операциями нельзя сделать пустыми все шесть узлов с наименьшей суммой координат.
\end{enumerate} 
\item Можно ли за круглым столом рассадить 12 сладкоежек и поставить 28 сахарниц на стол так, чтобы между любыми двумя сладкоежками стояла сахарница?
\item В вершинах правильного $ n $—угольника расположены лампочки. Изначально одна горит, остальные выключены. Разрешается выбрать правильный многоугольник с вершинами в точках, где стоят лампочки, все лампочки в котором имеют одинаковое состояние, и все их переключить. Докажите, что нельзя выключить все лампочки.
\item Квадрат разрезали на несколько треугольников. Докажите, что среди них найдётся два с общей стороной.

\end{enumerate}
\end{document}