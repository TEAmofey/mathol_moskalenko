\documentclass{article}

\usepackage[12pt]{extsizes}
\usepackage[T2A]{fontenc}
\usepackage[utf8]{inputenc}
\usepackage[english, russian]{babel}

\usepackage{mathrsfs}
\usepackage[dvipsnames]{xcolor}

\usepackage{amsmath}
\usepackage{amssymb}
\usepackage{amsthm}
\usepackage{indentfirst}
\usepackage{amsfonts}
\usepackage{enumitem}
\usepackage{graphics}
\usepackage{tikz}
\usepackage{tabu}
\usepackage{diagbox}
\usepackage{hyperref}
\usepackage{mathtools}
\usepackage{ucs}
\usepackage{lipsum}
\usepackage{geometry} % Меняем поля страницы
\usepackage{fancyhdr} % Headers and footers
\newcommand{\range}{\mathrm{range}}
\newcommand{\dom}{\mathrm{dom}}
\newcommand{\N}{\mathbb{N}}
\newcommand{\R}{\mathbb{R}}
\newcommand{\E}{\mathbb{E}}
\newcommand{\D}{\mathbb{D}}
\newcommand{\M}{\mathcal{M}}
\newcommand{\Prime}{\mathbb{P}}
\newcommand{\A}{\mathbb{A}}
\newcommand{\Q}{\mathbb{Q}}
\newcommand{\Z}{\mathbb{Z}}
\newcommand{\F}{\mathbb{F}}
\newcommand{\CC}{\mathbb{C}}

\DeclarePairedDelimiter\abs{\lvert}{\rvert}
\DeclarePairedDelimiter\floor{\lfloor}{\rfloor}
\DeclarePairedDelimiter\ceil{\lceil}{\rceil}
\DeclarePairedDelimiter\lr{(}{)}
\DeclarePairedDelimiter\set{\{}{\}}
\DeclarePairedDelimiter\norm{\|}{\|}

\renewcommand{\labelenumi}{(\alph{enumi})}

\newcommand{\smallindent}{
    \geometry{left=1cm}% левое поле
    \geometry{right=1cm}% правое поле
    \geometry{top=1.5cm}% верхнее поле
    \geometry{bottom=1cm}% нижнее поле
}

\newcommand{\header}[3]{
    \pagestyle{fancy} % All pages have headers and footers
    \fancyhead{} % Blank out the default header
    \fancyfoot{} % Blank out the default footer
    \fancyhead[L]{#1}
    \fancyhead[C]{#2}
    \fancyhead[R]{#3}
}

\newcommand{\dividedinto}{
    \,\,\,\vdots\,\,\,
}

\newcommand{\littletaller}{\mathchoice{\vphantom{\big|}}{}{}{}}

\newcommand\restr[2]{{
    \left.\kern-\nulldelimiterspace % automatically resize the bar with \right
    #1 % the function
    \littletaller % pretend it's a little taller at normal size
    \right|_{#2} % this is the delimiter
}}

\DeclareGraphicsExtensions{.pdf,.png,.jpg}

\newenvironment{enumerate_boxed}[1][enumi]{\begin{enumerate}[label*=\protect\fbox{\arabic{#1}}]}{\end{enumerate}}



\smallindent

\header{Математика}{\textit{Геометрия}}{7 сентября 2023}

%----------------------------------------------------------------------------------------

\begin{document}
    \large

    \begin{center}
        \textbf{Изогональное сопряжение 2}
    \end{center}


    \textbf{Изогональное сопряжение в четырёхугольнике:}

    \begin{enumerate_boxed}

        \item Точка $P$ лежит внутри выпуклого четырёхугольника $ABCD$ и проекции точки $P$ на прямые, содержащие стороны, попадают на стороны.
        Докажите, что для точки $P$ существует изогонально сопряжённая относительно четырёхугольника $ABCD$ тогда и только тогда, когда

        a) основания перпендикуляров из точки $P$ на стороны являются вершинами вписанного четырёхугольника;

        b) верно соотношение $\angle APB+\angle CPD=180^\circ$.

        \item  В выпуклом четырёхугольнике \(ABCD\) биссектрисы углов \(\angle BAC\) и \(\angle BDC\) пересекаются в точке \(P\).
        Кроме того, \(\angle APB = \angle CPD\).
        Докажите, что \(AB + BD = AC + CD\).

        \item  Дан выпуклый четырёхугольник \(ABCD\).
        Обозначим через \(I_A\), \(I_B\), \(I_C\) и \(I_D\) центры вписанных окружностей \(\omega_A\), \(\omega_B\), \(\omega_C\) и \(\omega_D\) треугольников \(DAB\), \(ABC\), \(BCD\) и \(CDA\) соответственно.
        Оказалось, что \(\angle BI_{A}A + \angle I_{C}I_{A}I_D = 180^\circ\).
        Докажите, что \(\angle BI_{B}A + \angle I_{C}I_{B}I_D = 180^\circ\).

        \item  В выпуклом четырёхугольнике \(ABCD\) диагональ \(BD\) не является биссектрисой ни угла \(ABC\), ни угла \(CDA\).
        Точка \(P\) внутри четырехугольника \(ABCD\) такова, что \(\angle PBC = \angle DBA\) и \(\angle PDC = \angle BDA\).
        Докажите, что \(ABCD\) вписан тогда и только тогда, когда \(AP = CP\).

        \textbf{Изогональное сопряжение окружности относительно треугольника:}

        \item  На окружности, проходящей через вершины \(B\) и \(C\) треугольника \(ABC\) и через центр его вписанной окружности, выбраны такие точки \(P\) и \(Q\), лежащие внутри треугольника, что \(\angle BAP = \angle CAQ\).
        Докажите, что точки \(P\) и \(Q\) изогонально сопряжены.

        \item  Окружность пересекает сторону \(BC\) треугольника \(ABC\) в точках \(A_1\) и \(A_2\), сторону \(CA\) — в точках \(B_1\) и \(B_2\), а сторону \(AB\) — в точках \(C_1\) и \(C_2\).
        Окружности, описанные около треугольников \(AB_{1}C_1\) и \(BC_{1}A_1\), пересекаются в точке \(P_1\).
        Окружности, описанные около треугольников \(AB_{2}C_2\) и \(BC_{2}A_2\), пересекаются в точке \(P_2\).
        Докажите, что точки \(P_1\) и \(P_2\) изогонально сопряжены относительно треугольника \(ABC\).

        \textbf{Изогональное сопряжение в подобных треугольниках:}

        \item Дан неравнобедренный треугольник \(ABC\).
        Пусть \(N\) — середина дуги \(BAC\) его описанной окружности, а \(M\) — середина стороны \(BC\).
        Обозначим через \(I_1\) и \(I_2\) центры вписанных окружностей треугольников \(ABM\) и \(ACM\) соответственно.
        Докажите, что точки \(I_1\), \(I_2\), \(A\) и \(N\) лежат на одной окружности.

        \item Четырёхугольник \(ABCD\) вписан в окружность \(\omega\).
        Окружность \(\omega_1\) касается прямых \(AB\) и \(CD\) в точках \(X\) и \(Y\) и пересекает дугу \(AD\) окружности \(\omega\) в точках \(K\) и \(L\).
        Прямая \(XY\) пересекает прямые \(AC\) и \(BD\) в точках \(Z\) и \(T\).
        Докажите, что \(K\), \(L\), \(Z\) и \(T\) лежат на одной окружности, касающейся прямых \(AC\) и \(BD\).

        \item Точка \(X\) вне треугольника \(ABC\) такова, что \(A\) лежит внутри треугольника \(BXC\).
        При этом \(2\angle BAX = \angle CBA\), \(2\angle CAX = \angle BCA\).
        Докажите, что центры описанной и вневписанной со стороны \(BC\) окружностей треугольника \(ABC\) и точка \(X\) лежат на одной прямой.

        \item Точки \(M\) и \(N\) — соответственно середины сторон \(AB\) и \(AC\) треугольника \(ABC\).
        На касательной в точке \(A\) к описанной окружности треугольника \(ABC\) выбрана точка \(X\).
        Окружность \(\omega_B\), проходящая через точки \(M\) и \(B\), касается прямой \(MX\), а окружность \(\omega_C\), проходящая через точки \(N\) и \(C\), касается прямой \(NX\).
        Докажите, что \(\omega_B\) и \(\omega_C\) пересекаются на прямой \(BC\).

        \textbf{Задачи посложнее:}

        \item Вписанная окружность треугольника \(ABC\) касается стороны \(BC\) в точке \(A_1\), точка \(I\) — центр этой окружности.
        Прямая, проходящая через точку \(A_1\) перпендикулярно \(AA_1\), пересекает прямые \(BI\) и \(CI\) в точках \(X\) и \(Y\) соответственно.
        Докажите, что \(AX = AY\).

        \item Пусть пары точек \(X\) и \(X'\), \(Y\) и \(Y'\) изогонально сопряжены относительно треугольника \(ABC\).
        Докажите, что точки пересечения пар прямых \(XY\) и \(X'Y'\), \(XY'\) и \(X'Y\) изогонально сопряжены относительно треугольника \(ABC\).

        \item Докажите, что проекция ортоцентра треугольника \(ABC\) на медиану, выходящую из вершины \(A\), и проекция центра описанной окружности на симедиану, выходящую из вершины \(A\), изогонально сопряжены.

        \item Чевианы \(AA_1\), \(BB_1\) и \(CC_1\) треугольника \(ABC\) пересекаются в точке \(P\), лежащей внутри треугольника.
        Известно, что \(PA_1 = PB_1 = PC_1\).
        Докажите, что перпендикуляры, восставленные в точках \(A_1\), \(B_1\) и \(C_1\) к сторонам треугольника \(ABC\), пересекаются в одной точке.

    \end{enumerate_boxed}
\end{document}