\documentclass{article}

\usepackage[12pt]{extsizes}
\usepackage[T2A]{fontenc}
\usepackage[utf8]{inputenc}
\usepackage[english, russian]{babel}

\usepackage{mathrsfs}
\usepackage[dvipsnames]{xcolor}

\usepackage{amsmath}
\usepackage{amssymb}
\usepackage{amsthm}
\usepackage{indentfirst}
\usepackage{amsfonts}
\usepackage{enumitem}
\usepackage{graphics}
\usepackage{tikz}
\usepackage{tabu}
\usepackage{diagbox}
\usepackage{hyperref}
\usepackage{mathtools}
\usepackage{ucs}
\usepackage{lipsum}
\usepackage{geometry} % Меняем поля страницы
\usepackage{fancyhdr} % Headers and footers
\newcommand{\range}{\mathrm{range}}
\newcommand{\dom}{\mathrm{dom}}
\newcommand{\N}{\mathbb{N}}
\newcommand{\R}{\mathbb{R}}
\newcommand{\E}{\mathbb{E}}
\newcommand{\D}{\mathbb{D}}
\newcommand{\M}{\mathcal{M}}
\newcommand{\Prime}{\mathbb{P}}
\newcommand{\A}{\mathbb{A}}
\newcommand{\Q}{\mathbb{Q}}
\newcommand{\Z}{\mathbb{Z}}
\newcommand{\F}{\mathbb{F}}
\newcommand{\CC}{\mathbb{C}}

\DeclarePairedDelimiter\abs{\lvert}{\rvert}
\DeclarePairedDelimiter\floor{\lfloor}{\rfloor}
\DeclarePairedDelimiter\ceil{\lceil}{\rceil}
\DeclarePairedDelimiter\lr{(}{)}
\DeclarePairedDelimiter\set{\{}{\}}
\DeclarePairedDelimiter\norm{\|}{\|}

\renewcommand{\labelenumi}{(\alph{enumi})}

\newcommand{\smallindent}{
    \geometry{left=1cm}% левое поле
    \geometry{right=1cm}% правое поле
    \geometry{top=1.5cm}% верхнее поле
    \geometry{bottom=1cm}% нижнее поле
}

\newcommand{\header}[3]{
    \pagestyle{fancy} % All pages have headers and footers
    \fancyhead{} % Blank out the default header
    \fancyfoot{} % Blank out the default footer
    \fancyhead[L]{#1}
    \fancyhead[C]{#2}
    \fancyhead[R]{#3}
}

\newcommand{\dividedinto}{
    \,\,\,\vdots\,\,\,
}

\newcommand{\littletaller}{\mathchoice{\vphantom{\big|}}{}{}{}}

\newcommand\restr[2]{{
    \left.\kern-\nulldelimiterspace % automatically resize the bar with \right
    #1 % the function
    \littletaller % pretend it's a little taller at normal size
    \right|_{#2} % this is the delimiter
}}

\DeclareGraphicsExtensions{.pdf,.png,.jpg}

\newenvironment{enumerate_boxed}[1][enumi]{\begin{enumerate}[label*=\protect\fbox{\arabic{#1}}]}{\end{enumerate}}


\usepackage[framemethod=TikZ]{mdframed}

\newcommand{\definebox}[3]{%
    \newcounter{#1}
    \newenvironment{#1}[1][]{%
        \stepcounter{#1}%
        \mdfsetup{%
            frametitle={%
            \tikz[baseline=(current bounding box.east),outer sep=0pt]
            \node[anchor=east,rectangle,fill=white]
            {\strut #2~\csname the#1\endcsname\ifstrempty{##1}{}{##1}};}}%
        \mdfsetup{innertopmargin=1pt,linecolor=#3,%
            linewidth=3pt,topline=true,
            frametitleaboveskip=\dimexpr-\ht\strutbox\relax,}%
        \begin{mdframed}[]
            \relax%
            }{
        \end{mdframed}}%
}

\definebox{theorem_boxed}{Теорема}{ForestGreen!24}
\definebox{definition_boxed}{Определение}{blue!24}
\definebox{task_boxed}{Задача}{orange!24}
\definebox{paradox_boxed}{Парадокс}{red!24}

\theoremstyle{plain}
\newtheorem{theorem}{Теорема}
\newtheorem{task}{Задача}
\newtheorem{lemma}{Лемма}
\newtheorem{definition}{Определение}
\newtheorem{statement}{Утверждение}
\newtheorem{corollary}{Следствие}

\theoremstyle{remark}
\newtheorem{remark}{Замечание}
\newtheorem{example}{Пример}

\smallindent

\header{Математика}{\textit{Геометрия}}{24 июля 2024}

%----------------------------------------------------------------------------------------

\begin{document}
    \large

    \begin{center}
        \textbf{Синусы в геометрии}
    \end{center}

    \begin{enumerate_boxed}
        \item На сторонах $BC$, $CA$, $AB$ треугольника $ABC$ во внешнюю сторону построены
        треугольники $BCA_1$, $CAB_1$, $ABC_1$ так, что $\angle BCA_1=\angle B_{1}CA=\varphi$,
        $\angle CAB_1=\angle BAC_1=\theta$, $\angle CBA_1=\angle ABC_1=\psi$.
        Докажите, что прямые $AA_1$, $BB_1$, $CC_1$ пересекаются в одной точке.

        \item  В треугольнике $ABC$ проведены биссектрисы $AA'$, $BB'$ и $CC'$.
        Пусть $P$ --- точка пересечения $A'B'$ и $CC'$, а $Q$ --- точка пересечения $A'C'$ и $BB'$.
        Докажите, что $\angle PAC = \angle QAB$.

        \item В окружность вписан выпуклый шестиугольник $ABCDEF$.
        Докажите, что прямые $AD$, $BE$ и $CF$ пересекаются в одной точке тогда и только тогда, когда $AB\cdot CD\cdot EF=BC\cdot DE\cdot FA$.

        \item Через точку $M$ проведены касательные $MA$ и $MB$ и две произвольные секущие $CD$ и $EF$.
        Докажите, что прямые $CF$ и $DE$ пересекаются на прямой $AB$.

        \item В треугольнике $ABC$ через внутреннюю точку $X$ проведены чевианы $AD$, $BE$, $CF$.
        В сегмент, отсекаемый прямой $AC$ от описанной окружности $\omega$ треугольника $ABC$ (не содержащий точку $B$), вписана окружность, касающася $AC$ в точке $E$ и $\omega$ в точке $B_1$.
        Аналогично определяются точки $A_1$ и $C_1$.
        Докажите, что прямые $AA_1$, $BB_1$, $CC_1$ пересекаются в одной точке.

        \item Противоположные стороны выпуклого шестиугольника попарно параллельны.
        Докажите, что прямые, соединяющие середины противоположных сторон, пересекаются в одной точке.

        \item Вписанная окружность треугольника $ABC$ касается его сторон в точках $A_1$, $B_1$ и $C_1$.
        Внутри треугольника $ABC$ взята точка $X$.
        Прямая $AX$ пересекает дугу $B_{1}C_1$ вписанной окружности в точке $A_2$; точки $B_2$ и $C_2$ определяются аналогично.
        Докажите, что прямые $A_{1}A_2$, $B_{1}B_2$ и $C_{1}C_2$ пересекаются в одной точке.

        \item Через точки $A$ и $D$, лежащие на окружности, проведены касательные, пересекающиеся в точке $S$.
        На дуге $AD$ взяты точки $B$ и $C$.
        Прямые $AC$ и $BD$ пересекаются в точке $P$, $AB$ и $CD$~--- в точке $Q$.
        Докажите, что прямая $PQ$ проходит через точку $S$.


    \end{enumerate_boxed}

\end{document}