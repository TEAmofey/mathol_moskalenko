\documentclass{article}
\usepackage[12pt]{extsizes}
\usepackage[T2A]{fontenc}
\usepackage[utf8]{inputenc}
\usepackage[english, russian]{babel}

\usepackage{amssymb}
\usepackage{amsfonts}
\usepackage{amsmath}
\usepackage{enumitem}
\usepackage{graphics}
\usepackage{graphicx}

\usepackage{lipsum}

\newtheorem{theorem}{Теорема}
\newtheorem{task}{Задача}
\newtheorem{lemma}{Лемма}
\newtheorem{definition}{Определение}
\newtheorem{example}{Пример}
\newtheorem{statement}{Утверждение}
\newtheorem{corollary}{Следствие}


\usepackage{geometry} % Меняем поля страницы
\geometry{left=1cm}% левое поле
\geometry{right=1cm}% правое поле
\geometry{top=1.5cm}% верхнее поле
\geometry{bottom=1cm}% нижнее поле


\usepackage{fancyhdr} % Headers and footers
\pagestyle{fancy} % All pages have headers and footers
\fancyhead{} % Blank out the default header
\fancyfoot{} % Blank out the default footer
\fancyhead[L]{Математика}
\fancyhead[C]{\textit{Разное}}
\fancyhead[R]{18 декабря}% Custom header text


%----------------------------------------------------------------------------------------

%\begin{document}\normalsize
\begin{document}\large
	
\begin{center}
	\textbf{Разнобой}
\end{center}


\begin{enumerate}[label*=\protect\fbox{\arabic{enumi}}]
	
\item Число $\dfrac{3}{2}$ является корнем многочлена $a_4x^4 + a_3x^3 + a_2x^2 + a_1x + a_0$. Найдите хотя
бы один корень многочлена $a_0x^4 + 3a_1x^3 + 9a_2x^2 + 27a_3x + 81a_4$.

\item Известно, что $abc = 1$, и что
$a + b + c = \dfrac{1}{a} + \dfrac{1}{b} + \dfrac{1}{c}$ . Докажите, что по меньшей мере одно из чисел $a, b$ и $c$ равно 1.

\item Докажите, что если $P(0)$ и $P(1)$ нечетные числа, то многочлен $P(x)$ не имеет целых корней.

\item Докажите, что уравнение $x^2 + y^2 - z^2 = 2024$ имеет бесконечно много решений.

\item Три стороны четырёхугольника в порядке обхода равны $7, 1$ и $4$. Найдите четвёртую сторону этого четырёхугольника, если известно, что его диагонали перпендикулярны.

\item На биссектрисе внешнего угла $C$ треугольника $ABC$ взята точка $M$, отличная от $C$. Докажите, что $MA+ MB > CA + CB$.

\item Рассмотрим треугольник $ABC$. Пусть $r$~--- центр вписанной окружности, $r_a$,$r_b$, $r_c$~--- центры соответствующих вневписанных окружностей. Докажите что $$\frac{1}{r} = \frac{1}{r_a}  + \frac{1}{r_b} + \frac{1}{r_c} $$

\item Рассмотрим треугольник $ABC$. Пусть $r$~--- центр вписанной окружности, $r_a$,$r_b$, $r_c$~--- центры соответствующих вневписанных окружностей. Докажите что $$S_\Delta = \sqrt{rr_ar_br_c} $$

\item За круглым столом совещались $2n$ депутатов. После перерыва эти же $2n$ депутатов расселись вокруг стола, но уже в другом порядке.
Доказать, что найдутся два депутата, между которыми как до, так и после перерыва сидело одинаковое число человек.

\item В классе 20 учеников, причём каждый дружит не менее, чем с 14 другими.
Можно ли утверждать, что найдутся четыре ученика, которые все дружат
между собой?

\item Можно ли расставить по кругу 100 цифр так, чтобы каждая двузначная комбинация от 00 до 99 при движении по часовой стрелке встречалась ровно один раз?

\item Город представляет из себя квадрат $5 \times 5$, в котором каждая сторона квартала квадратика участок улицы длины $500$ метров. Какой наименьший путь придется проделать катку, чтобы заасфальтировать улицы?

\item Натуральное число $n$ таково, что
$[n, n + 1] > [n, n + 2] > \dotsc > [n, n + 35]$.
Докажите, что $[n, n + 35] >[n, n + 36]$

\item На длинной полоске написана десятичная запись числа $3^{20202021}$. Саша разрезал полоску на три куска. Изучив числа, написанные на этих кусках, Саша заявил, что каждое из этих трех чисел является степенью тройки. Докажите, что он ошибается.

\item Три натуральных числа таковы, что произведение любых двух из них делится на сумму этих двух чисел. Докажите, что эти три числа имеют общий делитель, больший единицы.


\end{enumerate}
\end{document}