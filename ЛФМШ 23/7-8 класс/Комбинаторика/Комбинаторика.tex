\documentclass{article}
\usepackage[12pt]{extsizes}
\usepackage[T2A]{fontenc}
\usepackage[utf8]{inputenc}
\usepackage[english, russian]{babel}

\usepackage{amssymb}
\usepackage{amsfonts}
\usepackage{amsmath}
\usepackage{enumitem}
\usepackage{graphics}

\usepackage{lipsum}



\usepackage{geometry} % Меняем поля страницы
\geometry{left=1cm}% левое поле
\geometry{right=1cm}% правое поле
\geometry{top=1.5cm}% верхнее поле
\geometry{bottom=1cm}% нижнее поле


\usepackage{fancyhdr} % Headers and footers
\pagestyle{fancy} % All pages have headers and footers
\fancyhead{} % Blank out the default header
\fancyfoot{} % Blank out the default footer
\fancyhead[L]{ЦРОД $\bullet$ Математика}
\fancyhead[C]{\textit{Комбинаторика}}
\fancyhead[R]{ЛФМШ 2023}% Custom header text


%----------------------------------------------------------------------------------------

%\begin{document}\normalsize
\begin{document}\large
	
	
\begin{center}
	\textbf{Комбинаторика}
\end{center}



\begin{enumerate}[label*=\protect\fbox{\arabic{enumi}}]
	
\setcounter{enumi}{-2}
\item Сколько существует пятизначных чисел, цифры в которых не повторяются?
% 9 * 9 * 8 * 7 * 6 = 27216
\item Сколько существует пятизначных чисел, у которых все цифры нечётные?
% 5 * 5 * 5 * 5 * 5 = 3125
\item Сколько существует пятизначных чисел?
% 9 * 10 * 10 * 10 * 10 = 90000
\item Сколько существует пятизначных чисел, у которых хотя бы одна цифра чётная?
% 90000 - 3125 = 86875
\item Сколько существует различных пятизначных чисел, в которых цифры идут в порядке убывания?
% C_10^5 = 252
\item На каждой из двух параллельных прямых отмечено по 9 точек. Сколько различных (а) четырёхугольников (б) треугольников с вершинами в этих точках существует?
% (а) (9 * 8 / 2)^2 = 1296
% (б) 9 * 9 * 8 = 648
\item Сколькими способами можно покрасить клетки таблицы 2 на 5 в чёрный и белый цвета так, чтобы не было ни одноцветных строк, ни одноцветных столбцов?
%2^5 - 2 = 30
\item В математическом кружке занимается 10 ребят. Сколькими способами их можно
выстроить в ряд так, чтобы Аня стояла левее Бори, а Боря стоял левее Вовы?
%10!/61
\item Сколькими способами можно переставить буквы в слове
«ПОЛУПАРАЛЛЕЛОГРАММ», чтобы никакие две гласные не стояли рядом? А если согласные при этом должны стоять в алфавитном порядке?
%C_17^7*11!/4!/2!/2!/2!/1!
%C_17^7
\item На окружности отмечено десять точек. Сколько существует незамкнутых несамопересекающихся девятизвенных ломаных с вершинами в этих точках?
%10 * 2 ^ 8 / 2 = 1280
\item Сколькими способами можно на доске $30 \times 30$ расставить сорок одинаковых ладей так, чтобы каждая била ровно одну другую?

\item Дана полоска $1 \times 10$. В клетки записываются числа $1, 2, \dotsc , 10$ по следующему правилу: сначала в какую-нибудь клетку пишут число 1, затем число 2 записывают в соседнюю клетку, затем число 3 — в одну из соседних с уже занятыми, и так далее. Сколькими способами это можно сделать?

\item Сколькими способами можно выбрать две кости домино так, чтобы их можно было приложить друг к другу (то есть, чтобы какое-то число встречалось на обоих костях).

\item На танцы пришли $s$ юношей и $s$ девушек. Сколькими способами можно из них выбрать компанию, в которой было бы одинаковое число юношей и девушек?

\item На доску выписали все 10-значные числа, в записи которых использованы только цифры 3 и 7 (возможно только одна из цифр). Какова сумма цифр всех полученных чисел?
%2^9*11111111110
\item Клетки доски 4 × 4 заполнены числами 1 и 2. Сколько способов заполнить доску так, чтобы сумма чисел в каждой строке и каждом столбце была простым числом?
%2^9
\item Игральный кубик имеет 6 граней с цифрами 1, 2, 3, 4, 5, 6. Сколько различных иг- ральных кубиков существует, если считать различными два кубика, которые нельзя спутать, как ни переворачивай?
%6*5
\item В выпуклом $n$-угольнике проведены все диагонали. Они разбивают его на выпук- лые многоугольники. Возьмем среди них многоугольник с самым большим числом сторон. Сколько сторон он может иметь?


\end{enumerate}
\end{document}