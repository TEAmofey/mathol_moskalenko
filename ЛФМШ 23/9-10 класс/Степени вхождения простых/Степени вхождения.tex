\documentclass{article}
\usepackage[12pt]{extsizes}
\usepackage[T2A]{fontenc}
\usepackage[utf8]{inputenc}
\usepackage[english, russian]{babel}

\usepackage{amssymb}
\usepackage{amsfonts}
\usepackage{amsmath}
\usepackage{enumitem}
\usepackage{graphics}
\usepackage{graphicx}

\usepackage{lipsum}

\newtheorem{theorem}{Теорема}
\newtheorem{task}{Задача}
\newtheorem{lemma}{Лемма}
\newtheorem{definition}{Определение}
\newtheorem{example}{Пример}
\newtheorem{statement}{Утверждение}
\newtheorem{corollary}{Следствие}


\usepackage{geometry} % Меняем поля страницы
\geometry{left=1cm}% левое поле
\geometry{right=1cm}% правое поле
\geometry{top=1.5cm}% верхнее поле
\geometry{bottom=1cm}% нижнее поле


\usepackage{fancyhdr} % Headers and footers
\pagestyle{fancy} % All pages have headers and footers
\fancyhead{} % Blank out the default header
\fancyfoot{} % Blank out the default footer
\fancyhead[L]{ЦРОД $\bullet$ Математика}
\fancyhead[C]{\textit{Теория чисел}}
\fancyhead[R]{ЛФМШ 2023}% Custom header text


%----------------------------------------------------------------------------------------

%\begin{document}\normalsize
\begin{document}\large
	
\begin{center}
	\textbf{Степени вхождения простых чисел}
\end{center}

\textbf{Определение:} Степенью вхождения простого числа $p$ в натуральное число $n$ будем называть наибольшее такое $k$, что $n$ делится на $p^k$. Обозначать для краткости будем $\nu_p(n)$ (это греческая буква “ню”)

\textbf{Факт:}  $\nu_p(a + b) \ge \min\{\nu_p(a),\nu_p(b)\}$, причём если $\nu_p(a) \neq \nu_p(b)$, то $\nu_p(a + b) = \min\{\nu_p (a), \nu_p(b)\}$.

\begin{enumerate}[label*=\protect\fbox{\arabic{enumi}}]
	
\item Докажите \textbf{формулу Лежандра}: $\nu_p(n!)=\left\lfloor \dfrac{n}{p}\right\rfloor+\left\lfloor \dfrac{n}{p^2} \right\rfloor+\left\lfloor \dfrac{n}{p^3} \right\rfloor+...$
\item Докажите, что $n!$ не делится на $2^n$.

\item Натуральные числа $a$ и $b$ таковы, что сумма $\dfrac{b^2}{a} + \dfrac{a^2}{b}$ целая. Докажите, что оба слагаемых целые.
\item Взаимно простые в совокупности натуральные числа $a, b, c$ удовлетворяют условию
$ab = ac + bc$. Докажите, что $abc$ — точный квадрат.
\item Натуральные числа $a, b, c$ таковы, что число $\dfrac{a}{b} + \dfrac{b}{c} + \dfrac{c}{a}$ является целым. Верно ли,
что $abc$ — точный куб?
\item Докажите, что если числа $ab, cd$ и $ac + bd$ делятся на $k$ то $ac$ и $bd$ делятся на $k$.
\item Даны натуральные числа $a$ и $b$, причем $a < 1000$. Докажите, что если $a^{21}$ делится на $b^{10}$, то $a^2$ делится на $b$.
\item Натуральные числа $m, n$ таковы, что $m^2 + n^2 + m$ кратно $mn$. Докажите, что $m$ — квадрат натурального числа.
\item Докажите, что наименьшее общее кратное чисел от $n$ до $2n + 1$ делится на $\dfrac{(2n + 1)!}{n!\cdot n!}$
\item Докажите, что не существует трёх различных натуральных чисел, каждое из которых равно наименьшему общему кратному своих разностей с двумя другими.
\item Даны различные натуральные числа $a_1 , a_2 , \dotsc , a_n$ . Положим
$$b_i = (a_i - a_1)(a_i - a_2)\dotsc(a_i - a_{i-1})(a_i - a_{i+1})\dotsc(a_i - a_n).$$
Докажите, что наименьшее общее кратное $[b_1 , b_2 , \dotsc , b_n ]$ делится на $(n - 1)!$
\item Решите в натуральных числах уравнение $x^y = y^x$ при $x \neq y$.
\item Найдите все натуральные $x, y$ и простые $p$ такие что:
$$x^5 + y^4 = pxy$$
\end{enumerate}
\end{document}