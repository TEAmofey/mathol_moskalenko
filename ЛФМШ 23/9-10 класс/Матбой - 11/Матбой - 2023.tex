\documentclass{article}
\usepackage[12pt]{extsizes}
\usepackage[T2A]{fontenc}
\usepackage[utf8]{inputenc}
\usepackage[english, russian]{babel}

\usepackage{amssymb}
\usepackage{amsfonts}
\usepackage{amsmath}
\usepackage{enumitem}
\usepackage{graphics}

\usepackage{lipsum}



\usepackage{geometry} % Меняем поля страницы


\geometry{top=1.5cm}% верхнее поле
\geometry{bottom=1cm}% нижнее поле
\geometry{left=1cm}% левое поле
\geometry{right=1cm}% правое поле


\usepackage{fancyhdr} % Headers and footers
\pagestyle{fancy} % All pages have headers and footers
\fancyhead{} % Blank out the default header
\fancyfoot{} % Blank out the default footer
\fancyhead[L]{ЦРОД $\bullet$ Математика}
\fancyhead[C]{\textit{Не балуемся}}
\fancyhead[R]{ЛФМШ 2023}% Custom header text


%----------------------------------------------------------------------------------------

%\begin{document}\normalsize
\begin{document}\large
	

\begin{center}
\textbf{Математический бой }
\end{center}

\begin{enumerate}

\item Биссектрисы неравнобедренного треугольника $ABC$ пересекаются в точке $I$. Точки $I_A$, $I_B$ и $I_C$ симметричны $I$ относительно прямых $BC$, $AC$ и $AB$ соответственно. Докажите, что центры описанных окружностей треугольников $AI_AI$, $BI_BI$, $CI_CI$ лежат на одной прямой.

\item Решите уравнение в целых числах $3x^2 - y^2 = 3^{x+y}.$


\item В команде на ЧГК есть семь человек. Но как мы знаем, в ЧГК можно играть только вшестером. Поэтому каждый вопрос они меняют одного игрока на другого и все вместе пересаживаются. Известно, что любых шестерых участников можно посадить по кругу так, чтобы каждый сидел рядом с двумя своими друзьями. Докажите, что можно так посадить всех семерых, что каждый будет сидеть рядом со своими друзьями.

\item Найдите значение выражения

$$\frac{(3^4+4)\cdot(7^4+4)\cdot\dotsc\cdot(2019^4+4)\cdot(2023^4+4)}{(1^4+4)\cdot(5^4+4)\cdot\dotsc\cdot(2017^4+4)\cdot(2021^4+4)}$$

\item Хорда $CD$ окружности с центром $O$ перпендикулярна ее диаметру $AB$, а хорда $AE$ делит пополам радиус $OC$. Докажите, что хорда $DE$ делит пополам хорду $BC.$

\item Алиса, Вадим и Настя записали на доске 3 числа $a, b$ и $c$. Затем пришел Матвей и стёр с доски все числа, заменив их на попарные произведения. Из-за этого в классе начался сущий кошмар... До наших дней дошла информация от очевидцев, что на доске в итоге было записано два последовательных числа и два числа отличались на 1024. Восстановите числа, которые загадали изначально.

\item Пока Никита разбирался в теории чисел, он заметил, что число $$\underbrace{1\dotsc1}_{p}\underbrace{2\dotsc2}_{p}\underbrace{3\dotsc3}_{p}\dotsc \underbrace{9\dotsc9}_{p} - 123456789$$ делится на $p$ при любом простом $p$. Докажите это.

\item Сколькими способами можно переставить буквы в слове «СУПЕРИЗБАЛОВАННАЯ», чтобы никакие две гласные не стояли рядом? 

\end{enumerate}
\end{document}