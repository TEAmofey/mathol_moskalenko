\documentclass{article}

\usepackage[12pt]{extsizes}
\usepackage[T2A]{fontenc}
\usepackage[utf8]{inputenc}
\usepackage[english, russian]{babel}

\usepackage{mathrsfs}
\usepackage[dvipsnames]{xcolor}

\usepackage{amsmath}
\usepackage{amssymb}
\usepackage{amsthm}
\usepackage{indentfirst}
\usepackage{amsfonts}
\usepackage{enumitem}
\usepackage{graphics}
\usepackage{tikz}
\usepackage{tabu}
\usepackage{diagbox}
\usepackage{hyperref}
\usepackage{mathtools}
\usepackage{ucs}
\usepackage{lipsum}
\usepackage{geometry} % Меняем поля страницы
\usepackage{fancyhdr} % Headers and footers
\newcommand{\range}{\mathrm{range}}
\newcommand{\dom}{\mathrm{dom}}
\newcommand{\N}{\mathbb{N}}
\newcommand{\R}{\mathbb{R}}
\newcommand{\E}{\mathbb{E}}
\newcommand{\D}{\mathbb{D}}
\newcommand{\M}{\mathcal{M}}
\newcommand{\Prime}{\mathbb{P}}
\newcommand{\A}{\mathbb{A}}
\newcommand{\Q}{\mathbb{Q}}
\newcommand{\Z}{\mathbb{Z}}
\newcommand{\F}{\mathbb{F}}
\newcommand{\CC}{\mathbb{C}}

\DeclarePairedDelimiter\abs{\lvert}{\rvert}
\DeclarePairedDelimiter\floor{\lfloor}{\rfloor}
\DeclarePairedDelimiter\ceil{\lceil}{\rceil}
\DeclarePairedDelimiter\lr{(}{)}
\DeclarePairedDelimiter\set{\{}{\}}
\DeclarePairedDelimiter\norm{\|}{\|}

\renewcommand{\labelenumi}{(\alph{enumi})}

\newcommand{\smallindent}{
    \geometry{left=1cm}% левое поле
    \geometry{right=1cm}% правое поле
    \geometry{top=1.5cm}% верхнее поле
    \geometry{bottom=1cm}% нижнее поле
}

\newcommand{\header}[3]{
    \pagestyle{fancy} % All pages have headers and footers
    \fancyhead{} % Blank out the default header
    \fancyfoot{} % Blank out the default footer
    \fancyhead[L]{#1}
    \fancyhead[C]{#2}
    \fancyhead[R]{#3}
}

\newcommand{\dividedinto}{
    \,\,\,\vdots\,\,\,
}

\newcommand{\littletaller}{\mathchoice{\vphantom{\big|}}{}{}{}}

\newcommand\restr[2]{{
    \left.\kern-\nulldelimiterspace % automatically resize the bar with \right
    #1 % the function
    \littletaller % pretend it's a little taller at normal size
    \right|_{#2} % this is the delimiter
}}

\DeclareGraphicsExtensions{.pdf,.png,.jpg}

\newenvironment{enumerate_boxed}[1][enumi]{\begin{enumerate}[label*=\protect\fbox{\arabic{#1}}]}{\end{enumerate}}



\smallindent

\header{ЦРОД $\bullet$ Математика}{\textit{Геометрия}}{Май 2022}

%----------------------------------------------------------------------------------------

\begin{document}
    \large

    \begin{center}
        \textbf{Сумма углов треугольника}
    \end{center}

    \begin{enumerate_boxed}

        \item Чему равна сумма углов пятиугольника?

        \item Чему равна сумма углов $n$-угольника?

        \item Внешние углы при вершинах $A$ и $B$ треугольника $ABC$ равны $134^\circ$ и $99^\circ$ соответственно.
        Чему равна величина внешнего угла при вершине $C$ этого треугольника?

        \item В равнобедренном треугольнике один из углов равен $40^\circ$.
        Чему может быть равна величина наибольшего угла треугольника?

        \item В равнобедренном треугольнике один из углов в два раза больше другого.
        Чему может быть равна величина наименьшего угла этого треугольника?

        \item Два угла треугольника равны $10^\circ$ и $70^\circ$ соответственно.
        Найдите величину угла между высотой и биссектрисой, проведёнными из вершины третьего угла треугольника.

        \item На стороне $AB$ равнобедренного треугольника $ABC$ ($AB= AC$) нашлись такие точки $D$ и $E$ (точка $D$ лежит между точками $A$ и $E$), а на стороне $AC$ — такая точка $F$, что $BC=CE=EF=FD=DA$.
        Найдите величину угла $ABC$.

        \item Дан треугольник $ABC$.
        На продолжении стороны $AC$ за точку $A$ отложен отрезок $AD=AB$, а за точку $C$ — отрезок $CE = CB$.
        Выразите углы треугольника  $DBE$, через углы треугольника  $ABC$.

        \item Точки $M$ и $N$ лежат на стороне $AC$ треугольника $ABC$, причём $\angle ABM = \angle C$ и $\angle CBN=\angle A$.
        Докажите, что треугольник $BMN$ равнобедренный.

        \item Выразите угол между биссектрисой угла $A$ и биссектрисой внешнего угла $B$ через величину угла $C$.

        \item В четырёхугольнике $ABCD$ биссектрисы углов $A$ и $C$ параллельны.
        Докажите, что углы $B$ и $D$ четырёхугольника равны.

        \item Найдите сумму острых углов пятиугольной звезды

    \end{enumerate_boxed}
\end{document}