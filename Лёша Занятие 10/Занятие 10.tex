\documentclass{article}
\usepackage[12pt]{extsizes}
\usepackage[T2A]{fontenc}
\usepackage[utf8]{inputenc}
\usepackage[english, russian]{babel}

\usepackage{amssymb}
\usepackage{amsfonts}
\usepackage{amsmath}
\usepackage{enumitem}
\usepackage{graphics}
\usepackage{graphicx}

\usepackage{lipsum}

\newtheorem{theorem}{Теорема}
\newtheorem{task}{Задача}
\newtheorem{lemma}{Лемма}
\newtheorem{definition}{Определение}
\newtheorem{example}{Пример}
\newtheorem{statement}{Утверждение}
\newtheorem{corollary}{Следствие}


\usepackage{geometry} % Меняем поля страницы
\geometry{left=1cm}% левое поле
\geometry{right=1cm}% правое поле
\geometry{top=1.5cm}% верхнее поле
\geometry{bottom=1cm}% нижнее поле


\usepackage{fancyhdr} % Headers and footers
\pagestyle{fancy} % All pages have headers and footers
\fancyhead{} % Blank out the default header
\fancyfoot{} % Blank out the default footer
\fancyhead[L]{Математика}
\fancyhead[C]{\textit{Разнобой}}
\fancyhead[R]{4 декабря 2023}% Custom header text


%----------------------------------------------------------------------------------------

%\begin{document}\normalsize
\begin{document}\large
	
\begin{center}
	\textbf{Разнобой}
\end{center}


\begin{enumerate}[label*=\protect\fbox{\arabic{enumi}}]

\item На Васильевском острове 100 перекрёстков, и из каждого выходит ровно 4 дороги. А каждая дорога соединяет 2 перекрёстка. Сколько всего дорог на васильевском острове?

\item В городе Калининград 15 телефонов. Можно ли их соединить проводами так, чтобы было четыре телефона, каждый из которых соединен с тремя другими, восемь телефонов, каждый из которых соединен с шестью, и три телефона, каждый из которых соединен с пятью другими?

\item В кружке художественного свиста у каждого ровно один друг и ровно один враг. Докажите, что в кружке четное число людей.

\item В стране Цифра есть 9 городов с названиями 1, 2, 3, 4, 5, 6, 7, 8, 9. Путешественник обнаружил, что два города соединены авиалинией в том и только в том случае, если двузначное число, составленное из цифр-названий этих городов, делится на 3. Можно ли добраться из города 1 в город 9?

\item В кружке художественного свиста у каждого ровно один друг и ровно один враг среди членов кружка.
Докажите, что кружок можно разделить на два кружка равной численности так, чтобы ни в одном из них не было ни друзей, ни врагов.

\item 20 школьников решили 20 задач, причем каждую задачу решило ровно 2 школьника, и каждый школьник решил ровно 2 задачи. Докажите, что Тимофей Дмитриевич может организовать разбор так, чтобы каждая задача была рассказана ровно по одному разу, а каждый школьник рассказал ровно одну из решенных им задач.

\end{enumerate}


\end{document}