\documentclass{article}

\usepackage[12pt]{extsizes}
\usepackage[T2A]{fontenc}
\usepackage[utf8]{inputenc}
\usepackage[english, russian]{babel}

\usepackage{mathrsfs}
\usepackage[dvipsnames]{xcolor}

\usepackage{amsmath}
\usepackage{amssymb}
\usepackage{amsthm}
\usepackage{indentfirst}
\usepackage{amsfonts}
\usepackage{enumitem}
\usepackage{graphics}
\usepackage{tikz}
\usepackage{tabu}
\usepackage{diagbox}
\usepackage{hyperref}
\usepackage{mathtools}
\usepackage{ucs}
\usepackage{lipsum}
\usepackage{geometry} % Меняем поля страницы
\usepackage{fancyhdr} % Headers and footers
\newcommand{\range}{\mathrm{range}}
\newcommand{\dom}{\mathrm{dom}}
\newcommand{\N}{\mathbb{N}}
\newcommand{\R}{\mathbb{R}}
\newcommand{\E}{\mathbb{E}}
\newcommand{\D}{\mathbb{D}}
\newcommand{\M}{\mathcal{M}}
\newcommand{\Prime}{\mathbb{P}}
\newcommand{\A}{\mathbb{A}}
\newcommand{\Q}{\mathbb{Q}}
\newcommand{\Z}{\mathbb{Z}}
\newcommand{\F}{\mathbb{F}}
\newcommand{\CC}{\mathbb{C}}

\DeclarePairedDelimiter\abs{\lvert}{\rvert}
\DeclarePairedDelimiter\floor{\lfloor}{\rfloor}
\DeclarePairedDelimiter\ceil{\lceil}{\rceil}
\DeclarePairedDelimiter\lr{(}{)}
\DeclarePairedDelimiter\set{\{}{\}}
\DeclarePairedDelimiter\norm{\|}{\|}

\renewcommand{\labelenumi}{(\alph{enumi})}

\newcommand{\smallindent}{
    \geometry{left=1cm}% левое поле
    \geometry{right=1cm}% правое поле
    \geometry{top=1.5cm}% верхнее поле
    \geometry{bottom=1cm}% нижнее поле
}

\newcommand{\header}[3]{
    \pagestyle{fancy} % All pages have headers and footers
    \fancyhead{} % Blank out the default header
    \fancyfoot{} % Blank out the default footer
    \fancyhead[L]{#1}
    \fancyhead[C]{#2}
    \fancyhead[R]{#3}
}

\newcommand{\dividedinto}{
    \,\,\,\vdots\,\,\,
}

\newcommand{\littletaller}{\mathchoice{\vphantom{\big|}}{}{}{}}

\newcommand\restr[2]{{
    \left.\kern-\nulldelimiterspace % automatically resize the bar with \right
    #1 % the function
    \littletaller % pretend it's a little taller at normal size
    \right|_{#2} % this is the delimiter
}}

\DeclareGraphicsExtensions{.pdf,.png,.jpg}

\newenvironment{enumerate_boxed}[1][enumi]{\begin{enumerate}[label*=\protect\fbox{\arabic{#1}}]}{\end{enumerate}}



\smallindent

\header{Математика}{\textit{Алгебра}}{17 мая 2024}

%----------------------------------------------------------------------------------------

\begin{document}
    \large

    \begin{center}
        \textbf{Периодические функции}
    \end{center}

    \begin{enumerate_boxed}

        \item Функции $f$ и $g$ определены на всей числовой прямой и взаимно обратны.
        Известно, что $f$ представляется в виде суммы линейной и периодической функций: $f(x) = kx + h(x)$,  где $k$~--- число, $h$~--- периодическая функция.
        Доказать, что $g$ также представляется в таком виде.

        \item Функция $f(x)$, определённая при всех действительных
        $x$, является чётной.
        Кроме того, при любом действительном $x$ выполняется равенство
        \[f(x) + f(10 - x) = 4.\]
        \begin{enumerate}
            \item Приведите пример такой функции, отличной от константы.
            \item Докажите, что любая такая функция является периодической.
        \end{enumerate}

        \item Найдите все периодические функции $y = f(x)$, удовлетворяющие уравнению
        \[f(x) - 0,5f(x - \pi) = \sin x.\]

        \item Показать, что если функция $f(x)$ определена при всех действительных $x$ и \[f(x + a) = \frac{1 + f(x)}{1 - f(x)}\] при некотором $a > 0$, то эта функция — периодическая, и найти ее период.

        \item Доказать, что производная периодической функции — периодическая функция.

        \item Каждое следующее число в последовательности целых чисел получается из предыдущего так: число возводится в квадрат, из него вычеркиваются все цифры, кроме последних
        четырёх.
        Докажите, что последовательность периодическая (возможно, с предпериодом),
        причём длина периода не больше (а) 10000; (б) 625.

        \item Дана бесконечная последовательность чисел $a_1, a_2, a_3, \dotsc$
        Известно, что для любого номера $k$ можно указать такое натуральное число $t$, что $a_k = a_{k+t} = a_{k+2t} = \dotsc$
        Обязательно ли тогда эта последовательность периодическая?

        \item Дана бесконечная вправо последовательность букв русского алфавита.
        Известно, что в ней
        различных подслов длины 100 столько же, сколько различных подслов длины 101.
        Докажите, что последовательность периодична, возможно, с предпериодом.

        \item На проволоку в форме окружности насажено несколько разноцветных шариков.
        В некоторый момент шарики начинают двигаться с одинаковыми скоростями: некоторые по часовой стрелке, а некоторые против.
        Сталкиваясь, шарики разлетаются с теми же скоростями в противоположные стороны.
        Докажите, что рано или поздно расположение шариков на окружности повторится с исходным.
    \end{enumerate_boxed}
\end{document}