\documentclass{article}
\usepackage[12pt]{extsizes}
\usepackage[T2A]{fontenc}
\usepackage[utf8]{inputenc}
\usepackage[english, russian]{babel}

\usepackage{amssymb}
\usepackage{amsfonts}
\usepackage{amsmath}
\usepackage{enumitem}
\usepackage{graphics}
\usepackage{graphicx}

\usepackage{lipsum}

\newtheorem{theorem}{Теорема}
\newtheorem{task}{Задача}
\newtheorem{lemma}{Лемма}
\newtheorem{definition}{Определение}
\newtheorem{example}{Пример}
\newtheorem{statement}{Утверждение}
\newtheorem{corollary}{Следствие}


\usepackage{geometry} % Меняем поля страницы
\geometry{left=1cm}% левое поле
\geometry{right=1cm}% правое поле
\geometry{top=1.5cm}% верхнее поле
\geometry{bottom=1cm}% нижнее поле


\usepackage{fancyhdr} % Headers and footers
\pagestyle{fancy} % All pages have headers and footers
\fancyhead{} % Blank out the default header
\fancyfoot{} % Blank out the default footer
\fancyhead[L]{Математика}
\fancyhead[C]{\textit{Олимпиадная подготовка}}
\fancyhead[R]{10 октября 2023}% Custom header text


%----------------------------------------------------------------------------------------

%\begin{document}\normalsize
\begin{document}\large
	
\begin{center}
	\textbf{Занятие 3}
\end{center}


\begin{enumerate}[label*=\protect\fbox{\arabic{enumi}}]
	
\item Докажите, что уравнение $$\frac{1}{x} + \frac{1}{y} = \frac{1}{7}$$ имеет конечное число решений в целых числах.
	
\item Докажите, что уравнение $$\frac{1}{x} + \frac{1}{y} + \frac{1}{z} = \frac{1}{7}$$ имеет конечное число решений в натуральных числах.

\item Любые два из тысячи квадратных трёхчленов имеют общий корень. Верно ли, что они все имеют общий корень?

\item Докажите, что для любого $n > 2$ можно придумать $n$ натуральных чисел так, чтобы сумма квадратов первых $n - 1$ числа равнялась квадрату последнего.

\item Докажите, что биссектрисы двух внутренних углов треугольника и биссектриса
внешнего угла, не смежного с ними, пересекают прямые, содержащие соответственные стороны треугольника, в трех коллинеарных точках.

\item Высоты остроугольного треугольника $ABC$ пересекаются в точке $H$. Окружность,
описанная около треугольника $ABH$, пересекает окружность, построенную на отрезке $AC$ как на диаметре, в точке $K$, отличной от $A$. Докажите, что прямая $CK$ делит
отрезок $BH$ пополам.
	
\end{enumerate}
\end{document}