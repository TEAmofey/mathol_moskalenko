\documentclass{article}
\usepackage[12pt]{extsizes}
\usepackage[T2A]{fontenc}
\usepackage[utf8]{inputenc}
\usepackage[english, russian]{babel}

\usepackage{amssymb}
\usepackage{amsfonts}
\usepackage{amsmath}
\usepackage{enumitem}
\usepackage{graphics}
\usepackage{graphicx}

\usepackage{lipsum}

\newtheorem{theorem}{Теорема}
\newtheorem{task}{Задача}
\newtheorem{lemma}{Лемма}
\newtheorem{definition}{Определение}
\newtheorem{example}{Пример}
\newtheorem{statement}{Утверждение}
\newtheorem{corollary}{Следствие}


\usepackage{geometry} % Меняем поля страницы
%\geometry{left=1cm}% левое поле
%\geometry{right=1cm}% правое поле
\geometry{top=3cm}% верхнее поле
%\geometry{bottom=1cm}% нижнее поле


\usepackage{fancyhdr} % Headers and footers
\pagestyle{fancy} % All pages have headers and footers
\fancyhead{} % Blank out the default header
\fancyfoot{} % Blank out the default footer
\fancyhead[L]{\textit{\textbf{XIII Олимпиада Эйлера}}}
\fancyhead[C]{}
\fancyhead[R]{26 декабря}% Custom header text


%----------------------------------------------------------------------------------------

%\begin{document}\normalsize
\begin{document}\large
	
\begin{center}
	\LARGE\textbf{8 класс}
\end{center}
\begin{center}
	\large\textbf{Второй день}
\end{center}


\begin{enumerate}[label*=8.{\arabic{enumi}}]
\setcounter{enumi}{5}
\item У уголка из трёх клеток центральной назовём клетку, соседнюю по стороне с двумя другими. Существует ли клетчатая фигура, которую можно разбить на уголки из трех клеток тремя способами так, чтобы каждая ее клетка в одном из разбиений была центральной в своем уголке?
\item Точка $M$~--- середина стороны $AC$ равностороннего треугольника $ABC$. Точки $P$ и $R$ на отрезках $AM$ и $BC$ соответственно выбраны так, что $AP = BR$. Найдите сумму углов $ARM$, $PBM$ и $BMR$.
\item Сначала Саша прямолинейными разрезами, каждый из которых соединяет две точки на сторонах квадрата, делит квадрат со стороной 2 на 2020 частей. Затем Дима вырезает из каждой части по кругу. Докажите, что Дима всегда может добиться того, чтобы сумма радиусов этих кругов была не меньше 1.
\item Дано натуральное число $n$, большее 2. Докажите, что если число $n!+n^3+1$ — простое, то число $n^2+2$ представляется в виде суммы двух простых чисел.
\item В квадратной таблице $2021 \times 2021$ стоят натуральные числа. Можно выбрать любой столбец или любую строку в таблице и выполнить одно из следующих действий: 

1) Прибавить к каждому выбранному числу 1. 

2) Разделить каждое из выбранных чисел на какое-нибудь натуральное число. 

Можно ли за несколько таких действий добиться того, чтобы каждое число в таблице было равно 1?

\end{enumerate}
\end{document}