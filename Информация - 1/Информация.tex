 \documentclass{article}
\usepackage[12pt]{extsizes}
\usepackage[T2A]{fontenc}
\usepackage[utf8]{inputenc}
\usepackage[english, russian]{babel}

\usepackage{amssymb}
\usepackage{amsfonts}
\usepackage{amsmath}
\usepackage{enumitem}
\usepackage{graphics}

\usepackage{lipsum}



\usepackage{geometry} % Меняем поля страницы
\geometry{left=1cm}% левое поле
\geometry{right=1cm}% правое поле
\geometry{top=1.5cm}% верхнее поле
\geometry{bottom=1cm}% нижнее поле

\newtheorem{definition}{Опредление}
\usepackage{fancyhdr} % Headers and footers
\pagestyle{fancy} % All pages have headers and footers
\fancyhead{} % Blank out the default header
\fancyfoot{} % Blank out the default footer
\fancyhead[L]{Математика}
\fancyhead[C]{\textit{Комбинаторика}}
\fancyhead[R]{20 октября 2022}% Custom header text


%----------------------------------------------------------------------------------------

%\begin{document}\normalsize
\begin{document}\large
	
	
\begin{center}
	\textbf{Информация и весы}
\end{center}

\textbf{Загадка:} Я загадал целое число от 1 до 3. На один вопрос я могу ответить <<Да>>, <<Нет>> и <<Не знаю>>. За один вопрос угадайте, какое число я загадал.
%Вопрос: верно ли, что любое совершенное число делится на загаданное Вами?


\begin{enumerate}[label*=\protect\fbox{\arabic{enumi}}]

\item Есть (а) 3; (б) 9 монет, из которых ровно одна более лёгкая фальшивка. За наименьшее число взвешиваний на чашечных весах найдите её.


\item Есть 4 гири разных масс, за наименьшее число взвешиваний на чашечных весах упорядочите их по массе.

\item Есть 5 гирь разных масс, за наименьшее число взвешиваний на чашечных весах упорядочите их по массе.

\item Имеется (a) 5; (b) 6 с виду одинаковых шаров, из которых два радиоактивные. Дозиметром можно проверить на радиоактивность любую группу шаров. За какое наименьшее число проверок можно выявить оба радиоактивных шара?

\item Среди $81$ монеты есть только золотые и серебряные. Одна из этих монет фальшивая. Все настоящие монеты весят одинаково, а фальшивая монета тяжелее настоящей, если она золотая, и легче, если она серебряная. Какое наименьшее количество взвешиваний на двухчашечных весах без гирь необходимо, чтобы однозначно определить фальшивую монету.

\item Есть 10 мешков монет (в любом либо настоящие, весят по 10 г; либо фальшивые, по 11 г). Как за наименьшее число взвешиваний на весах со стрелкой узнать, в каких из них фальшивые монеты, если:

а) мешок с фальшивками ровно один;

б) неизвестно число мешков с фальшивками.

\item Даны $n > 1$ чашечных весов без гирь, из которых ровно одни сломаны: их показания произвольны. К сожалению, не известно, какие именно весы неисправны.

(a) Докажите, что из $3^k$ монет нельзя гарантированно определить фальшивую монету за $k$ взвешиваний.

(b) Докажите, что из $3^k$ монет нельзя гарантированно определить фальшивую монету за $k + 1$ взвешиваний.

\item При каком наименьшем $n$ среди $n$ весов, из которых ровно $k$ сломанных, можно из $10$ монет определить одну фальшивую (количество взвешиваний не ограничено)?

\item Даны трое чашечных весов без гирь, из которых ровно одни сломаны: их показания произвольны, и мы не знаем, какие весы неисправны. Докажите, что из $3k$ монет можно определить одну фальшивую (легче настоящих) не более, чем за $2k + 1$ взвешивание.

\item У Васи есть два запасных телефона Nokia 3310, которые ему не жалко. В васином городе есть 101-этажный небоскреб, с которого Вася может скидывать свои телефоны. Какое наименьшее количество скидываний потребуется Васе, чтобы узнать, упав с какого этажа Nokia 3310 перестает работать навсегда, или убедиться в исключительной прочности телефона?

%\item Фокусник и ассистент показывают фокус. Пока фокусника нет, зритель выкладывает в ряд 6 монет, после чего ассистент закрывает непрозрачной тканью $k$ монет. Наконец входит фокусник, который должен угадать, какой стороной вверх лежат закрытые монеты. При каком наибольшем $k$ ассистент и фокусник могут договориться, чтобы фокус удался?

%\item В финале телешоу требуется угадать число от 1 до 144 с помощью вопросов, на которые ведущий отвечает <<Да>> или <<Нет>>. Однако за каждый ответ <<Да>> нужно платить по 1000 рублей из выигранных денег, а за ответ <<Нет>> --- 2000. Какой наименьшей суммы участник может лишиться, чтобы гарантированно угадать число и выиграть суперприз?

%\item
%\begin{enumerate}
	
	%\item Есть 15 монет, одна из которых фальшивая. Все настоящие монеты весят одинаково, а фальшивая весит иначе, но неизвестно, тяжелее она или легче. Какое наименьшее количество взвешиваний на двухчашечных весах без гирь необходимо, чтобы гарантированно найти фальшивую монету и сказать, тяжелее она или легче?
	
	%\item Та же задача, но теперь не надо говорить тяжелее фальшивая монета или легче.
	
	%\item Та же задача, но теперь про одну из монет вам известно, что она настоящая, и не надо говорить тяжелее фальшивая монета или легче. Изменится ли ответ, если отобрать у вас гарантированно настоящую монету?
	
%\end{enumerate}


%\item Алиса и ее младший брат Боб играют в игру. Боб загадывает число от 1 до 1000, а Алиса пытается его угадать. Алиса называет Бобу число, а Боб говорит, верно ли, что оно больше загаданного. Алиса знает, что Боб, чтобы запутать Алису, может соврать один раз за игру (а может и не соврать). За какое наименьшее количество вопросов Алиса может гарантированно угадать загаданное Бобом число?

\end{enumerate}

\end{document}