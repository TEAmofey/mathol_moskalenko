\documentclass{article}

\usepackage[12pt]{extsizes}
\usepackage[T2A]{fontenc}
\usepackage[utf8]{inputenc}
\usepackage[english, russian]{babel}

\usepackage{mathrsfs}
\usepackage[dvipsnames]{xcolor}

\usepackage{amsmath}
\usepackage{amssymb}
\usepackage{amsthm}
\usepackage{indentfirst}
\usepackage{amsfonts}
\usepackage{enumitem}
\usepackage{graphics}
\usepackage{tikz}
\usepackage{tabu}
\usepackage{diagbox}
\usepackage{hyperref}
\usepackage{mathtools}
\usepackage{ucs}
\usepackage{lipsum}
\usepackage{geometry} % Меняем поля страницы
\usepackage{fancyhdr} % Headers and footers
\usepackage[framemethod=TikZ]{mdframed}

\newcommand{\definebox}[3]{%
    \newcounter{#1}
    \newenvironment{#1}[1][]{%
        \stepcounter{#1}%
        \mdfsetup{%
            frametitle={%
            \tikz[baseline=(current bounding box.east),outer sep=0pt]
            \node[anchor=east,rectangle,fill=white]
            {\strut #2~\csname the#1\endcsname\ifstrempty{##1}{}{##1}};}}%
        \mdfsetup{innertopmargin=1pt,linecolor=#3,%
            linewidth=3pt,topline=true,
            frametitleaboveskip=\dimexpr-\ht\strutbox\relax,}%
        \begin{mdframed}[]
            \relax%
            }{
        \end{mdframed}}%
}

\definebox{theorem_boxed}{Теорема}{ForestGreen!24}
\definebox{definition_boxed}{Определение}{blue!24}
\definebox{task_boxed}{Задача}{orange!24}
\definebox{paradox_boxed}{Парадокс}{red!24}

\theoremstyle{plain}
\newtheorem{theorem}{Теорема}
\newtheorem{task}{Задача}
\newtheorem{lemma}{Лемма}
\newtheorem{statement}{Утверждение}
\newtheorem{corollary}{Следствие}

\theoremstyle{remark}
\newtheorem{remark}{Замечание}
\newtheorem{example}{Пример}
\newcommand{\range}{\mathrm{range}}
\newcommand{\dom}{\mathrm{dom}}
\newcommand{\N}{\mathbb{N}}
\newcommand{\R}{\mathbb{R}}
\newcommand{\E}{\mathbb{E}}
\newcommand{\D}{\mathbb{D}}
\newcommand{\M}{\mathcal{M}}
\newcommand{\Prime}{\mathbb{P}}
\newcommand{\A}{\mathbb{A}}
\newcommand{\Q}{\mathbb{Q}}
\newcommand{\Z}{\mathbb{Z}}
\newcommand{\F}{\mathbb{F}}
\newcommand{\CC}{\mathbb{C}}

\DeclarePairedDelimiter\abs{\lvert}{\rvert}
\DeclarePairedDelimiter\floor{\lfloor}{\rfloor}
\DeclarePairedDelimiter\ceil{\lceil}{\rceil}
\DeclarePairedDelimiter\lr{(}{)}
\DeclarePairedDelimiter\set{\{}{\}}
\DeclarePairedDelimiter\norm{\|}{\|}

\renewcommand{\labelenumi}{(\alph{enumi})}

\newcommand{\smallindent}{
    \geometry{left=1cm}% левое поле
    \geometry{right=1cm}% правое поле
    \geometry{top=1.5cm}% верхнее поле
    \geometry{bottom=1cm}% нижнее поле
}

\newcommand{\header}[3]{
    \pagestyle{fancy} % All pages have headers and footers
    \fancyhead{} % Blank out the default header
    \fancyfoot{} % Blank out the default footer
    \fancyhead[L]{#1}
    \fancyhead[C]{#2}
    \fancyhead[R]{#3}
}

\newcommand{\dividedinto}{
    \,\,\,\vdots\,\,\,
}

\newcommand{\littletaller}{\mathchoice{\vphantom{\big|}}{}{}{}}

\newcommand\restr[2]{{
    \left.\kern-\nulldelimiterspace % automatically resize the bar with \right
    #1 % the function
    \littletaller % pretend it's a little taller at normal size
    \right|_{#2} % this is the delimiter
}}

\DeclareGraphicsExtensions{.pdf,.png,.jpg}

\newenvironment{enumerate_boxed}[1][enumi]{\begin{enumerate}[label*=\protect\fbox{\arabic{#1}}]}{\end{enumerate}}



\smallindent

\header{Математика}{\textit{Теория вероятностей}}{16 апреля 2022}

%----------------------------------------------------------------------------------------

\begin{document}
    \large

    \begin{center}
        \textbf{Случайные события}
    \end{center}


    \begin{enumerate_boxed}

        \item Брошено 6 игральных кубиков.
        Найти вероятность следующих событий:
        \begin{itemize}
            \item $A= \{\text{среди выпавших нет единиц}\}.$
            \item $B= \{\text{выпало ровно 3 двойки}\}.$
            \item $C= \{\text{выпала хотя бы 1 единица}\}.$
            \item $D= \{\text{все цифры выпали хотя бы по одному разу}\}.$
        \end{itemize}

        \item Карточка <<спортлото>> содержит 36 чисел.
        Игрок может выбрать 6, а выигрышных номеров в тираже определяется тоже 6.
        Какова вероятность того, что верно будет угадано ровно 3 числа?

        \item Подбрасывают симметричную монету.
        \begin{itemize}
            \item С какой вероятностью при $n$ подбрасываниях выпадет ровно $k$ орлов?
            \item Какое $k$ наиболее вероятно?
        \end{itemize}

        \item  При бросании неправильной монеты орел выпадает с вероятностью $p$, решка — с вероятностью $q=  1-p$.
        С какой вероятностью после $n$ бросков выпадет четное число орлов?

        \item Из колоды карт (52 карты) наугад вытаскивают 5.
        Что более вероятно — вытащить ровно3 карты одного номинала или вытащить ровно 2 пары карт одного номинала?

        \item Колоду карт случайным образом делят на 2 части (необязательно равного размера).
        С какой вероятностью в каждой части будет по 2 туза?

        \item Из контейнера $A$, в котором было 1000 зеленых яблок и 3000 красных яблок, взяли половину яблок и перенесли в контейнер $B$, в котором к тому времени уже лежало 3000 зеленых и 1000 красных яблок.
        Затем из контейнера $B$ извлекли одно яблоко.
        Найти вероятность того, что оно зеленое.

        \item Рассмотрим простейшее случайное блуждание на прямой: частица находится в 0 и каждым шагом сдвигается на 1 вправо или влево.
        Всего сделано $n$ шагов, все траектории равновероятны.
        Найти вероятность события \[A_k = \{\text{блуждание завершилось в точке с координатой $k$}\}.\]

        \item Каждый из двух игроков подбрасывает симметричную монету $n$ раз.
        С какой вероятностью у них выпадет одинаковое число орлов?

    \end{enumerate_boxed}
\end{document}