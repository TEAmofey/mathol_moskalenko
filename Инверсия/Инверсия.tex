\documentclass{article}

\usepackage[12pt]{extsizes}
\usepackage[T2A]{fontenc}
\usepackage[utf8]{inputenc}
\usepackage[english, russian]{babel}

\usepackage{mathrsfs}
\usepackage[dvipsnames]{xcolor}

\usepackage{amsmath}
\usepackage{amssymb}
\usepackage{amsthm}
\usepackage{indentfirst}
\usepackage{amsfonts}
\usepackage{enumitem}
\usepackage{graphics}
\usepackage{tikz}
\usepackage{tabu}
\usepackage{diagbox}
\usepackage{hyperref}
\usepackage{mathtools}
\usepackage{ucs}
\usepackage{lipsum}
\usepackage{geometry} % Меняем поля страницы
\usepackage{fancyhdr} % Headers and footers
\newcommand{\range}{\mathrm{range}}
\newcommand{\dom}{\mathrm{dom}}
\newcommand{\N}{\mathbb{N}}
\newcommand{\R}{\mathbb{R}}
\newcommand{\E}{\mathbb{E}}
\newcommand{\D}{\mathbb{D}}
\newcommand{\M}{\mathcal{M}}
\newcommand{\Prime}{\mathbb{P}}
\newcommand{\A}{\mathbb{A}}
\newcommand{\Q}{\mathbb{Q}}
\newcommand{\Z}{\mathbb{Z}}
\newcommand{\F}{\mathbb{F}}
\newcommand{\CC}{\mathbb{C}}

\DeclarePairedDelimiter\abs{\lvert}{\rvert}
\DeclarePairedDelimiter\floor{\lfloor}{\rfloor}
\DeclarePairedDelimiter\ceil{\lceil}{\rceil}
\DeclarePairedDelimiter\lr{(}{)}
\DeclarePairedDelimiter\set{\{}{\}}
\DeclarePairedDelimiter\norm{\|}{\|}

\renewcommand{\labelenumi}{(\alph{enumi})}

\newcommand{\smallindent}{
    \geometry{left=1cm}% левое поле
    \geometry{right=1cm}% правое поле
    \geometry{top=1.5cm}% верхнее поле
    \geometry{bottom=1cm}% нижнее поле
}

\newcommand{\header}[3]{
    \pagestyle{fancy} % All pages have headers and footers
    \fancyhead{} % Blank out the default header
    \fancyfoot{} % Blank out the default footer
    \fancyhead[L]{#1}
    \fancyhead[C]{#2}
    \fancyhead[R]{#3}
}

\newcommand{\dividedinto}{
    \,\,\,\vdots\,\,\,
}

\newcommand{\littletaller}{\mathchoice{\vphantom{\big|}}{}{}{}}

\newcommand\restr[2]{{
    \left.\kern-\nulldelimiterspace % automatically resize the bar with \right
    #1 % the function
    \littletaller % pretend it's a little taller at normal size
    \right|_{#2} % this is the delimiter
}}

\DeclareGraphicsExtensions{.pdf,.png,.jpg}

\newenvironment{enumerate_boxed}[1][enumi]{\begin{enumerate}[label*=\protect\fbox{\arabic{#1}}]}{\end{enumerate}}



\smallindent

\header{ЦРОД $\bullet$ Математика}{\textit{Геометрия}}{ЛФМШ 2022}

%----------------------------------------------------------------------------------------

\begin{document}
    \large

    \begin{center}
        \textbf{Инверсия}
    \end{center}

    \textbf{Определение} \textit{Инверсией} относительно окружности $S(O; R)$ называют преобразование, переводящее произвольную точку $A$, отличную от $O$, в точку $A'$, лежащую на луче $OA$ такую, что $OA \cdot OA' = R^2$.

    Отметим, что если при инверсии точка $X$ переходит в точку $Y$, то $Y$ переходит в $X$.

    Инверсию относительно $S$ будем также называть инверсией с центром $O$ и степенью $R^2$, а окружность $S$ — \textit{окружностью инверсии.}

    \begin{enumerate_boxed}

        \item Докажите, что при инверсии относительно окружности $\omega$ с центром $O$
        \begin{enumerate}[label=\alph*)]
            \item точка $M$, лежащая внутри окружности $\omega$, переходит в точку $M'$, лежащую снаружи;
            \item прямая, проходящая через $O$, переходит в себя.
        \end{enumerate}

        \item Пусть при инверсии с центром $O$ точка $A$ переходит в $A'$, а точка $B$ — в $B'$. Доказать:
        \begin{enumerate}[label=\alph*)]
            \item треугольники $OAB$ и $OB'A'$ подобны;
            \item точки $A$, $B$, $A'$ и $B'$ лежат на одной окружности.
        \end{enumerate}

        \item Докажите, что при инверсии с центром $O$:
        \begin{enumerate}[label=\alph*)]
            \item прямая, не проходящая через $O$, переходит в окружность, проходящую через $O$;
            \item окружность, не проходящая через $O$, переходит в окружность, не про- ходящую через $O$.
        \end{enumerate}

        \item Точки $A$ и $B$ лежат на окружности $\omega$.
        Что является образом прямой $AB$ при инверсии относительно $\omega$?

        \item Докажите, что касающиеся окружности (окружность и прямая) переходят при инверсии в касающиеся окружности или в касающиеся окружность и прямую, или в пару параллельных прямых.

        \item Докажите, что инверсия с центром в вершине $A$ равнобедренного треугольника $ABC$ ($AB = AC$) и степенью $AB^2$ переводит основание $BC$ треугольника в дугу $BC$ описанной окружности.

        \item Точки $X'$ и $Y'$ — образы точек $X$ и $Y$ при инверсии относительно окружности с центром $O$ радиуса $R$, причём точки $X$ и $Y$ отличны от $O$.
        Докажите, что $X'Y' =XY \cdot \dfrac{R^2}{OX\cdot OY}$.

        \item Пусть окружность $\omega$ вписана в угол $BAC$, $B$ и $C$ — точки касания $\omega$ со лучами $AB$ и $AC$.
        Докажите, что точка $A$ при инверсии относительно $\omega$ переходит в середину отрезка $BC$.

        \item Четырёхугольник $ABCD$ вписан в окружность с центром в точке $O$.
        Окружности, описанные около треугольников $AOB$ и $COD$, вторично пересекаются в точке $Y$, прямые $AB$ и $CD$ пересекаются в точке $X$.
        Докажите, что точки $X$, $O$ и $Y$ лежат на одной прямой.

        \item В сегмент вписываются всевозможные пары пересекающихся окружностей, и для каждой пары через точки их пересечения проводится прямая.
        Докажите, что все эти прямые проходят через одну точку.

        \item В сегмент вписываются всевозможные пары касающихся окружностей, точки касания отмечаются.
        Докажите, что все отмеченные точки лежат на одной окружности.

        \item Что является образом описанной окружности треугольника при инверсии относительно вписанной окружности?

        \item Две окружности пересекаются в точках $A$ и $B$.
        Их общая касательная (та, которая ближе к точке $B$) касается окружностей в точках $E$ и $F$.
        Прямая $AB$ пересекает прямую $EF$ в точке $M$.
        На продолжении $AM$ за точку $M$ выбрана точка $K$ так, что $KM = MA$.
        Прямая $KE$ вторично пересекает окружность, содержащую точку $E$, в точке $C$.
        Прямая $KF$ вторично пересекает окружность, содержащую точку $F$, в точке $D$.
        Докажите, что точки $C$, $D$ и $A$ лежат на одной прямой.

        \item Пусть $AH$ — высота остроугольного треугольника $ABC$, а точки $K$ и $L$ — проекции $H$ на стороны $AB$ и $AC$.
        Описанная окружность $\Omega$ треугольника $ABC$ пересекает прямую $KL$ в точках $P$ и $Q$, а прямую $AH$ — в точках $A$ и $T$.
        Докажите, что точка $H$ является центром вписанной окружности треугольника $PQT$.

        \item Четырёхугольник $ABCD$ вписан в окружность $\Omega$ с центром $O$, причём $O$ не лежит на диагоналях четырёхугольника.
        Описанная окружность $\Omega_1$ треугольника $AOC$ проходит через середину диагонали $BD$.
        Докажите, что описанная окружность $\Omega_2$ треугольника $BOD$ проходит через середину диагонали $AC$.

        \item В угол $\alpha$ вписаны окружности $\omega$ и $\Omega$, причём окружность $\Omega$ проходит через центр окружности $\omega$ и касается сторон угла $\alpha$ в точках $P$ и $Q$.
        Докажите, что $PQ$ касается $\omega$.

        \item В треугольнике $A_{1}A_{2}A_3$ провели окружности $\omega_1$, $\omega_2$, $\omega_3$, вписанные в углы $\angle A_1$, $\angle A_2$ и $\angle A_3$ соответственно и проходящие через центр вписанной окружности $I$.
        Эти окружности вторично пересекаются в точках $B_1$, $B_2$ и $B_3$ ($B_i$ не лежит на $\omega_i$). Докажите, что центры описанных окружностей треугольников $A_{i}B_{i}I$ лежат на одной прямой.

%(Всеросс-2013, 11.8)
        \item В треугольник $ABC$ вписана окружность $\omega$ с центром в точке $I$.
        Около треугольника $AIB$ описана окружность $\Gamma$.
        Окружности $\omega$ и $\Gamma$ пересекаются в точках $X$ и $Y$ .
        Общие касательные к окружностям $\omega$ и $\Gamma$ пересекаются в точке $Z$.
        Докажите, что описанные окружности треугольников $ABC$ и $XYZ$ касаются.

        \item Пусть $O$ — одна из точек пересечения окружностей $\omega_1$ и $\omega_2$.
        Окружность $\omega$ с центром $O$ пересекает $\omega_1$ в точках $A$ и $B$,а $\omega_2$ — в точках $C$ и $D$.
        Пусть $X$— точка пересечения прямых $AC$ и $BD$.
        Докажите, что все такие точки $X$ лежат на одной прямой.

        \item Четырёхугольник $ABCD$ описан около окружности с центром $I$.
        Касательные к описанной окружности треугольника $AIC$ в точках $A$, $C$ пересекаются в точке $X$.
        Касательные к описанной окружности треугольника $BID$ в точках $B$, $D$ пересекаются в точке $Y$.
        Докажите, что точки $X$, $I$, $Y$ лежат на одной прямой.

        \item В четырёхугольнике $ABCD$ вписанная окружность $\omega$ касается сторон $BC$ и $DA$ в точках $E$ и $F$ соответственно.
        Оказалось, что прямые $AB$, $FE$ и $CD$ пересекаются в одной точке $S$.
        Описанные окружности $\Omega$ и $\Omega_1$ треугольников $AED$ и $BFC$, вторично пересекают окружность $\omega$ в точках $E_1$ и $F_1$.
        Докажите, что прямые $EF$ и $E_{1}F_1$ параллельны.

    \end{enumerate_boxed}
\end{document}