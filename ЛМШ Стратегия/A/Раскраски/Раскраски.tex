\documentclass[b5paper,usehyperref, twoside]{article}
\usepackage[utf8]{inputenc}
\usepackage[T2A]{fontenc}
\usepackage[russian]{babel}
\usepackage{indentfirst}
\usepackage{amssymb}
\usepackage{amsmath}
\usepackage{tikz}
\usepackage{amsthm}
\usepackage{hyperref}
\usepackage{eurosym}
\usepackage{graphicx}
\usepackage{enumitem}
\usepackage{icomma}
\usepackage{hhline}
\newcommand{\sol}{\textbf{Решение.\ }}
\newcommand{\dzz}{\medskip\textbf{Для самостоятельного решения\\ }}
\newcommand{\ds}{\displaystyle}
\newcommand{\al}{\alpha}

\newcommand*{\hm}[1]{#1\nobreak\discretionary{}%
	{\hbox{$\mathsurround=0pt #1$}}{}}
%\renewcommand{\labelenumii}{\arabic{enumi}.\arabic{enumii}.}
%\renewcommand{\thefigure}{\thesection.\arabic{figure}}
\graphicspath{{Pics/}}

\newcommand{\be}{\beta}
\newcommand{\ga}{\gamma}

\newcommand{\ra}{\rightarrow}
\newcommand{\Ra}{\Rightarrow}
\newcommand{\Lra}{\Leftrightarrow}
\newcommand{\cd}{\cdot}
\newcommand{\gs}{\geqslant}
\newcommand{\ls}{\leqslant}
\newcommand{\tx}{\text}
\newcommand{\ti}{\times}
\newcommand{\opr}{\textbf{Определение.\ }}
\newcommand{\ov}{\overline}
\newcommand{\df}{\ds\frac}
\newcommand{\an}{\angle}
\newcommand{\tr}{\triangle}
\newcommand{\pa}{\parallel}
\newcommand{\ig}{\includegraphics}
\newcommand{\eq}{\equiv}
\renewcommand{\mod}{\operatorname{mod}}
\newcommand{\ba}{\begin{array}{c}}
	\newcommand{\ea}{\end{array}}
\renewcommand{\sb}{\left\{\ba}
\newcommand{\se}{\ea\right.}
\newcommand{\ovr}{\overrightarrow}
\newcommand{\om}{\omega}
\renewcommand{\tg}{\operatorname{tg}}
\renewcommand{\ctg}{\operatorname{ctg}}
\renewcommand{\phi}{\varphi}
\newcommand{\Z}{\mathbb{Z}}
\newcommand{\de}{\mathop{\raisebox{-2pt}{\vdots}}}
\newcommand{\ub}{\underbrace}
\newcommand{\arc}{\smile\!}

\usepackage{multirow}
\usepackage{epigraph}

\tikzset{
	treenode/.style = {shape=rectangle, rounded corners,
		draw, align=center,
		top color=white, bottom color=blue!20},
	root/.style     = {treenode, font=\Large, bottom color=red!30},
	env/.style      = {treenode, font=\ttfamily\normalsize},
	dummy/.style    = {circle,draw}
}


\usepackage[paperheight=20.5cm,paperwidth=14cm, left=0.6cm,right=0.9cm,top=1.8cm,bottom=1.5cm]{geometry}

\parindent=0pt
\parskip=1pt
\tolerance=1100

\addto\captionsrussian{\renewcommand{\partname}{Раздел}}

\def\thepart{\arabic{part}}

\hypersetup{pdfstartview=FitH,  linkcolor=linkcolor,urlcolor=urlcolor, colorlinks=true}

\definecolor{linkcolor}{HTML}{000000} % цвет ссылок
\definecolor{urlcolor}{HTML}{000000} % цвет гиперссылок


\newtheorem{Lemma}{Лемма}
\theoremstyle{definition}
\newtheorem{Remark}{Замечание}
\newtheorem{Claim}{Утверждение}
\newtheorem{Cor}{Следствие}
\newtheorem{Theorem}{Теорема}
\newtheorem*{known}{Теорема}
\def\proofname{Доказательство}
\theoremstyle{definition}
\newtheorem{Def}{Определение}
%\newcounter{partsec}[section]
%\def\thepartsec{\arabic{part}.\arabic{section}}
\newtheorem{Task}{}


\newtheorem{Example}{}


\usepackage{fancyhdr} %загрузим пакет
\pagestyle{fancy} %применим колонтитул
\fancyhead[LO]{Летняя математическая школа Стратегия, группа A} %очистим хидер на всякий случай
\fancyhead[RO]{2023.07.09}
%\fancyhead[LE,RO]{\thepage} %номер страницы слева сверху на четных и справа на нечетных
%\fancyhead[CO]{текст-центр-нечетные}  %текст-центр-нечетные
%\fancyhead[LO]{текст-слева-нечетные}  %текст-слева-нечетные
%\fancyhead[CE]{текст-центр-четные}    %текст-центр-четные
\fancyfoot{} %футер будет пустой

\begin{document}
	\newcounter{totalpics}
	\newcounter{totaltables}
	%\setcounter{tocdepth}{2}
	
	%
	%\begin{document}
	%\begin{center}
	
	
	%\begin{figure}[bh]
	%\noindent\centering{
		%\includegraphics[width=70mm]{logo}
		% }
	%\label{figCurves}
	%\end{figure}
	%\end{center}
	\centerline{\LARGE Раскраска}
	
	%\begin{center}
	%Преподаватель Подаев М.
	%\end{center}
	
	% \epigraph{Он стал поэтом - для математика у него не хватало фантазии. }{Давид Гильберт}
	%	\centerline{\LARGE }
	
	%\textit{Инвариант} - параметр, который остается неизменным в ходе любой операции.
	
	
\begin{Task}
	Докажите, что клетчатую доску $10 \times 10$ нельзя разрезать по линиям сетки на прямоугольники $1\times 4$.
\end{Task}

\begin{Task}
	На каждой из клеток доски размером $9 \times 9$ находится фишка. Петя хочет передвинуть каждую фишку на соседнюю по стороне клетку так, чтобы снова в каждой из клеток оказалось по одной фишке. Сможет ли Петя это сделать?
\end{Task}

\centerline{\LARGE Основные задачи}
	
\begin{Task}
	Можно ли разбить квадрат $8 \times 8$ с отрезанным уголком на прямоугольники $1 \times 3$
\end{Task}

\begin{Task}
	Можно ли шахматным конем обойти все клетки доски $5 \times 5$, побывав на каждой клетке по одному разу и вернуться последним ходом в исходное положение?
\end{Task}

\begin{Task}
	Мышка грызет куб сыра с ребром 3, разбитый на 27 единичных кубиков. Когда мышка съедает какой-либо кубик, она переходит к другому кубику, имеющему общую грань с предыдущим. Может ли мышка съесть весь куб, кроме центрального кубика?
\end{Task}

\begin{Task}
	На каждой клетке доски $9 \times 9$ сидело по жуку. По сигналу каждый жук переполз на одну из соседних клеток а) по стороне; б) по диагонали. При этом в каких-то клетках могло оказаться несколько жуков, а какие-то могли оказаться пустыми. Найдите наименьшее возможное количество пустых клеток.
\end{Task}

\begin{Task}
	На каждой клетке-треугольничке треугольной доски со стороной 5 сидит жук. В некоторый момент все жуки взлетают и приземляются на соседние (по стороне) клетки этой доски. Докажите, что тогда найдутся по крайней мере 5 пустых клеток.
\end{Task}


\begin{Task}
	На шахматной доске стоят несколько (не менее четырех) королей. Докажите, что их можно разбить на четыре группы так, чтобы короли каждой группы друг друга не били.
\end{Task}

\begin{Task}
	Можно ли клетчатую доску $10 \times 10$ разрезать по линиям сетки на \texttt{T}-тетрамино.
\end{Task}

\centerline{\LARGE Посложнее}

\begin{Task}
	Из квадрата $7\times 7$ по линиям сетки вырезали 8 квадратов $2 \times 2$. Докажите, что можно вырезать еще один квадрат $2 \times 2$.
\end{Task}

\begin{Task}
	Можно ли клетчатую доску $10 \times 10$ разрезать по линиям сетки на \texttt{L}-тетрамино.
\end{Task}

%\begin{Task}
%	В левом нижнем углу доски $9 \times 9$ стоят 9 шашек, образуя квадрат $3 \times 3$. За один ход можно выбрать какие- то две шашки и переставить одну из них симметрично относительно другой (не выходя при этом за пределы доски). Можно ли за несколько ходов переместить эти шашки так, чтобы они образовали квадрат $3 \times 3$: а) в левом верхнем углу; б) в правом верхнем углу; в) в центральном квадрате $3 \times 3$?
%\end{Task}
%
%\begin{Task}
%	Можно ли три попарно соседние грани кубика $4\times 4\times 4$ оклеить 16 полосками $3\times 1$?
%\end{Task}


\begin{Task}
	Можно ли шахматную доску разрезать на 15 вертикальных и 17 горизонтальных доминошек?
\end{Task}

\begin{Task}
	Прямоугольное дно коробки было выложено квадратами $2 \times 2$ и прямоугольниками $1 \times 4$. Один квадрат потеряли и вместо него нашли прямоугольник. Можно ли теперь сложить дно прямоугольной коробки?
\end{Task}
	
\end{document}
