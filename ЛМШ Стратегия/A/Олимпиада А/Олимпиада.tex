\documentclass[b5paper,usehyperref, twoside]{article}
\usepackage[utf8]{inputenc}
\usepackage[T2A]{fontenc}
\usepackage[russian]{babel}
\usepackage{indentfirst}
\usepackage{amssymb}
\usepackage{amsmath}
\usepackage{tikz}
\usepackage{amsthm}
\usepackage{hyperref}
\usepackage{eurosym}
\usepackage{graphicx}
\usepackage{enumitem}
\usepackage{icomma}
\usepackage{hhline}
\newcommand{\sol}{\textbf{Решение.\ }}
\newcommand{\dzz}{\medskip\textbf{Для самостоятельного решения\\ }}
\newcommand{\ds}{\displaystyle}
\newcommand{\al}{\alpha}

\newcommand*{\hm}[1]{#1\nobreak\discretionary{}%
	{\hbox{$\mathsurround=0pt #1$}}{}}
%\renewcommand{\labelenumii}{\arabic{enumi}.\arabic{enumii}.}
%\renewcommand{\thefigure}{\thesection.\arabic{figure}}
\graphicspath{{Pics/}}

\newcommand{\be}{\beta}
\newcommand{\ga}{\gamma}

\newcommand{\ra}{\rightarrow}
\newcommand{\Ra}{\Rightarrow}
\newcommand{\Lra}{\Leftrightarrow}
\newcommand{\cd}{\cdot}
\newcommand{\gs}{\geqslant}
\newcommand{\ls}{\leqslant}
\newcommand{\tx}{\text}
\newcommand{\ti}{\times}
\newcommand{\opr}{\textbf{Определение.\ }}
\newcommand{\ov}{\overline}
\newcommand{\df}{\ds\frac}
\newcommand{\an}{\angle}
\newcommand{\tr}{\triangle}
\newcommand{\pa}{\parallel}
\newcommand{\ig}{\includegraphics}
\newcommand{\eq}{\equiv}
\renewcommand{\mod}{\operatorname{mod}}
\newcommand{\ba}{\begin{array}{c}}
	\newcommand{\ea}{\end{array}}
\renewcommand{\sb}{\left\{\ba}
\newcommand{\se}{\ea\right.}
\newcommand{\ovr}{\overrightarrow}
\newcommand{\om}{\omega}
\renewcommand{\tg}{\operatorname{tg}}
\renewcommand{\ctg}{\operatorname{ctg}}
\renewcommand{\phi}{\varphi}
\newcommand{\Z}{\mathbb{Z}}
\newcommand{\de}{\mathop{\raisebox{-2pt}{\vdots}}}
\newcommand{\ub}{\underbrace}
\newcommand{\arc}{\smile\!}

\usepackage{multirow}
\usepackage{epigraph}

\tikzset{
	treenode/.style = {shape=rectangle, rounded corners,
		draw, align=center,
		top color=white, bottom color=blue!20},
	root/.style     = {treenode, font=\Large, bottom color=red!30},
	env/.style      = {treenode, font=\ttfamily\normalsize},
	dummy/.style    = {circle,draw}
}


\usepackage[paperheight=20.5cm,paperwidth=14cm, left=0.6cm,right=0.9cm,top=1.8cm,bottom=1.5cm]{geometry}

\parindent=0pt
\parskip=1pt
\tolerance=1100

\addto\captionsrussian{\renewcommand{\partname}{Раздел}}

\def\thepart{\arabic{part}}

\hypersetup{pdfstartview=FitH,  linkcolor=linkcolor,urlcolor=urlcolor, colorlinks=true}

\definecolor{linkcolor}{HTML}{000000} % цвет ссылок
\definecolor{urlcolor}{HTML}{000000} % цвет гиперссылок


\newtheorem{Lemma}{Лемма}
\theoremstyle{definition}
\newtheorem{Remark}{Замечание}
\newtheorem{Claim}{Утверждение}
\newtheorem{Cor}{Следствие}
\newtheorem{Theorem}{Теорема}
\newtheorem*{known}{Теорема}
\def\proofname{Доказательство}
\theoremstyle{definition}
\newtheorem{Def}{Определение}
%\newcounter{partsec}[section]
%\def\thepartsec{\arabic{part}.\arabic{section}}
\newtheorem{Task}{}


\newtheorem{Example}{}


\usepackage{fancyhdr} %загрузим пакет
\pagestyle{fancy} %применим колонтитул
\fancyhead[LO]{Летняя математическая школа Стратегия, группа A} %очистим хидер на всякий случай
\fancyhead[RO]{2023.07.10}
%\fancyhead[LE,RO]{\thepage} %номер страницы слева сверху на четных и справа на нечетных
%\fancyhead[CO]{текст-центр-нечетные}  %текст-центр-нечетные
%\fancyhead[LO]{текст-слева-нечетные}  %текст-слева-нечетные
%\fancyhead[CE]{текст-центр-четные}    %текст-центр-четные
\fancyfoot{} %футер будет пустой
\begin{document}\large
	

\begin{center}
\textbf{Финальная олимпиада}
\end{center}


\begin{Task}
	 Лиза нарисовала на бесконечном листе бумаги несколько окружностей, которые разделили весь лист на части. Покажите, что она может покрасить эти части в белый и чёрный цвета так, чтобы соседние части, это те, что имеют общую дугу границы, были бы разного цвета.
\end{Task}


\begin{Task}
	На каждой клетке-треугольничке треугольной доски со стороной 7 сидит лягушка. В некоторый момент все лягушки прыгают и приземляются на соседние (по стороне) клетки этой доски. Докажите, что тогда найдутся по крайней мере 7 пустых клеток.
\end{Task}

\begin{Task}
	Васе очень сильно понравилась задача про кубик сыра и мышь поэтому, придя домой, он захотел сделать 15 кубических котлет. Дома была всего одна большая сковородка. Котлеты оказались довольно маленькими, так что на одну сковородку помещается целых 10 кубо-котлет. Васе уже не терпится их съесть и он хочет приготовить их как можно быстрее. Он посчитал, что каждую котлету нужно пожарить с каждой стороны в течение одной минуты. Объясните Васе, как ему нужно действовать, и докажите, что быстрее точно не получится.
\end{Task}

\begin{Task}
	Сколько существует различных пятизначных чисел, в которых цифры идут в порядке убывания?
\end{Task}

\begin{Task}
	Сколькими способами можно выбрать две кости домино так, чтобы их можно было приложить друг к другу (то есть, чтобы какое-то число встречалось на обоих костях).
\end{Task}

\begin{Task}
	В стране несколько мегаполисов (из которых выходит по 43 дороги) и несколько провинциальных городов (из которых выходит по 4 дороги). Докажите, что из любого мегаполиса можно проехать в некоторый другой мегаполис.
\end{Task}

\begin{Task}
	На доске написаны числа 2, 3, 9. Разрешается взять любые два из написанных чисел $a$ и $b$, стереть их, а вместо них написать числа $3a-2b$ и $3b-2a$. Можно ли несколькими такими операциями получить числа 5, 6, 7?
\end{Task}


\begin{Task}
	Все жители острова прошли социальный опрос. Некоторые из них заявили, что на острове четное число рыцарей, а остальные — что на острове нечетное число лжецов. Может ли число жителей острова быть равно $2023$? Известно, что хотя бы один рыцарь и хотя бы один лжец на острове есть.
\end{Task}

\end{document}