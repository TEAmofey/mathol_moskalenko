\documentclass[b5paper,usehyperref, twoside]{article}
\usepackage[utf8]{inputenc}
\usepackage[T2A]{fontenc}
\usepackage[russian]{babel}
\usepackage{indentfirst}
\usepackage{amssymb}
\usepackage{amsmath}
\usepackage{tikz}
\usepackage{amsthm}
\usepackage{hyperref}
\usepackage{eurosym}
\usepackage{graphicx}
\usepackage{enumitem}
\usepackage{icomma}
\usepackage{hhline}
\newcommand{\sol}{\textbf{Решение.\ }}
\newcommand{\dzz}{\medskip\textbf{Для самостоятельного решения\\ }}
\newcommand{\ds}{\displaystyle}
\newcommand{\al}{\alpha}

\newcommand*{\hm}[1]{#1\nobreak\discretionary{}%
	{\hbox{$\mathsurround=0pt #1$}}{}}
%\renewcommand{\labelenumii}{\arabic{enumi}.\arabic{enumii}.}
%\renewcommand{\thefigure}{\thesection.\arabic{figure}}
\graphicspath{{Pics/}}

\newcommand{\be}{\beta}
\newcommand{\ga}{\gamma}

\newcommand{\ra}{\rightarrow}
\newcommand{\Ra}{\Rightarrow}
\newcommand{\Lra}{\Leftrightarrow}
\newcommand{\cd}{\cdot}
\newcommand{\gs}{\geqslant}
\newcommand{\ls}{\leqslant}
\newcommand{\tx}{\text}
\newcommand{\ti}{\times}
\newcommand{\opr}{\textbf{Определение.\ }}
\newcommand{\ov}{\overline}
\newcommand{\df}{\ds\frac}
\newcommand{\an}{\angle}
\newcommand{\tr}{\triangle}
\newcommand{\pa}{\parallel}
\newcommand{\ig}{\includegraphics}
\newcommand{\eq}{\equiv}
\renewcommand{\mod}{\operatorname{mod}}
\newcommand{\ba}{\begin{array}{c}}
	\newcommand{\ea}{\end{array}}
\renewcommand{\sb}{\left\{\ba}
\newcommand{\se}{\ea\right.}
\newcommand{\ovr}{\overrightarrow}
\newcommand{\om}{\omega}
\renewcommand{\tg}{\operatorname{tg}}
\renewcommand{\ctg}{\operatorname{ctg}}
\renewcommand{\phi}{\varphi}
\newcommand{\Z}{\mathbb{Z}}
\newcommand{\de}{\mathop{\raisebox{-2pt}{\vdots}}}
\newcommand{\ub}{\underbrace}
\newcommand{\arc}{\smile\!}

\usepackage{multirow}
\usepackage{epigraph}

\tikzset{
	treenode/.style = {shape=rectangle, rounded corners,
		draw, align=center,
		top color=white, bottom color=blue!20},
	root/.style     = {treenode, font=\Large, bottom color=red!30},
	env/.style      = {treenode, font=\ttfamily\normalsize},
	dummy/.style    = {circle,draw}
}


\usepackage[paperheight=20.5cm,paperwidth=14cm, left=0.6cm,right=0.9cm,top=1.8cm,bottom=1.5cm]{geometry}

\parindent=0pt
\parskip=1pt
\tolerance=1100

\addto\captionsrussian{\renewcommand{\partname}{Раздел}}

\def\thepart{\arabic{part}}

\hypersetup{pdfstartview=FitH,  linkcolor=linkcolor,urlcolor=urlcolor, colorlinks=true}

\definecolor{linkcolor}{HTML}{000000} % цвет ссылок
\definecolor{urlcolor}{HTML}{000000} % цвет гиперссылок


\newtheorem{Lemma}{Лемма}
\theoremstyle{definition}
\newtheorem{Remark}{Замечание}
\newtheorem{Claim}{Утверждение}
\newtheorem{Cor}{Следствие}
\newtheorem{Theorem}{Теорема}
\newtheorem*{known}{Теорема}
\def\proofname{Доказательство}
\theoremstyle{definition}
\newtheorem{Def}{Определение}
%\newcounter{partsec}[section]
%\def\thepartsec{\arabic{part}.\arabic{section}}
\newtheorem{Task}{}


\newtheorem{Example}{}


\usepackage{fancyhdr} %загрузим пакет
\pagestyle{fancy} %применим колонтитул
\fancyhead[LO]{Летняя математическая школа Стратегия, группа C} %очистим хидер на всякий случай
\fancyhead[RO]{2023.07.09}
%\fancyhead[LE,RO]{\thepage} %номер страницы слева сверху на четных и справа на нечетных
%\fancyhead[CO]{текст-центр-нечетные}  %текст-центр-нечетные
%\fancyhead[LO]{текст-слева-нечетные}  %текст-слева-нечетные
%\fancyhead[CE]{текст-центр-четные}    %текст-центр-четные
\fancyfoot{} %футер будет пустой

\begin{document}
	\newcounter{totalpics}
	\newcounter{totaltables}
	%\setcounter{tocdepth}{2}
	
	%
	%\begin{document}
	%\begin{center}
	
	
	%\begin{figure}[bh]
	%\noindent\centering{
		%\includegraphics[width=70mm]{logo}
		% }
	%\label{figCurves}
	%\end{figure}
	%\end{center}
	\centerline{\LARGE Принцип Дирихле}
	
	%\begin{center}
	%Преподаватель Подаев М.
	%\end{center}
	\textit{Принцип Дирихле}~--- если кролики рассажены в клетки, причём число кроликов больше числа клеток, то хотя бы в одной из клеток находится более одного кролика.
	
	\begin{Task} 
		 В ковре размером $4 \times 4$ метра моль проела 15 дырок. Всегда ли можно вырезать коврик размером $1 \times 1$, не содержащий внутри дырок? (Дырки считаются точечными).
	 \end{Task}

	\begin{Task} 
		В лагере "Стратегия" учатся 38 человек. Докажите, что среди них найдутся четверо, родившихся в один месяц.
	 \end{Task}
	
	%с подлянкой
	\begin{Task} 
		Даны $n$ точек. Некоторые из них соединены отрезками. Докажите, что найдутся две точки, из которых выходит поровну отрезков.
	\end{Task}

\centerline{\LARGE Основные задачи}
	\begin{Task} 
		 Обязательно ли среди двадцати пяти монет достоинством 1, 2, 5 и 10 рублей найдётся семь монет одинакового достоинства?
	 \end{Task}
 
	 \begin{Task} 
	 	В мешке лежат 16 шариков черного цвета и 8 белого. Какое наименьшее число шариков нужно вынуть из мешка вслепую так, чтобы среди них заведомо оказались два шарика одного цвета?
	 \end{Task}
 
	\begin{Task} 
		В квадрат со стороной 2 метра бросили 76 точек. Докажите, что какие-то четыре из них можно накрыть квадратом со стороной 40 см.
	\end{Task}
	
	\begin{Task} 
		 Какое наибольшее число королей можно поставить на шахматной доске так, чтобы никакие два из них не били друг друга?
	 \end{Task}
	
	\begin{Task} 
		В клетках таблицы $3 \times 3$ расставлены числа –1, 0, 1. Докажите, что какие-то две из восьми сумм по всем строкам, всем столбцам и двум главным диагоналям будут равны.
	 \end{Task}

	\begin{Task} 
		В мешке 70 шаров, отличающихся только цветом: 20 красных, 20 синих, 20 жёлтых, остальные – чёрные и белые. 
		Какое наименьшее число шаров надо вынуть из мешка, не видя их, чтобы среди них было не менее 10 шаров одного цвета?
	\end{Task}
	
	
%	\begin{Task} 
%		10 школьников на олимпиаде решили 35 задач, причем известно, что среди них есть школьники, решившие ровно одну задачу, школьники, решившие ровно две задачи и школьники, решившие ровно три задачи. Докажите, что есть школьник, решивший не менее пяти задач.
%	 \end{Task}
	
	
	\centerline{\LARGE Посложнее}
	
	\begin{Task} 
		В каждой вершине куба написано число 1 или число 0. На каждой грани куба написана сумма четырёх чисел, написанных в вершинах этой грани. Может ли оказаться, что все числа, написанные на гранях, различны? 
	\end{Task}

	%с подлянкой
%	\begin{Task} 
%		 Дано 8 различных натуральных чисел, не больших 15. Докажите, что среди их положительных попарных разностей есть три одинаковых.
%	 \end{Task}
%	
	
	\begin{Task} 
		Докажите, что в любой компании найдутся два человека, имеющие одинаковое число друзей (из этой компании).
	 \end{Task}
	
	%хард
	\begin{Task} 
		Докажите, что среди любых шести человек есть либо трое попарно знакомых, либо трое попарно незнакомых.
	 \end{Task}
	
%	\begin{Task} 
%		В дискуссии приняли участие 15 депутатов. Каждый из них в своем выступлении раскритиковал ровно $k$ из оставшихся 14 депутатов. При каком наименьшем $k$ можно утверждать, что найдутся два депутата, которые раскритиковали друг друга?
%	 \end{Task}

	
\end{document}