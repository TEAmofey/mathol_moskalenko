\documentclass[b5paper,usehyperref, twoside]{article}
\usepackage[utf8]{inputenc}
\usepackage[T2A]{fontenc}
\usepackage[russian]{babel}
\usepackage{indentfirst}
\usepackage{amssymb}
\usepackage{amsmath}
\usepackage{tikz}
\usepackage{amsthm}
\usepackage{hyperref}
\usepackage{eurosym}
\usepackage{graphicx}
\usepackage{enumitem}
\usepackage{icomma}
\usepackage{hhline}
\newcommand{\sol}{\textbf{Решение.\ }}
\newcommand{\dzz}{\medskip\textbf{Для самостоятельного решения\\ }}
\newcommand{\ds}{\displaystyle}
\newcommand{\al}{\alpha}

\newcommand*{\hm}[1]{#1\nobreak\discretionary{}%
	{\hbox{$\mathsurround=0pt #1$}}{}}
%\renewcommand{\labelenumii}{\arabic{enumi}.\arabic{enumii}.}
%\renewcommand{\thefigure}{\thesection.\arabic{figure}}
\graphicspath{{Pics/}}

\newcommand{\be}{\beta}
\newcommand{\ga}{\gamma}

\newcommand{\ra}{\rightarrow}
\newcommand{\Ra}{\Rightarrow}
\newcommand{\Lra}{\Leftrightarrow}
\newcommand{\cd}{\cdot}
\newcommand{\gs}{\geqslant}
\newcommand{\ls}{\leqslant}
\newcommand{\tx}{\text}
\newcommand{\ti}{\times}
\newcommand{\opr}{\textbf{Определение.\ }}
\newcommand{\ov}{\overline}
\newcommand{\df}{\ds\frac}
\newcommand{\an}{\angle}
\newcommand{\tr}{\triangle}
\newcommand{\pa}{\parallel}
\newcommand{\ig}{\includegraphics}
\newcommand{\eq}{\equiv}
\renewcommand{\mod}{\operatorname{mod}}
\newcommand{\ba}{\begin{array}{c}}
	\newcommand{\ea}{\end{array}}
\renewcommand{\sb}{\left\{\ba}
\newcommand{\se}{\ea\right.}
\newcommand{\ovr}{\overrightarrow}
\newcommand{\om}{\omega}
\renewcommand{\tg}{\operatorname{tg}}
\renewcommand{\ctg}{\operatorname{ctg}}
\renewcommand{\phi}{\varphi}
\newcommand{\Z}{\mathbb{Z}}
\newcommand{\de}{\mathop{\raisebox{-2pt}{\vdots}}}
\newcommand{\ub}{\underbrace}
\newcommand{\arc}{\smile\!}

\usepackage{multirow}
\usepackage{epigraph}

\tikzset{
	treenode/.style = {shape=rectangle, rounded corners,
		draw, align=center,
		top color=white, bottom color=blue!20},
	root/.style     = {treenode, font=\Large, bottom color=red!30},
	env/.style      = {treenode, font=\ttfamily\normalsize},
	dummy/.style    = {circle,draw}
}


\usepackage[paperheight=20.5cm,paperwidth=14cm, left=0.6cm,right=0.9cm,top=1.8cm,bottom=1.5cm]{geometry}

\parindent=0pt
\parskip=1pt
\tolerance=1100

\addto\captionsrussian{\renewcommand{\partname}{Раздел}}

\def\thepart{\arabic{part}}

\hypersetup{pdfstartview=FitH,  linkcolor=linkcolor,urlcolor=urlcolor, colorlinks=true}

\definecolor{linkcolor}{HTML}{000000} % цвет ссылок
\definecolor{urlcolor}{HTML}{000000} % цвет гиперссылок


\newtheorem{Lemma}{Лемма}
\theoremstyle{definition}
\newtheorem{Remark}{Замечание}
\newtheorem{Claim}{Утверждение}
\newtheorem{Cor}{Следствие}
\newtheorem{Theorem}{Теорема}
\newtheorem*{known}{Теорема}
\def\proofname{Доказательство}
\theoremstyle{definition}
\newtheorem{Def}{Определение}
%\newcounter{partsec}[section]
%\def\thepartsec{\arabic{part}.\arabic{section}}
\newtheorem{Task}{}


\newtheorem{Example}{}


\usepackage{fancyhdr} %загрузим пакет
\pagestyle{fancy} %применим колонтитул
\fancyhead[LO]{Летняя математическая школа Стратегия, группа C} %очистим хидер на всякий случай
\fancyhead[RO]{2023.07.08}
%\fancyhead[LE,RO]{\thepage} %номер страницы слева сверху на четных и справа на нечетных
%\fancyhead[CO]{текст-центр-нечетные}  %текст-центр-нечетные
%\fancyhead[LO]{текст-слева-нечетные}  %текст-слева-нечетные
%\fancyhead[CE]{текст-центр-четные}    %текст-центр-четные
\fancyfoot{} %футер будет пустой

\begin{document}
	\newcounter{totalpics}
	\newcounter{totaltables}
	%\setcounter{tocdepth}{2}
	
	%
	%\begin{document}
	%\begin{center}
	
	
	%\begin{figure}[bh]
	%\noindent\centering{
		%\includegraphics[width=70mm]{logo}
		% }
	%\label{figCurves}
	%\end{figure}
	%\end{center}
	\centerline{\LARGE Оценка + Пример + Добавка}
	
	%\begin{center}
	%Преподаватель Подаев М.
	%\end{center}
	
	% \epigraph{Он стал поэтом - для математика у него не хватало фантазии. }{Давид Гильберт}
%	\centerline{\LARGE }
	
	%\textit{Инвариант} - параметр, который остается неизменным в ходе любой операции.
%	
%	\begin{Task}
%		Какое наибольшее число трёхклеточных уголков можно вырезать из клетчатого квадрата $8 \times 8$?
%	\end{Task}
%	
%	
%	\textbf{Решение:} В квадрате $8 \time 8=64$ клетки. Поэтому вырезать $22$ и более уголков не получится: ведь тогда суммарное число клеток в них будет не меньше $22\cdot 3 = 66$. Значит, число уголков не больше $21$ (\textit{оценка}).
%	Вырезать $21$ уголок можно --- \textit{пример} будет на доске. Следовательно, наибольшее возможное количество уголков равно $21$.
%	
%	Логика рассуждения ясна: мы показали, что количество уголков не превосходит числа $21$
%	(\textit{оценка}) и иногда ему равно (\textit{пример}). Значит, $21$ и есть максимум числа уголков.
%	
%	
%	\begin{Task}
%		Какое наименьшее число ладей могут побить всю шахматную доску?
%	\end{Task}
%		
%	\centerline{\LARGE Основная часть}
%	
%	\begin{Task}
%		Как можно набрать сумму 37 рублей используя только монеты по 3 рубля и 5 рублей так, чтобы суммарно было наименьшее число монет?
%	\end{Task} 
%	
%	\begin{Task}
%		Какое наибольшее число трёхклеточных уголков можно вырезать из клетчатого прямоугольника $5 \times 7$?
%	\end{Task}
%
%	\begin{Task}
%		111 человек встали в хоровод. Оказалось, что среди любых двух соседей есть по меньшей мере одна девушка. Какое наименьшее количество девушек может участвовать в этом хороводе?
%	\end{Task}
%	
%	\begin{Task}
%		Каково наименьшее натуральное $n$ такое, что $n!$ делится на 18, на 19, на 20 и на 21?
%	\end{Task}
%	
%	\begin{Task}
%		У вас есть три котлеты и две сковороды. Каждая сторона котлеты жарится одну минуту. На одну сковороду одновременно помещается лишь одна котлета. За какое наименьшее время можно пожарить все котлеты с обеих сторон?
%	\end{Task}
%	
%	
%	\begin{Task}
%		Какое наименьшее число клеточек на доске $8\times8$ можно закрасить в чёрный цвет так, чтобы была хотя бы одна закрашенная клетка в любом квадратике $2\times2$?
%	\end{Task}
	
	\begin{Task}
		Сложите квадрат из наименьшего возможного количества трёхклеточных уголков. 
	\end{Task}
	
	
	\begin{Task}
		Михаил Валерьевич зашел в аудиторию, где вокруг большого круглого стола стояло 30 стульев. На некоторых из стульев сидели информаты. Оказалось, что Михаил Валерьевич не может сесть так, чтобы рядом с ним никто не сидел. Какое наименьшее число информатов могло быть за столом?
	\end{Task}
	
	
	\begin{Task} 
		$72$ кузнецов должны подковать $90$ лошадей. Какое наименьшее время они затратят на работу, если каждый кузнец тратит на одну подкову пять минут? (Лошадь не может стоять на двух ногах.)
	\end{Task}

	\begin{Task}
		Парты в 307 аудитории расположены так, что образуют таблицу 8 × 8.
		В момент, когда Тимофей Дмитриевич отвернулся, каждый ребенок решил
		подойти к парте своего друга. Чтобы не быть замеченными ученики дошли лишь до соседней по стороне парты (в таблице $8\times 8$ каждый попал в соседнюю по стороне клетку). Когда преподаватель посмотрел обратно в класс он заметил, что занято минимально возможное количество парт. Сколько парт оказалось занято?
	\end{Task}
	
	
\end{document}
