\documentclass{article}
\usepackage[12pt]{extsizes}
\usepackage[T2A]{fontenc}
\usepackage[utf8]{inputenc}
\usepackage[english, russian]{babel}

\usepackage{amssymb}
\usepackage{amsfonts}
\usepackage{amsmath}
\usepackage{enumitem}
\usepackage{graphics}

\usepackage{lipsum}



\usepackage{geometry} % Меняем поля страницы
\geometry{left=1cm}% левое поле
\geometry{right=1cm}% правое поле
\geometry{top=1.5cm}% верхнее поле
\geometry{bottom=1cm}% нижнее поле


\usepackage{fancyhdr} % Headers and footers
\pagestyle{fancy} % All pages have headers and footers
\fancyhead{} % Blank out the default header
\fancyfoot{} % Blank out the default footer
\fancyhead[L]{Математика}
\fancyhead[C]{\textit{Вся домашка}}
\fancyhead[R]{7 Августа 2022 --- 6 Августа 2023}% Custom header text


%----------------------------------------------------------------------------------------

%\begin{document}\normalsize
\begin{document}\large
	
	\begin{center}
		\textbf{Процессы}
	\end{center}
	
	\begin{enumerate}
		
		\item[\protect\fbox{10}] На бесконечной шахматной доске стоят ферзь и невидимый король. Известно, что ферзь дал шах по горизонтали, и король ушел из под шаха. Докажите, что ферзь может ходить так, чтобы король наверняка ещё раз попал под шах.
		
	\end{enumerate}

	\begin{center}
		\textbf{Питерский город}
	\end{center}
	\begin{enumerate}[label*=\protect\fbox{\arabic{enumi}}]
		\setcounter{enumi}{6}
		
		\item Существует ли выпуклый многоугольник, который можно разрезать непересекающимися (во внутренних точках) диагоналями на треугольники равной площади хотя бы тремя разными способами?
		
	\end{enumerate}

	
	\begin{center}
		\textbf{Информация --- 2}
	\end{center}
	
	\begin{enumerate}[label*=\protect\fbox{\arabic{enumi}}]
	
		\setcounter{enumi}{6}
		\item Есть 15 монет, одна из которых фальшивая. Все настоящие монеты весят одинаково, а фальшивая весит иначе, но неизвестно, тяжелее она или легче. Какое наименьшее количество взвешиваний на двухчашечных весах без гирь необходимо, чтобы гарантированно найти фальшивую монету и сказать, тяжелее она или легче?
		
		\setcounter{enumi}{7}
		\item Та же задача, но теперь не надо говорить тяжелее фальшивая монета или легче.
		
		\setcounter{enumi}{8}
		\item Та же задача, но теперь про одну из монет вам известно, что она настоящая, и не надо говорить тяжелее фальшивая монета или легче. Изменится ли ответ, если отобрать у вас гарантированно настоящую монету?
		
		\setcounter{enumi}{9}
		\item Алиса и ее младший брат Боб играют в игру. Боб загадывает число от 1 до 1000, а Алиса пытается его угадать. Алиса называет Бобу число, а Боб говорит, верно ли, что оно больше загаданного. Алиса знает, что Боб, чтобы запутать Алису, может соврать один раз за игру (а может и не соврать). За какое наименьшее количество вопросов Алиса может гарантированно угадать загаданное Бобом число?
	\end{enumerate}
	
	\begin{center}
		\textbf{Равносоставленные многоугольники}
	\end{center}
	
	
	\begin{enumerate}[label*=\protect\fbox{\arabic{enumi}}]
		
		\setcounter{enumi}{13}
		\item Перекроите квадрат в 3 равных квадрата, разрезав его не более чем на а) 10 частей; б) 7 частей.
		
		\setcounter{enumi}{14}
		\item Докажите, что правильный пятиугольник можно разрезать на 4 части, из которых без просветов и наложений можно сложить прямоугольник.
	\end{enumerate}
	
\begin{center}
	\textbf{Инверсия}
\end{center}

\begin{enumerate}[label*=\protect\fbox{\arabic{enumi}}]
	
	
	\setcounter{enumi}{18}
	\item Пусть $O$ — одна из точек пересечения окружностей $\omega_1$ и $\omega_2$. Окружность $\omega$ с центром $O$ пересекает $\omega_1$ в точках $A$ и $B$,а $\omega_2$ — в точках $C$ и $D$. Пусть $X$— точка пересечения прямых $AC$ и $BD$. Докажите, что все такие точки $X$ лежат на одной прямой.
	
	\setcounter{enumi}{19}
	\item Четырёхугольник $ABCD$ описан около окружности с центром $I$. Касательные к описанной окружности треугольника $AIC$ в точках $A$, $C$ пересекаются в точке $X$. Касательные к описанной окружности треугольника $BID$ в точках $B$, $D$ пересекаются в точке $Y$. Докажите, что точки $X$, $I$, $Y$ лежат на одной прямой.
	
	\setcounter{enumi}{20}
	\item В четырёхугольнике $ABCD$ вписанная окружность $\omega$ касается сторон $BC$ и $DA$ в точках $E$ и $F$ соответственно. Оказалось, что прямые $AB$, $FE$ и $CD$ пересекаются в одной точке $S$. Описанные окружности $\Omega$ и $\Omega_1$ треугольников $AED$ и $BFC$, вторично пересекают окружность $\omega$ в точках $E_1$ и $F_1$. Докажите, что прямые $EF$ и $E_1F_1$ параллельны.
	
\end{enumerate}
	
	
\end{document}