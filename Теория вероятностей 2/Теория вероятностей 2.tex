\documentclass{article}
\usepackage[12pt]{extsizes}
\usepackage[T2A]{fontenc}
\usepackage[utf8]{inputenc}
\usepackage[english, russian]{babel}

\usepackage{amssymb}
\usepackage{amsfonts}
\usepackage{amsmath}
\usepackage{enumitem}
\usepackage{graphics}
\usepackage{graphicx}

\usepackage{lipsum}

\newtheorem{theorem}{Теорема}
\newtheorem{task}{Задача}
\newtheorem{lemma}{Лемма}
\newtheorem{definition}{Определение}
\newtheorem{example}{Пример}
\newtheorem{statement}{Утверждение}
\newtheorem{corollary}{Следствие}


\usepackage{geometry} % Меняем поля страницы
\geometry{left=1cm}% левое поле
\geometry{right=1cm}% правое поле
\geometry{top=1.5cm}% верхнее поле
\geometry{bottom=1cm}% нижнее поле


\usepackage{fancyhdr} % Headers and footers
\pagestyle{fancy} % All pages have headers and footers
\fancyhead{} % Blank out the default header
\fancyfoot{} % Blank out the default footer
\fancyhead[L]{Математика}
\fancyhead[C]{\textit{Теория вероятностей}}
\fancyhead[R]{23 апреля}% Custom header text


%----------------------------------------------------------------------------------------

%\begin{document}\normalsize
\begin{document}\large
	
\begin{center}
	\textbf{Формула Байеса}
\end{center}


\begin{enumerate}[label*=\protect\fbox{\arabic{enumi}}]
	
\item Считается, что доля больных туберкулёзом в составе населения составляет $0.001$. При флюорографическом обследовании туберкулёз распознаётся с вероятностью $0.9,$ а здоровый человек диагностируется как больной с вероятностью $0.01$

\item В двух коробках находятся однотипные диоды. В первой – 20 шт., из них 2 неисправных; во второй – 10 шт., из них 4 неисправных. Наугад была вы- брана коробка, а затем из нее наугад был выбран диод. Он оказался неисправным. Найти вероятность того, что он был взят из второй коробки.

\item Три гимнаста стоят друг на друге. Вероятность падения нижнего составляет 0,08, второго – 0,5, третьего – 0,31. Найти вероятность того, что трюк
не удастся.

\item Имеется три закрытых двери, за одной из которых спрятан клад. Игрок выбирает дверь,за которой, как он думает, спрятан клад и сообщает о своём выборе ведущему игры. Затем ведущий (он знает, где клад) открывает дверь, за которой ничего нет, причём нету, которую выбрал игрок (у ведущего на выбор есть либо одна, либо две таких двери). Далее, ведущий предлагает игроку передумать: выбрать другую из оставшихся двух дверей. Имеет ли для игрока смысл принять предложение ведущего?

\item При бросании неправильной монеты орел выпадает с вероятностью $p$, решка — с вероятностью $q= 1-p$. С какой вероятностью после $n$ бросков выпадет четное число орлов?

\item С какой вероятностью ни один член случайной перестановки $N$ чисел не стоит на своем месте?

\item При двух бросаниях игрального кубика вероятность того, что выпадет одинаковое число очков равна $\frac{1}{6}.$ Докажите, что все числа от 1 до 6 выпадают с одинаковой вероятностью, то есть кубик правильный.

\item В семье с двумя детьми считаем равновероятными все четыре сочетания полов: ММ, МД, ДМ, ДД. Какова вероятность того, что в семье оба ребёнка мальчики, если известно, что 

а) в семье есть мальчик 

б) старший из детей — мальчик.

\item 100 пассажиров садятся по очереди в 100-местный самолёт. Первый занимает случайное место. Второй садится на своё место, если оно свободно, а если нет, то садится на случайное место из оставшихся. Остальные делают то же самое. С какой вероятностью последний пассажир окажется на своём месте?


\end{enumerate}
\end{document}