\documentclass{article}

\usepackage[12pt]{extsizes}
\usepackage[T2A]{fontenc}
\usepackage[utf8]{inputenc}
\usepackage[english, russian]{babel}

\usepackage{mathrsfs}
\usepackage[dvipsnames]{xcolor}

\usepackage{amsmath}
\usepackage{amssymb}
\usepackage{amsthm}
\usepackage{indentfirst}
\usepackage{amsfonts}
\usepackage{enumitem}
\usepackage{graphics}
\usepackage{tikz}
\usepackage{tabu}
\usepackage{diagbox}
\usepackage{hyperref}
\usepackage{mathtools}
\usepackage{ucs}
\usepackage{lipsum}
\usepackage{geometry} % Меняем поля страницы
\usepackage{fancyhdr} % Headers and footers
\newcommand{\range}{\mathrm{range}}
\newcommand{\dom}{\mathrm{dom}}
\newcommand{\N}{\mathbb{N}}
\newcommand{\R}{\mathbb{R}}
\newcommand{\E}{\mathbb{E}}
\newcommand{\D}{\mathbb{D}}
\newcommand{\M}{\mathcal{M}}
\newcommand{\Prime}{\mathbb{P}}
\newcommand{\A}{\mathbb{A}}
\newcommand{\Q}{\mathbb{Q}}
\newcommand{\Z}{\mathbb{Z}}
\newcommand{\F}{\mathbb{F}}
\newcommand{\CC}{\mathbb{C}}

\DeclarePairedDelimiter\abs{\lvert}{\rvert}
\DeclarePairedDelimiter\floor{\lfloor}{\rfloor}
\DeclarePairedDelimiter\ceil{\lceil}{\rceil}
\DeclarePairedDelimiter\lr{(}{)}
\DeclarePairedDelimiter\set{\{}{\}}
\DeclarePairedDelimiter\norm{\|}{\|}

\renewcommand{\labelenumi}{(\alph{enumi})}

\newcommand{\smallindent}{
    \geometry{left=1cm}% левое поле
    \geometry{right=1cm}% правое поле
    \geometry{top=1.5cm}% верхнее поле
    \geometry{bottom=1cm}% нижнее поле
}

\newcommand{\header}[3]{
    \pagestyle{fancy} % All pages have headers and footers
    \fancyhead{} % Blank out the default header
    \fancyfoot{} % Blank out the default footer
    \fancyhead[L]{#1}
    \fancyhead[C]{#2}
    \fancyhead[R]{#3}
}

\newcommand{\dividedinto}{
    \,\,\,\vdots\,\,\,
}

\newcommand{\littletaller}{\mathchoice{\vphantom{\big|}}{}{}{}}

\newcommand\restr[2]{{
    \left.\kern-\nulldelimiterspace % automatically resize the bar with \right
    #1 % the function
    \littletaller % pretend it's a little taller at normal size
    \right|_{#2} % this is the delimiter
}}

\DeclareGraphicsExtensions{.pdf,.png,.jpg}

\newenvironment{enumerate_boxed}[1][enumi]{\begin{enumerate}[label*=\protect\fbox{\arabic{#1}}]}{\end{enumerate}}



\smallindent

\header{Математика}{\textit{Алгебра}}{27 сентября 2023}

%----------------------------------------------------------------------------------------

\begin{document}
    \large

    \begin{center}
        \textbf{Тригонометрия}
    \end{center}

    \begin{table}[h]
        \label{tab:table-2}
        \centering
        \begin{tabular}{|c|c|c|c|c|}
            \hline
            \textbf{$\alpha$} & \textbf{$0^\circ$} & \textbf{$30^\circ$}       & \textbf{$45^\circ$}       & \textbf{$60^\circ$}   \\\hline
            $\sin(\alpha)$    & $0$            & $\dfrac{1}{2}$        & $\dfrac{\sqrt{2}}{2}$ & $\dfrac{\sqrt{3}}{2}$ \\[3mm]\hline
            $\cos(\alpha)$    & $1$            & $\dfrac{\sqrt{3}}{2}$ & $\dfrac{\sqrt{2}}{2}$ & $\dfrac{1}{2}$ \\[3mm]\hline
            $\tg(\alpha)$     & $0$            & $\dfrac{\sqrt{3}}{3}$ & $1$                   & $\sqrt{3}$ \\[3mm]\hline
        \end{tabular}
        \caption{Тригонометрическая таблица (Обязательно запомнить)}
    \end{table}


    \textbf{Задачи:}

    \begin{enumerate}[label*=\protect\fbox{\arabic{enumi}}]

        \setcounter{enumi}{0}

        \item Вычислите значение выражение $\sin 20^\circ \sin 40^\circ \sin 60^\circ \sin 80^\circ$.

        \item Вычислите значение выражение $\cos 20^\circ \cos 40^\circ \cos 60^\circ \cos 80^\circ$.

        \item Упростите выражение $\cos (a) \cdot \cos (2a) \cdot \cos (4a) \cdot \dotsc \cdot \cos (2^{n-1}a)$.

        \item Известно, что  $\tg \alpha + \tg \beta = 2$ и $\ctg \alpha + \ctg \beta = 3$.
        Найдите $ \tg (\alpha + \beta) $.

        \item Найдите $ \sin(15^\circ) $ и $ \cos(15^\circ) $

        \item Найдите $ \sin(10^\circ) $ и $ \cos(10^\circ) $

        \item Что больше $\dfrac{\sin1^\circ}{\sin2^\circ}$ или $\dfrac{\sin3^\circ}{\sin4^\circ}$

        \item При каких значениях $c$ числа $ \sin \alpha $ и $ \cos \alpha $ являются корнями квадратного уравнения  $5x^2 - 3x + c = 0$  ($ \alpha $~--- некоторый угол)?
    \end{enumerate}

\end{document}