\documentclass{article}

\usepackage[12pt]{extsizes}
\usepackage[T2A]{fontenc}
\usepackage[utf8]{inputenc}
\usepackage[english, russian]{babel}

\usepackage{mathrsfs}
\usepackage[dvipsnames]{xcolor}

\usepackage{amsmath}
\usepackage{amssymb}
\usepackage{amsthm}
\usepackage{indentfirst}
\usepackage{amsfonts}
\usepackage{enumitem}
\usepackage{graphics}
\usepackage{tikz}
\usepackage{tabu}
\usepackage{diagbox}
\usepackage{hyperref}
\usepackage{mathtools}
\usepackage{ucs}
\usepackage{lipsum}
\usepackage{geometry} % Меняем поля страницы
\usepackage{fancyhdr} % Headers and footers
\newcommand{\range}{\mathrm{range}}
\newcommand{\dom}{\mathrm{dom}}
\newcommand{\N}{\mathbb{N}}
\newcommand{\R}{\mathbb{R}}
\newcommand{\E}{\mathbb{E}}
\newcommand{\D}{\mathbb{D}}
\newcommand{\M}{\mathcal{M}}
\newcommand{\Prime}{\mathbb{P}}
\newcommand{\A}{\mathbb{A}}
\newcommand{\Q}{\mathbb{Q}}
\newcommand{\Z}{\mathbb{Z}}
\newcommand{\F}{\mathbb{F}}
\newcommand{\CC}{\mathbb{C}}

\DeclarePairedDelimiter\abs{\lvert}{\rvert}
\DeclarePairedDelimiter\floor{\lfloor}{\rfloor}
\DeclarePairedDelimiter\ceil{\lceil}{\rceil}
\DeclarePairedDelimiter\lr{(}{)}
\DeclarePairedDelimiter\set{\{}{\}}
\DeclarePairedDelimiter\norm{\|}{\|}

\renewcommand{\labelenumi}{(\alph{enumi})}

\newcommand{\smallindent}{
    \geometry{left=1cm}% левое поле
    \geometry{right=1cm}% правое поле
    \geometry{top=1.5cm}% верхнее поле
    \geometry{bottom=1cm}% нижнее поле
}

\newcommand{\header}[3]{
    \pagestyle{fancy} % All pages have headers and footers
    \fancyhead{} % Blank out the default header
    \fancyfoot{} % Blank out the default footer
    \fancyhead[L]{#1}
    \fancyhead[C]{#2}
    \fancyhead[R]{#3}
}

\newcommand{\dividedinto}{
    \,\,\,\vdots\,\,\,
}

\newcommand{\littletaller}{\mathchoice{\vphantom{\big|}}{}{}{}}

\newcommand\restr[2]{{
    \left.\kern-\nulldelimiterspace % automatically resize the bar with \right
    #1 % the function
    \littletaller % pretend it's a little taller at normal size
    \right|_{#2} % this is the delimiter
}}

\DeclareGraphicsExtensions{.pdf,.png,.jpg}

\newenvironment{enumerate_boxed}[1][enumi]{\begin{enumerate}[label*=\protect\fbox{\arabic{#1}}]}{\end{enumerate}}



\smallindent

\header{Математика}{\textit{Геометрия}}{12 марта 2024}

%----------------------------------------------------------------------------------------

\begin{document}
    \large

    \begin{center}
        \textbf{Двигающиеся точки}
    \end{center}

    \begin{enumerate_boxed}

        \item В середине лестницы, приставленной к стене, сидит котенок.
        Лестница начинает скользить по стене и полу.
        Какую траекторию будет описывать котенок?

        \item На окружности зафиксированы точки $A$ и $B$ Точка $C$ движется по дуге окружности.
        Докажите, что ортоцентр $H$ треугольника $ABC$ движется по некоторой окружности.

        \item Две окружности пересекаются в точках $A$ и $B$.
        Из точки $A$ одновременно с одинаковыми угловыми скоростями стартуют два велосипедиста.
        Каждый из них едет по своей окружности против часовой стрелки.
        Докажите, что прямая, соединяющая их, всё время проходит через точку $B$

        \item Через точку $P$ вне окружности $\Omega$ провели произвольную прямую $l$, пересекающую $\Omega$.
        Прямую $l$ начинают вращать вокруг точки $p$ Докажите, что середина хорды, высекаемой на прямой $l$ окружностью $\Omega$ всё время лежит на одной окружности

        \item На окружности зафиксированы точки $A$ и $B$.
        Точка $C$ движется по дуге окружности.
        Докажите, что $I, I_a, I_b$ и $I_c$ двигаются по дугам окружностей

    \end{enumerate_boxed}
\end{document}