\documentclass{article}
\usepackage[12pt]{extsizes}
\usepackage[T2A]{fontenc}
\usepackage[utf8]{inputenc}
\usepackage[english, russian]{babel}

\usepackage{amssymb}
\usepackage{amsfonts}
\usepackage{amsmath}
\usepackage{enumitem}
\usepackage{graphics}

\usepackage{lipsum}



\usepackage{geometry} % Меняем поля страницы
\geometry{left=1cm}% левое поле
\geometry{right=1cm}% правое поле
\geometry{top=1.5cm}% верхнее поле
\geometry{bottom=1cm}% нижнее поле


\newtheorem{definition}{Опредление}
\usepackage{fancyhdr} % Headers and footers
\pagestyle{fancy} % All pages have headers and footers
\fancyhead{} % Blank out the default header
\fancyfoot{} % Blank out the default footer
\fancyhead[L]{Математика}
\fancyhead[C]{\textit{Геометрия}}
\fancyhead[R]{12 марта 2024}% Custom header text


%----------------------------------------------------------------------------------------

%\begin{document}\normalsize
\begin{document}\large



\begin{center}
\textbf{Двигающиеся точки}
\end{center}

\begin{enumerate}[label*=\protect\fbox{\arabic{enumi}}]

\item В середине лестницы, приставленной к стене, сидит котенок. Лестница начинает
скользить по стене и полу. Какую траекторию будет описывать котенок?

\item На окружности зафиксированы точки $A$ и $B$ Точка $C$ движется по дуге окружности. Докажите, что ортоцентр $H$ треугольника $ABC$ движется по некоторой окружности.

\item Две окружности пересекаются в точках $A$ и $B$. Из точки $A$ одновременно с одинаковыми угловыми скоростями стартуют два велосипедиста. Каждый из них едет по
своей окружности против часовой стрелки. Докажите, что прямая, соединяющая их,
всё время проходит через точку $B$

\item Через точку $P$ вне окружности $\Omega$ провели произвольную прямую $l$, пересекающую $\Omega$. Прямую $l$ начинают вращать вокруг точки $p$ Докажите, что середина хорды, высекаемой на прямой $l$ окружностью $\Omega$ всё время лежит на одной окружности

\item
На окружности зафиксированы точки $A$ и $B$. Точка $C$ движется по дуге окружности. 
Докажите, что $I, I_a, I_b$ и $I_c$ двигаются по дугам окружностей

\end{enumerate}
\end{document}