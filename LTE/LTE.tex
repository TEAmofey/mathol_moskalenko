\documentclass{article}

\usepackage[12pt]{extsizes}
\usepackage[T2A]{fontenc}
\usepackage[utf8]{inputenc}
\usepackage[english, russian]{babel}

\usepackage{mathrsfs}
\usepackage[dvipsnames]{xcolor}

\usepackage{amsmath}
\usepackage{amssymb}
\usepackage{amsthm}
\usepackage{indentfirst}
\usepackage{amsfonts}
\usepackage{enumitem}
\usepackage{graphics}
\usepackage{tikz}
\usepackage{tabu}
\usepackage{diagbox}
\usepackage{hyperref}
\usepackage{mathtools}
\usepackage{ucs}
\usepackage{lipsum}
\usepackage{geometry} % Меняем поля страницы
\usepackage{fancyhdr} % Headers and footers
\newcommand{\range}{\mathrm{range}}
\newcommand{\dom}{\mathrm{dom}}
\newcommand{\N}{\mathbb{N}}
\newcommand{\R}{\mathbb{R}}
\newcommand{\E}{\mathbb{E}}
\newcommand{\D}{\mathbb{D}}
\newcommand{\M}{\mathcal{M}}
\newcommand{\Prime}{\mathbb{P}}
\newcommand{\A}{\mathbb{A}}
\newcommand{\Q}{\mathbb{Q}}
\newcommand{\Z}{\mathbb{Z}}
\newcommand{\F}{\mathbb{F}}
\newcommand{\CC}{\mathbb{C}}

\DeclarePairedDelimiter\abs{\lvert}{\rvert}
\DeclarePairedDelimiter\floor{\lfloor}{\rfloor}
\DeclarePairedDelimiter\ceil{\lceil}{\rceil}
\DeclarePairedDelimiter\lr{(}{)}
\DeclarePairedDelimiter\set{\{}{\}}
\DeclarePairedDelimiter\norm{\|}{\|}

\renewcommand{\labelenumi}{(\alph{enumi})}

\newcommand{\smallindent}{
    \geometry{left=1cm}% левое поле
    \geometry{right=1cm}% правое поле
    \geometry{top=1.5cm}% верхнее поле
    \geometry{bottom=1cm}% нижнее поле
}

\newcommand{\header}[3]{
    \pagestyle{fancy} % All pages have headers and footers
    \fancyhead{} % Blank out the default header
    \fancyfoot{} % Blank out the default footer
    \fancyhead[L]{#1}
    \fancyhead[C]{#2}
    \fancyhead[R]{#3}
}

\newcommand{\dividedinto}{
    \,\,\,\vdots\,\,\,
}

\newcommand{\littletaller}{\mathchoice{\vphantom{\big|}}{}{}{}}

\newcommand\restr[2]{{
    \left.\kern-\nulldelimiterspace % automatically resize the bar with \right
    #1 % the function
    \littletaller % pretend it's a little taller at normal size
    \right|_{#2} % this is the delimiter
}}

\DeclareGraphicsExtensions{.pdf,.png,.jpg}

\newenvironment{enumerate_boxed}[1][enumi]{\begin{enumerate}[label*=\protect\fbox{\arabic{#1}}]}{\end{enumerate}}



\smallindent

\header{Математика}{\textit{Теория чисел}}{14 февраля 2024}


%----------------------------------------------------------------------------------------

\begin{document}
    \large

    \begin{center}
        \textbf{Лемма об уточнении показателя}
    \end{center}

    \textbf{Лемма об уточнении показателя } состоит из трёх частей

    Пусть $x$ и $y$ — различные ненулевые целые числа, $p$ — нечетное простое число, не являющееся делителем $x$ и $y$ и такое, что $x - y \,\,\,\vdots\,\,\, p$.
    Тогда для любого натурального $n$ выполнено равенство $\nu_p (x^n - y^n) = \nu_p (x - y) + \nu_p(n).$

    Пусть $x$ и $y$ — различные целые числа, $p$ — нечетное простое число, не являющееся делителем $x$ и $y$ и такое, что $x + y \,\,\,\vdots\,\,\, p$.
    Тогда для любого нечетного натурального $n$ выполнено $\nu_p (x^n + y^n) = \nu_p(x + y) + \nu_p (n).$

    Пусть $x$ и $y$ — различные нечетные целые числа.
    Тогда для любого натурального $n$ выполнено $\nu_2 (x^n - y^n)  = \nu_2 (x - y) + \nu_2 (x + y) + \nu_2 (n) - 1.$

    \begin{enumerate_boxed}

%\item Докажите, что $\nu_p(C_n^k) > \nu_p(n) - \dfrac{k}{p-1}.$

        \item Пусть $p$ — простое число, числа $a$ и $b$ не делятся на $p$, a $a - b$ --- делится.
        \begin{enumerate}
            \item Докажите, что $\nu_p(a^p - b^p) > \nu_p(a - b)$
            \item Докажите, что $\nu_p(a^s - b^s) = \nu_p(a - b)$, если $s$ не делится на $p$
            \item Докажите, что $\nu_p(a^k - b^k) \geqslant \nu_p(a - b) + \nu_p(k)$
            \item Докажите, что если $p > 2$, то $\nu_p(a^p - b^p) = \nu_p(a - b) + 1$
            \item Докажите лемму в случае, когда $p > 2$.
            \item Докажите лемму в случае, когда $p = 2$ и $a - b \,\,\,\vdots\,\,\, 4$.
            \item Докажите лемму в случае, когда $p = 2$ и $a - b \,\,\,\vdots\,\,\, 2$ но не делится на 4.
%	\item Известно, что $x \equiv y \bmod p$. Докажите, что $x^p \equiv y^p \bmod p^2$
%	\item Докажите, что LTE-лемма справедлива для случая, когда $n$ не делится на $p$.
%	\item Чему может равняться НОД чисел $x - y$ и $\dfrac{x^p - y^p}{x - y}$, если $(x, y) = 1$?
%	\item Докажите LTE-лемму.
        \end{enumerate}

        \item Используя \textit{лемму об уточнении показателя}, найдите степень вхождения тройки в число $5^{18} - 2^{18}$.
        \item Найдите все тройки натуральных чисел $x, y, p$ такие, что $p$ — простое и $p^x - y^p = 1.$
        \item Дано простое число $p$ и натуральные числа $a$ и $n$.
        Докажите, что если $2^p + 3^p = a^n$, то $n = 1$.

        \item Докажите, что показатель числа 2 по модулю $3^n$ равен $\varphi(3^n)$.
        \item Решить уравнение $3^x = 2^x \cdot y + 1$ в натуральных числах.
        \item Дано натуральное число n, не делящееся ни на один точный квадрат, больший 1.
        Докажите, что не существует пары взаимно простых чисел $(x, y)$ такой, что $x^n + y^n$ делится на $(x + y)^3$.
        \item Пусть натуральные числа $x, y, p, n, k$ таковы, что $x^n +y^n = p^k$.
        Докажите, что если число $n > 1$ — нечетное, а число $p$ — простое нечетное, то $n$ является степенью числа $p$ с натуральным показателем.
        \item Докажите, что если $3^n - 2^n = p^a$ для некоторых натуральных $n$, a и простого $p$, то тогда $n$ — простое.
        \item Пусть положительные числа $a$ и $b$ таковы, что $a^k - b^k$ является натуральным числом для любого натурального $k$.
        Докажите, что $a$ и $b$ — натуральные
        \item Найдите все такие натуральные $n$, что при некоторых взаимно простых $x$ и $y$ и натуральном $k > 1$ выполняется равенство $3^n = x^k + y^k$.
        \item На сколько нулей заканчивается число $4^{5^6} + 6^{5^4}$?
        \item Пусть $m$ — нечетное натуральное число, $m > 3$.
        Найти наименьшее натуральное $n$, такое, что $m^n - 1 \dividedinto 2^{2024}$.
        \item Найдите все такие натуральные $n > 1$, что $2^n + 1$ делится на $n^2$.

    \end{enumerate_boxed}

\end{document}
