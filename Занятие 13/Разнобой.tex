\documentclass{article}
\usepackage[12pt]{extsizes}
\usepackage[T2A]{fontenc}
\usepackage[utf8]{inputenc}
\usepackage[english, russian]{babel}

\usepackage{amssymb}
\usepackage{amsfonts}
\usepackage{amsmath}
\usepackage{enumitem}
\usepackage{graphics}
\usepackage{graphicx}

\usepackage{lipsum}
\DeclareGraphicsExtensions{.pdf,.png,.jpg}



\usepackage{geometry} % Меняем поля страницы
\geometry{left=1cm}% левое поле
\geometry{right=1cm}% правое поле
\geometry{top=1.5cm}% верхнее поле
\geometry{bottom=1cm}% нижнее поле


\usepackage{fancyhdr} % Headers and footers
\pagestyle{fancy} % All pages have headers and footers
\fancyhead{} % Blank out the default header
\fancyfoot{} % Blank out the default footer
\fancyhead[L]{Математика}
\fancyhead[C]{\textit{Алгебра}}
\fancyhead[R]{20 февраля 2023}% Custom header text

%----------------------------------------------------------------------------------------

%\begin{document}\normalsize
\begin{document}\large
	

\begin{center}
\textbf{Разнобой}
\end{center}

\begin{enumerate}[label*=\protect\fbox{\arabic{enumi}}]
	
	\item Последовательность из пяти цифр $a_1, a_2, a_3, a_4, a_5$ будем называть «горой», если $a_1 < a_2 < a_3 > a_4 > a_5$, и «ямой», если $a_1 > a_2 > a_3 < a_4 < a_5$. Чего больше: «гор» или
	«ям»?
	
	\item Кубик Рубика $3 \times 3 \times 3$ надо распилить на единичные кубики. После распила части
	можно перекладывать и прикладывать так, чтобы можно было пилить несколько
	частей одновременно. Какое наименьшее число распилов нам понадобится?
	
	\item Докажите, что хотя бы одно из следующих чисел 
	\begin{center}
		 \texttt{ИКС, МИКС, ПЯТИКЛАССНИК и ПЛЯСКИТИМАТИНАТАТАМИ}
	\end{center}
	 составное (разным буквами соответствуют разные цифры, одинаковым — одинаковые, первые цифры чисел не равны 0).
	
	\item 100 гирек веса 1, 2, $\dotsc$, 100г разложили на две чаши весов так, что есть равновесие. Докажите, что можно убрать по 2 гирьки с каждой чаши так что равновесие не нарушится.
	
	\item $n$ гирек веса 1, 2, $\dotsc$, $n$г разложили на две чаши весов так, что есть равновесие. Верно ли, что для любого $n > 3$ можно убрать по 2 гирьки с каждой чаши так что равновесие не нарушится?
	
	\item На окружности сидят 239 птиц, причем в одной точке могут сидеть несколько птиц. Две птицы видят друг друга, если соединяющая их дуга не больше $10^\circ$. Найдите наименьшее возможное количество пар птиц, видящих друг друга.
	
	\item В клетчатом квадрате $N\times N$ отмечены центры $3N$ клеток. Докажите, что среди попарных расстояний между отмеченными точками какие-то два отличаются в два раза.
\end{enumerate}
\end{document}