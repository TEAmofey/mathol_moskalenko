\documentclass{article}
\usepackage[12pt]{extsizes}
\usepackage[T2A]{fontenc}
\usepackage[utf8]{inputenc}
\usepackage[english, russian]{babel}

\usepackage{amssymb}
\usepackage{amsfonts}
\usepackage{amsmath}
\usepackage{enumitem}
\usepackage{graphics}
\usepackage{graphicx}

\usepackage{lipsum}

\newtheorem{theorem}{Теорема}
\newtheorem{task}{Задача}
\newtheorem{lemma}{Лемма}
\newtheorem{definition}{Определение}
\newtheorem{example}{Пример}
\newtheorem{statement}{Утверждение}
\newtheorem{corollary}{Следствие}


\usepackage{geometry} % Меняем поля страницы
\geometry{left=1cm}% левое поле
\geometry{right=1cm}% правое поле
\geometry{top=1.5cm}% верхнее поле
\geometry{bottom=1cm}% нижнее поле


\usepackage{fancyhdr} % Headers and footers
\pagestyle{fancy} % All pages have headers and footers
\fancyhead{} % Blank out the default header
\fancyfoot{} % Blank out the default footer
\fancyhead[L]{Математика}
\fancyhead[C]{\textit{Разнобой}}
\fancyhead[R]{15 января 2024}% Custom header text


%----------------------------------------------------------------------------------------

%\begin{document}\normalsize
\begin{document}\large
	
\begin{center}
	\textbf{Разнобой}
\end{center}


\begin{enumerate}[label*=\protect\fbox{\arabic{enumi}}]

\item Серия для шестиклассников, выданная на первом занятии январской математической смены, состоит из 7 задач. Из 16 ребят каждый сдал больше половины серии. Докажите, что какую-то задачу решило не менее 10 человек.

\item Найдите последнюю цифру числа $7^{49}$.


\item Одним взмахом меча Богатырь И. Муромец может отсечь З. Горынычу одну голову или сразу 10 голов (конечно, если их было не меньше 10). Однако, если после отсечения осталось чётное число голов, то число голов сразу же удваивается. Изначально, у З. Горыныча 444 головы. Может ли Богатырь отрубить З. Горынычу все его головы?

\item Вова забыл номер телефона ($8\,931\star\star\star\star\star\star\,\star$) своей девушки Маши, но помнит, что дальше в нем были только девятки и двойки, причем двоек было больше, чем девяток. Также он вспомнил, что номер (как 11-значное число) делился на три и на четыре. Помогите Вове вспомнить номер Маши.

\item Ы и Ь --- только такие две буквы есть в алфавите некоторого племени. Каждая конечная последовательность из этих букв является словом и что-то да значит. От замен следующих буквосочетаний в словах смысл слова не меняется: ЫЫЬ\,$\Leftrightarrow$\,ЬЬ, ЬЫЫ\,$\Leftrightarrow$\,ЫЬЫ, ЬЫЬ\,$\Leftrightarrow$\,ЫЬ, ЬЬЬ\,$\Leftrightarrow$\,ЫЫ (замену можно делать в любом месте слова). Обязательно ли смысл у слов ЫЬЫ и ЬЫЬ одинаков?

\item Могут ли все три числа $a, b, c$ быть меньше $-1$, если известно, что $$ab+a+b = c?$$

\item Гриша вычислил сумму первых 2024 чисел из ряда $9, 99, 999, \ldots$. Сколько различных цифр содержит результат?

\item Олег лёг спать в 10 вечера и завел будильник (со стрелками и~циферблатом на 12 делений) на 7 утра. Ночью в некоторый момент будильник, до этого работавший исправно, сломался, и его стрелки пошли в обратную сторону (с прежней скоростью). Тем не менее, утром будильник прозвенел точно в положенное время. Во сколько сломался будильник? 

\item Докажите, что из чисел $x+y-2z$, $x+z-2y$, $y+z-2x$ хотя бы одно неотрицательно.

\item Одно утверждение среди следующих неверно. Найдите его: 

- если точный квадрат делится на 6, то он делится на 36;

- если точный квадрат делится на 7, то он делится на 49;

- если точный квадрат делится на 8, то он делится на 64.

\item Максим отметил несколько клеток в квадрате $12 \times 12$ так, что в любом a) прямоугольнике $1 \times 3$; b) $L$--тетрамино есть отмеченная клетка. Какое наименьшее число клеток мог отметить Максим?

\item Проверь свою наблюдательность. В условиях задач скрыто некоторое \emph{послание}. Какое? 

\end{enumerate}


\end{document}