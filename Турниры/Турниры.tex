\documentclass{article}

\usepackage[12pt]{extsizes}
\usepackage[T2A]{fontenc}
\usepackage[utf8]{inputenc}
\usepackage[english, russian]{babel}

\usepackage{mathrsfs}
\usepackage[dvipsnames]{xcolor}

\usepackage{amsmath}
\usepackage{amssymb}
\usepackage{amsthm}
\usepackage{indentfirst}
\usepackage{amsfonts}
\usepackage{enumitem}
\usepackage{graphics}
\usepackage{tikz}
\usepackage{tabu}
\usepackage{diagbox}
\usepackage{hyperref}
\usepackage{mathtools}
\usepackage{ucs}
\usepackage{lipsum}
\usepackage{geometry} % Меняем поля страницы
\usepackage{fancyhdr} % Headers and footers
\newcommand{\range}{\mathrm{range}}
\newcommand{\dom}{\mathrm{dom}}
\newcommand{\N}{\mathbb{N}}
\newcommand{\R}{\mathbb{R}}
\newcommand{\E}{\mathbb{E}}
\newcommand{\D}{\mathbb{D}}
\newcommand{\M}{\mathcal{M}}
\newcommand{\Prime}{\mathbb{P}}
\newcommand{\A}{\mathbb{A}}
\newcommand{\Q}{\mathbb{Q}}
\newcommand{\Z}{\mathbb{Z}}
\newcommand{\F}{\mathbb{F}}
\newcommand{\CC}{\mathbb{C}}

\DeclarePairedDelimiter\abs{\lvert}{\rvert}
\DeclarePairedDelimiter\floor{\lfloor}{\rfloor}
\DeclarePairedDelimiter\ceil{\lceil}{\rceil}
\DeclarePairedDelimiter\lr{(}{)}
\DeclarePairedDelimiter\set{\{}{\}}
\DeclarePairedDelimiter\norm{\|}{\|}

\renewcommand{\labelenumi}{(\alph{enumi})}

\newcommand{\smallindent}{
    \geometry{left=1cm}% левое поле
    \geometry{right=1cm}% правое поле
    \geometry{top=1.5cm}% верхнее поле
    \geometry{bottom=1cm}% нижнее поле
}

\newcommand{\header}[3]{
    \pagestyle{fancy} % All pages have headers and footers
    \fancyhead{} % Blank out the default header
    \fancyfoot{} % Blank out the default footer
    \fancyhead[L]{#1}
    \fancyhead[C]{#2}
    \fancyhead[R]{#3}
}

\newcommand{\dividedinto}{
    \,\,\,\vdots\,\,\,
}

\newcommand{\littletaller}{\mathchoice{\vphantom{\big|}}{}{}{}}

\newcommand\restr[2]{{
    \left.\kern-\nulldelimiterspace % automatically resize the bar with \right
    #1 % the function
    \littletaller % pretend it's a little taller at normal size
    \right|_{#2} % this is the delimiter
}}

\DeclareGraphicsExtensions{.pdf,.png,.jpg}

\newenvironment{enumerate_boxed}[1][enumi]{\begin{enumerate}[label*=\protect\fbox{\arabic{#1}}]}{\end{enumerate}}



\smallindent

\header{ЦРОД $\bullet$ Математика}{\textit{Логика}}{Май 2022}

%----------------------------------------------------------------------------------------

\begin{document}
    \large

    \begin{center}
        \textbf{Турниры}
    \end{center}

    \begin{enumerate_boxed}

        \item В однокруговом шахматном турнире (каждый играет с каждым ровно 1 раз) участвовало 20 человек.
        Сколько всего было сыграно партий?

        \item 20 команд сыграли турнир по олимпийской системе (тот, кто проиграл - выбывает).
        Сколько всего было сыграно матчей?

        \item В однокруговом шахматном турнире было сыграно 105 партий.
        Сколько всего участников на этом турнире?

        \item Трое друзей играли в шашки.
        Один из них сыграл 32 игр, а другой – 18 игр.
        Мог ли третий участник сыграть a) 36; b) 37; c) 58 игр?

        \item По окончании шахматного турнира Незнайка сказал: <<Я набрал на $3.5$ очка больше, чем потерял>>.
        Могут ли его слова быть правдой?
        (Победа – 1 очко, ничья – $\frac{1}{2}$ очка, поражение – 0.)

        \item В турнире по волейболу, прошедшем в один круг, $5\%$ всех команд не выиграли ни одной игры.
        Сколько было команд?

        \item Каждый участник шахматных соревнований выиграл чёрными столько же партий, сколько все остальные вместе взятые – белыми.
        Докажите, что все участники выиграли поровну партий.

        \item  Пять футбольных команд провели турнир – каждая команда сыграла с каждой по разу.
        За победу начислялось 3 очка, за ничью – 1 очко, за проигрыш очков не давалось.
        Четыре команды набрали соответственно 1, 2, 5 и 12 очков.
        А сколько очков набрала пятая команда?

        \item В однокруговом шахматном турнире участвовало 8 человек, и в итоге они набрали разное количество очков (победа – 1 очко, ничья – 0,5 очка, поражение – 0).
        Шахматист, занявший второе место, набрал столько же очков, сколько четверо последних набрали вместе.
        Как сыграли между собой шахматисты, занявшие третье и шестое места?

        \item В гандбольном турнире в один круг (победа – 2 очка, ничья – 1 очко, поражение – 0) приняло участие 16 команд.
        Все команды набрали разное количество очков, причём команда, занявшая седьмое место, набрала 21 очко.
        Докажите, что победившая команда хотя бы один раз сыграла вничью.

        \item 20 шахматистов сыграли турнир в один круг.
        Корреспондент <<Спортивной газеты>> написал в своей заметке, что каждый участник этого турнира выиграл столько же партий, сколько и свёл вничью.
        Докажите, что корреспондент ошибся.

        \item Восемь волейбольных команд провели турнир в один круг (каждая команда сыграла с каждой один раз).
        Доказать, что можно выделить такие четыре команды $A, B, C$ и $D$, что $A$ выиграла у $B, C$ и $D$; B выиграла у $C$ и $D$, $C$ выиграла у $D$.

    \end{enumerate_boxed}

\end{document}