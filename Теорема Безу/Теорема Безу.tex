\documentclass{article}

\usepackage[12pt]{extsizes}
\usepackage[T2A]{fontenc}
\usepackage[utf8]{inputenc}
\usepackage[english, russian]{babel}

\usepackage{mathrsfs}
\usepackage[dvipsnames]{xcolor}

\usepackage{amsmath}
\usepackage{amssymb}
\usepackage{amsthm}
\usepackage{indentfirst}
\usepackage{amsfonts}
\usepackage{enumitem}
\usepackage{graphics}
\usepackage{tikz}
\usepackage{tabu}
\usepackage{diagbox}
\usepackage{hyperref}
\usepackage{mathtools}
\usepackage{ucs}
\usepackage{lipsum}
\usepackage{geometry} % Меняем поля страницы
\usepackage{fancyhdr} % Headers and footers
\newcommand{\range}{\mathrm{range}}
\newcommand{\dom}{\mathrm{dom}}
\newcommand{\N}{\mathbb{N}}
\newcommand{\R}{\mathbb{R}}
\newcommand{\E}{\mathbb{E}}
\newcommand{\D}{\mathbb{D}}
\newcommand{\M}{\mathcal{M}}
\newcommand{\Prime}{\mathbb{P}}
\newcommand{\A}{\mathbb{A}}
\newcommand{\Q}{\mathbb{Q}}
\newcommand{\Z}{\mathbb{Z}}
\newcommand{\F}{\mathbb{F}}
\newcommand{\CC}{\mathbb{C}}

\DeclarePairedDelimiter\abs{\lvert}{\rvert}
\DeclarePairedDelimiter\floor{\lfloor}{\rfloor}
\DeclarePairedDelimiter\ceil{\lceil}{\rceil}
\DeclarePairedDelimiter\lr{(}{)}
\DeclarePairedDelimiter\set{\{}{\}}
\DeclarePairedDelimiter\norm{\|}{\|}

\renewcommand{\labelenumi}{(\alph{enumi})}

\newcommand{\smallindent}{
    \geometry{left=1cm}% левое поле
    \geometry{right=1cm}% правое поле
    \geometry{top=1.5cm}% верхнее поле
    \geometry{bottom=1cm}% нижнее поле
}

\newcommand{\header}[3]{
    \pagestyle{fancy} % All pages have headers and footers
    \fancyhead{} % Blank out the default header
    \fancyfoot{} % Blank out the default footer
    \fancyhead[L]{#1}
    \fancyhead[C]{#2}
    \fancyhead[R]{#3}
}

\newcommand{\dividedinto}{
    \,\,\,\vdots\,\,\,
}

\newcommand{\littletaller}{\mathchoice{\vphantom{\big|}}{}{}{}}

\newcommand\restr[2]{{
    \left.\kern-\nulldelimiterspace % automatically resize the bar with \right
    #1 % the function
    \littletaller % pretend it's a little taller at normal size
    \right|_{#2} % this is the delimiter
}}

\DeclareGraphicsExtensions{.pdf,.png,.jpg}

\newenvironment{enumerate_boxed}[1][enumi]{\begin{enumerate}[label*=\protect\fbox{\arabic{#1}}]}{\end{enumerate}}



\smallindent

\header{Математика}{\textit{Алгебра}}{13 октября 2022}

%----------------------------------------------------------------------------------------

\begin{document}
    \large


    \begin{center}
        \textbf{Теорема Безу}
    \end{center}

    \begin{definition}
        Многочлен $A$ \textit{делится} на ненулевой многочлен $B$, если
        существует многочлен $Q$, называемый \textit{частным} такой, что $A=B\cdot Q$.
    \end{definition}


    \begin{definition}
        Разделить многочлен $A$ на ненулевой многочлен $B$ с
        остатком --- это найти многочлены $Q$, $R$ такие, что выполнено равенство
        $A=B\cdot Q+R$, причем $\deg R<\deg B$\footnote{ или $R=0$}.
        Многочлен $Q$ называется \textit{неполным частным}, многочлен $R$ называется \textit{остатком}.
    \end{definition}

    \begin{enumerate_boxed}

        \item Поделите многочлен $x^4-3x^3+3x^2+ax+b$ в столбик на $x^2-3x+2$.
        При каких $a$
        и $b$ остаток будет равен нулю?

        \item При каких $n$ многочлен $x^n+x+1$ делится на многочлен $x^2+x+1$? \textit{Указание: начните
        делить в столбик.}

        \item Алгоритм деления <<в столбик>> даёт нам пример каких-то $Q$ и $R$, для которых
        $A=B\cdot Q+R$.
        Докажите, что других $Q$ и $R$ быть не может, т.е. докажите, что
        неполное частное и остаток при делении $A$ на $B$ определяются однозначно. \textit{Указание:
        предположите противное.}

        \item  a) Разделите многочлен $x^{100}$ на $x-1$ и $x+1$ с остатком.

        b) Чему равны остатки при делении многочлена $P(x)=a_{n}x^n+\ldots+a_0$
        на многочлены $x-1$ и $x+1$? Сформулируйте и докажите признаки делимости на эти многочлены.
        Признаки делимости на какие числа они вам напоминают?

        \item \textbf {Теорема Безу.} Докажите, что остаток от деления многочлена $P(x)$ на $x-a$ равен $P(a)$,
        т.е. $P(x)=Q(x)(x-a)+P(a)$.
        Выведите из этого, что число $a$ является корнем многочлена $P(x)$
        тогда и только тогда, когда $P(x)$ делится на $x-a$.

        \item Докажите, что если $x_1$, $x_2$, \ldots, $x_n$ --- различные корни многочлена~$P$,
        то он делится на многочлен $(x-x_1)(x-x_2)\ldots (x-x_n)$.
        Верно ли обратное утверждение?

        \item При каких $a$ и $b$ многочлен $x^4-3x^3+3x^2+ax+b$  делится на многочлен $(x-1)(x-2)$?

        \item Докажите, что многочлен степени $n$ имеет не более $n$ различных корней.

        \item Докажите, что если значения двух многочленов степени не выше $n$ совпадают в $n+1$ точке, то и сами многочлены равны.

        \item Пусть $a$, $b$, $c$ --- натуральные числа.
        Верно ли, что обязательно существует квадратный трёхчлен с целыми коэффициентами, который в некоторых целых точках принимает значения $a^3$, $b^3$, $c^3$?


    \end{enumerate_boxed}
\end{document}