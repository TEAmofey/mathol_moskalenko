\documentclass{article}

\usepackage[12pt]{extsizes}
\usepackage[T2A]{fontenc}
\usepackage[utf8]{inputenc}
\usepackage[english, russian]{babel}

\usepackage{mathrsfs}
\usepackage[dvipsnames]{xcolor}

\usepackage{amsmath}
\usepackage{amssymb}
\usepackage{amsthm}
\usepackage{indentfirst}
\usepackage{amsfonts}
\usepackage{enumitem}
\usepackage{graphics}
\usepackage{tikz}
\usepackage{tabu}
\usepackage{diagbox}
\usepackage{hyperref}
\usepackage{mathtools}
\usepackage{ucs}
\usepackage{lipsum}
\usepackage{geometry} % Меняем поля страницы
\usepackage{fancyhdr} % Headers and footers
\newcommand{\range}{\mathrm{range}}
\newcommand{\dom}{\mathrm{dom}}
\newcommand{\N}{\mathbb{N}}
\newcommand{\R}{\mathbb{R}}
\newcommand{\E}{\mathbb{E}}
\newcommand{\D}{\mathbb{D}}
\newcommand{\M}{\mathcal{M}}
\newcommand{\Prime}{\mathbb{P}}
\newcommand{\A}{\mathbb{A}}
\newcommand{\Q}{\mathbb{Q}}
\newcommand{\Z}{\mathbb{Z}}
\newcommand{\F}{\mathbb{F}}
\newcommand{\CC}{\mathbb{C}}

\DeclarePairedDelimiter\abs{\lvert}{\rvert}
\DeclarePairedDelimiter\floor{\lfloor}{\rfloor}
\DeclarePairedDelimiter\ceil{\lceil}{\rceil}
\DeclarePairedDelimiter\lr{(}{)}
\DeclarePairedDelimiter\set{\{}{\}}
\DeclarePairedDelimiter\norm{\|}{\|}

\renewcommand{\labelenumi}{(\alph{enumi})}

\newcommand{\smallindent}{
    \geometry{left=1cm}% левое поле
    \geometry{right=1cm}% правое поле
    \geometry{top=1.5cm}% верхнее поле
    \geometry{bottom=1cm}% нижнее поле
}

\newcommand{\header}[3]{
    \pagestyle{fancy} % All pages have headers and footers
    \fancyhead{} % Blank out the default header
    \fancyfoot{} % Blank out the default footer
    \fancyhead[L]{#1}
    \fancyhead[C]{#2}
    \fancyhead[R]{#3}
}

\newcommand{\dividedinto}{
    \,\,\,\vdots\,\,\,
}

\newcommand{\littletaller}{\mathchoice{\vphantom{\big|}}{}{}{}}

\newcommand\restr[2]{{
    \left.\kern-\nulldelimiterspace % automatically resize the bar with \right
    #1 % the function
    \littletaller % pretend it's a little taller at normal size
    \right|_{#2} % this is the delimiter
}}

\DeclareGraphicsExtensions{.pdf,.png,.jpg}

\newenvironment{enumerate_boxed}[1][enumi]{\begin{enumerate}[label*=\protect\fbox{\arabic{#1}}]}{\end{enumerate}}



\smallindent

\header{Математика}{\textit{Алгебра}}{2 октября 2022}

%----------------------------------------------------------------------------------------

\begin{document}
    \large

    \begin{center}
        \textbf{Квадратный трёхчлен}
    \end{center}

    \begin{enumerate_boxed}

        \item Решите уравнение: а) $x^2 + 2023x + 2022 = 0$; б) $1999x^2 + 1000x-2999=0$.

        \item При каких значениях $a$ уравнение $2x^2 + (a - 3)x + 81 = 0$ имеет 1 корень?

        \item Пусть $x_1, x_2$~--- корни уравнения  $x^2 + px + q = 0$.
        Выразите через $p$ и $q$ следующие выражения:
        \begin{enumerate}

            \item $\dfrac{1}{x_1} + \dfrac{1}{x_2}$

            \item $\dfrac{1}{x_1^2} + \dfrac{1}{x_2^2}$

        \end{enumerate}

        \item Докажите, что при любых $a$ и $b$ уравнение имеет решение:
        \[(a^2 - b^2)x^2 + 2(a^3 - b^3)x + (a^4 -b^4) =0.\]

        \item Рассматриваются квадратичные функции  $y = x^2 + px + q$,  для которых  $p + q = 2021$.
        Покажите, что параболы, являющиеся графиками этих функций, пересекаются в одной точке.

        \item Сумма четырех корней двух квадратных трехчленов $f(x)$ и $g(x)$ с одинаковыми старшими коэффициентами равна нулю.
        Известно, что квадратный трехчлен $f(x)+g(x)$ имеет корни.
        Докажите, что их сумма также равна нулю.

        \item Два различных числа $x$ и $y$ (не обязательно целых) таковы, что \[x^2 - 2000x = y^2 - 2000y\].
        Найдите сумму чисел $x$ и $y$.

        \item Докажите, что если $x_1 , x_2$ корни приведенного квадратного трехчлена, то выполнено равенство $(x_2 - x_1)^2 = D$, где $D$ – его дискриминант.

        \item Дискриминанты трёх приведённых квадратных трёхчленов равны 1, 4 и 9.
        Докажите, что можно выбрать по одному корню каждого из них так, чтобы их сумма равнялась сумме оставшихся корней.

        \item Существуют ли такие три квадратных трёхчлена, что каждый из них имеет корень, а сумма любых двух из них корней не имеет?

        \item Квадратный трехчлен  $y = ax^2 + bx + c$  не имеет корней и  $a + b + c > 0$.
        Найдите знак коэффициента $c$.

        \item Ненулевые числа $a$ и $b$ таковы, что уравнение $a(x - a)^2 + b(x - b)^2 = 0$ имеет единственное решение.
        Докажите, что $|a| = |b|$.

        \item Квадратный трёхчлен $f(x) = ax^2 + bx + c$ принимает в точках $\frac{1}{a}$ и $c$ значения разных знаков.
        Докажите, что корни трёхчлена $f(x)$ имеют разные знаки.

        \item Верно ли, что если  $b > a + c > 0$,  то квадратное уравнение $ax^2 + bx + c = 0$ имеет два корня?

        \item Про действительные числа $a, b, c$ известно, что $(a + b + c)\cdot c < 0$.
        Докажите, что $b^2 - 4ac>0$.

        \item Найдите все целые $a$, при которых уравнение $x^2 + ax + a = 0$ имеет целый корень.

    \end{enumerate_boxed}
\end{document}