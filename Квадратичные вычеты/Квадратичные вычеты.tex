\documentclass{article}

\usepackage[12pt]{extsizes}
\usepackage[T2A]{fontenc}
\usepackage[utf8]{inputenc}
\usepackage[english, russian]{babel}

\usepackage{mathrsfs}
\usepackage[dvipsnames]{xcolor}

\usepackage{amsmath}
\usepackage{amssymb}
\usepackage{amsthm}
\usepackage{indentfirst}
\usepackage{amsfonts}
\usepackage{enumitem}
\usepackage{graphics}
\usepackage{tikz}
\usepackage{tabu}
\usepackage{diagbox}
\usepackage{hyperref}
\usepackage{mathtools}
\usepackage{ucs}
\usepackage{lipsum}
\usepackage{geometry} % Меняем поля страницы
\usepackage{fancyhdr} % Headers and footers
\newcommand{\range}{\mathrm{range}}
\newcommand{\dom}{\mathrm{dom}}
\newcommand{\N}{\mathbb{N}}
\newcommand{\R}{\mathbb{R}}
\newcommand{\E}{\mathbb{E}}
\newcommand{\D}{\mathbb{D}}
\newcommand{\M}{\mathcal{M}}
\newcommand{\Prime}{\mathbb{P}}
\newcommand{\A}{\mathbb{A}}
\newcommand{\Q}{\mathbb{Q}}
\newcommand{\Z}{\mathbb{Z}}
\newcommand{\F}{\mathbb{F}}
\newcommand{\CC}{\mathbb{C}}

\DeclarePairedDelimiter\abs{\lvert}{\rvert}
\DeclarePairedDelimiter\floor{\lfloor}{\rfloor}
\DeclarePairedDelimiter\ceil{\lceil}{\rceil}
\DeclarePairedDelimiter\lr{(}{)}
\DeclarePairedDelimiter\set{\{}{\}}
\DeclarePairedDelimiter\norm{\|}{\|}

\renewcommand{\labelenumi}{(\alph{enumi})}

\newcommand{\smallindent}{
    \geometry{left=1cm}% левое поле
    \geometry{right=1cm}% правое поле
    \geometry{top=1.5cm}% верхнее поле
    \geometry{bottom=1cm}% нижнее поле
}

\newcommand{\header}[3]{
    \pagestyle{fancy} % All pages have headers and footers
    \fancyhead{} % Blank out the default header
    \fancyfoot{} % Blank out the default footer
    \fancyhead[L]{#1}
    \fancyhead[C]{#2}
    \fancyhead[R]{#3}
}

\newcommand{\dividedinto}{
    \,\,\,\vdots\,\,\,
}

\newcommand{\littletaller}{\mathchoice{\vphantom{\big|}}{}{}{}}

\newcommand\restr[2]{{
    \left.\kern-\nulldelimiterspace % automatically resize the bar with \right
    #1 % the function
    \littletaller % pretend it's a little taller at normal size
    \right|_{#2} % this is the delimiter
}}

\DeclareGraphicsExtensions{.pdf,.png,.jpg}

\newenvironment{enumerate_boxed}[1][enumi]{\begin{enumerate}[label*=\protect\fbox{\arabic{#1}}]}{\end{enumerate}}



\smallindent

\header{Математика}{\textit{Теория чисел}}{25 ноября 2023}

%----------------------------------------------------------------------------------------

%\begin{document}\normalsize
\begin{document}
    \large

    \begin{center}
        \textbf{Квадратичные вычеты}
    \end{center}

    \textbf{Определение:}
    Зафиксируем простое число $p$.
    Для числа $a$, не делящегося на $p$, рассмотрим сравнение $x^2 \equiv a \pmod{p}$.
    Если это сравнение имеет решение, то число $a$ называется квадратичным вычетом по модулю $p$, в противном случае — квадратичным невычетом по модулю $p$.
    Достаточно часто слово «квадратичный» мы будем опускать.

    \textbf{Свойства:}

    \begin{enumerate_boxed}

        \item Пусть $p > 2$.
        Докажите, что

        \begin{enumerate}

            \item по модулю $p$ существует ровно $\frac{p-1}{2}$ квадратичных вычетов и столько же невычетов;

            \item произведение двух квадратичных вычетов — вычет;

            \item произведение вычета на невычет — невычет;

            \item произведение двух невычетов — вычет.

        \end{enumerate}

        \item Докажите, что все квадратичные вычеты являются корнями многочлена $x^{\frac{p-1}{2}} - 1 \in \mathbb{F}_p[x]$, а все невычеты — корнями многочлена $x^{\frac{p-1}{2}} + 1 \in \mathbb{F}_p[x]$.

    \end{enumerate_boxed}

    \textbf{Определение:} \textit{Символом Лежандра} называется выражение, обозначаемое $\left(\frac{a}{p}\right)$, равное 1, если $a$ — квадратичный вычет по модулю $p$; равное $-1$, если $a$ — невычет по модулю $p$ и 0, если $a$ кратно $p$.

    Из свойств 1 и 2 следует, что $\left(\frac{a}{p}\right) \cdot \left(\frac{b}{p}\right) = \left(\frac{ab}{p}\right)$, а также $\left(\frac{a}{p}\right) \equiv a^{\frac{p-1}{2}} \pmod{p}$.

    \textbf{Задачи:}

    \begin{enumerate_boxed}

        \item Пусть $p = 163$, $\left(\frac{a}{p}\right)$ — символ Лежандра.
        Чему равно $\sum\limits_{a = 1}^p\left(\frac{a}{p}\right)$?

        \item Докажите, что если $x^2 + 1$ делится на $p$, то $p$ имеет вид $4k + 1$.

        \item Покажите, что для каждого простого числа $ p $ существуют целые числа $ a $ и $ b $, такие что $ a^2 + b^2 +1 $ кратно $ p $.

        \item Докажите, что $\left(\frac{-1}{p}\right) = (-1)^{\frac{p - 1}{2}}$.

        \item Решите в целых числах уравнение $z(y^2 - 5) = x^2 + 1$.

        \item Докажите, что уравнение $4xy - x - y = z^2$ (a) не имеет решений в натуральных числах; (b) имеет бесконечно много решений в целых числах.

        \item Решите в целых числах уравнение $x^3 + 7 = y^2$.

        \item Докажите, что $\left(\frac{2}{p}\right) = (-1)^{\frac{p^2 - 1}{8}}$.

        \item \textbf{Лемма Эйзенштейна} Докажите, что
        \[\left(\frac{a}{p}\right) = (-1)^{\sum\limits_{n=1}^{(p-1)/2}\left\lfloor\frac{2an}{p}\right\rfloor}.\]

        \item Четность числа $\varepsilon(q)$ совпадает с четностью числа целых точек в треугольнике, заданном неравенствами $0 < x<\frac{p}{2} , 0 < y < \frac{q}{2}, y < \frac{qx}{p}$

        \item \textbf{Квадратичный закон взаимности} Для различных нечетных простых чисел имеет место равенство
        \[\left(\frac{p}{q}\right) \cdot \left(\frac{q}{p}\right) = (-1)^{\frac{p-1}{2}\cdot\frac{q-1}{2}}\]

        \item Является ли число 74 квадратичным вычетом по модулю 131?

        \item Целое число $a$ таково, что $a^2 - 6a + 3$ делится на некоторое простое $p$.
        Докажите, что существует целое число $b$ такое, что $b^2 - 2b - 53$ делится на $p$.

        \item Дано натуральное $a$, не делящееся на простое $p$.
        Рассмотрим перестановку чисел $0, 1, \ldots, p-1$, на $i$-м месте которой стоит остаток $ai$ от деления на $p$.
        Докажите, что эта перестановка четна при $\left(\frac{a}{p}\right) = 1$ и нечетна при $\left(\frac{a}{p}\right) = -1$

        \item Для простого $p$ найдите значение выражения \[\sum\limits_{a = 1}^{p-1}\left(\frac{a^2 + a}{p}\right)\]

        \item Докажите, что для простого числа $p > 2$ наименьший квадратичный невычет по модулю $p$ меньше $1 + \sqrt{p}$.

    \end{enumerate_boxed}
\end{document}