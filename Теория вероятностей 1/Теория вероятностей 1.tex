\documentclass{article}
\usepackage[12pt]{extsizes}
\usepackage[T2A]{fontenc}
\usepackage[utf8]{inputenc}
\usepackage[english, russian]{babel}

\usepackage{amssymb}
\usepackage{amsfonts}
\usepackage{amsmath}
\usepackage{enumitem}
\usepackage{graphics}
\usepackage{graphicx}

\usepackage{lipsum}

\newtheorem{theorem}{Теорема}
\newtheorem{task}{Задача}
\newtheorem{lemma}{Лемма}
\newtheorem{definition}{Определение}
\newtheorem{example}{Пример}
\newtheorem{statement}{Утверждение}
\newtheorem{corollary}{Следствие}


\usepackage{geometry} % Меняем поля страницы
\geometry{left=1cm}% левое поле
\geometry{right=1cm}% правое поле
\geometry{top=1.5cm}% верхнее поле
\geometry{bottom=1cm}% нижнее поле


\usepackage{fancyhdr} % Headers and footers
\pagestyle{fancy} % All pages have headers and footers
\fancyhead{} % Blank out the default header
\fancyfoot{} % Blank out the default footer
\fancyhead[L]{Математика}
\fancyhead[C]{\textit{Теория вероятностей}}
\fancyhead[R]{16 апреля}% Custom header text


%----------------------------------------------------------------------------------------

%\begin{document}\normalsize
\begin{document}\large
	
\begin{center}
	\textbf{Случайные события}
\end{center}


\begin{enumerate}[label*=\protect\fbox{\arabic{enumi}}]
	
\item Брошено 6 игральных кубиков. Найти вероятность следующих событий:
\begin{itemize}
	\item $A= \{\text{среди выпавших нет единиц}\}.$
	\item $B= \{\text{выпало ровно 3 двойки}\}.$
	\item $C= \{\text{выпала хотя бы 1 единица}\}.$
	\item $D= \{\text{все цифры выпали хотя бы по одному разу}\}.$
\end{itemize}

\item Карточка <<спортлото>> содержит 36 чисел. Игрок может выбрать 6, а выигрышных номеров в тираже определяется тоже 6. Какова вероятность того, что верно будет угадано ровно 3 числа?

\item Подбрасывают симметричную монету.
\begin{itemize}
	\item С какой вероятностью при $n$ подбрасываниях выпадет ровно $k$ орлов?
	\item Какое $k$ наиболее вероятно?
\end{itemize}

\item  При бросании неправильной монеты орел выпадает с вероятностью $p$, решка — с вероятностью $q=  1-p$. С какой вероятностью после $n$ бросков выпадет четное число орлов?

\item Из колоды карт (52 карты) наугад вытаскивают 5. Что более вероятно — вытащить ровно3 карты одного номинала или вытащить ровно 2 пары карт одного номинала?

\item Колоду карт случайным образом делят на 2 части (необязательно равного размера). С какой вероятностью в каждой части будет по 2 туза?

\item Из контейнера $A$, в котором было 1000 зеленых яблок и 3000 красных яблок, взяли половину яблок и перенесли в контейнер $B$, в котором к тому времени уже лежало 3000 зеленых и 1000 красных яблок. Затем из контейнера $B$ извлекли одно яблоко. Найти вероятность того, что оно зеленое.

\item Рассмотрим простейшее случайное блуждание на прямой: частица находится в 0 и каждым шагом сдвигается на 1 вправо или влево. Всего сделано $n$ шагов, все траектории равновероятны. Найти вероятность события $$A_k = \{\text{блуждание завершилось в точке с координатой $k$}\}.$$

\item Каждый из двух игроков подбрасывает симметричную монету $n$ раз. С какой вероятно-стью у них выпадет одинаковое число орлов?


\end{enumerate}
\end{document}