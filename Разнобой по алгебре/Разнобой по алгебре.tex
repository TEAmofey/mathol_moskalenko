\documentclass{article}

\usepackage[12pt]{extsizes}
\usepackage[T2A]{fontenc}
\usepackage[utf8]{inputenc}
\usepackage[english, russian]{babel}

\usepackage{mathrsfs}
\usepackage[dvipsnames]{xcolor}

\usepackage{amsmath}
\usepackage{amssymb}
\usepackage{amsthm}
\usepackage{indentfirst}
\usepackage{amsfonts}
\usepackage{enumitem}
\usepackage{graphics}
\usepackage{tikz}
\usepackage{tabu}
\usepackage{diagbox}
\usepackage{hyperref}
\usepackage{mathtools}
\usepackage{ucs}
\usepackage{lipsum}
\usepackage{geometry} % Меняем поля страницы
\usepackage{fancyhdr} % Headers and footers
\newcommand{\range}{\mathrm{range}}
\newcommand{\dom}{\mathrm{dom}}
\newcommand{\N}{\mathbb{N}}
\newcommand{\R}{\mathbb{R}}
\newcommand{\E}{\mathbb{E}}
\newcommand{\D}{\mathbb{D}}
\newcommand{\M}{\mathcal{M}}
\newcommand{\Prime}{\mathbb{P}}
\newcommand{\A}{\mathbb{A}}
\newcommand{\Q}{\mathbb{Q}}
\newcommand{\Z}{\mathbb{Z}}
\newcommand{\F}{\mathbb{F}}
\newcommand{\CC}{\mathbb{C}}

\DeclarePairedDelimiter\abs{\lvert}{\rvert}
\DeclarePairedDelimiter\floor{\lfloor}{\rfloor}
\DeclarePairedDelimiter\ceil{\lceil}{\rceil}
\DeclarePairedDelimiter\lr{(}{)}
\DeclarePairedDelimiter\set{\{}{\}}
\DeclarePairedDelimiter\norm{\|}{\|}

\renewcommand{\labelenumi}{(\alph{enumi})}

\newcommand{\smallindent}{
    \geometry{left=1cm}% левое поле
    \geometry{right=1cm}% правое поле
    \geometry{top=1.5cm}% верхнее поле
    \geometry{bottom=1cm}% нижнее поле
}

\newcommand{\header}[3]{
    \pagestyle{fancy} % All pages have headers and footers
    \fancyhead{} % Blank out the default header
    \fancyfoot{} % Blank out the default footer
    \fancyhead[L]{#1}
    \fancyhead[C]{#2}
    \fancyhead[R]{#3}
}

\newcommand{\dividedinto}{
    \,\,\,\vdots\,\,\,
}

\newcommand{\littletaller}{\mathchoice{\vphantom{\big|}}{}{}{}}

\newcommand\restr[2]{{
    \left.\kern-\nulldelimiterspace % automatically resize the bar with \right
    #1 % the function
    \littletaller % pretend it's a little taller at normal size
    \right|_{#2} % this is the delimiter
}}

\DeclareGraphicsExtensions{.pdf,.png,.jpg}

\newenvironment{enumerate_boxed}[1][enumi]{\begin{enumerate}[label*=\protect\fbox{\arabic{#1}}]}{\end{enumerate}}



\smallindent

\header{ЦРОД $\bullet$ Математика}{\textit{Алгебра}}{ЛФМШ 2022}

%----------------------------------------------------------------------------------------

\begin{document}
    \large

    \begin{center}
        \textbf{Алгебраический разнобой}
    \end{center}

    \begin{enumerate_boxed}

        \item Докажите, что если $a^2 + b^2 + c^2 = ab + bc + ca$, то $a = b = c$.

        \item  Даны квадратные трёхчлены $f_1(x) = x^2 + 2a_{1}x + b_1$, $f_2(x) = x^2 + 2a_{2}x + b_2$, $f_3(x) = x^2 + 2a_{3}x + b_3$, причем $a_{1}a_{2}a_3 = b_{1}b_{2}b_3 > 1$.
        Докажите, что хотя бы один из этих трёхчленов имеет два корня.

        \item Даны квадратные трехчлены $f_1(x), f_2(x), \ldots, f_{100}(x)$ с одинаковыми коэффициентами при $x^2$, одинаковыми коэффициентами при $x$, но различными свободными членами.
        У каждого из этих трехчленов есть по два корня.
        У каждого трехчлена $f_i(x)$ выбрали один корень и обозначили его через $x_i$.
        Какие значения может принимать выражение $f_2(x_1) + f_3(x_2) + \ldots + f_{100}(x_{99}) + f_1(x_{100})$?

        \item Многочлен $P(x)$ дает остаток 5 при делении на $(x-2)$ и остаток 7 при делении на $(x-3)$.
        Какой остаток $P(x)$ дает при делении на $x^2 - 5x + 6$?

        \item Пусть \[P(x) = (2x^2 - 2x + 1)^{17} (3x^2 - 3x + 1)^{17}.\] Найдите

        а) сумму коэффициентов многочлена $P(x)$;

        б) знакопеременную сумму коэффициентов многочлена $P(x)$;

        в) сумму коэффициентов при четных степенях многочлена $P(x)$;

        г) сумму коэффициентов при нечетных степенях многочлена $P(x)$.

        \item Петя сложил $100$ последовательных степеней двойки, начиная с некоторой, а Вася сложил некоторое количество последовательных натуральных чисел, начиная с $1$.
        Могли ли они получить один и тот же результат?

        \item Числа $a, b, c, d$ таковы, что $a + b = c + d$ и $a^2 + b^2 = c^2 + d^2$.
        Докажите, что $a^3 + b^3 = c^3 + d^3$.

        \item Про различные числа $x$, $y$, $z$ известно, что выполняются равенства $x^3 - 3x = y^3 - 3y = z^3 - 3z$.
        Чему может равняться значение выражения $x^2 + y^2 + z^2$?

        \item Даны различные действительные числа $a$, $b$, $c$.
        Докажите, что хотя бы два из уравнений $(x - a)(x - b) = x - c$, $(x - b)(x - c) = x - a$, $(x - c)(x - a) = x - b$ имеют решение.

        \item Пусть $P(x)$~--- произвольный многочлен с целыми коэффициентами, причём известно, что многочлены $P(x)$ и $P(P(P(x)))$ имеют общий вещественный корень.
        Докажите, что эти многочлены имеют общий целый корень.

        \item Докажите, что если $P(x^8)$ делится на $x-1$, то $P(x^8)$ делится на $x^4+1$.

    \end{enumerate_boxed}
\end{document}