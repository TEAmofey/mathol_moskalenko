\documentclass{article}
\usepackage[12pt]{extsizes}
\usepackage[T2A]{fontenc}
\usepackage[utf8]{inputenc}
\usepackage[english, russian]{babel}

\usepackage{amssymb}
\usepackage{amsfonts}
\usepackage{amsmath}
\usepackage{enumitem}
\usepackage{graphics}
\usepackage{graphicx}

\usepackage{lipsum}

\newtheorem{theorem}{Теорема}
\newtheorem{task}{Задача}
\newtheorem{lemma}{Лемма}
\newtheorem{definition}{Определение}
\newtheorem{example}{Пример}
\newtheorem{statement}{Утверждение}
\newtheorem{corollary}{Следствие}


\usepackage{geometry} % Меняем поля страницы
\geometry{left=1cm}% левое поле
\geometry{right=1cm}% правое поле
\geometry{top=1.5cm}% верхнее поле
\geometry{bottom=1cm}% нижнее поле


\usepackage{fancyhdr} % Headers and footers
\pagestyle{fancy} % All pages have headers and footers
\fancyhead{} % Blank out the default header
\fancyfoot{} % Blank out the default footer
\fancyhead[L]{ЦРОД \textbullet Математика}
\fancyhead[C]{\textit{Алгебра}}
\fancyhead[R]{ЛФМШ 2022}% Custom header text


%----------------------------------------------------------------------------------------

%\begin{document}\normalsize
\begin{document}\large
	
	
	\begin{center}
		\textbf{Алгебра}
	\end{center}

\begin{enumerate}[label*=\protect\fbox{\arabic{enumi}}]
	
\item Докажите, что если $a^2 + b^2 + c^2 = ab + bc + ca$, то $a = b = c$.

\item  Даны квадратные трёхчлены $f_1(x) = x^2 + 2a_1x + b_1$, $f_2(x) = x^2 + 2a_2x + b_2$, $f_3(x) = x^2 + 2a_3x + b_3$, причем $a_1a_2a_3 = b_1b_2b_3 > 1$. Докажите, что хотя бы один из этих трёхчленов имеет два корня.

\item Даны квадратные трехчлены $f_1(x), f_2(x), \ldots, f_{100}(x)$ с одинаковыми коэффициентами при $x^2$, одинаковыми коэффициентами при $x$, но различными свободными членами. У каждого из этих трехчленов есть по два корня. У каждого трехчлена $f_i(x)$ выбрали один корень и обозначили его через $x_i$. Какие значения может принимать выражение $f_2(x_1) + f_3(x_2) + \ldots + f_{100}(x_{99}) + f_1(x_{100})$?

\item Многочлен $P(x)$ дает остаток 5 при делении на $(x-2)$ и остаток 7 при делении на $(x-3)$. Какой остаток $P(x)$ дает при делении на $x^2 - 5x + 6$?

\item Пусть $$P(x) = (2x^2 - 2x + 1)^{17} (3x^2 - 3x + 1)^{17}.$$ Найдите 

а) сумму коэффициентов многочлена $P(x)$;

б) знакопеременную сумму коэффициентов многочлена $P(x)$;

в) сумму коэффициентов при четных степенях многочлена $P(x)$;

г) сумму коэффициентов при нечетных степенях многочлена $P(x)$.

\item Петя сложил $100$ последовательных степеней двойки, начиная с некоторой, а Вася сложил некоторое количество последовательных натуральных чисел, начиная с $1$. Могли ли они получить один и тот же результат?

\item Числа $a, b, c, d$ таковы, что $a + b = c + d$ и $a^2 + b^2 = c^2 + d^2$. Докажите, что $a^3 + b^3 = c^3 + d^3$.

\item Про различные числа $x$, $y$, $z$ известно, что выполняются равенства $x^3 - 3x = y^3 - 3y = z^3 - 3z$. Чему может равняться значение выражения $x^2 + y^2 + z^2$?

\item Даны различные действительные числа $a$, $b$, $c$. Докажите, что хотя бы два из уравнений $(x - a)(x - b) = x - c$, $(x - b)(x - c) = x - a$, $(x - c)(x - a) = x - b$ имеют решение.

\item Пусть $P(x)$~--- произвольный многочлен с целыми коэффициентами, причём известно, что многочлены $P(x)$ и $P(P(P(x)))$ имеют общий вещественный корень. Докажите, что эти многочлены имеют общий целый корень.

\item Докажите, что если $P(x^8)$ делится на $x-1$, то $P(x^8)$ делится на $x^4+1$.



\end{enumerate}
\end{document}