\documentclass{article}
\usepackage[12pt]{extsizes}
\usepackage[T2A]{fontenc}
\usepackage[utf8]{inputenc}
\usepackage[english, russian]{babel}

\usepackage{amssymb}
\usepackage{amsfonts}
\usepackage{amsmath}
\usepackage{enumitem}
\usepackage{graphics}
\usepackage{graphicx}

\usepackage{lipsum}

\newtheorem{theorem}{Теорема}
\newtheorem{task}{Задача}
\newtheorem{lemma}{Лемма}
\newtheorem{definition}{Определение}
\newtheorem{example}{Пример}
\newtheorem{statement}{Утверждение}
\newtheorem{corollary}{Следствие}


\usepackage{geometry} % Меняем поля страницы
\geometry{left=1cm}% левое поле
\geometry{right=1cm}% правое поле
\geometry{top=1.5cm}% верхнее поле
\geometry{bottom=1cm}% нижнее поле


\usepackage{fancyhdr} % Headers and footers
\pagestyle{fancy} % All pages have headers and footers
\fancyhead{} % Blank out the default header
\fancyfoot{} % Blank out the default footer
\fancyhead[L]{Математика}
\fancyhead[C]{\textit{Разнобой}}
\fancyhead[R]{16 октября 2023}% Custom header text


%----------------------------------------------------------------------------------------

%\begin{document}\normalsize
\begin{document}\large
	
\begin{center}
	\textbf{Разнобой}
\end{center}


\begin{enumerate}[label*=\protect\fbox{\arabic{enumi}}]
	
\item Сколько треугольников изображено на картинке ниже?

\begin{figure}[h]
	\centering
	\includegraphics[width=0.3\linewidth]{img.png}
\end{figure}

\item Можно ли на шести книжных полках длиной по 1 м каждая расставить 150 книг, из которых a) 51; b) 50; c) 49 книг имеют толщину 6 см, а остальные – 3 см?

\item В ковре размером $4 \times 4$ метра моль проела 15 дырок. Всегда ли можно вырезать коврик размером $1 \times 1$, не содержащий внутри дырок? (Дырки считаются точечными).

\item Обязательно ли среди двадцати пяти монет достоинством 1, 2, 5 и 10 рублей найдётся семь монет одинакового достоинства?

\item Можно ли из семи прямоугольников $1 \times1, 1 \times2, 1 \times3, ..., 1 \times6, 1 \times7$ сложить какой-нибудь прямоугольник, обе стороны которого больше 1?

\item От шахматной доски $8 \times 8$ отрезали a) угловую клетку (например, a1) b) две соседние угловые клетки (a1 и a8) c) две противоположные угловые клетки (a1 и h8). Можно ли оставшуюся часть разрезать на доминошки

\item В кружке художественного свиста у каждого ровно один друг и ровно один враг. Докажите, что в кружке четное число людей.


\end{enumerate}


\end{document}