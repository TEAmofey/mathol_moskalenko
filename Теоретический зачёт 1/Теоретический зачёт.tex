\documentclass{article}
\usepackage[12pt]{extsizes}
\usepackage[T2A]{fontenc}
\usepackage[utf8]{inputenc}
\usepackage[english, russian]{babel}

\usepackage{amssymb}
\usepackage{amsfonts}
\usepackage{amsmath}
\usepackage{enumitem}
\usepackage{graphics}

\usepackage{lipsum}



\usepackage{geometry} % Меняем поля страницы
\geometry{left=1cm}% левое поле
\geometry{right=1cm}% правое поле
\geometry{top=1.5cm}% верхнее поле
\geometry{bottom=1cm}% нижнее поле


\usepackage{fancyhdr} % Headers and footers
\pagestyle{fancy} % All pages have headers and footers
\fancyhead{} % Blank out the default header
\fancyfoot{} % Blank out the default footer
\fancyhead[L]{Математика}
\fancyhead[C]{\textit{Зачёт}}
\fancyhead[R]{6 октября 2022}% Custom header text


%----------------------------------------------------------------------------------------

%\begin{document}\normalsize
\begin{document}\large

\begin{center}
	\textbf{Теоретический зачёт}
\end{center}

\begin{enumerate}[label*=\protect\fbox{\arabic{enumi}}]

\item Теорема Виета для квадратного трёхчлена (формулировка + доказательство)

\item Теорема Виета для кубического многочлена (формулировка + доказательство)

\item Сравнение по модулю. Определение. Формулировка и доказательство всех свойств.

\item Обратный остаток. Какие остатки могут быть обратными по модулю $m$.

\item Всё про графы (вершины, рёбра, связность, компонента связности, цикл, мост, двудольный граф)

\item Параллелограмм (определение, признаки, свойства)

\item $a^n - b^n$

\item $(a-b)^n$

\item Алгоритм Евклида (определение, доказательство корректности)

\item Линейное представление НОДа (формулировка, доказательство)

\item Неравенство о средних для $n$ переменных (формулировка, доказательство)

\item Китайская теорема об остатках (формулировка, 2 доказательства)

\item Малая теорема Ферма (формулировка, 2 доказательства)

\item Функция Эйлера (определение, формула для любого $n$)

\item Теорема Эйлера (формулировка, доказательство)

\end{enumerate}

\end{document}