\documentclass{article}
\usepackage[12pt]{extsizes}
\usepackage[T2A]{fontenc}
\usepackage[utf8]{inputenc}
\usepackage[english, russian]{babel}

\usepackage{amssymb}
\usepackage{amsfonts}
\usepackage{amsmath}
\usepackage{enumitem}
\usepackage{graphics}
\usepackage{graphicx}

\usepackage{lipsum}

\newtheorem{theorem}{Теорема}
\newtheorem{task}{Задача}
\newtheorem{lemma}{Лемма}
\newtheorem{definition}{Определение}
\newtheorem{example}{Пример}
\newtheorem{statement}{Утверждение}
\newtheorem{corollary}{Следствие}


\usepackage{geometry} % Меняем поля страницы
%\geometry{left=1cm}% левое поле
%\geometry{right=1cm}% правое поле
\geometry{top=3cm}% верхнее поле
%\geometry{bottom=1cm}% нижнее поле


\usepackage{fancyhdr} % Headers and footers
\pagestyle{fancy} % All pages have headers and footers
\fancyhead{} % Blank out the default header
\fancyfoot{} % Blank out the default footer
\fancyhead[L]{\textit{\textbf{Региональный этап ВСоШ по Математике}}}
\fancyhead[C]{}
\fancyhead[R]{1 января 2024}% Custom header text


%----------------------------------------------------------------------------------------

%\begin{document}\normalsize
\begin{document}\large
	
\begin{center}
	\LARGE\textbf{9 класс}
\end{center}
\begin{center}
	\large\textbf{Второй день}
\end{center}


\begin{enumerate}[label*=9.{\arabic{enumi}}]
\setcounter{enumi}{5}

%22.9.6
\item Последовательность чисел $a_1, a_2, \dotsc , a_{2022}$ такова, что $a_n - a_k \geqslant n^3 - k^3$ для любых $n$ и $k$ таких, что $1 \leqslant n \leqslant 2022$ и $1 \leqslant k \leqslant 2022$. При этом $a_{1011} = 0$. Какие значения может принимать $a_{2022}$? 

%22.9.7
\item  Петя разбил клетчатый квадрат $100 \times 100$ некоторым образом на домино — клетчатые прямоугольники $1 \times 2$, и в каждом домино соединил центры двух его клеток синим отрезком. Вася хочет разбить этот же квадрат на домино вторым способом, и в каждом своём домино соединить две клетки красным отрезком. Вася хочет добиться того, чтобы из каждой клетки можно было пройти в любую другую, идя по синим и красным отрезкам. Обязательно ли у него будет возможность это сделать?

%22.9.8
\item  В трапеции $ABCD$ диагональ $BD$ равна основанию $AD$. Диагонали $AC$ и $BD$ пересекаются в точке $E$. Точка $F$ на отрезке $AD$ выбрана так, что $EF \parallel CD$. Докажите, что $BE = DF$.

%22.9.9
\item На плоскости отмечены $N$ точек. Любые три из них образуют треугольник, величины углов которого в градусах выражаются натуральными числами. При каком наибольшем $N$ это возможно?

%22.9.10
\item Докажите, что существует натуральное число $b$ такое, что при любом натуральном $n > b$ сумма цифр числа $n!$ не меньше $10^{100}$.

\end{enumerate}
\end{document}