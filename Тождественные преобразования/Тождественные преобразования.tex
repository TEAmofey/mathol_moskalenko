\documentclass{article}
\usepackage[12pt]{extsizes}
\usepackage[T2A]{fontenc}
\usepackage[utf8]{inputenc}
\usepackage[english, russian]{babel}

\usepackage{amssymb}
\usepackage{amsfonts}
\usepackage{amsmath}
\usepackage{enumitem}
\usepackage{graphics}
\usepackage{graphicx}

\usepackage{lipsum}

\newtheorem{theorem}{Теорема}
\newtheorem{task}{Задача}
\newtheorem{lemma}{Лемма}
\newtheorem{definition}{Определение}
\newtheorem{example}{Пример}
\newtheorem{statement}{Утверждение}
\newtheorem{corollary}{Следствие}


\usepackage{geometry} % Меняем поля страницы
\geometry{left=1cm}% левое поле
\geometry{right=1cm}% правое поле
\geometry{top=1.5cm}% верхнее поле
\geometry{bottom=1cm}% нижнее поле


\usepackage{fancyhdr} % Headers and footers
\pagestyle{fancy} % All pages have headers and footers
\fancyhead{} % Blank out the default header
\fancyfoot{} % Blank out the default footer
\fancyhead[L]{Математика}
\fancyhead[C]{\textit{Алгебра}}
\fancyhead[R]{Август 2022}% Custom header text


%----------------------------------------------------------------------------------------

%\begin{document}\normalsize
\begin{document}\large
	
	
	\begin{center}
		\textbf{Тождественные преобразования}
	\end{center}

\begin{enumerate}[label*=\protect\fbox{\arabic{enumi}}]

\item Раскройте скобки в выражении $(a + b)^4$.

\item Раскройте скобки в выражении $(a - b)^5$.

\item Раскройте скобки в выражении $(a - b)^n$.

\item Разложите на скобки выражение $a^4- b^4$.

\item Разложите на скобки выражение $a^n- b^n$.

\item Раскройте скобки в выражении $(a + b + c)^3$.

\item Раскройте скобки в выражении $(x-y)(y-z)(z-x)$.

\item Докажите, что если $a^2 + b^2 + c^2 = ab + bc + ca$, то $a = b = c$.

\item Числа $a, b, c, d$ таковы, что $a + b = c + d$ и $a^2 + b^2 = c^2 + d^2$. Докажите, что $a^3 + b^3 = c^3 + d^3$.

\item Действительные числа $x$ и $y$ таковы, что $x^3 + y^3 + 3xy = 1$. Докажите, что или $x + y = 1$, или $x = y = -1$.

\item Целые числа $x$, $y$, $z$ таковы, что $$(x-y)^2 + (y-z)^2 + (z-x)^2 = xyz$$
Докажите, что $x^3 + y^3 + z^3$ делится на $x + y + z + 6$.

\item Пусть $a, b, c$ - целые числа. Докажите, что если $a + b \sqrt[3]{2} + c \sqrt[3]{4} = 0$, то $a = b = c = 0$.

\item Для каждого натурального $n \ge 2$ вычислите сумму $$ \dfrac{1}{1 \cdot 2} + \dfrac{1}{2 \cdot 3} + \dotso +  \dfrac{1}{(n-1) \cdot n}$$.

\item Докажите, что если действительные числа $a$, $b$, $c$ удовлетворяют условию $$\dfrac{1}{a} + \dfrac{1}{b} + \dfrac{1}{c} = \dfrac{1}{a + b + c},$$ то сумма каких-то двух из них равна 0.

\item Для каждого натурального $n \ge 2$ вычислите сумму $$\dfrac{1}{1} + \dfrac{1}{2} + \dotso +  \dfrac{1}{n} + \dfrac{1}{1 \cdot 2} + \dfrac{1}{1 \cdot 3} + \dotso +  \dfrac{1}{(n-1) \cdot n} + \dotso + \dfrac{1}{1 \cdot 2 \cdot \dotso \cdot n}$$.

\item Пусть $a$, $b$, $c$ — попарно различные числа. Докажите, что выражение $$a^2 (c-b)+b^2 (a-c)+c^2 (b-a)$$
не равно нулю.

\end{enumerate}
\end{document}