\documentclass{article}

\usepackage[12pt]{extsizes}
\usepackage[T2A]{fontenc}
\usepackage[utf8]{inputenc}
\usepackage[english, russian]{babel}

\usepackage{mathrsfs}
\usepackage[dvipsnames]{xcolor}

\usepackage{amsmath}
\usepackage{amssymb}
\usepackage{amsthm}
\usepackage{indentfirst}
\usepackage{amsfonts}
\usepackage{enumitem}
\usepackage{graphics}
\usepackage{tikz}
\usepackage{tabu}
\usepackage{diagbox}
\usepackage{hyperref}
\usepackage{mathtools}
\usepackage{ucs}
\usepackage{lipsum}
\usepackage{geometry} % Меняем поля страницы
\usepackage{fancyhdr} % Headers and footers
\usepackage[framemethod=TikZ]{mdframed}

\newcommand{\definebox}[3]{%
    \newcounter{#1}
    \newenvironment{#1}[1][]{%
        \stepcounter{#1}%
        \mdfsetup{%
            frametitle={%
            \tikz[baseline=(current bounding box.east),outer sep=0pt]
            \node[anchor=east,rectangle,fill=white]
            {\strut #2~\csname the#1\endcsname\ifstrempty{##1}{}{##1}};}}%
        \mdfsetup{innertopmargin=1pt,linecolor=#3,%
            linewidth=3pt,topline=true,
            frametitleaboveskip=\dimexpr-\ht\strutbox\relax,}%
        \begin{mdframed}[]
            \relax%
            }{
        \end{mdframed}}%
}

\definebox{theorem_boxed}{Теорема}{ForestGreen!24}
\definebox{definition_boxed}{Определение}{blue!24}
\definebox{task_boxed}{Задача}{orange!24}
\definebox{paradox_boxed}{Парадокс}{red!24}

\theoremstyle{plain}
\newtheorem{theorem}{Теорема}
\newtheorem{task}{Задача}
\newtheorem{lemma}{Лемма}
\newtheorem{statement}{Утверждение}
\newtheorem{corollary}{Следствие}

\theoremstyle{remark}
\newtheorem{remark}{Замечание}
\newtheorem{example}{Пример}
\newcommand{\range}{\mathrm{range}}
\newcommand{\dom}{\mathrm{dom}}
\newcommand{\N}{\mathbb{N}}
\newcommand{\R}{\mathbb{R}}
\newcommand{\E}{\mathbb{E}}
\newcommand{\D}{\mathbb{D}}
\newcommand{\M}{\mathcal{M}}
\newcommand{\Prime}{\mathbb{P}}
\newcommand{\A}{\mathbb{A}}
\newcommand{\Q}{\mathbb{Q}}
\newcommand{\Z}{\mathbb{Z}}
\newcommand{\F}{\mathbb{F}}
\newcommand{\CC}{\mathbb{C}}

\DeclarePairedDelimiter\abs{\lvert}{\rvert}
\DeclarePairedDelimiter\floor{\lfloor}{\rfloor}
\DeclarePairedDelimiter\ceil{\lceil}{\rceil}
\DeclarePairedDelimiter\lr{(}{)}
\DeclarePairedDelimiter\set{\{}{\}}
\DeclarePairedDelimiter\norm{\|}{\|}

\renewcommand{\labelenumi}{(\alph{enumi})}

\newcommand{\smallindent}{
    \geometry{left=1cm}% левое поле
    \geometry{right=1cm}% правое поле
    \geometry{top=1.5cm}% верхнее поле
    \geometry{bottom=1cm}% нижнее поле
}

\newcommand{\header}[3]{
    \pagestyle{fancy} % All pages have headers and footers
    \fancyhead{} % Blank out the default header
    \fancyfoot{} % Blank out the default footer
    \fancyhead[L]{#1}
    \fancyhead[C]{#2}
    \fancyhead[R]{#3}
}

\newcommand{\dividedinto}{
    \,\,\,\vdots\,\,\,
}

\newcommand{\littletaller}{\mathchoice{\vphantom{\big|}}{}{}{}}

\newcommand\restr[2]{{
    \left.\kern-\nulldelimiterspace % automatically resize the bar with \right
    #1 % the function
    \littletaller % pretend it's a little taller at normal size
    \right|_{#2} % this is the delimiter
}}

\DeclareGraphicsExtensions{.pdf,.png,.jpg}

\newenvironment{enumerate_boxed}[1][enumi]{\begin{enumerate}[label*=\protect\fbox{\arabic{#1}}]}{\end{enumerate}}



\smallindent

\header{Математика}{\textit{Алгебра}}{21 августа 2022}

%----------------------------------------------------------------------------------------

\begin{document}
    \large

    \begin{center}
        \textbf{Тождественные преобразования}
    \end{center}

    \begin{enumerate_boxed}

        \item Раскройте скобки в выражении $(a + b)^4$.

        \item Раскройте скобки в выражении $(a - b)^5$.

        \item Раскройте скобки в выражении $(a - b)^n$.

        \item Разложите на скобки выражение $a^4- b^4$.

        \item Разложите на скобки выражение $a^n- b^n$.

        \item Раскройте скобки в выражении $(a + b + c)^3$.

        \item Раскройте скобки в выражении $(x-y)(y-z)(z-x)$.

        \item Докажите, что если $a^2 + b^2 + c^2 = ab + bc + ca$, то $a = b = c$.

        \item Числа $a, b, c, d$ таковы, что $a + b = c + d$ и $a^2 + b^2 = c^2 + d^2$.
        Докажите, что $a^3 + b^3 = c^3 + d^3$.

        \item Действительные числа $x$ и $y$ таковы, что $x^3 + y^3 + 3xy = 1$.
        Докажите, что или $x + y = 1$, или $x = y = -1$.

        \item Целые числа $x$, $y$, $z$ таковы, что \[(x-y)^2 + (y-z)^2 + (z-x)^2 = xyz\]
        Докажите, что $x^3 + y^3 + z^3$ делится на $x + y + z + 6$.

        \item Пусть $a, b, c$ - целые числа.
        Докажите, что если $a + b \sqrt[3]{2} + c \sqrt[3]{4} = 0$, то $a = b = c = 0$.

        \item Для каждого натурального $n \ge 2$ вычислите сумму
        \[\dfrac{1}{1 \cdot 2} + \dfrac{1}{2 \cdot 3} + \dotso +  \dfrac{1}{(n-1) \cdot n}\].

        \item Докажите, что если действительные числа $a$, $b$, $c$ удовлетворяют условию
        \[\dfrac{1}{a} + \dfrac{1}{b} + \dfrac{1}{c} = \dfrac{1}{a + b + c},\] то сумма каких-то двух из них равна 0.

        \item Для каждого натурального $n \ge 2$ вычислите сумму
        \[\dfrac{1}{1} + \dfrac{1}{2} + \dotso +  \dfrac{1}{n} + \dfrac{1}{1 \cdot 2} + \dfrac{1}{1 \cdot 3} + \dotso +  \dfrac{1}{(n-1) \cdot n} + \dotso + \dfrac{1}{1 \cdot 2 \cdot \dotso \cdot n}\].

        \item Пусть $a$, $b$, $c$ — попарно различные числа.
        Докажите, что выражение
        \[a^2 (c-b)+b^2 (a-c)+c^2 (b-a)\]
        не равно нулю.

    \end{enumerate_boxed}
\end{document}