\documentclass{article}
\usepackage[12pt]{extsizes}
\usepackage[T2A]{fontenc}
\usepackage[utf8]{inputenc}
\usepackage[english, russian]{babel}

\usepackage{amssymb}
\usepackage{amsfonts}
\usepackage{amsmath}
\usepackage{enumitem}
\usepackage{graphics}
\usepackage{graphicx}

\usepackage{lipsum}

\newtheorem{theorem}{Теорема}
\newtheorem{task}{Задача}
\newtheorem{lemma}{Лемма}
\newtheorem{definition}{Определение}
\newtheorem{example}{Пример}
\newtheorem{statement}{Утверждение}
\newtheorem{corollary}{Следствие}


\usepackage{geometry} % Меняем поля страницы
\geometry{left=1cm}% левое поле
\geometry{right=1cm}% правое поле
\geometry{top=1.5cm}% верхнее поле
\geometry{bottom=1cm}% нижнее поле


\usepackage{fancyhdr} % Headers and footers
\pagestyle{fancy} % All pages have headers and footers
\fancyhead{} % Blank out the default header
\fancyfoot{} % Blank out the default footer
\fancyhead[L]{Математика}
\fancyhead[C]{\textit{Разнобой}}
\fancyhead[R]{29 января 2024}% Custom header text


%----------------------------------------------------------------------------------------

%\begin{document}\normalsize
\begin{document}\large
	
\begin{center}
	\textbf{Разнобой}
\end{center}


\begin{enumerate}[label*=\protect\fbox{\arabic{enumi}}]

\item Про каждое из следующих чисел скажите, является ли оно простым: 0, 1, 12, 13, 17, 97, 899, 43 825 539 123, 621 212 229, 100020001. Обоснуйте каждый свой ответ.

\item В ряд выписаны числа от 1 до 25. Можно ли расставить между
ними знаки <<+>> и <<-->> так, чтобы значение полученного выражения
было равно нулю?

\item Имеется много одинаковых квадратов. В вершинах каждого из них в произвольном порядке написаны числа 1, 2, 3 и 4. Квадраты сложили в стопку и написали сумму чисел, попавших в каждый из четырех углов стопки. Может ли оказаться так, что в каждом углу стопки сумма равна $2024$?


\item Может ли число, десятичная запись которого состоит из 2024 нулей, 2024 единиц и 2024 двоек, быть точным квадратом?


\item Найдите наибольшее четырёхзначное число, все цифры которого различны и которое делится на 2, 5, 9 и 11.

\end{enumerate}


\end{document}