\documentclass{article}

\usepackage[12pt]{extsizes}
\usepackage[T2A]{fontenc}
\usepackage[utf8]{inputenc}
\usepackage[english, russian]{babel}

\usepackage{mathrsfs}
\usepackage[dvipsnames]{xcolor}

\usepackage{amsmath}
\usepackage{amssymb}
\usepackage{amsthm}
\usepackage{indentfirst}
\usepackage{amsfonts}
\usepackage{enumitem}
\usepackage{graphics}
\usepackage{tikz}
\usepackage{tabu}
\usepackage{diagbox}
\usepackage{hyperref}
\usepackage{mathtools}
\usepackage{ucs}
\usepackage{lipsum}
\usepackage{geometry} % Меняем поля страницы
\usepackage{fancyhdr} % Headers and footers
\newcommand{\range}{\mathrm{range}}
\newcommand{\dom}{\mathrm{dom}}
\newcommand{\N}{\mathbb{N}}
\newcommand{\R}{\mathbb{R}}
\newcommand{\E}{\mathbb{E}}
\newcommand{\D}{\mathbb{D}}
\newcommand{\M}{\mathcal{M}}
\newcommand{\Prime}{\mathbb{P}}
\newcommand{\A}{\mathbb{A}}
\newcommand{\Q}{\mathbb{Q}}
\newcommand{\Z}{\mathbb{Z}}
\newcommand{\F}{\mathbb{F}}
\newcommand{\CC}{\mathbb{C}}

\DeclarePairedDelimiter\abs{\lvert}{\rvert}
\DeclarePairedDelimiter\floor{\lfloor}{\rfloor}
\DeclarePairedDelimiter\ceil{\lceil}{\rceil}
\DeclarePairedDelimiter\lr{(}{)}
\DeclarePairedDelimiter\set{\{}{\}}
\DeclarePairedDelimiter\norm{\|}{\|}

\renewcommand{\labelenumi}{(\alph{enumi})}

\newcommand{\smallindent}{
    \geometry{left=1cm}% левое поле
    \geometry{right=1cm}% правое поле
    \geometry{top=1.5cm}% верхнее поле
    \geometry{bottom=1cm}% нижнее поле
}

\newcommand{\header}[3]{
    \pagestyle{fancy} % All pages have headers and footers
    \fancyhead{} % Blank out the default header
    \fancyfoot{} % Blank out the default footer
    \fancyhead[L]{#1}
    \fancyhead[C]{#2}
    \fancyhead[R]{#3}
}

\newcommand{\dividedinto}{
    \,\,\,\vdots\,\,\,
}

\newcommand{\littletaller}{\mathchoice{\vphantom{\big|}}{}{}{}}

\newcommand\restr[2]{{
    \left.\kern-\nulldelimiterspace % automatically resize the bar with \right
    #1 % the function
    \littletaller % pretend it's a little taller at normal size
    \right|_{#2} % this is the delimiter
}}

\DeclareGraphicsExtensions{.pdf,.png,.jpg}

\newenvironment{enumerate_boxed}[1][enumi]{\begin{enumerate}[label*=\protect\fbox{\arabic{#1}}]}{\end{enumerate}}



\smallindent

\header{ЦРОД $\bullet$ Математика}{\textit{Методы}}{Стратегия 2021}

%----------------------------------------------------------------------------------------

\begin{document}
    \large

    \begin{center}
        \textbf{Двойной подсчёт}
    \end{center}

    \begin{enumerate_boxed}

        \item Можно ли расставить числа в таблице $6 \times 9$ так, чтобы в каждом столбце была сумма по $10$, а в каждой строке — по $20$?

        \item В прямоугольной таблице $8$ столбцов, сумма в каждом столбце — по $10$, а в каждой строке — по $20$.
        Сколько в таблице строк?

        \item В конференции участвовали $19$ ученых.
        После конференции каждый из них отправил $2$ или $4$ письма участникам этой конференции.
        Могло ли получиться так, что каждый участник получил по $3$ письма, если письма на почте не теряют?

        \item Дано $25$ чисел.
        Какие бы три из них мы ни выбрали, среди оставшихся найдётся такое четвёртое, что сумма этих четырёх чисел будет положительна.
        Верно ли, что сумма всех чисел положительна?

        \item Несколько восьмиклассников и девятиклассников обменялись рукопожатиями.
        При этом каждый восьмиклассник пожал руку девяти девятиклассникам, а каждый девятиклассник — восьми восьмиклассникам.

        Кого среди них было больше — восьмиклассников или девятиклассников?

        \item Игорь закрасил в квадрате $6 \times 6$ несколько клеток.
        После этого оказалось, что во всех квадратиках $2 \times 2$ одинаковое число закрашенных клеток и во всех полосках $1 \times 3$ одинаковое число закрашенных клеток.
        Докажите, что старательный Игорь закрасил все клетки.

        \item Можно ли занумеровать рёбра куба числами $1, 2, \dotsc, 11, 12$ так, чтобы для каждой вершины сумма номеров трёх выходящих из неё рёбер была одной и той же.

        \item Футбольный мяч сшит из $32$ лоскутов: белых шестиугольников и черных пятиугольников.
        Каждый черный лоскут граничит с пятью белыми, а каждый белый — с тремя черными и тремя белыми.
        Сколько лоскутов белого цвета?

        \item В городе от каждой площади отходит ровно $5$ улиц.
        Докажите, что число площадей четно, а число улиц делится на $5$.
        (Улицы соединяют площади.)

        \item Взяли несколько одинаковых равносторонних треугольников.
        Вершины каждого из них пометили цифрами $1, 2$ и $3$.
        Затем их сложили в стопку.
        Могло ли оказаться, что сумма чисел, находящихся в каждом углу, равна $55$?

        \item Дано $2021$ число.
        Известно, что сумма любых четырёх чисел положительна.
        Верно ли, что сумма всех чисел положительна?

        \item Можно ли в таблицу $5 \times 5$ записать числа $1, 2, 3,
        \dots, 25$ так, чтобы в каждой строке сумма нескольких
        записанных чисел была равна сумме остальных чисел
        этой строки?

        \item По окружности отметили $40$ красных, $30$ синих и $20$ зеленых точек.
        На каждой дуге между соседними красной и синей точками поставили цифру $1$, на каждой дуге между соседними красной и зеленой – цифру $2$, а на каждой дуге между соседними синей и зеленой – цифру $3$.
        (На дугах между одноцветными точками поставили $0$.) Найдите максимальную возможную сумму поставленных чисел.

        \item Дан набор, состоящий из таких $2021$ числа, что если каждое число в наборе заменить на сумму остальных, то получится тот же набор.
        Докажите, что произведение чисел в наборе равно $0$.

        \item По кругу расставлены красные и синие числа.
        Каждое красное число равно сумме соседних чисел, а каждое синее~--- полусумме соседних чисел.
        Докажите, что сумма красных чисел равна нулю.

    \end{enumerate_boxed}
\end{document}