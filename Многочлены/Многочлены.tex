\documentclass{article}

\usepackage[12pt]{extsizes}
\usepackage[T2A]{fontenc}
\usepackage[utf8]{inputenc}
\usepackage[english, russian]{babel}

\usepackage{mathrsfs}
\usepackage[dvipsnames]{xcolor}

\usepackage{amsmath}
\usepackage{amssymb}
\usepackage{amsthm}
\usepackage{indentfirst}
\usepackage{amsfonts}
\usepackage{enumitem}
\usepackage{graphics}
\usepackage{tikz}
\usepackage{tabu}
\usepackage{diagbox}
\usepackage{hyperref}
\usepackage{mathtools}
\usepackage{ucs}
\usepackage{lipsum}
\usepackage{geometry} % Меняем поля страницы
\usepackage{fancyhdr} % Headers and footers
\newcommand{\range}{\mathrm{range}}
\newcommand{\dom}{\mathrm{dom}}
\newcommand{\N}{\mathbb{N}}
\newcommand{\R}{\mathbb{R}}
\newcommand{\E}{\mathbb{E}}
\newcommand{\D}{\mathbb{D}}
\newcommand{\M}{\mathcal{M}}
\newcommand{\Prime}{\mathbb{P}}
\newcommand{\A}{\mathbb{A}}
\newcommand{\Q}{\mathbb{Q}}
\newcommand{\Z}{\mathbb{Z}}
\newcommand{\F}{\mathbb{F}}
\newcommand{\CC}{\mathbb{C}}

\DeclarePairedDelimiter\abs{\lvert}{\rvert}
\DeclarePairedDelimiter\floor{\lfloor}{\rfloor}
\DeclarePairedDelimiter\ceil{\lceil}{\rceil}
\DeclarePairedDelimiter\lr{(}{)}
\DeclarePairedDelimiter\set{\{}{\}}
\DeclarePairedDelimiter\norm{\|}{\|}

\renewcommand{\labelenumi}{(\alph{enumi})}

\newcommand{\smallindent}{
    \geometry{left=1cm}% левое поле
    \geometry{right=1cm}% правое поле
    \geometry{top=1.5cm}% верхнее поле
    \geometry{bottom=1cm}% нижнее поле
}

\newcommand{\header}[3]{
    \pagestyle{fancy} % All pages have headers and footers
    \fancyhead{} % Blank out the default header
    \fancyfoot{} % Blank out the default footer
    \fancyhead[L]{#1}
    \fancyhead[C]{#2}
    \fancyhead[R]{#3}
}

\newcommand{\dividedinto}{
    \,\,\,\vdots\,\,\,
}

\newcommand{\littletaller}{\mathchoice{\vphantom{\big|}}{}{}{}}

\newcommand\restr[2]{{
    \left.\kern-\nulldelimiterspace % automatically resize the bar with \right
    #1 % the function
    \littletaller % pretend it's a little taller at normal size
    \right|_{#2} % this is the delimiter
}}

\DeclareGraphicsExtensions{.pdf,.png,.jpg}

\newenvironment{enumerate_boxed}[1][enumi]{\begin{enumerate}[label*=\protect\fbox{\arabic{#1}}]}{\end{enumerate}}



\smallindent

\header{Математика}{\textit{Алгебра}}{17 октября 2023}

%----------------------------------------------------------------------------------------

\begin{document}
    \large

    \begin{center}
        \textbf{Многочлены}
    \end{center}

    \begin{enumerate_boxed}

        \item Докажите, что любой многочлен нечётной степени имеет хотя бы один корень

        \item Дан многочлен $P(x)$ с вещественными коэффициентами нечетной степени.
        Докажите, что уравнение $P(P(x)) = 0$ имеет не меньше различных вещественных корней, чем уравнение $P(x) = 0$.

        \item Последовательность многочленов $P_1(x), P_2(x), \ldots, P_n(x), \ldots$ удовлетворяет равенствам $P_1(x) = x$ и $P_{n+1}(x) = P_n(x - 1)P_n(x + 1)$.
        Найдите наибольшее натуральное $k$, для которого $P_{2021}(x)$ делится на $x^k$.

        \item В выражении $(x^4 + x^3 - 3x^2 + x + 2)^{2019}$ раскрыли скобки и привели подобные слагаемые.
        Докажите, что при некоторой степени переменной $x$ получился отрицательный коэффициент.

        \item Даны два различных приведенных кубических многочлена $F(x)$ и $G(x)$.
        Выписали все корни уравнений $F(x) = 0$, $G(x) = 0$, $F(x) = G(x)$.
        Оказалось, что выписаны 8 различных чисел.
        Докажите, что наибольшее и наименьшее из них не могут одновременно являться корнями многочлена $F(x)$.

        \item Многочлен $P(x)$ с целыми коэффициентами имеет 100 различных целых корней.
        Многочлен $Q(x)$ степени не ниже первой с целыми коэффициентами — делитель $P(x) + 2021$.
        Докажите, что степень $Q(x)$ не меньше 13.

        \item Дан непостоянный многочлен $P(x)$ с натуральными коэффициентами.
        Докажите, что найдется целое число $k$ такое, что числа $P(k), P(k+1), \ldots, P(k+2021)$ — составные.

        \item Дано натуральное число $k$.

        (а) Докажите, что найдется такое $n$ и расстановка знаков, что $1^k \pm 2^k \pm \ldots \pm n^k = 0$.

        (б) Для каждого натурального $m$ обозначим через $f(m)$ наименьшее значение выражения $|1^k \pm 2^k \pm \ldots \pm m^k|$ по всем расстановкам знаков.
        Докажите, что функция $f(m)$ периодична, начиная с некоторого места.


    \end{enumerate_boxed}
\end{document}