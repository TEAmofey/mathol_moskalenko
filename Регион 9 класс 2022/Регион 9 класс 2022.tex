\documentclass{article}

\usepackage[12pt]{extsizes}
\usepackage[T2A]{fontenc}
\usepackage[utf8]{inputenc}
\usepackage[english, russian]{babel}

\usepackage{mathrsfs}
\usepackage[dvipsnames]{xcolor}

\usepackage{amsmath}
\usepackage{amssymb}
\usepackage{amsthm}
\usepackage{indentfirst}
\usepackage{amsfonts}
\usepackage{enumitem}
\usepackage{graphics}
\usepackage{tikz}
\usepackage{tabu}
\usepackage{diagbox}
\usepackage{hyperref}
\usepackage{mathtools}
\usepackage{ucs}
\usepackage{lipsum}
\usepackage{geometry} % Меняем поля страницы
\usepackage{fancyhdr} % Headers and footers
\newcommand{\range}{\mathrm{range}}
\newcommand{\dom}{\mathrm{dom}}
\newcommand{\N}{\mathbb{N}}
\newcommand{\R}{\mathbb{R}}
\newcommand{\E}{\mathbb{E}}
\newcommand{\D}{\mathbb{D}}
\newcommand{\M}{\mathcal{M}}
\newcommand{\Prime}{\mathbb{P}}
\newcommand{\A}{\mathbb{A}}
\newcommand{\Q}{\mathbb{Q}}
\newcommand{\Z}{\mathbb{Z}}
\newcommand{\F}{\mathbb{F}}
\newcommand{\CC}{\mathbb{C}}

\DeclarePairedDelimiter\abs{\lvert}{\rvert}
\DeclarePairedDelimiter\floor{\lfloor}{\rfloor}
\DeclarePairedDelimiter\ceil{\lceil}{\rceil}
\DeclarePairedDelimiter\lr{(}{)}
\DeclarePairedDelimiter\set{\{}{\}}
\DeclarePairedDelimiter\norm{\|}{\|}

\renewcommand{\labelenumi}{(\alph{enumi})}

\newcommand{\smallindent}{
    \geometry{left=1cm}% левое поле
    \geometry{right=1cm}% правое поле
    \geometry{top=1.5cm}% верхнее поле
    \geometry{bottom=1cm}% нижнее поле
}

\newcommand{\header}[3]{
    \pagestyle{fancy} % All pages have headers and footers
    \fancyhead{} % Blank out the default header
    \fancyfoot{} % Blank out the default footer
    \fancyhead[L]{#1}
    \fancyhead[C]{#2}
    \fancyhead[R]{#3}
}

\newcommand{\dividedinto}{
    \,\,\,\vdots\,\,\,
}

\newcommand{\littletaller}{\mathchoice{\vphantom{\big|}}{}{}{}}

\newcommand\restr[2]{{
    \left.\kern-\nulldelimiterspace % automatically resize the bar with \right
    #1 % the function
    \littletaller % pretend it's a little taller at normal size
    \right|_{#2} % this is the delimiter
}}

\DeclareGraphicsExtensions{.pdf,.png,.jpg}

\newenvironment{enumerate_boxed}[1][enumi]{\begin{enumerate}[label*=\protect\fbox{\arabic{#1}}]}{\end{enumerate}}



\header{\textit{\textbf{Региональный этап ВСоШ по Математике}}}{}{31 декабря 2023--1 января 2024}

%----------------------------------------------------------------------------------------

\begin{document}
    \large

    \begin{center}
        \LARGE\textbf{9 класс}
    \end{center}
    \begin{center}
        \large\textbf{Первый день}
    \end{center}


    \begin{enumerate}[label*=9.{\arabic{enumi}}]
        \setcounter{enumi}{0}

%22.9.1
        \item Петя написал на доске десять натуральных чисел, среди которых нет двух равных.
        Известно, что из этих десяти чисел можно выбрать три числа, делящихся на 5.
        Также известно, что из написанных десяти чисел можно выбрать четыре числа, делящихся на 4.
        Может ли сумма всех написанных на доске чисел быть меньше 75?

%22.9.2
        \item На доске девять раз (друг под другом) написали некоторое натуральное число $N$.
        Петя к каждому из 9 чисел приписал слева или справа одну ненулевую цифру; при этом все приписанные цифры различны.
        Какое наибольшее количество простых чисел могло оказаться среди 9 полученных чисел?

%22.9.3
        \item  Дан квадратный трёхчлен $P(x)$, не обязательно с целыми коэффициентами.
        Известно, что при некоторых целых $a$ и $b$ разность $P(a) - P(b)$ является квадратом натурального числа.
        Докажите, что существует более миллиона таких пар целых чисел $(c, d)$, что разность $P(c) - P(d)$ также является квадратом натурального числа.

%22.9.4
        \item  В компании некоторые пары людей дружат (если $A$ дружит с $B$, то и $B$ дружит с $A$). Оказалось, что среди каждых 100 человек в компании количество пар дружащих людей нечётно.
        Найдите наибольшее возможное количество человек в такой компании.

%20.9.5
        \item Четырёхугольник $ABCD$ описан около окружности $\omega$.
        Докажите, что диаметр окружности $\omega$ не превосходит длины отрезка, соединяющего середины сторон $BC$ и $AD$.

    \end{enumerate}

    \newpage

    \begin{center}
        \LARGE\textbf{9 класс}
    \end{center}
    \begin{center}
        \large\textbf{Второй день}
    \end{center}


    \begin{enumerate}[label*=9.{\arabic{enumi}}]
        \setcounter{enumi}{5}

%22.9.6
        \item Последовательность чисел $a_1, a_2, \dotsc , a_{2022}$ такова, что $a_n - a_k \geqslant n^3 - k^3$ для любых $n$ и $k$ таких, что $1 \leqslant n \leqslant 2022$ и $1 \leqslant k \leqslant 2022$.
        При этом $a_{1011} = 0$.
        Какие значения может принимать $a_{2022}$?

%22.9.7
        \item  Петя разбил клетчатый квадрат $100 \times 100$ некоторым образом на домино — клетчатые прямоугольники $1 \times 2$, и в каждом домино соединил центры двух его клеток синим отрезком.
        Вася хочет разбить этот же квадрат на домино вторым способом, и в каждом своём домино соединить две клетки красным отрезком.
        Вася хочет добиться того, чтобы из каждой клетки можно было пройти в любую другую, идя по синим и красным отрезкам.
        Обязательно ли у него будет возможность это сделать?

%22.9.8
        \item  В трапеции $ABCD$ диагональ $BD$ равна основанию $AD$.
        Диагонали $AC$ и $BD$ пересекаются в точке $E$.
        Точка $F$ на отрезке $AD$ выбрана так, что $EF \parallel CD$.
        Докажите, что $BE = DF$.

%22.9.9
        \item На плоскости отмечены $N$ точек.
        Любые три из них образуют треугольник, величины углов которого в градусах выражаются натуральными числами.
        При каком наибольшем $N$ это возможно?

%22.9.10
        \item Докажите, что существует натуральное число $b$ такое, что при любом натуральном $n > b$ сумма цифр числа $n!$ не меньше $10^{100}$.

    \end{enumerate}
\end{document}