\documentclass{article}

\usepackage[12pt]{extsizes}
\usepackage[T2A]{fontenc}
\usepackage[utf8]{inputenc}
\usepackage[english, russian]{babel}

\usepackage{mathrsfs}
\usepackage[dvipsnames]{xcolor}

\usepackage{amsmath}
\usepackage{amssymb}
\usepackage{amsthm}
\usepackage{indentfirst}
\usepackage{amsfonts}
\usepackage{enumitem}
\usepackage{graphics}
\usepackage{tikz}
\usepackage{tabu}
\usepackage{diagbox}
\usepackage{hyperref}
\usepackage{mathtools}
\usepackage{ucs}
\usepackage{lipsum}
\usepackage{geometry} % Меняем поля страницы
\usepackage{fancyhdr} % Headers and footers
\usepackage[framemethod=TikZ]{mdframed}

\newcommand{\definebox}[3]{%
    \newcounter{#1}
    \newenvironment{#1}[1][]{%
        \stepcounter{#1}%
        \mdfsetup{%
            frametitle={%
            \tikz[baseline=(current bounding box.east),outer sep=0pt]
            \node[anchor=east,rectangle,fill=white]
            {\strut #2~\csname the#1\endcsname\ifstrempty{##1}{}{##1}};}}%
        \mdfsetup{innertopmargin=1pt,linecolor=#3,%
            linewidth=3pt,topline=true,
            frametitleaboveskip=\dimexpr-\ht\strutbox\relax,}%
        \begin{mdframed}[]
            \relax%
            }{
        \end{mdframed}}%
}

\definebox{theorem_boxed}{Теорема}{ForestGreen!24}
\definebox{definition_boxed}{Определение}{blue!24}
\definebox{task_boxed}{Задача}{orange!24}
\definebox{paradox_boxed}{Парадокс}{red!24}

\theoremstyle{plain}
\newtheorem{theorem}{Теорема}
\newtheorem{task}{Задача}
\newtheorem{lemma}{Лемма}
\newtheorem{definition}{Определение}
\newtheorem{statement}{Утверждение}
\newtheorem{corollary}{Следствие}

\theoremstyle{remark}
\newtheorem{remark}{Замечание}
\newtheorem{example}{Пример}
\newcommand{\range}{\mathrm{range}}
\newcommand{\dom}{\mathrm{dom}}
\newcommand{\N}{\mathbb{N}}
\newcommand{\R}{\mathbb{R}}
\newcommand{\E}{\mathbb{E}}
\newcommand{\D}{\mathbb{D}}
\newcommand{\M}{\mathcal{M}}
\newcommand{\Prime}{\mathbb{P}}
\newcommand{\A}{\mathbb{A}}
\newcommand{\Q}{\mathbb{Q}}
\newcommand{\Z}{\mathbb{Z}}
\newcommand{\F}{\mathbb{F}}
\newcommand{\CC}{\mathbb{C}}

\DeclarePairedDelimiter\abs{\lvert}{\rvert}
\DeclarePairedDelimiter\floor{\lfloor}{\rfloor}
\DeclarePairedDelimiter\ceil{\lceil}{\rceil}
\DeclarePairedDelimiter\lr{(}{)}
\DeclarePairedDelimiter\set{\{}{\}}
\DeclarePairedDelimiter\norm{\|}{\|}

\renewcommand{\labelenumi}{(\alph{enumi})}

\newcommand{\smallindent}{
    \geometry{left=1cm}% левое поле
    \geometry{right=1cm}% правое поле
    \geometry{top=1.5cm}% верхнее поле
    \geometry{bottom=1cm}% нижнее поле
}

\newcommand{\header}[3]{
    \pagestyle{fancy} % All pages have headers and footers
    \fancyhead{} % Blank out the default header
    \fancyfoot{} % Blank out the default footer
    \fancyhead[L]{#1}
    \fancyhead[C]{#2}
    \fancyhead[R]{#3}
}

\newcommand{\dividedinto}{
    \,\,\,\vdots\,\,\,
}

\newcommand{\littletaller}{\mathchoice{\vphantom{\big|}}{}{}{}}

\newcommand\restr[2]{{
    \left.\kern-\nulldelimiterspace % automatically resize the bar with \right
    #1 % the function
    \littletaller % pretend it's a little taller at normal size
    \right|_{#2} % this is the delimiter
}}

\DeclareGraphicsExtensions{.pdf,.png,.jpg}

\newenvironment{enumerate_boxed}[1][enumi]{\begin{enumerate}[label*=\protect\fbox{\arabic{#1}}]}{\end{enumerate}}



\smallindent

\header{ЦРОД $\bullet$ Математика}{\textit{Комбинаторика}}{ЛФМШ 2022}

%----------------------------------------------------------------------------------------

\begin{document}
    \large

    \begin{center}
        \textbf{Сочетания}
    \end{center}

    \begin{definition}
        Количество способов выбрать $k$ предметов из $n$ имеющихся называется числом сочетаний из $n$ элементов по $k$ и обозначается $C^k_n = \binom{n}{k}$ .
    \end{definition}

    \begin{enumerate_boxed}
        \item Докажите, что $C^k_n = \frac{n!}{k!(n - k)!}$

        \item У семиклассника Пети есть $7$ детективов, а у восьмиклассника Васи – $8$ книг по математике.
        Сколькими способами они могут обменять три книги одного на три книги другого?

        \item $15$ человек нужно разбить на баскетбольную, волейбольную и футбольную команды по пять человек.
        Сколькими способами это можно сделать?

        \item \textbf{(a)} Что можно выбрать большим числом способов: двух преподавателей из $15$ для работы
        в группе <<Олимпиадная математика - $4$>> или $13$ преподавателей, которые не решаются работать с группой <<Олимпиадная математика - $4$>>? \textbf{(b)} Докажите тождество алгебраически, комбинаторно и через треугольник Паскаля. $C^k_n = C^{n-k}_n$

        \item \textbf{(a)} В группе $15$ человек.
        Сколькими способами из них можно выбрать шестерых, которые пойдут на лекцию, если Даниил категорически отказывается туда идти, так как ему нужно дорешать задачи? \textbf{(b)} Для проведения матбоя нужна команда из шести человек, в которой Саша будет капитаном.
        Сколькими способами можно собрать такую команду из группы в $15$ человек? \textbf{(c)} Докажите тождество алгебраически, комбинаторно и через треугольник Паскаля $C^k_n = C^{k-1}_{n - 1} + C^k_{n- 1}$

        \item \textbf{(a)} В классе $15$ человек.
        Учитель по физкультуре Андрей Леопольдович выбирает команду в футбол на кубок ЦРОДа (из 6 человек).
        Капитаном он выберет самого высокого из них (в классе все дети разного роста).
        Но Андрей Леопольдович решил сначала выбрать капитана, а потом всю оставшуюся команду.
        Помогите ему посчитать количество возможных команд. \textbf{(b)} Докажите тождество комбинаторно и через треугольник Паскаля $C^{k + 1}_{n + 1} = C^{k}_{k} + C^{k}_{k + 1} + \dotso + C^k_{n}$

        \item \textbf{(a)} У акулы было $100$ зубов.
        Сколько различных улыбок могло у неё остаться после встречи
        с катером? \textbf{(b)} Докажите, что $C^0_n + C^1_n + \dotsc + C^n_n = 2^n$

        \item \textbf{(a)} Сколькими способами из $15$ человек можно выбрать команду из шести человек, возглавляемую капитаном? \textbf{(b)} Сколькими способами можно выбрать из $n$ человек $k$ человек в парламент, возглавляемого президентом парламента? \textbf{(c)} Докажите тождество двумя способами – комбинаторно и алгебраически: $k \cdot C^k_n = n \cdot C^{k-1}_{n-1}$

        \item Сколькими способами можно выбрать положительные числа $x_1, \dotso , x_k$ такие, что $x_1 + \dotso + x_k = n$.

        \item Сколькими способами можно выбрать неотрицательные числа $x_1, \dotso , x_k$ такие, что $x_1 + \dotso + x_k = n$.

        \item В коробке лежат $n$ синих и $n$ красных шариков (все шарики разные).
        Сформулируйте
        вопрос, позволяющий доказать, что $C^0_n \cdot C^k_n +C^1_n \cdot C^{k-1}_n+\dotso+C^i_n \cdot C^{k-i}_n +\dotso + C^k_n \cdot C^0_n =C^k_{2n}$

        \item Найдите, значение выражения $C^0_n \cdot C^k_m +C^1_n \cdot C^{k-1}_m+\dotso+C^i_n \cdot C^{k-i}_m +\dotso + C^k_n \cdot C^0_m$

        \item Найдите, значение выражения $(C^0_n)^2 +(C^1_n)^2 +\dotso+(C^i_n)^2 +\dotso + (C^n_n)^2$

        \item Найдите, значение выражения $C^0_n +C^1_n \cdot 2 +\dotso+C^i_n \cdot 2^i +\dotso + C^n_n \cdot 2^n$

        \item Для натурального числа $n$ оказалось, что каждое из чисел $C^1_n, \dotsc, C^{k - 1}_n$
        делится на $n$, а число $C^k_n$ — нет.
        Докажите, что $k$ — простое число.

    \end{enumerate_boxed}
\end{document}