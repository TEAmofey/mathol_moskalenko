\documentclass{article}
\usepackage[12pt]{extsizes}
\usepackage[T2A]{fontenc}
\usepackage[utf8]{inputenc}
\usepackage[english, russian]{babel}

\usepackage{amssymb}
\usepackage{amsfonts}
\usepackage{amsmath}
\usepackage{enumitem}
\usepackage{graphics}
\usepackage{graphicx}

\usepackage{lipsum}

\newtheorem{theorem}{Теорема}
\newtheorem{task}{Задача}
\newtheorem{lemma}{Лемма}
\newtheorem{definition}{Определение}
\newtheorem{example}{Пример}
\newtheorem{statement}{Утверждение}
\newtheorem{corollary}{Следствие}


\usepackage{geometry} % Меняем поля страницы
\geometry{left=1cm}% левое поле
\geometry{right=1cm}% правое поле
\geometry{top=1.5cm}% верхнее поле
\geometry{bottom=1cm}% нижнее поле


\usepackage{fancyhdr} % Headers and footers
\pagestyle{fancy} % All pages have headers and footers
\fancyhead{} % Blank out the default header
\fancyfoot{} % Blank out the default footer
\fancyhead[L]{ЦРОД $\bullet$ Математика}
\fancyhead[C]{\textit{Комбинаторика}}
\fancyhead[R]{1 ноября 2023}% Custom header text


%----------------------------------------------------------------------------------------

%\begin{document}\normalsize
\begin{document}\large
	
\begin{center}
	\textbf{Индукция в графах}
\end{center}


\begin{enumerate}[label*=\protect\fbox{\arabic{enumi}}]
	
\item В одной далёкой-далёкой стране некоторые (соседние) города соединены дорогами. Докажите, что президент может в каждый город назначить мэра (рыцаря или лжеца), чтобы на вопрос <<есть ли лжецы среди мэров соседних городов?>> любой мэр отвечал утвердительно.

\item 
\begin{enumerate}
	\item Докажите, что в связном графе существует цикл, проходящий по каждому ребру ровно один раз (Эйлеров
	цикл), тогда и только тогда, когда степени всех вершин данного графа чётны; 
	\item Докажите, что в связном графе
	существует путь, проходящий по каждому ребру ровно один раз (Эйлеров путь), тогда и только тогда, когда в
	данном графе есть не более двух вершин нечётной степени.
\end{enumerate}

\item Дан набор из $n$ натуральных чисел. Сумма всех чисел равна $2n-2$. Докажите, что существует
дерево, для которого набор степеней вершин совпадает с данным набором.

\item Докажите, что в планарном графе с $V$ вершинами, $E$ рёбрами и $F$ гранями $$V - E + F = 2$$

\item В королевстве $N$ городов, некоторые пары которых соединены непересекающимися дорогами с двусторонним движением (города из такой пары называются соседними). При этом известно, что из каждого города можно доехать до любого другого, но невозможно, выехав из некоторого города и двигаясь по различным дорогам, вернуться в исходный город.
Однажды Король провел такую реформу: каждый из $N$ мэров городов стал снова мэром одного из $N$ городов, но, возможно, не того города, в котором он работал до реформы. Оказалось, что каждые два мэра, работавшие в соседних городах до реформы, оказались в соседних городах и после реформы. Докажите, что либо найдётся город, в котором мэр после реформы не поменялся, либо найдётся пара соседних городов, обменявшихся мэрами.

\item В компании из $n > 2$ человек среди любых четырех есть знакомый с тремя остальными.
Докажите, что есть человек, который знает всех.

\item Дан планарный граф, степени всех вершин которого чётны. Докажите, что её грани можно раскрасить в два цвета так, чтобы любые две грани, имеющие общее ребро, были покрашены в разные цвета.

\item В связном графе $n$ вершин. В каждой из них лежит некоторое количество монет, в сумме
$kn$. За один ход разрешается переложить некоторое количество монет из одной вершины
в соседнюю. Докажите, что из любого расположения монет можно разложить монеты поровну во все вершины не более чем за $n - 1$ ходов.
 
\end{enumerate}
\end{document}