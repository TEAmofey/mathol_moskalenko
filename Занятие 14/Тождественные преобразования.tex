\documentclass{article}
\usepackage[12pt]{extsizes}
\usepackage[T2A]{fontenc}
\usepackage[utf8]{inputenc}
\usepackage[english, russian]{babel}

\usepackage{amssymb}
\usepackage{amsfonts}
\usepackage{amsmath}
\usepackage{enumitem}
\usepackage{graphics}
\usepackage{graphicx}

\usepackage{lipsum}
\DeclareGraphicsExtensions{.pdf,.png,.jpg}



\usepackage{geometry} % Меняем поля страницы
\geometry{left=1cm}% левое поле
\geometry{right=1cm}% правое поле
\geometry{top=1.5cm}% верхнее поле
\geometry{bottom=1cm}% нижнее поле


\usepackage{fancyhdr} % Headers and footers
\pagestyle{fancy} % All pages have headers and footers
\fancyhead{} % Blank out the default header
\fancyfoot{} % Blank out the default footer
\fancyhead[L]{Математика}
\fancyhead[C]{\textit{Алгебра}}
\fancyhead[R]{16 апреля 2024}% Custom header text

%----------------------------------------------------------------------------------------

%\begin{document}\normalsize
\begin{document}\large
	

\begin{center}
\textbf{Тождественные преобразования}
\end{center}

\begin{enumerate}[label*=\protect\fbox{\arabic{enumi}}]
	
	\item Чему равняется сумма всех натуральных делителей числа $2^2 \cdot 3^3 \cdot 5^5.$
	
	\item Для каждого натурального $n \ge 2$ вычислите сумму $$\dfrac{1}{1} + \dfrac{1}{2} + \dotso +  \dfrac{1}{n} + \dfrac{1}{1 \cdot 2} + \dfrac{1}{1 \cdot 3} + \dotso +  \dfrac{1}{(n-1) \cdot n} + \dotso + \dfrac{1}{1 \cdot 2 \cdot \dotso \cdot n}.$$
	(В знаменателях стоят все возможные произведения нескольких из чисел $1, 2, ..., n$. Произведение одного числа равно самому этому числу).
	
	\item Упростите выражение $(1+3+3^2)\cdot (1+3^3 +3^6)\cdot (1+3^9 +3^{18})\cdot \dotsc \cdot  (1+3^{3^n} +3^{2\cdot 3^n})$
	
	\item На доске записаны 10 различных чисел. Профессор Odd вычислил всевозможные произведения нескольких записанных чисел, взятых в нечетном количестве (по 1, по 3, по 5, по 7, по 9), сложил все эти произведения и полученную сумму записал на листок. Аналогично профессор Even вычислил все возможные произведения нескольких чисел, записанных на доске, взятых в четном количестве (по 2, по 4, по 6, по 8, по 10), сложил все эти произведения и полученную сумму записал на свой листок. Оказалось, что сумма на листке профессора Odd на 1 больше, чем сумма на листке профессора Even. Докажите, что одно из чисел, выписанных на доске, равно 1.
	
	\item Рассмотрим все произведения некоторого количества чисел из $2, 3, ..., n$ (произведение одного числа равно самому этому числу). Найдите сумму всех таких произведений, в которых чётное количество чётных сомножителей.
	
	\item Про натуральное число $n$ известно, что сумма его натуральных делителей является степенью двойки. Докажите, что тогда и количество его делителей является степенью двойки.
	
	
	
	
\end{enumerate}
\end{document}