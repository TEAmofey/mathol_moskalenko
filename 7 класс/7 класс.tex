\documentclass{article}

\usepackage[12pt]{extsizes}
\usepackage[T2A]{fontenc}
\usepackage[utf8]{inputenc}
\usepackage[english, russian]{babel}

\usepackage{mathrsfs}
\usepackage[dvipsnames]{xcolor}

\usepackage{amsmath}
\usepackage{amssymb}
\usepackage{amsthm}
\usepackage{indentfirst}
\usepackage{amsfonts}
\usepackage{enumitem}
\usepackage{graphics}
\usepackage{tikz}
\usepackage{tabu}
\usepackage{diagbox}
\usepackage{hyperref}
\usepackage{mathtools}
\usepackage{ucs}
\usepackage{lipsum}
\usepackage{geometry} % Меняем поля страницы
\usepackage{fancyhdr} % Headers and footers
\newcommand{\range}{\mathrm{range}}
\newcommand{\dom}{\mathrm{dom}}
\newcommand{\N}{\mathbb{N}}
\newcommand{\R}{\mathbb{R}}
\newcommand{\E}{\mathbb{E}}
\newcommand{\D}{\mathbb{D}}
\newcommand{\M}{\mathcal{M}}
\newcommand{\Prime}{\mathbb{P}}
\newcommand{\A}{\mathbb{A}}
\newcommand{\Q}{\mathbb{Q}}
\newcommand{\Z}{\mathbb{Z}}
\newcommand{\F}{\mathbb{F}}
\newcommand{\CC}{\mathbb{C}}

\DeclarePairedDelimiter\abs{\lvert}{\rvert}
\DeclarePairedDelimiter\floor{\lfloor}{\rfloor}
\DeclarePairedDelimiter\ceil{\lceil}{\rceil}
\DeclarePairedDelimiter\lr{(}{)}
\DeclarePairedDelimiter\set{\{}{\}}
\DeclarePairedDelimiter\norm{\|}{\|}

\renewcommand{\labelenumi}{(\alph{enumi})}

\newcommand{\smallindent}{
    \geometry{left=1cm}% левое поле
    \geometry{right=1cm}% правое поле
    \geometry{top=1.5cm}% верхнее поле
    \geometry{bottom=1cm}% нижнее поле
}

\newcommand{\header}[3]{
    \pagestyle{fancy} % All pages have headers and footers
    \fancyhead{} % Blank out the default header
    \fancyfoot{} % Blank out the default footer
    \fancyhead[L]{#1}
    \fancyhead[C]{#2}
    \fancyhead[R]{#3}
}

\newcommand{\dividedinto}{
    \,\,\,\vdots\,\,\,
}

\newcommand{\littletaller}{\mathchoice{\vphantom{\big|}}{}{}{}}

\newcommand\restr[2]{{
    \left.\kern-\nulldelimiterspace % automatically resize the bar with \right
    #1 % the function
    \littletaller % pretend it's a little taller at normal size
    \right|_{#2} % this is the delimiter
}}

\DeclareGraphicsExtensions{.pdf,.png,.jpg}

\newenvironment{enumerate_boxed}[1][enumi]{\begin{enumerate}[label*=\protect\fbox{\arabic{#1}}]}{\end{enumerate}}



\smallindent

\header{Математика}{\textit{Разное}}{7 класс}

%----------------------------------------------------------------------------------------

\begin{document}
    \large

    \begin{center}
        \textbf{Разнобой}
    \end{center}


    \begin{enumerate_boxed}
        \item Три лягушки на болоте прыгнули по очереди.
        Каждая приземлялась точно в середину отрезка между двумя другими.
        Длина прыжка второй лягушки 60 см.
        Найдите длину прыжка третьей лягушки.

        \item На прямой отмечены 5 точек $P, Q, R, S, T$ именно в таком порядке.
        Известно, что сумма расстояний от $P$ до остальных 4 точек равна 67, а сумма расстояний
        от $Q$ до остальных 4 точек равна 34.
        Найдите длину отрезка $P Q$.

        \item Полый кубик размером $2 \times 2 \times 2$, изготовленный из листового материала, весит 5 грамм.
        Сколько весит кубик размером $6 \times 6 \times 6$, изготовленный из того же листового материала?

        \item Из десяти одинаковых прямоугольников с периметром 42 Коля взял пять и выложил их в ряд.
        Получился прямоугольник с периметром 106.
        Из оставшихся пяти прямоугольников Коля сложил ещё один прямоугольник.
        Чему может быть равен его периметр?

        \item Разрежьте клетчатый квадрат $6 \times 6$ на различные клетчатые фигурки, каждая из которых состоит не более чем из 5 клеток и не является прямоугольником (или квадратом).

        \item Картонный квадрат $10 \times 10$ расчерчен красным фломастером на клетки со стороной 1.
        В каждой его клетке зелёным фломастером провели обе диагонали, и разрезали большой квадрат по зелёным линиям.
        В результате картонный квадрат $10 \times 10$ распался на части.
        Сколько частей получилось?

        \item Можно ли какой-нибудь клетчатый квадрат разрезать на трёхклеточные уголки и вертикальные доминошки так, чтобы фигурок каждого вида было поровну?

        \item Найдутся ли семь различных правильных несократимых дробей со знаменателями от 2 до 6 и с суммой 4?

        \item На какую цифру заканчивается $6^{2024} + 2023^{2024}$

        \item В турнире участвовали десять шахматистов.
        Каждый сыграл с каждым два раза: один раз белыми и один раз чёрными, причём какую-то из этих партий он выиграл, а другую проиграл (ничьих не было).
        Могло ли оказаться так, что половину всех партий выиграли белые, а половину – чёрные?

        \item Придумайте пять различных натуральных чисел, произведение которых равно 1000.

        \item У числа $100!$ вычеркнули все нули в конце.
        Четна или нечетна цифра в получившемся числе?

        \item Числа $p$ и $p^p - p! + 1$~--- простые.
        Найдите $p$

        \item Саша нарисовал квадрат $8\times 8$.
        Его сестренка Настя поставила кляксы в 15 клеток этого квадрата.
        Докажите, что Саша может вырезать прямоугольник со сторонами 1 и 4, не содержащий ни одной кляксы.

        \item На острове 2/3 всех мужчин женаты и 3/5 всех женщин замужем.
        Какая доля населения острова состоит в браке?

        \item Известно, что $20!$ равно одному из следующих чисел: 2432902008176640000 или 2432902008146640000.
        Какому?

        \item В клетках квадрата $3 \times 3$ расставлены числа (не обязательно целые) так, что в любой строчке и в любом столбце сумма чисел равна 2, а в любом квадрате $2 \times 2$ сумма чисел равна 3.
        Какие числа стоят в квадрате?


    \end{enumerate_boxed}
\end{document}