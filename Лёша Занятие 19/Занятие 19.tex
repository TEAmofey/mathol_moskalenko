\documentclass{article}
\usepackage[12pt]{extsizes}
\usepackage[T2A]{fontenc}
\usepackage[utf8]{inputenc}
\usepackage[english, russian]{babel}

\usepackage{amssymb}
\usepackage{amsfonts}
\usepackage{amsmath}
\usepackage{enumitem}
\usepackage{graphics}
\usepackage{graphicx}

\usepackage{lipsum}

\newtheorem{theorem}{Теорема}
\newtheorem{task}{Задача}
\newtheorem{lemma}{Лемма}
\newtheorem{definition}{Определение}
\newtheorem{example}{Пример}
\newtheorem{statement}{Утверждение}
\newtheorem{corollary}{Следствие}


\usepackage{geometry} % Меняем поля страницы
\geometry{left=1cm}% левое поле
\geometry{right=1cm}% правое поле
\geometry{top=1.5cm}% верхнее поле
\geometry{bottom=1cm}% нижнее поле


\usepackage{fancyhdr} % Headers and footers
\pagestyle{fancy} % All pages have headers and footers
\fancyhead{} % Blank out the default header
\fancyfoot{} % Blank out the default footer
\fancyhead[L]{Математика}
\fancyhead[C]{\textit{Разнобой}}
\fancyhead[R]{19 марта 2024}% Custom header text


%----------------------------------------------------------------------------------------

%\begin{document}\normalsize
\begin{document}\large
	
\begin{center}
	\textbf{Теория чисел}
\end{center}


\begin{enumerate}[label*=\protect\fbox{\arabic{enumi}}]
	
\item Найдите $n$ из равенства $5^{47} \cdot 5^{19} = 5^n$.

\item На острове живут рыцари, которые всегда говорят правду, и лжецы, которые всегда лгут. На улице встретились два жителя острова. Один из них сказал: "По крайней мере один из нас рыцарь". Второй ему ответил: "Ты лжец". Кто из них кто?

\item Можно ли в выражении $1 * 2 * 3*4*5*6*7*8*9$ заменить все звездочки на $+$ или $-$ так, чтобы в ответе получилось a) 1? b) 0?

\item Натуральное число умножили на произведение всех его цифр. Получилось 1995. Найдите исходное
число.

\item Вася возводил число в квадрат и получил 1234567. Его друг Петя сказал: <<Пересчитай, а то получишь двойку>>. Докажите, что Петя прав.


\end{enumerate}


\end{document}