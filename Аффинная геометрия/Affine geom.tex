\documentclass{article}

\usepackage[12pt]{extsizes}
\usepackage[T2A]{fontenc}
\usepackage[utf8]{inputenc}
\usepackage[english, russian]{babel}

\usepackage{mathrsfs}
\usepackage[dvipsnames]{xcolor}

\usepackage{amsmath}
\usepackage{amssymb}
\usepackage{amsthm}
\usepackage{indentfirst}
\usepackage{amsfonts}
\usepackage{enumitem}
\usepackage{graphics}
\usepackage{tikz}
\usepackage{tabu}
\usepackage{diagbox}
\usepackage{hyperref}
\usepackage{mathtools}
\usepackage{ucs}
\usepackage{lipsum}
\usepackage{geometry} % Меняем поля страницы
\usepackage{fancyhdr} % Headers and footers
\newcommand{\range}{\mathrm{range}}
\newcommand{\dom}{\mathrm{dom}}
\newcommand{\N}{\mathbb{N}}
\newcommand{\R}{\mathbb{R}}
\newcommand{\E}{\mathbb{E}}
\newcommand{\D}{\mathbb{D}}
\newcommand{\M}{\mathcal{M}}
\newcommand{\Prime}{\mathbb{P}}
\newcommand{\A}{\mathbb{A}}
\newcommand{\Q}{\mathbb{Q}}
\newcommand{\Z}{\mathbb{Z}}
\newcommand{\F}{\mathbb{F}}
\newcommand{\CC}{\mathbb{C}}

\DeclarePairedDelimiter\abs{\lvert}{\rvert}
\DeclarePairedDelimiter\floor{\lfloor}{\rfloor}
\DeclarePairedDelimiter\ceil{\lceil}{\rceil}
\DeclarePairedDelimiter\lr{(}{)}
\DeclarePairedDelimiter\set{\{}{\}}
\DeclarePairedDelimiter\norm{\|}{\|}

\renewcommand{\labelenumi}{(\alph{enumi})}

\newcommand{\smallindent}{
    \geometry{left=1cm}% левое поле
    \geometry{right=1cm}% правое поле
    \geometry{top=1.5cm}% верхнее поле
    \geometry{bottom=1cm}% нижнее поле
}

\newcommand{\header}[3]{
    \pagestyle{fancy} % All pages have headers and footers
    \fancyhead{} % Blank out the default header
    \fancyfoot{} % Blank out the default footer
    \fancyhead[L]{#1}
    \fancyhead[C]{#2}
    \fancyhead[R]{#3}
}

\newcommand{\dividedinto}{
    \,\,\,\vdots\,\,\,
}

\newcommand{\littletaller}{\mathchoice{\vphantom{\big|}}{}{}{}}

\newcommand\restr[2]{{
    \left.\kern-\nulldelimiterspace % automatically resize the bar with \right
    #1 % the function
    \littletaller % pretend it's a little taller at normal size
    \right|_{#2} % this is the delimiter
}}

\DeclareGraphicsExtensions{.pdf,.png,.jpg}

\newenvironment{enumerate_boxed}[1][enumi]{\begin{enumerate}[label*=\protect\fbox{\arabic{#1}}]}{\end{enumerate}}


\usepackage[framemethod=TikZ]{mdframed}

\newcommand{\definebox}[3]{%
    \newcounter{#1}
    \newenvironment{#1}[1][]{%
        \stepcounter{#1}%
        \mdfsetup{%
            frametitle={%
            \tikz[baseline=(current bounding box.east),outer sep=0pt]
            \node[anchor=east,rectangle,fill=white]
            {\strut #2~\csname the#1\endcsname\ifstrempty{##1}{}{##1}};}}%
        \mdfsetup{innertopmargin=1pt,linecolor=#3,%
            linewidth=3pt,topline=true,
            frametitleaboveskip=\dimexpr-\ht\strutbox\relax,}%
        \begin{mdframed}[]
            \relax%
            }{
        \end{mdframed}}%
}

\definebox{theorem_boxed}{Теорема}{ForestGreen!24}
\definebox{definition_boxed}{Определение}{blue!24}
\definebox{task_boxed}{Задача}{orange!24}
\definebox{paradox_boxed}{Парадокс}{red!24}

\theoremstyle{plain}
\newtheorem{theorem}{Теорема}
\newtheorem{task}{Задача}
\newtheorem{lemma}{Лемма}
\newtheorem{statement}{Утверждение}
\newtheorem{corollary}{Следствие}

\theoremstyle{remark}
\newtheorem{remark}{Замечание}
\newtheorem{example}{Пример}

\smallindent

\header{Математика}{\textit{Геометрия}}{22 июля 2024}

%----------------------------------------------------------------------------------------

\begin{document}
    \large

    \begin{center}
        \textbf{Аффинная геометрия}
    \end{center}

    \begin{enumerate_boxed}
        \item
        \begin{enumerate}
            \item Докажите, что любой треугольник можно спроектировать так, чтобы получился треугольник любой заданной формы, то есть изображением любого треугольника может служить произвольный заданный треугольник.
            \item Изображаются ли при этом высоты высотами?
            Медианы~--- медианами?
            Биссектрисы~--- биссектрисами?
        \end{enumerate}
        \item Какие четырехугольники могут служить изображениями квадрата; ромба; дельтоида; равнобокой трапеции; трапеции, отсекаемой от правильного треугольника средней линией?
        Как описать множество всевозможных изображений произвольного заданного четырехугольника?

        \item Покажите, что задачу 9 (из листика про площади) достаточно решить для частного случая, когда данный параллелограмм~--- квадрат.
        Найдите решение для этого случая.
        \item В треугольнике $C_{1}C_{0}C$ провели 3 чевианы $C_{1}A, C_{0}B, CC_1$ так, что $AB \parallel C_{1}C_{0}$.
        Докажите, что $C{0}C_{1} : C_{n - 1}C_{n} = n : 1$
%        (Используйте центральную проекцию.)

        \begin{center}

            \begin{asy}
                import geometry;


                size(7cm,0);

                point pA = (-2,0);
                point pB = (2,0);
                point pC = (-1,4);


                real alpha = 0.3;

                point pB0 = scale(alpha, pC) * pA;
                point pA0 = scale(alpha, pC) * pB;
                point pC0 = (pA + pB) / 2;

                point pC1 = scale(2/3, pB) * pC0;
                point pC2 = scale(3/4, pB) * pC1;


                dot("$C$", pC, N);
                dot("$C_1$", pA, SW);
                dot("$C_0$", pB, SE);
                dot("$A$", pA0, E);
                dot("$B$", pB0, W);
                dot("$C_2$", pC0, S);
                dot("$C_3$", pC1, S);
                dot("$C_4$", pC2, S);


                draw(pA -- pB -- pC -- cycle);

                draw(pA -- pA0);
                draw(pB -- pB0);
                draw(pC -- pC0 -- pA0 -- pC1);
                draw(pC -- pC1 -- pA0);
                draw(pC -- pC2);
            \end{asy}
        \end{center}

        \item Докажите теорему Чевы сведением к частному случаю~--- теореме о высотах треугольника.
        Можно ли аналогичным образом использовать медианы?
        биссектрисы?

        \item На изображении прямоугольного треугольника $ABC$, у которого $\angle C=90^\circ, AC:CB=3:1$, постройте изображение его а) биссектрисы, б) медианы, в) высоты, проведенных из $\angle C$.
        \item Дано изображение треугольника $ABC$.
        Постройте изображения центров его вписанной и описанной окружностей, если известно, что $AC = BC$ и высота $CH = AB$.
        \item Известно, что данная трапеция является изображением
        \begin{enumerate}
            \item трапеции с острым углом $45^\circ$, вписанной в окружность с центром О;
            \item прямоугольной трапеции с острым углом $60^\circ$, описанной около окружности с центром $O$.
        \end{enumerate}
        Постройте изображение точки $O$.
        \item Дано изображение фигуры, состоящей из прямоугольного треугольника и квадрата, построенного на его катете.
        Постройте изображение квадрата, построенного
        \begin{enumerate}
            \item на другом катете;
            \item на гипотенузе.
        \end{enumerate}
        \item
        \begin{enumerate}
            \item На изображении прямоугольного треугольника $ABC$, в котором проведен отрезок $CD$~--- изображение биссектрисы прямого угла, постройте изображение $CH$ высоты.
            \item Пусть $A_1$ и $B_1$~--- точки на продолжениях сторон $BC$ и $AC$ треугольника $ABC$ такие, что $AA_1 \parallel BB_1 \parallel CD$.
            Докажите, что прямая $HC$ (изображение перпендикуляра к $AB$) делит пополам отрезок $A_{1}B_1$.
        \end{enumerate}
        \item Дан треугольник, изображающий некоторый треугольник $T$.
        Где могут располагаться изображения а) центроида, б) центра описанной окружности, в) центра вписанной окружности, г) ортоцентра треугольника $T$?
        \item Как изображается правильный шестиугольник?
        Постройте его изображение, если даны изображения трёх вершин; вершины и середин двух сторон (рассмотрите разные случаи).
        \item Докажите, что любой пятиугольник, у которого каждая сторона параллельна одной из диагоналей, является изображением правильного пятиугольника.

        \item Из концов основания треугольника проведены медианы, а из произвольной точки основания - параллельные им прямые.
        Докажите, что отрезок, соединяющий точки пересечения этих прямых с боковыми сторонами, делится медианами на три равные части.
        \item Каждая диагональ выпуклого пятиугольника, кроме, может быть, одной, параллельна одной из его сторон.
        Докажите, что а) это верно и для пятой диагонали; б) отношение каждой диагонали к параллельной ей стороне одно и то же; найдите это отношение.
        \item В выпуклом пятиугольнике проведены «медианы» - прямые, соединяющие каждую вершину с серединой противоположной стороны.
        Докажите, что если четыре из них проходят через одну точку, то и пятая тоже.
        \item Докажите, что шестиугольник с вершинами в серединах «малых» диагоналей данного выпуклого шестиугольника имеет в четыре раза меньшую площадь.
        \item Середины трёх сторон правильного шестиугольника соединены с концами противоположных сторон как на рисунке.
        Какую часть от площади шестиугольника составляет площадь треугольника, ограниченного проведенными отрезками?
        \item В шестиугольнике $ABCDEF$ противоположные стороны параллельны.
        Докажите, что
        \begin{enumerate}
            \item если две пары противоположных сторон параллельны соответствующим диагоналям, то это верно и для третьей пары;
            \item отрезки, соединяющие середины противоположных сторон шестиугольника, пересекаются в одной точке;
            \item треугольники $ACE$ и $BDF$ равновелики;
            \item если противоположные стороны не только параллельны, но и равны, то площадь $\Delta ACE$ равна половине площади шестиугольника.
        \end{enumerate}
        \item Докажите, что в произвольном шестиугольнике условия b) и c) из задачи 19 эквивалентны.
        \item Докажите, что если каждый из отрезков, соединяющих середины противоположных сторон выпуклого шестиугольника, делит его площадь пополам, то эти отрезки пересекаются в одной точке.
        \item Докажите, что если каждая из «больших» диагоналей выпуклого шестиугольника делит его площадь пополам, то они пересекаются в одной точке.
        \item Даны площади трех треугольников найдите площадь $\Delta AFN$.

        \begin{center}
            \begin{asy}
                import geometry;


                size(7cm,0);

                point pA = (5,-4);
                point pB = (6,4);
                point pC = (-5,4);
                point pD = (-6,-4);

                point pE = (pC + pB) / 2;
                point pF = (2 * pA + pB) / 3;

                point pK = intersectionpoint(pC -- pF, pA -- pE);
                point pM = intersectionpoint(pC -- pF, pD -- pE);
                point pN = intersectionpoint(pE -- pA, pD -- pF);

                draw(pA -- pB -- pC -- pD -- cycle);
                draw(pC -- pF);
                draw(pA -- pE);
                draw(pD -- pE);
                draw(pD -- pF);

                filldraw(pC -- pE -- pM -- cycle, grey);
                filldraw(pK -- pE -- pM -- cycle, grey);
                filldraw(pK -- pN -- pF -- cycle, grey);
                filldraw(pA -- pN -- pF -- cycle, grey);

                label(Label("21"), (pM + pE + pC) / 3);
                label(Label("21"), (pM + pE + pK) / 3);
                label(Label("6"), (pN + pF + pK) / 3);
                label(Label("?"), (pN + pF + pA) / 3);

                dot("$A$", pA, SE);
                dot("$B$", pB, NE);
                dot("$C$", pC, NW);
                dot("$D$", pD, SW);
                dot("$E$", pE, NW);
                dot("$F$", pF, SE);
                dot("$K$", pK, NE);
                dot("$M$", pM, 2S);
                dot("$N$", pN, 2S + W);
            \end{asy}
        \end{center}

    \end{enumerate_boxed}

\end{document}