\documentclass{article}
\usepackage[12pt]{extsizes}
\usepackage[T2A]{fontenc}
\usepackage[utf8]{inputenc}
\usepackage[english, russian]{babel}

\usepackage{amssymb}
\usepackage{amsfonts}
\usepackage{amsmath}
\usepackage{enumitem}
\usepackage{graphics}
\usepackage{graphicx}

\usepackage{lipsum}
\DeclareGraphicsExtensions{.pdf,.png,.jpg}



\usepackage{geometry} % Меняем поля страницы
\geometry{left=1cm}% левое поле
\geometry{right=1cm}% правое поле
\geometry{top=1.5cm}% верхнее поле
\geometry{bottom=1cm}% нижнее поле


\usepackage{fancyhdr} % Headers and footers
\pagestyle{fancy} % All pages have headers and footers
\fancyhead{} % Blank out the default header
\fancyfoot{} % Blank out the default footer
\fancyhead[L]{Математика}
\fancyhead[C]{\textit{Алгебра}}
\fancyhead[R]{28 ноября 2023}% Custom header text

%----------------------------------------------------------------------------------------

%\begin{document}\normalsize
\begin{document}\large
	

\begin{center}
\textbf{Разнобой по алгебре}
\end{center}

\begin{enumerate}[label*=\protect\fbox{\arabic{enumi}}]
	
	\item  Рациональные числа $a$ и $b$ удовлетворяют равенству 
	$$a^3b + ab^3 + 2a^2b^2+2a+2b+1 = 0$$
	Докажите, что $1 - ab$ — квадрат рационального числа.
	
	\item Дано натуральное $n > 1$. Для каждого делителя $d$ числа $n + 1$ Петя разделил
	число $n$ на $d$ с остатком и записал на доску неполное частное, а в тетрадь —
	остаток. Докажите, что наборы чисел на доске и в тетради совпадают.
	
	\item Докажите, что для любого простого числа $p$ найдется число вида $2023^n - n$,
	делящееся на $p$.
	
	\item Найдите все тройки простых чисел $p, q, r$ такие, что четвёртая степень любого из них, уменьшенная на 1, делится на произведение двух остальных.
	
	\item Докажите, что существует бесконечно много таких пар различных натуральных чисел $k, n > 1$, что $(k! + 1, n! + 1) > 1.$
	
	\item Числа $a, b, c$ являются длинами сторон треугольника. Докажите, что
	$$\frac{a^2 + 2bc}{b^2 + c^2} + \frac{b^2 + 2ca}{c^2 + a^2} +\frac{c^2 + 2ab}{a^2 + b^2} > 3$$
	
	\item Попарно различные натуральные числа $a, b, c$ таковы, что $b + c + bc$ делится
	на $a$, $a + c + ac$ делится на $b$, $a + b + ab$ делится на $c$. Докажите, что хотя бы
	одно из чисел $a, b, c$ не является простым.

	\item Дано натуральное число $a$. Известно, что для любого натурального $n$ у числа $n^2 a - 1$ найдётся натуральный делитель $d > 1$ такой, что $d \equiv 1 \pmod n$. Докажите, что $a$ — точный квадрат.

\end{enumerate}
\end{document}