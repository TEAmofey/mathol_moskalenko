\documentclass{article}
\usepackage[12pt]{extsizes}
\usepackage[T2A]{fontenc}
\usepackage[utf8]{inputenc}
\usepackage[english, russian]{babel}

\usepackage{amssymb}
\usepackage{amsfonts}
\usepackage{amsmath}
\usepackage{enumitem}
\usepackage{graphics}
\usepackage{graphicx}

\usepackage{lipsum}

\newtheorem{theorem}{Теорема}
\newtheorem{task}{Задача}
\newtheorem{lemma}{Лемма}
\newtheorem{definition}{Определение}
\newtheorem{example}{Пример}
\newtheorem{statement}{Утверждение}
\newtheorem{corollary}{Следствие}


\usepackage{geometry} % Меняем поля страницы
\geometry{left=1cm}% левое поле
\geometry{right=1cm}% правое поле
\geometry{top=1.5cm}% верхнее поле
\geometry{bottom=1cm}% нижнее поле


\usepackage{fancyhdr} % Headers and footers
\pagestyle{fancy} % All pages have headers and footers
\fancyhead{} % Blank out the default header
\fancyfoot{} % Blank out the default footer
\fancyhead[L]{Математика}
\fancyhead[C]{\textit{Геометрия}}
\fancyhead[R]{3 ноября 2023}% Custom header text


%----------------------------------------------------------------------------------------

%\begin{document}\normalsize
\begin{document}\large

\begin{center}
\textbf{Геометрия масс}
\end{center}


\textbf{Определение:} Пусть $M$ некоторая точка плоскости и $m$ ненулевое число. Материальной точкой $(m,M)$ называется точка $M$ с числом (массой) $m$, причем число $m$ называется массой материальной точки $(m, M)$, а точка $M$ носителем этой материальной точки.

\textbf{Определение:} Центром масс системы материальных точек $(m_1, M_1), (m_2, M_2), \dotsc, (m_n,M_n)$ с ненулевой суммой масс называется такая точка $Z$, для которой имеет место
равенство $m_1\overrightarrow{ZM_1}+m_2\overrightarrow{ZM_2}+ \dotsc + m_n\overrightarrow{ZM_n} = 0.$

\begin{enumerate}[label*=\protect\fbox{\arabic{enumi}}]


\item \textbf{(Основная теорема)} Точка $Z$ является центром масс системы точек $(m_1,M_1),$ $(m_2,M_2), \dotsc, (m_n,M_n)$, причем $m_1 + m_2 + \dotsc + m_n \neq 0$. Докажите для любой точки плоскости $O$, что
$$(m_1 +m_2 +...+m_n)\overrightarrow{OZ} =m_1\overrightarrow{OM_1} +m_2\overrightarrow{OM_2} +\dotsc +m_n\overrightarrow{OM_n}$$

\textbf{Следствие:} У системы точек с ненулевой суммой масс центр масс существует и единственен.
\item \textbf{(Правило группировки)} Есть $n$ материальных точек с ненулевой суммой масс $(m_1, M_1), \dotsc , (m_n, M_n)$. Точка $Z$ является центром масс системы первых $k$ из этих материальных точек $(m_1, M_1), \dotsc , (m_k, M_k)$. Докажите, что центры масс исходной системы и системы, в которой первые $k$ материальных точек были заменены на материальную точку $(m_1 + \dotsc + m_k, Z)$, совпадают.

\item На сторонах $AB$ и $AC$ треугольника $ABC$ выбраны точки $D$ и $E$ соответственно. $X$ — точка пересечения отрезков $BE$ и $CD$.
В точке $A$ находится масса 1.
Какие массы надо поместить в точки $B$ и $C$, чтобы центр масс системы попал в точку $X$, если $AD:DB=1:2$, $AE:EC=3:1$?

\item Стороны треугольника $ABC$ равны $AB=3, BC = 4, AC = 5$. В вершине $C$ находится масса 5. Какие массы нужно поместить в вершины $A$ и $B$, чтобы центр масс попал в точку пересечения медиан треугольника $ABC$?

\item Стороны треугольника $ABC$ равны $AB=5, BC = 10, AC = 7$. В вершине $C$ находится масса 10. Какие массы нужно поместить в вершины $A$ и $B$, чтобы центр масс попал в точку пересечения биссектрис треугольника $ABC$?

\item Дан треугольник $ABC$ со сторонами $|AB| = c, |BC| = a, |CA| = b$. Какие массы нужно поставить в его вершины, чтобы центр масс попал в
\begin{enumerate}
	\item точку пересечения медиан;
	\item центр вписанной окружности;
	\item центр вневписанной окружности, касающейся стороны $BC$;
	\item вершину $D$ параллелограмма $ABCD$;
	\item точку \textit{Нагеля}, то есть точку пересечения отрезков, соединяющих вершину с точкой касания противоположной вневписанной окружности;
	\item точку \textit{Жергонна}, то есть точку пересечения отрезков, соединяющих вершину с противоположной точкой касания вписанной окружности;
\end{enumerate} 
Эти массы называются \textbf{барицентрическими координатами} точки, относительно треугольника.

\item Пусть $K, L, M$ — точки пересечения медиан треугольников $PAB, PBC, PCA$ соответственно. Докажите, точка $P$ и точки пересечения медиан треугольников $ABC$ и $KLM$ лежат на одной прямой.

\item Дан треугольник $ABC$ и точка $P$ на плоскости. Обозначим через $S_{BPC}$ ориентированную площадь треугольника $BPC$ с положительным знаком, если $P$ и $A$ в одной полуплоскости относительно $BC$, и с отрицательным иначе. Аналогично определим $S_{CPA}$ и $S_{APB}$. Докажите, что $P$ является центром масс системы материальных точек $(S_{BPC},A), (S_{CPA},B), (S_{APB},C)$.

\item В шестиугольнике точки $A_1, A_2, A_3, A_4, A_5, A_6$
являются серединами последовательных сторон. Докажите, что точки пересечения медиан треугольников $A_1A_3A_5$ и $A_2A_4A_6$ совпадают.

\item \textbf{(Теорема ван Обеля)} Пусть в треугольнике $ABC$
проведены чевианы $AA_1, BB_1, CC_1$, пересекающиеся в точке $X$. Тогда
$$\frac{BX}{XB_1}=\frac{BC_1}{C_1A}+\frac{BA_1}{A_1C}.$$

\item Докажите, что уравнение $ax + by + cz = 0$ с фиксированными $a, b$ и $c$ $(a + b + c \neq 0)$ задает прямую, состоящую из точек, являющихся центром масс системы материальных точек $(x, A), (y, B), (z, C)$, не лежащих на одной прямой.

\item На биссектрисе угла $A$ неравнобедренного треугольника $ ABC $ выбрана точка $ K $. Прямые $ BK $ и $ CK $ пересекают стороны $ AC $ и $ AB $ в точках $ L $ и $ M $. Докажите, что прямая $ LM $ проходит через основание биссектрисы внешнего угла $ A $ треугольника.

\item Чевианы $ AA_1 $ и $ CC_1 $ в треугольнике $ ABC $ пересекаются на высоте $ BH $. Докажите, что $ BH $ является биссектрисой угла $ A_1HC_1 $.
\item Вписанная окружность треугольника $ ABC $ с центром в точке $ I $ касается сторон $ BC $, $ CA $ и $ AB $ в точках $ A' $ , $ B' $  и $ C' $  соответственно. Докажите, что точка пересечения прямых $ A' B' $  и $ C' I $ лежит на медиане из вершины $ C $.
\item \textbf{(Прямая Гаусса)} Прямая пересекает стороны $ AB $, $ BC $ и продолжение стороны $ AC $ треугольника $ ABC $ в точках $ D $, $ E $ и $ F $ соответственно. Докажите, что середины отрезков $ CD $, $ AE $ и $ BF $ лежат на одной прямой.

\item Пусть $ BM $ медиана прямоугольного треугольника $ ABC $ $(\angle B = 90^{\circ} )$. Окружность, вписанная в треугольник $ABM$, касается сторон $ AB $ и $ AM $ в точках $ A_1 $ и $ A_2 $ соответственно; аналогично определяются точки $ C_1 $ и $ C_2 $. Докажите, что прямые $ A_1A_2 $ и $ C_1C_2 $ пересекаются на биссектрисе угла $ ABC $.
\item Пусть $ BD $ биссектриса треугольника $ ABC $. Точки $ I_a, I_c $ центры вписанных окружностей треугольников $ ABD $, $ CBD $. Прямая $ I_aI_c $ пересекает прямую $ AC $ в точке $ Q $. Докажите, что  $ \angle DBQ = 90^{\circ} $ .
\item На окружности расположены $ n $ точек с единичными массами. Вася выбирает произвольные $ n - 2 $ из них, рисует их центр масс и опускает из него перпендикуляр на прямую, проходящую через две оставшиеся точки. Докажите, что все Васины прямые пройдут через одну точку.
\item Вневписанная окружность треугольника $ ABC $, вписанная в угол $ C $, касается продолжения стороны $ BC $ в точке $ X $. Прямая, параллельная $ AX $ и проходящая через вершину $ B $ пересекает $ AC $ в точке $ Y $. Точка $ I $ центр вписанной окружности треугольника $ ABC $. Докажите, что $\angle YIC = 90^\circ $ .
	\end{enumerate} 
\end{document}