\documentclass{article}
\usepackage[12pt]{extsizes}
\usepackage[T2A]{fontenc}
\usepackage[utf8]{inputenc}
\usepackage[english, russian]{babel}

\usepackage{amssymb}
\usepackage{amsfonts}
\usepackage{amsmath}
\usepackage{enumitem}
\usepackage{graphics}
\usepackage{graphicx}

\usepackage{lipsum}

\newtheorem{theorem}{Теорема}
\newtheorem{task}{Задача}
\newtheorem{lemma}{Лемма}
\newtheorem{definition}{Определение}
\newtheorem{example}{Пример}
\newtheorem{statement}{Утверждение}
\newtheorem{corollary}{Следствие}


\usepackage{geometry} % Меняем поля страницы
\geometry{left=1cm}% левое поле
\geometry{right=1cm}% правое поле
\geometry{top=1.5cm}% верхнее поле
\geometry{bottom=1cm}% нижнее поле


\usepackage{fancyhdr} % Headers and footers
\pagestyle{fancy} % All pages have headers and footers
\fancyhead{} % Blank out the default header
\fancyfoot{} % Blank out the default footer
\fancyhead[L]{ЦРОД $\bullet$ Математика}
\fancyhead[C]{\textit{Теория чисел}}
\fancyhead[R]{25 марта 2024}% Custom header text


%----------------------------------------------------------------------------------------

%\begin{document}\normalsize
\begin{document}\large
	
\begin{center}
	\textbf{Сравнения по модулю}
\end{center}


\textbf{Определение 1}
Число $a$ делится на натуральное $b$ с остатком $r$, если $a = bk + r$, причем $0 \le r < b$.

\textbf{Определение 2}
Целые числа, разность которых делится на $m$, называются сравнимыми по модулю $m$. Запись: $a \equiv b$  (mod $m)$.

\textbf{Свойства сравнений}

\begin{itemize}
	\item $a \equiv b$  (mod $m) \Leftrightarrow$ числа $a$ и $b$ дают одинаковые остатки по модулю $m$.
	
	\item $a \equiv b$  (mod $m) \Rightarrow$ $ka \equiv kb$ (mod $m)$.
	
	\item $a \equiv b$  (mod $m),\, c \equiv d$ (mod $m) \Rightarrow a\pm c \equiv b\pm d$ (mod $m)$.
	
	\item $a \equiv b$  (mod $m),\, c \equiv d$ (mod $m) \Rightarrow ac \equiv bd$ (mod $m)$.
	
	\item $a \equiv b$  (mod $m) \Rightarrow a^k \equiv b^k$ (mod $m)$.
\end{itemize}


\begin{enumerate}[label*=\protect\fbox{\arabic{enumi}}]
	
	\item Докажите, что число $1000 \cdot 1001 \cdot 1002 \cdot 1003 - 24$ 
	\begin{enumerate}
		\item делится  на 999;
		
		\item на 1004.
	\end{enumerate}
	
	\item Известно, что $a - 2b$ делится на $m$ и $c - 3d$ делится на $m$. Докажите, что $ac - 6bd$
	делится на $m$.
	
	\textbf{Идея:} Давайте посмотрим на остатки $1, a, a^2, a^3, \dotsc$ при делении на $p$. Они в какой-то момент зациклятся (докажите это)
	
	\item Найдите остаток от деления:
	\begin{enumerate}
		\item $4^{2020}$ на $3$;
		
		\item $7^{2021}$ на $8$;
		
		\item $13^{555}$ на $9$.
	\end{enumerate}
	
	\item Докажите, что  $30^{99} + 61^{100}$  делится на $31$
	
	\item Найдите остатки от деления числа $2^{2021}$ на $3, 5, 7, 9, 11, 13, 15, 17$.
	
	\item Какой остаток дает $x + y$ при делении на $17$, если
	\begin{enumerate}
		\item $x-16y\equiv 2$ (mod $17$);
		
		\item $3x \equiv 5+14y$ (mod $17$);
		
		\item $-10x \equiv 100+27y$ (mod $17$);
		
		\item $28x + 10\equiv -11y$ (mod $17$);
		
		\item $34x - 8\equiv 14(y + x)$ (mod $17$);
		
		\item $1000x \equiv -1085y - 90$ (mod $17$)?
	\end{enumerate}
	
	\item Докажите, что если при некоторых натуральных числах $a$ и $b$ сумма $a^2 + b^2\; \vdots \; 7$,
	то она делится и на $49$.
	
	\item Докажите, что число $9^{2021} + 7^{2020}$ делится на $10$;
	
	\item Найдите наименьшее число, дающее следующие остатки: 1 – при делении на 2, 2 – при делении на 3, 3 – при делении на 4, 4 – при делении на 5, 5 – при делении на 6.
	
	\item Докажите, что число $5^{2021} + 28$ — составное.
	
	\item Пусть $s(x)$~--- сумма цифр в десятичной записи числа $x$. Докажите, что
	\begin{enumerate}
		\item $x \equiv s(x)$ (mod $3$);
		
		\item $x \equiv s(x)$ (mod $9$).
	\end{enumerate}
	
	\item Про натуральные числа $a, b, c$ известно, что $a^2 + b^2 = c^2$. Докажите, что $abc$
	делится на $60$.
	
	\item Докажите, что если $2^k - 1$ делится на $11$, то оно делится и на $31$.
	
	\item Целые числа $x, y$ и $z$ таковы, что  $(x - y)(y - z)(z - x) = x + y + z$.  Докажите, что число  $x + y + z$  делится на $27$.
	
	\item Докажите, что для любого натурального $n$, $4^n + 15n - 1$ делится на $9$.
	
	\item Докажите, что если $a, b, c$ – нечётные числа, то хотя бы одно из чисел  $ab - 1,  bc - 1,  ca - 1$ делится на $4$.
	
	\item Решите сравнения
	
	\begin{enumerate}
		\item $5x \equiv 2$ (mod $3$);
		
		\item $3x \equiv 2$ (mod $11$);
		
		\item $6x \equiv 1$ (mod $13$).
	\end{enumerate}
	
	

\end{enumerate}
\end{document}