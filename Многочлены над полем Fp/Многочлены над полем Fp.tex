\documentclass{article}
\usepackage[12pt]{extsizes}
\usepackage[T2A]{fontenc}
\usepackage[utf8]{inputenc}
\usepackage[english, russian]{babel}

\usepackage{amssymb}
\usepackage{amsfonts}
\usepackage{amsmath}
\usepackage{enumitem}
\usepackage{graphics}
\usepackage{graphicx}

\usepackage{lipsum}
\DeclareGraphicsExtensions{.pdf,.png,.jpg}



\usepackage{geometry} % Меняем поля страницы
\geometry{left=1cm}% левое поле
\geometry{right=1cm}% правое поле
\geometry{top=1.5cm}% верхнее поле
\geometry{bottom=1cm}% нижнее поле


\usepackage{fancyhdr} % Headers and footers
\pagestyle{fancy} % All pages have headers and footers
\fancyhead{} % Blank out the default header
\fancyfoot{} % Blank out the default footer
\fancyhead[L]{Математика}
\fancyhead[C]{\textit{Алгебра и теория чисел}}
\fancyhead[R]{15 октября 2023}% Custom header text


%----------------------------------------------------------------------------------------

%\begin{document}\normalsize
\begin{document}\large
	

\begin{center}
\textbf{Многочлены над полем $\mathbb{F}_p$}
\end{center}

Пусть $p$ — простое число. Обозначим через $\mathbb{F}_p$ множество (поле) остатков от деления на $p$. Через $0 \in \mathbb{F}_p$ будем обозначать нулевой остаток. Множество $\mathbb{F}_p$ состоит из $p$ элементов, которые можно умножать, складывать и вычитать. Более того, любой элемент $a \in \mathbb{F}_p$ можно поделить на любой $0 \neq b \in \mathbb{F}_p$. Сложение и умножение являются \textit{ассоциативными} и \textit{коммутативными} операциями, \textit{дистрибутивность} также выполняется.

Многочленом $P(x)$ с коэффициентами в $\mathbb{F}_p$ назовем формальное выражение $P(x) = a_0 + a_1 x + \ldots + a_k x^k + \ldots$, где $x$ — формальная переменная, $a_0, \ldots, a_k, \ldots \in \mathbb{F}_p$ и только конечное число $a_i$ ненулевые. Многочлены можно складывать и умножать, как обычно:
\[
(a_0 + \ldots + a_k x^k + \ldots) \pm (b_0 + \ldots + b_k x^k + \ldots) = (a_0 \pm b_0) + (a_1 \pm b_1) x + \ldots + (a_k \pm b_k) x^k + \ldots
\]
\[
\begin{gathered}
(a_0 + \ldots + a_k x^k + \ldots) \cdot (b_0 \ldots + b_k x^k + \ldots) = (a_0 \cdot b_0) + (a_1 b_0 + a_0 b_1) x +\ldots \\ \ldots + (a_k b_0 + a_{k-1} b_1 + \ldots + a_0 b_k) x^k + \ldots
\end{gathered}
\]

Часто для краткости мы будем пропускать нулевые слагаемые и записывать многочлены в виде
\[
P(x) = a_0 + a_1 x + \ldots + a_n x^n.
\]
Множество многочленов с коэффициентами в $\mathbb{F}_p$ мы будем обозначать через $\mathbb{F}_p[x]$.

{\textit{Степенью}} многочлена $P(x) = a_0 + a_1 x + \ldots + a_k x^k + \ldots$ называется наибольшее целое $d$ такое, что $a_d \neq 0$. Будем обозначать ее через $\deg P(x)$. У нулевого многочлена степень не определена.

Многочлены $P(x) \in \mathbb{F}_p[x]$ можно вычислять на остатках. Иными словами, если $P(x) = a_0 + a_1 x + \ldots + a_n x^n \in \mathbb{F}_p[x]$ и $c \in \mathbb{F}_p$ — остаток, то $P(c) = a_0 + a_1 c + \ldots + a_n c^n \in \mathbb{F}_p$ — также остаток.

\begin{enumerate}[label*=\protect\fbox{\arabic{enumi}}]


\item Для многочленов $P(x), Q(x) \in \mathbb{F}_p[x]$ докажите, что 
\begin{enumerate}
	\item  $\deg(P(x) + Q(x)) \leq \max(\deg P(x), \deg Q(x))$; 
	\item  $\deg(P(x) \cdot Q(x)) = \deg P(x) + \deg Q(x)$.
\end{enumerate}

\item Пусть $P(x), Q(x) \in \mathbb{F}_p [x]$. Докажите по индукции по $\deg P (x)$, что многочлен $P(x)$ можно поделить на $Q(x)$ с остатком. А именно, что существуют многочлены $S(x), R(x) \in \mathbb{F}_p[x]$ такие, что $\deg R(x) < \deg Q(x)$ и $P(x) = Q(x)S(x) + R(x)$.

\item Поделите с остатком многочлен $P(x)$ на $Q(x)$ в случае
\begin{enumerate}
	\item  $P(x),Q(x) \in \mathbb{F}_{13}[x]: P(x) = x^7, Q(x) = x^2 - 1$
	\item  $P(x),Q(x) \in \mathbb{F}_{11}[x]: P(x) = x^3, Q(x) = 6x^2 + x + 1$
	\item $P(x),Q(x) \in \mathbb{F}_{7}[x]: P(x) = x^7 + 2x + 1, Q(x) = x - 3$
\end{enumerate}

\item \textbf{Теорема Безу.} Дан остаток $a \in \mathbb{F}_p$. Докажите, что многочлен $P(x) \in \mathbb{F}_p[x]$ даёт остаток $P(a)$ при делении на $x - a$.

\item Дан остаток $a \in \mathbb{F}_p$. Докажите, что многочлен $P(x) \in \mathbb{F}_p[x]$ делится на $x - a$ тогда и только тогда, когда $a$ является его корнем, то есть остаток $P (a)$ — нулевой.

\item
\begin{enumerate}
	\item  Пусть $a_1, \ldots, a_k$ — различные остатки. Докажите, что многочлен $P (x) \in \mathbb{F}_p[x]$ делится на произведение $(x - a_1) \cdot \ldots \cdot (x - a_k)$ тогда и только тогда, когда все $a_i$ являются корнями $P(x)$.
	\item Докажите, что у многочлена степени $n > 0$ над $\mathbb{F}_p$ не более $n$ различных корней.
\end{enumerate} 

\item  Разложите на множители многочлены: 
\begin{enumerate}
	\item $x^p - x \in \mathbb{F}_p[x]$;
	\item $x^{p} - 2 \in \mathbb{F}_p[x]$; 
	\item $1 + x + \ldots + x^{p-1} \in \mathbb{F}_p[x]$.
\end{enumerate}

\item \textbf{Теорема Виета.} Пусть различные остатки $a_1, \ldots, a_n$ — корни многочлена $b_n x^n + \ldots + b_1 x + b_0$. Докажите, что

$$a_1 + \ldots + a_n = - \frac{b_{n-1}}{b_n},$$
$$a_1 a_2 + a_1 a_3 + \ldots + a_{n-1} a_n = \frac{b_{n-2}}{b_n},$$
$$\vdots$$
$$a_1 a_2 \ldots a_n = (-1)^n \frac{b_0}{b_n}.$$

\item Петя выписал в тетрадку все наборы из трёх натуральных чисел $1 \leq k \leq p$. Затем он перемножил числа в каждой тройке, а результаты сложил. Какой остаток даёт получившееся число при делении на $p$?

\end{enumerate}
\end{document}