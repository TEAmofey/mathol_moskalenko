\documentclass{article}

\usepackage[12pt]{extsizes}
\usepackage[T2A]{fontenc}
\usepackage[utf8]{inputenc}
\usepackage[english, russian]{babel}

\usepackage{mathrsfs}
\usepackage[dvipsnames]{xcolor}

\usepackage{amsmath}
\usepackage{amssymb}
\usepackage{amsthm}
\usepackage{indentfirst}
\usepackage{amsfonts}
\usepackage{enumitem}
\usepackage{graphics}
\usepackage{tikz}
\usepackage{tabu}
\usepackage{diagbox}
\usepackage{hyperref}
\usepackage{mathtools}
\usepackage{ucs}
\usepackage{lipsum}
\usepackage{geometry} % Меняем поля страницы
\usepackage{fancyhdr} % Headers and footers
\newcommand{\range}{\mathrm{range}}
\newcommand{\dom}{\mathrm{dom}}
\newcommand{\N}{\mathbb{N}}
\newcommand{\R}{\mathbb{R}}
\newcommand{\E}{\mathbb{E}}
\newcommand{\D}{\mathbb{D}}
\newcommand{\M}{\mathcal{M}}
\newcommand{\Prime}{\mathbb{P}}
\newcommand{\A}{\mathbb{A}}
\newcommand{\Q}{\mathbb{Q}}
\newcommand{\Z}{\mathbb{Z}}
\newcommand{\F}{\mathbb{F}}
\newcommand{\CC}{\mathbb{C}}

\DeclarePairedDelimiter\abs{\lvert}{\rvert}
\DeclarePairedDelimiter\floor{\lfloor}{\rfloor}
\DeclarePairedDelimiter\ceil{\lceil}{\rceil}
\DeclarePairedDelimiter\lr{(}{)}
\DeclarePairedDelimiter\set{\{}{\}}
\DeclarePairedDelimiter\norm{\|}{\|}

\renewcommand{\labelenumi}{(\alph{enumi})}

\newcommand{\smallindent}{
    \geometry{left=1cm}% левое поле
    \geometry{right=1cm}% правое поле
    \geometry{top=1.5cm}% верхнее поле
    \geometry{bottom=1cm}% нижнее поле
}

\newcommand{\header}[3]{
    \pagestyle{fancy} % All pages have headers and footers
    \fancyhead{} % Blank out the default header
    \fancyfoot{} % Blank out the default footer
    \fancyhead[L]{#1}
    \fancyhead[C]{#2}
    \fancyhead[R]{#3}
}

\newcommand{\dividedinto}{
    \,\,\,\vdots\,\,\,
}

\newcommand{\littletaller}{\mathchoice{\vphantom{\big|}}{}{}{}}

\newcommand\restr[2]{{
    \left.\kern-\nulldelimiterspace % automatically resize the bar with \right
    #1 % the function
    \littletaller % pretend it's a little taller at normal size
    \right|_{#2} % this is the delimiter
}}

\DeclareGraphicsExtensions{.pdf,.png,.jpg}

\newenvironment{enumerate_boxed}[1][enumi]{\begin{enumerate}[label*=\protect\fbox{\arabic{#1}}]}{\end{enumerate}}



\smallindent

\header{Математика}{\textit{Алгебра и теория чисел}}{15 октября 2023}

%----------------------------------------------------------------------------------------

\begin{document}
    \large


    \begin{center}
        \textbf{Многочлены над полем $\F_p$}
    \end{center}

    Пусть $p$ — простое число.
    Обозначим через $\F_p$ множество (поле) остатков от деления на $p$.
    Через $0 \in \F_p$ будем обозначать нулевой остаток.
    Множество $\F_p$ состоит из $p$ элементов, которые можно умножать, складывать и вычитать.
    Более того, любой элемент $a \in \F_p$ можно поделить на любой $0 \neq b \in \F_p$.
    Сложение и умножение являются \textit{ассоциативными} и \textit{коммутативными} операциями, \textit{дистрибутивность} также выполняется.

    Многочленом $P(x)$ с коэффициентами в $\F_p$ назовем формальное выражение $P(x) = a_0 + a_1 x + \ldots + a_k x^k + \ldots$, где $x$ — формальная переменная, $a_0, \ldots, a_k, \ldots \in \F_p$ и только конечное число $a_i$ ненулевые.
    Многочлены можно складывать и умножать, как обычно:
    \begin{gather*}
    (a_0 + \ldots + a_k x^k + \ldots)
        \pm (b_0 + \ldots + b_k x^k + \ldots) = (a_0 \pm b_0) + (a_1 \pm b_1) x + \ldots + (a_k \pm b_k) x^k + \ldots\\
        \begin{gathered}
        (a_0 + \ldots + a_k x^k + \ldots)
            \cdot (b_0 \ldots + b_k x^k + \ldots) = (a_0 \cdot b_0) + (a_1 b_0 + a_0 b_1) x +\ldots \\ \ldots + (a_k b_0 + a_{k-1} b_1 + \ldots + a_0 b_k) x^k + \ldots
        \end{gathered}\\
    \end{gather*}

    Часто для краткости мы будем пропускать нулевые слагаемые и записывать многочлены в виде
    \[
        P(x) = a_0 + a_1 x + \ldots + a_n x^n.
    \]
    Множество многочленов с коэффициентами в $\F_p$ мы будем обозначать через $\F_p[x]$.

        {\textit{Степенью}} многочлена $P(x) = a_0 + a_1 x + \ldots + a_k x^k + \ldots$ называется наибольшее целое $d$ такое, что $a_d \neq 0$.
    Будем обозначать ее через $\deg P(x)$.
    У нулевого многочлена степень не определена.

    Многочлены $P(x) \in \F_p[x]$ можно вычислять на остатках.
    Иными словами, если $P(x) = a_0 + a_1 x + \ldots + a_n x^n \in \F_p[x]$ и $c \in \F_p$ — остаток, то $P(c) = a_0 + a_1 c + \ldots + a_n c^n \in \F_p$ — также остаток.

    \begin{enumerate_boxed}


        \item Для многочленов $P(x), Q(x) \in \F_p[x]$ докажите, что
        \begin{enumerate}
            \item  $\deg(P(x) + Q(x)) \leq \max(\deg P(x), \deg Q(x))$;
            \item  $\deg(P(x) \cdot Q(x)) = \deg P(x) + \deg Q(x)$.
        \end{enumerate}

        \item Пусть $P(x), Q(x) \in \F_p [x]$.
        Докажите по индукции по $\deg P (x)$, что многочлен $P(x)$ можно поделить на $Q(x)$ с остатком.
        А именно, что существуют многочлены $S(x), R(x) \in \F_p[x]$ такие, что $\deg R(x) < \deg Q(x)$ и $P(x) = Q(x)S(x) + R(x)$.

        \item Поделите с остатком многочлен $P(x)$ на $Q(x)$ в случае
        \begin{enumerate}
            \item  $P(x),Q(x) \in \F_{13}[x]: P(x) = x^7, Q(x) = x^2 - 1$
            \item  $P(x),Q(x) \in \F_{11}[x]: P(x) = x^3, Q(x) = 6x^2 + x + 1$
            \item $P(x),Q(x) \in \F_{7}[x]: P(x) = x^7 + 2x + 1, Q(x) = x - 3$
        \end{enumerate}

        \item \textbf{Теорема Безу.} Дан остаток $a \in \F_p$.
        Докажите, что многочлен $P(x) \in \F_p[x]$ даёт остаток $P(a)$ при делении на $x - a$.

        \item Дан остаток $a \in \F_p$.
        Докажите, что многочлен $P(x) \in \F_p[x]$ делится на $x - a$ тогда и только тогда, когда $a$ является его корнем, то есть остаток $P (a)$ — нулевой.

        \item
        \begin{enumerate}
            \item  Пусть $a_1, \ldots, a_k$ — различные остатки.
            Докажите, что многочлен $P (x) \in \F_p[x]$ делится на произведение $(x - a_1) \cdot \ldots \cdot (x - a_k)$ тогда и только тогда, когда все $a_i$ являются корнями $P(x)$.
            \item Докажите, что у многочлена степени $n > 0$ над $\F_p$ не более $n$ различных корней.
        \end{enumerate}

        \item  Разложите на множители многочлены:
        \begin{enumerate}
            \item $x^p - x \in \F_p[x]$;
            \item $x^{p} - 2 \in \F_p[x]$;
            \item $1 + x + \ldots + x^{p-1} \in \F_p[x]$.
        \end{enumerate}

        \item \textbf{Теорема Виета.} Пусть различные остатки $a_1, \ldots, a_n$ — корни многочлена $b_n x^n + \ldots + b_1 x + b_0$.
        Докажите, что

        \begin{eqnarray*}
            a_1 + \ldots + a_n &=& - \frac{b_{n-1}}{b_n},\\
            a_1 a_2 + a_1 a_3 + \ldots + a_{n-1} a_n &=& \frac{b_{n-2}}{b_n},\\
            &\vdots&\\
            a_1 a_2 \ldots a_n &=& (-1)^n \frac{b_0}{b_n}.\\
        \end{eqnarray*}

        \item Петя выписал в тетрадку все наборы из трёх натуральных чисел $1 \leq k \leq p$.
        Затем он перемножил числа в каждой тройке, а результаты сложил.
        Какой остаток даёт получившееся число при делении на $p$?

    \end{enumerate_boxed}
\end{document}