\documentclass{article}
\usepackage[12pt]{extsizes}
\usepackage[T2A]{fontenc}
\usepackage[utf8]{inputenc}
\usepackage[english, russian]{babel}

\usepackage{amssymb}
\usepackage{amsfonts}
\usepackage{amsmath}
\usepackage{enumitem}
\usepackage{graphics}

\usepackage{lipsum}



\usepackage{geometry} % Меняем поля страницы
\geometry{left=1cm}% левое поле
\geometry{right=1cm}% правое поле
\geometry{top=1.5cm}% верхнее поле
\geometry{bottom=1cm}% нижнее поле


\usepackage{fancyhdr} % Headers and footers
\pagestyle{fancy} % All pages have headers and footers
\fancyhead{} % Blank out the default header
\fancyfoot{} % Blank out the default footer
\fancyhead[L]{ЦРОД $\bullet$ Математика}
\fancyhead[C]{\textit{Геометрия}}
\fancyhead[R]{Май 2022}% Custom header text


%----------------------------------------------------------------------------------------

%\begin{document}\normalsize
\begin{document}\large


\begin{center}
\textbf{Решаем задачи с прошлых листиков!}
\end{center}


\begin{enumerate}[label*=\protect\fbox{\arabic{enumi}}]

\item Точка $H$ является ортоцентром остроугольного треугольника $ABC$ ($AB> AC$). Точка $E$ симметрична $C$ относительно высоты $AH$. Обозначим за $F$ точку пересечения прямых $EH$ и $AC$. Докажите, что центр описанной окружности треугольника $AEF$ лежит на прямой $AB$.

\item В остроугольном треугольнике угол $A$ равен $60^\circ$. Докажите, что прямая, соединяющая центр описанной окружности с ортоцентром, отсекает от треугольника равносторонний треугольник.

\item Пусть $H'$ — проекция ортоцентра на касательную в точке $A$ к описанной окружности треугольника $ABC$. Докажите, что середина стороны $BC$ равноудалена от точек $A$ и $H'$.

\item Остроугольный треугольник $ABC$ ($AB < AC$) вписан в окружность $\Omega$. Пусть $M$ — точка пересечения его медиан, а $AH$ — высота этого треугольника. Луч $MH$ пересекает $\Omega$ в точке $A'$. Докажите, что окружность, описанная около треугольника $A'HB$,
касается $AB$.

\item Окружность $\omega$ касается сторон угла $BAC$ в точках $B$ и $C$. Прямая $l$ пересекает отрезки $AB$ и $AC$ в точках $K$ и $L$ соответственно. Окружность $\omega$ пересекает $l$ в точках $P$ и $Q$. Точки $S$ и $T$ выбраны на отрезке $BC$ так, что $KS \parallel AC$ и $LT \parallel AB$.
Докажите, что точки $P, Q, S$ и $T$ лежат на одной окружности


\end{enumerate}
\end{document}