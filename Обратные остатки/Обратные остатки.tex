\documentclass{article}

\usepackage[12pt]{extsizes}
\usepackage[T2A]{fontenc}
\usepackage[utf8]{inputenc}
\usepackage[english, russian]{babel}

\usepackage{mathrsfs}
\usepackage[dvipsnames]{xcolor}

\usepackage{amsmath}
\usepackage{amssymb}
\usepackage{amsthm}
\usepackage{indentfirst}
\usepackage{amsfonts}
\usepackage{enumitem}
\usepackage{graphics}
\usepackage{tikz}
\usepackage{tabu}
\usepackage{diagbox}
\usepackage{hyperref}
\usepackage{mathtools}
\usepackage{ucs}
\usepackage{lipsum}
\usepackage{geometry} % Меняем поля страницы
\usepackage{fancyhdr} % Headers and footers
\newcommand{\range}{\mathrm{range}}
\newcommand{\dom}{\mathrm{dom}}
\newcommand{\N}{\mathbb{N}}
\newcommand{\R}{\mathbb{R}}
\newcommand{\E}{\mathbb{E}}
\newcommand{\D}{\mathbb{D}}
\newcommand{\M}{\mathcal{M}}
\newcommand{\Prime}{\mathbb{P}}
\newcommand{\A}{\mathbb{A}}
\newcommand{\Q}{\mathbb{Q}}
\newcommand{\Z}{\mathbb{Z}}
\newcommand{\F}{\mathbb{F}}
\newcommand{\CC}{\mathbb{C}}

\DeclarePairedDelimiter\abs{\lvert}{\rvert}
\DeclarePairedDelimiter\floor{\lfloor}{\rfloor}
\DeclarePairedDelimiter\ceil{\lceil}{\rceil}
\DeclarePairedDelimiter\lr{(}{)}
\DeclarePairedDelimiter\set{\{}{\}}
\DeclarePairedDelimiter\norm{\|}{\|}

\renewcommand{\labelenumi}{(\alph{enumi})}

\newcommand{\smallindent}{
    \geometry{left=1cm}% левое поле
    \geometry{right=1cm}% правое поле
    \geometry{top=1.5cm}% верхнее поле
    \geometry{bottom=1cm}% нижнее поле
}

\newcommand{\header}[3]{
    \pagestyle{fancy} % All pages have headers and footers
    \fancyhead{} % Blank out the default header
    \fancyfoot{} % Blank out the default footer
    \fancyhead[L]{#1}
    \fancyhead[C]{#2}
    \fancyhead[R]{#3}
}

\newcommand{\dividedinto}{
    \,\,\,\vdots\,\,\,
}

\newcommand{\littletaller}{\mathchoice{\vphantom{\big|}}{}{}{}}

\newcommand\restr[2]{{
    \left.\kern-\nulldelimiterspace % automatically resize the bar with \right
    #1 % the function
    \littletaller % pretend it's a little taller at normal size
    \right|_{#2} % this is the delimiter
}}

\DeclareGraphicsExtensions{.pdf,.png,.jpg}

\newenvironment{enumerate_boxed}[1][enumi]{\begin{enumerate}[label*=\protect\fbox{\arabic{#1}}]}{\end{enumerate}}



\smallindent

\header{ЦРОД $\bullet$ Математика}{\textit{Теория чисел}}{ЦРОД 2022}

%----------------------------------------------------------------------------------------

\begin{document}
    \large

    \begin{center}
        \textbf{Обратный остаток}
    \end{center}

    Обратное по модулю целого $a$ --- это такое целое число $x$, что произведение $ax$ сравнимо с 1 по модулю $m$.

    \textbf{Теорема.} Если $(a,m) = 1$, то у $a$ есть обратный остаток по модулю $m$.


    \begin{enumerate_boxed}

        \item Дано простое число $p$ и его некоторый ненулевой остаток $a$.
        \begin{enumerate}
            \item Докажите, что в последовательности $0 \cdot a, 1 \cdot a, 2 \cdot a, \dots ,(p - 1) \cdot a$ все числа дают разные остатки по модулю $p$.

            \item Докажите, что существует и при том единственный обратный остаток $b$

            \item Какие остатки совпадают со своими обратными остатками?
        \end{enumerate}
% 2 7
% 3 9
% 4 10
% 5 8
% 6 11
        \item Какой остаток даёт $x$ при делении $13$, если:
        \begin{enumerate}
            \item $3x \equiv 4 + x$ (mod $13$);

            \item $7x \equiv 8 + 3x$ (mod $13$);

            \item $10x + 2 \equiv -x$ (mod $13$);

            \item $6 \equiv 11 + 3x$ (mod $13$);

            \item $-2x \equiv 1 + 3x$ (mod $13$);

        \end{enumerate}
        \item \textbf{(Теорема Вильсона.)} Докажите, что $(p-1)!\equiv -1\pmod p$, если и только если $p$ является простым числом.

        \item Пусть $p$ -- простое число и $k \leqslant p$.
        Докажите, что $(p - k)!(k - 1)! \underset{p}{\equiv} (-1)^k$.

        \item Пусть $p \geqslant 3$~--- простое число.
        Докажите, что если сумму $\frac{1}{1} + \frac{1}{2} + \ldots+\frac{1}{p-1}$ привести к общему знаменателю, то числитель получившейся дроби будет делиться на $p$.

        \item Пусть числа $p$ и $p + 2$ являются простыми числами-близнецами.
        Докажите,
        что справедливо $4((p - 1)! + 1) + p \equiv 0 \pmod{p^2+2p}$.

        \item  На доске написаны числа $\frac{100}{1}, \frac{99}{2}, \ldots, \frac{1}{100}$.
        Можно ли выбрать какие-то девять из них, произведение которых равняется единице?

        \item Докажите, что $5^{70} + 6^{70}$ делится на $61$.
        
    \end{enumerate_boxed}
\end{document}