\documentclass{article}

\usepackage[12pt]{extsizes}
\usepackage[T2A]{fontenc}
\usepackage[utf8]{inputenc}
\usepackage[english, russian]{babel}

\usepackage{mathrsfs}
\usepackage[dvipsnames]{xcolor}

\usepackage{amsmath}
\usepackage{amssymb}
\usepackage{amsthm}
\usepackage{indentfirst}
\usepackage{amsfonts}
\usepackage{enumitem}
\usepackage{graphics}
\usepackage{tikz}
\usepackage{tabu}
\usepackage{diagbox}
\usepackage{hyperref}
\usepackage{mathtools}
\usepackage{ucs}
\usepackage{lipsum}
\usepackage{geometry} % Меняем поля страницы
\usepackage{fancyhdr} % Headers and footers
\newcommand{\range}{\mathrm{range}}
\newcommand{\dom}{\mathrm{dom}}
\newcommand{\N}{\mathbb{N}}
\newcommand{\R}{\mathbb{R}}
\newcommand{\E}{\mathbb{E}}
\newcommand{\D}{\mathbb{D}}
\newcommand{\M}{\mathcal{M}}
\newcommand{\Prime}{\mathbb{P}}
\newcommand{\A}{\mathbb{A}}
\newcommand{\Q}{\mathbb{Q}}
\newcommand{\Z}{\mathbb{Z}}
\newcommand{\F}{\mathbb{F}}
\newcommand{\CC}{\mathbb{C}}

\DeclarePairedDelimiter\abs{\lvert}{\rvert}
\DeclarePairedDelimiter\floor{\lfloor}{\rfloor}
\DeclarePairedDelimiter\ceil{\lceil}{\rceil}
\DeclarePairedDelimiter\lr{(}{)}
\DeclarePairedDelimiter\set{\{}{\}}
\DeclarePairedDelimiter\norm{\|}{\|}

\renewcommand{\labelenumi}{(\alph{enumi})}

\newcommand{\smallindent}{
    \geometry{left=1cm}% левое поле
    \geometry{right=1cm}% правое поле
    \geometry{top=1.5cm}% верхнее поле
    \geometry{bottom=1cm}% нижнее поле
}

\newcommand{\header}[3]{
    \pagestyle{fancy} % All pages have headers and footers
    \fancyhead{} % Blank out the default header
    \fancyfoot{} % Blank out the default footer
    \fancyhead[L]{#1}
    \fancyhead[C]{#2}
    \fancyhead[R]{#3}
}

\newcommand{\dividedinto}{
    \,\,\,\vdots\,\,\,
}

\newcommand{\littletaller}{\mathchoice{\vphantom{\big|}}{}{}{}}

\newcommand\restr[2]{{
    \left.\kern-\nulldelimiterspace % automatically resize the bar with \right
    #1 % the function
    \littletaller % pretend it's a little taller at normal size
    \right|_{#2} % this is the delimiter
}}

\DeclareGraphicsExtensions{.pdf,.png,.jpg}

\newenvironment{enumerate_boxed}[1][enumi]{\begin{enumerate}[label*=\protect\fbox{\arabic{#1}}]}{\end{enumerate}}



\smallindent

\header{Математика}{\textit{Теория чисел}}{9 марта 2023}

%----------------------------------------------------------------------------------------

\begin{document}
    \large

    \begin{center}
        \textbf{Степени вхождения простых чисел}
    \end{center}

    \textbf{Определение:} Степенью вхождения простого числа $p$ в натуральное число $n$ будем называть наибольшее такое $k$, что $n$ делится на $p^k$.
    Обозначать для краткости будем $\nu_p(n)$ (это греческая буква “ню”)

    \textbf{Факт:}  $\nu_p(a + b) \ge \min\{\nu_p(a),\nu_p(b)\}$, причём если $\nu_p(a) \neq \nu_p(b)$, то $\nu_p(a + b) = \min\{\nu_p (a), \nu_p (b)\}$.

    \begin{enumerate_boxed}

        \item Докажите \textbf{формулу Лежандра}: $\nu_p(n!)=\left\lfloor \dfrac{n}{p}\right\rfloor+\left\lfloor \dfrac{n}{p^2} \right\rfloor+\left\lfloor \dfrac{n}{p^3} \right\rfloor+\dotsc$

        \item Натуральные числа $a$ и $b$ таковы, что сумма $\dfrac{b^2}{a} + \dfrac{a^2}{b}$ целая.
        Докажите, что оба слагаемых целые.

        \item Взаимно простые в совокупности натуральные числа $a, b, c$ удовлетворяют условию $ab = ac + bc$.
        Докажите, что $abc$ — точный квадрат.

        \item Натуральные числа $a, b, c$ таковы, что число $\dfrac{a}{b} + \dfrac{b}{c} + \dfrac{a}{c}$ является целым.
        Верно ли, что $abc$ — точный куб?

        \item Докажите, что если числа $ab, cd$ и $ac + bd$ делятся на $k$ то $ac$ и $bd$ делятся на $k$.

        \item Натуральные числа $m, n$ таковы, что $m^2 + n^2 + m$ кратно $mn$. Докажите, что $m$ — квадрат натурального числа.

        \item Докажите, что наименьшее общее кратное чисел от $n$ до $2n + 1$ делится на $\dfrac{(2n + 1)!}{n!\cdot n!}$.

        \item Докажите, что не существует трёх различных натуральных чисел, каждое из которых равно наименьшему общему кратному своих разностей с двумя другими.

        \item Даны различные натуральные числа $a_1 , a_2 , \dotsc , a_n$.
        Положим
        \[b_i = (a_i - a_1)(a_i - a_2)\dotsc(a_i - a_{i-1})(a_i - a_{i+1})\dotsc(a_i - a_n).\]
        Докажите, что наименьшее общее кратное $[b_1 , b_2 , \dotsc , b_n ]$ делится на $(n - 1)!$.

        \item Найдите все натуральные $x, y$ и простые $p$ такие что:
        \[x^5 + y^4 = pxy\]

    \end{enumerate_boxed}
\end{document}