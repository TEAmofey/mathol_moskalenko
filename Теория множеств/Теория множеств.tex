\documentclass{article}
\usepackage[12pt]{extsizes}
\usepackage[T2A]{fontenc}
\usepackage[utf8]{inputenc}
\usepackage[english, russian]{babel}

\usepackage{amssymb}
\usepackage{amsfonts}
\usepackage{amsmath}
\usepackage{enumitem}
\usepackage{graphics}
\usepackage{amsthm}
\usepackage{mathrsfs}
\usepackage[dvipsnames]{xcolor}

\usepackage{lipsum}

\usepackage[framemethod=TikZ]{mdframed}

\newcommand{\definebox}[3]{%
	\newcounter{#1}
	\newenvironment{#1}[1][]{%
		\stepcounter{#1}%
		\mdfsetup{%
			frametitle={%
				\tikz[baseline=(current bounding box.east),outer sep=0pt]
				\node[anchor=east,rectangle,fill=white]
				{\strut #2~\csname the#1\endcsname\ifstrempty{##1}{}{##1}};}}%
		\mdfsetup{innertopmargin=1pt,linecolor=#3,%
			linewidth=3pt,topline=true,
			frametitleaboveskip=\dimexpr-\ht\strutbox\relax,}%
		\begin{mdframed}[]\relax%
		}{\end{mdframed}}%
}

\newtheorem{lemma}{Лемма}
\newtheorem{statement}{Утверждение}
\newtheorem{corollary}{Следствие}
%\newtheorem{definition}{Определение}
%\newtheorem{task}{Задача}
%\newtheorem{paradox}{Парадокс}


\newcommand{\range}{\mathrm{range}}
\newcommand{\dom}{\mathrm{dom}}
\newcommand{\N}{\mathbb{N}}
\newcommand{\R}{\mathbb{R}}
\newcommand{\Prime}{\mathbb{P}}
\newcommand{\A}{\mathbb{A}}
\newcommand{\Q}{\mathbb{Q}}
\newcommand{\Z}{\mathbb{Z}}

\theoremstyle{remark}
\newtheorem{example}{Пример}

\definebox{theorem}{Теорема}{ForestGreen!24}
\definebox{definition}{Определение}{blue!24}
\definebox{task}{Задача}{orange!24}
\definebox{paradox}{Парадокс}{red!24}

\renewcommand{\labelenumi}{(\alph{enumi})}

\usepackage{geometry} % Меняем поля страницы
\geometry{left=1cm}% левое поле
\geometry{right=1cm}% правое поле
\geometry{top=1.5cm}% верхнее поле
\geometry{bottom=1cm}% нижнее поле


\usepackage{fancyhdr} % Headers and footers
\pagestyle{fancy} % All pages have headers and footers
\fancyhead{} % Blank out the default header
\fancyfoot{} % Blank out the default footer
\fancyhead[L]{Математика}
\fancyhead[C]{Теория множеств}
\fancyhead[R]{}% Custom header text


%----------------------------------------------------------------------------------------

%\begin{document}\normalsize
\begin{document}\large


\setcounter{task}{0}

\begin{task}
	Обязательно ли старейший математик среди шахматистов и старейший шахматист среди математиков один и тот же человек?
\end{task}

\begin{task}
	Обязательно ли лучший математик среди шахматистов и лучший шахматист среди математиков один и тот же человек?
\end{task}

\begin{definition}
	\textit{Множество} - неупорядоченная совокупность элементов.
\end{definition}

\begin{example}
$\{1,2,3\} = \{3,2,1\} = \{1,1,3,2,3,3\}$
\end{example}

\begin{example}
	 $\{\} =\varnothing$
 \end{example}

\begin{example}
	$\{a,\{b,c\},\{\{d\},e\},\varnothing\}$
\end{example}

\begin{definition}
	$a \in A$ - означает, что во множестве $A$ есть элемент $a$
\end{definition}

\begin{definition}
	$B \subseteq A$ - означает, что если $b \in B$, то $b \in A$.
\end{definition}

\begin{example}
	$1\in\{1,2,3\}; \;\;\;\;\; \{1,2\}\subseteq\{1,2,3\}$.
\end{example}

\begin{example}
	$1\notin\varnothing$.
\end{example}

\begin{example}
	$\{b,c\}\in \{a,\{b,c\},\{\{d\},e\},\varnothing\}$.
\end{example}

\begin{task}
	Какие из выражений верны для произвольного $x$?\\
	\\
	\begin{minipage}[c]{0.3\textwidth}
		\begin{itemize}
			\item $x \in \{x\}$
			\item $\{x\} \subseteq \{x\}$
			\item $\{x\} \in \{x\}$
		\end{itemize}
	\end{minipage}
	\begin{minipage}[c]{0.3\textwidth}
		\begin{itemize}
			\item $\{x,x\} \subseteq \{x\}$
			\item $\{x\} \in \{\{x\}\}$
			\item $\varnothing \in \varnothing$
		\end{itemize}
	\end{minipage}
	\begin{minipage}[c]{0.3\textwidth}
		\begin{itemize}
			\item $\varnothing \in \{x\}$
			\item $\varnothing \subseteq \{x\}$
			\item $\varnothing \subseteq \varnothing$
		\end{itemize}
	\end{minipage}
\end{task}

\begin{definition}
	Предикат от $x$ --- это такая функция $\varphi(x)$, которая принимает два возможных значения: либо истина, либо ложь (0 или 1)
\end{definition}

\begin{example}
	$\varphi(x) = (x = 239)$~--- принимает истинное значение только, если $x = 239$
\end{example}


\begin{definition}
	Чтобы работать с предикатами, нужно познакомиться с \textit{логическими связками}:
	\begin{itemize}
		\item $\neg$~--- \textbf{\textit{отрицание}} \texttt{"не ..."},
	
		\item $\wedge$~--- \textbf{\textit{конъюнкция}}  \texttt{"... и ..."},
	
		\item $\vee$~--- \textbf{\textit{дизъюнкция}}  \texttt{"... или ..."},
	
		\item $\rightarrow$~--- \textbf{\textit{импликация}}  \texttt{"если ..., то ..."},
	
		\item $\leftrightarrow$~--- \textbf{\textit{эквивалентность}}  \texttt{"... тогда и только тогда, когда ..."};
	\end{itemize}
	и {\textit{кванторами}}:
	\begin{itemize}
		\item $\forall x$~--- \textbf{\textit{квантор всеобщности}}  \texttt{"для любого х верно ..."}
	
		\item $\exists x$~--- \textbf{\textit{квантор существования}}  \texttt{"существует х, такой что верно ..."}
	\end{itemize}
\end{definition}

\begin{example}
	$\varphi(x) = (\forall z \exists y: z + y = x)$~--- \texttt{"для любого z найдется y, такие что они в сумме дают x"}. Это утверждение всегда верно, если мы рассматриваем предикат на множестве вещественных чисел.
\end{example}

\begin{example}
	А вот записано утверждение \texttt{"множества равны тогда и только тогда, когда у них совпадает набор элементов"}
	$\forall X\forall Y(\forall u(u \in X \leftrightarrow u \in Y) \rightarrow X = Y)$
\end{example}

\begin{definition}
	$A = \underbrace{\{}_{\text{множество}}\underbrace{x}_{\text{всех иксов}} \underbrace{\mid}_{\text{таких, что}} \underbrace{\varphi(x)}_{\text{выполнено условие} \ \varphi(x)}\}$, где $\varphi(x)$ предикат от $x$.

\end{definition}


\begin{paradox}[ (Рассел)]
	Рассмотрим множество $A = \{x\mid x \notin x\}$. То есть множество всех множеств, которые не содержат себя в качестве своего собственного элемента. Что тогда можно сказать про утверждение $A \in A$?
\end{paradox}

\begin{definition}
	Пусть даны множества $A$ и $B$. Тогда их \textit{\textbf{пересечением}} называется множество: $$A \cap B = \{x \mid x\in A \wedge x \in B\}.$$
	Пусть дано семейство множеств $\{M_{\alpha}\}_{\alpha \in A}.$ Тогда его пересечением называется \textbf{\textit{множество}}, состоящее из элементов, которые входят во все множества семейства: $$\bigcap\limits_{\alpha \in A} M_{\alpha} = \{x \mid \forall \alpha \in A,\; x \in M_{\alpha}\}.$$
\end{definition}

\begin{example}
	$\{1,2,3\}\cap \{3,4,5\} = \{3\}$
\end{example}

\begin{example}
	$x \cap \varnothing = \varnothing$
\end{example}

\begin{example}
	$\bigcap \{\{1,2,3,4,5\}, \{2,3,4,9\}, \{0,2,3\}\} = \{2,3\}$
\end{example}

\begin{definition}
	Пусть даны два множества $A$ и $B$. Тогда их \textbf{\textit{объединением}} называется множество	
	$$A \cup B = \{ x \mid x\in A \vee x\in B\}.$$
	Пусть дано семейство множеств $\{M_{\alpha}\}_{\alpha \in A}.$ Тогда его объединением называется множество, состоящее из всех элементов всех множеств семейства: $$\bigcup\limits_{\alpha \in A} M_{\alpha} = \{x \mid \exists \alpha \in A,\; x \in M_{\alpha}\}.$$
\end{definition}

\begin{example}
	$\{1,2,3\}\cup \{3,4,5\} = \{1,2,3,4,5\}$
\end{example}

\begin{example}
	$x \cup \varnothing = x$
\end{example}

\begin{example}
	$\bigcup \{\{1,2,3,4,5\}, \{2,3,4,9\}, \{0,2,3\}\} = \{0,1,2,3,4,5,9\}$
\end{example}

\begin{definition}
	Пусть даны два множества $A$ и $B$. Тогда их \textbf{\textit{разностью}} называется множество	
	$$A \setminus B =\{x\in A\mid x\not\in B\}.$$
\end{definition}

\begin{example}
	$\{1,2,3\} \setminus \{3,4,5\} = \{1,2\}$
\end{example}
\begin{example}
	$\{3,4,5\} \setminus \{1,2,3\} = \{4,5\}$
\end{example}

\begin{definition}
Пусть даны два множества $A$ и $B$. Тогда их \textbf{\textit{симметрической разностью}} называется множество
$$A \bigtriangleup B =\left( A \setminus B \right) \cup \left ( B \setminus A \right) = \left(A \cup B\right) \setminus \left(A \cap B\right).$$

\end{definition}

\begin{example}
$\{1,2,3\} \bigtriangleup \{3,4,5\} = \{1,2,4,5\}$
\end{example}

\begin{definition}
	Равенство называется \textit{\textbf{тождественно верным}}, или \textit{\textbf{тождеством}}, если оно истинно для любых значений входящих в него переменных.
\end{definition}

\begin{task} 
	Какие из равенств тождественно верны для множеств $X,Y,Z$? Приведите контрпримеры к неверным тождествам.
	\newline
	
	\begin{minipage}[c]{0.5\textwidth}
		\begin{itemize}
			\item $(X \cup Y) \cup Z = X \cup (Y \cup Z)$
			\item $X \cup Y = Y \cup X$
			\item $X \cup (Y \cap Z) = (X \cup Y) \cap (X \cup Z)$
			\item $X \setminus (Y \cap Z) = (X \setminus Y) \cup (X \setminus Z)$
			\item $(X \cup Y) \setminus Z = (X \setminus Z) \cup Y$
			\item $X \setminus (X \setminus Y) = X \cap Y$
		\end{itemize}
	\end{minipage}
	\begin{minipage}[c]{0.5\textwidth}
		\begin{itemize}
			\item $(X \cap Y) \cap Z = X \cap (Y \cap Z)$
			\item $X \cap Y = Y \cap X$
			\item $X \cup (Y \cap Z) = (X \cup Y) \cap (X \cup Z)$
			\item $X \setminus (Y \cup Z) = (X \setminus Y) \cap (X \setminus Z)$
			\item $(X \cap Y) \setminus Z = (X \setminus Z) \cap Y$
			\item $(X \bigtriangleup Y) \bigtriangleup Z = X \bigtriangleup (Y \bigtriangleup Z)$
		\end{itemize}
	\end{minipage}
\end{task}
	
\begin{task}
	Докажите, что если какое-то равенство, содержащее переменные для множеств и операции $\cap, \cup, \setminus,$ не является тождеством, то можно найти контрпример к нему, в котором множества пусты или состоят из одного элемента.
\end{task}

\begin{task}[*]
Сколько различных выражений для множеств можно составить из переменных $A$  и $B$ с помощью операций пересечения, объединения и разности? Из $n$ переменных? (Два выражения считаются одинаковыми, если они тождественно равны.)
\end{task}


\begin{definition}
	Определим \textit{\textbf{упорядоченную}} пару $X_1$ и $X_2$ как $(X_1, X_2) := \{\{X_1\} , \{X_1, X_2\}\}$
\end{definition}

\begin{example}
	$(1,1) = \{\{1\}, \{1,1\}\} = \{\{1\}\}.$
\end{example}

\begin{definition}
	Пусть даны два множества $X$ и $Y$. Тогда их \textbf{\textit{декартовым произведением}} называется множество упорядоченных пар
	$$X\times Y := \{(x,y) \mid x\in X \wedge y\in Y\}$$
	
	Будем обозначать $A^n := \underbrace{A\times A \times \dotsc \times A}_{n \text{ раз}}.$
\end{definition}

\begin{example}
$\{a,b\}\times \{0,1,2\} = \{(a,0),(a,1),(a,2),(b,0),(b,1),(b,2)\}.$
\end{example}

\begin{example}
$\{a,b\}^2 = \{(a,a),(a,b),(b,a),(b,b)\}.$
\end{example}

\begin{example}
$\{1\}^6 = \{(1,1,1,1,1,1)\}.$
\end{example}

\begin{definition}
Под \textit{\textbf{бинарными отношениями}} между $X$ и $Y$ мы будем понимать произвольные подможества $X \times Y$ . В частности, при $X = Y$ мы будем называть их еще \textit{бинарными отношениями на $X$}.
Для удобства будем писать $xRy$ вместо $(x, y) \in R$.
\end{definition}

\begin{example}
\textbf{\textit{Тождественное}} отношение на $X$ $$id_x:= \{(x,x) \mid x\in X\}= \{(x,y)\in  X^2 \mid x=y\}.$$
\end{example}

\begin{definition}
Бинарное отношение $R \subseteq X \times X$ на $X$ будем называть: 
\begin{itemize}

	\item \textbf{\textit{рефлексивным}}, если $\forall x(x \in X \rightarrow xRx)$;
	\item \textbf{\textit{иррефлексивным}}, если $\neg\exists x(x \in X \wedge xRx)$;
	\item \textbf{\textit{транзитивным}}, если $\forall x\forall y\forall z((xRy \wedge yRz) \rightarrow xRz)$; 
	\item \textbf{\textit{симметричным}}, если $\forall x\forall y(xRy \rightarrow yRx)$;
	\item \textbf{\textit{антисимметричным}}, если $\forall x\forall y((xRy \wedge yRx) \rightarrow x = y)$.
\end{itemize}
\end{definition}

\begin{example}
	Отношение \texttt{"$=$"} на вещественных числах является \textit{рефлексивным}.
\end{example}

\begin{example}
	Отношение \texttt{"$>$"} на вещественных числах является \textit{иррефлексивным}.
\end{example}

\begin{example}
Отношение параллельности на множестве прямых на плоскости является \textit{транзитивным} и симметричным.
\end{example}

\begin{example}
Отношение \texttt{"$\leqslant$"} на вещественных числах является \textit{антисимметричным}.
\end{example}

\begin{definition}
Будем говорить, что отношение $R$ является:
\begin{itemize}
\item \textbf{\textit{предпорядком}} на $X$, если $R$ \textit{рефлексивно} и \textit{транзитивно};
\item \textbf{\textit{строгим частичным порядком}} на $X$, если $R$ \textit{иррефлексивно} и \textit{транзитивно};
\item \textbf{\textit{частичным порядком}} на $X$, если $R$ \textit{рефлексивно}, \textit{антисимметрично} и \textit{транзитивно}; 
\item \textbf{\textit{эквивалентностью}} на $X$, если $R$ \textit{рефлексивно}, \textit{симметрично} и \textit{транзитивно}.
\end{itemize}

\end{definition}

\begin{example}
	Отношение делимости на натуральных числах является \textit{предпорядком}.
\end{example}

\begin{example}
	Отношение \texttt{"$>$"} (строго больше) на вещественных числах является \textit{строгим частичным порядком}.
\end{example}

\begin{example}
	Отношение \texttt{"$\geqslant$"} (нестрого больше) на вещественных числах является 	\textit{частичным порядком}.
\end{example}


\begin{example}
	Отношение параллельности на множестве прямых на плоскости является \textit{эквивалентностью}.
\end{example}

\begin{definition}
	Пусть $\approx$~--- \textit{эквивалентность} на $X$. Для каждого $x \in X$ под \textit{\textbf{классом эквивалентности}} $x$ по $\approx$ понимается множество
	$[x]_\approx :=\{u\in X \mid x\approx u\}$
\end{definition}

\begin{definition}
	Будем называть $Y$ (взаимно, или попарно) \textit{\textbf{дизъюнктным}}, если оно удовлетворяет условию $\forall u\forall v((u\in Y \wedge v\in Y \wedge u\neq v)\rightarrow u \cap v=\varnothing)$\\
%\end{definition}
%
%\begin{definition}
	Будем говорить, что $Y$ является \textit{\textbf{разбиением}} $X$ , если $Y$ \textit{дизъюнктно}, $\varnothing \notin Y$ и $\bigcup Y = X$.
\end{definition}


\begin{task}
	Показать:\\
	(a) Если $Y$~--- \textit{разбиение} $X$, то
	$\mathscr{E}_Y :=\{(u,v)\in X^2 \mid \exists y(y\in Y \wedge u\in y\wedge v\in y)\}$~--- \textit{эквивалентность} на $X$, причём $X_{/\mathscr{E}_Y}$ равно $Y$.\\
	(b) Если $\approx$~--- \textit{эквивалентность} на $X$, то $X_{/\approx}$~--- \textit{разбиение} $X$, причём $\mathscr{E}_{X_{/\approx}}$ равно $\approx$.
	
\end{task}
	
\begin{task}
Пусть $R$~--- \textit{предпорядок} на $X$. Тогда\\
(a) $\mathscr{S}_R := \{(u, v) \in X^2 \mid  uRv \wedge vRu\}$~--- \textit{эквивалентность} на $X$;\\
(b) $R^\# :=\{([u]_{/\mathscr{S}_R},[v]_{/\mathscr{S}_R})\mid u\in X\wedge v\in X\wedge uRv\}$~--- \textit{частичный порядок }на $X_{/\mathscr{S}_R}.$ 
\end{task} 

\begin{task}
Доказать:\\
(a) Если $R$~--- \textit{строгий частичный порядок} на $X$ , то $R \cup id_X$~--- \textit{частичный порядок} на $X$; \\
(b) Если $R$~--- \textit{частичный порядок} на $X$, то $R \setminus id_X$~--- \textit{строгий частичный порядок} на $X$.
\end{task}

\begin{definition}
	Множество
	$\mathrm{dom}(R):=\{u\in X\mid\exists v: uRv\}$,
	называют \textit{\textbf{областью определения}} $R$.
\end{definition}

\begin{definition}
	Множество
	$\mathrm{range}(R) := \{v\in Y\mid\exists u: uRv\}$,
	называют и \textit{\textbf{областью значений}} $R$.
\end{definition}

\begin{definition}
	Для каждого $U \subseteq X$ множество
	$$R[U] := \mathrm{range}(R\cap U\times Y) = \{v \in Y  \mid \exists u(u\in U\wedge uRv)\}$$ называется \textit{\textbf{образом}} $U$ относительно $R$.
\end{definition}

\begin{example}
	Рассмотрим \textit{строгий частичный порядок} \texttt{"$<$"} на $\{0,1,2,3,4,5,6,7,8,9\}.$\\
	Тогда $<[\{4,5,8\}] = \{5,6,7,8,9\}.$
\end{example}

\begin{definition}
	\textit{\textbf{Обратное отношение}} к $R$ определяется как
	$R^{-1} := \{(y,x) \mid (x,y) \in R\}$
\end{definition}

\begin{example}
Рассмотрим \textit{частичный порядок} \texttt{"$\geqslant$"} на $\{0,1,2,3,4,5,6,7,8,9\}.$\\
Тогда $\geqslant^{-1}$ равно $\leqslant.$
\end{example}

\begin{definition}
	Для каждого $V \subseteq Y$ \textit{образ} $V$ под действием $R^{-1}$ называется \textbf{\textit{прообразом}} $V$ относительно $R$.
\end{definition}

\begin{example}
	$\range(R) = \dom(R^{-1}) = R [X].$
\end{example}
\begin{example}
	$\range(R^{-1}) =\dom(R) = R^{-1}[Y].$
\end{example}

\begin{definition}
	Бинарные отношения можно естественным образом комбинировать: \\
	для любых $R \subseteq X \times Y$ и $Q \subseteq Y \times Z$ множество
	$$R\circ Q := \{(x,z) \in X\times Z\mid \exists y(xRy \wedge yQz)\}$$
	называется \textbf{\textit{композицией}} $R$ и $Q$.
\end{definition}

\begin{definition}
	Говорят, что $R \subseteq X \times Y$ \textbf{\textit{функционально}}, если
	$$\forall x \forall y_1 \forall y_2 ((xRy_1 \wedge xRy_2) \rightarrow y_1 = y_2).$$
	Далее, $R$ называют \textbf{\textit{функцией}} из $X$ в $Y$ , и пишут $R : X \rightarrow Y$, если $\dom(R) = X$ и $R$ \textit{функционально}.
\end{definition}

\begin{definition}
	Пусть $f : X \rightarrow Y$. Значит, для любого $x \in X$ имеется единственное $y \in Y$ такое, что $(x , y) \in f$, которое называется \textbf{\textit{значением}} $f$ в $ x $ и обозначается через $ f (x) $. 
\end{definition}

\begin{example}
	$\range(f) = \{f(x)\mid x\in X\}.$
\end{example}


\begin{definition}
	Для каждого $U \subseteq X$ \textbf{\textit{ограничение}} (или \textit{\textbf{сужение}}) $f$ на $U$ определяется как
	$$f|_U := f \cap U \times Y.$$
	$f|_U$ будет функцией из $U$ в $Y$. Вообще, если $ f : X \rightarrow Y$ и $g : U \rightarrow Y $ таковы, что $ U \subseteq X  $ и $ f|_U= g $, то $g$ называют \textit{\textbf{ограничением}} $ f $, a $ f $ \textit{\textbf{расширением}} $ g $. 
\end{definition}

\begin{definition}
	Обозначим $Y^X :=\{f\mid f:X\rightarrow Y\}.$
Под \textit{двухместными}, \textit{трехместными} и так далее функциями из $X$ в $Y$ понимают элементы $Y^{X^2}$, $Y^{X^3}$ и так далее.
\end{definition}

\begin{definition}
	Функцию $f$ из $X$ в $Y$ называют:
	
	\textbf{\textit{сюрьективной}}, если $\range (f) = Y$;
	
	\textbf{\textit{инъективной}}, если $f^{-1}$ функционально. 
	
	\textbf{\textit{биективной}}, если $ f $ \textit{сюрьективна} и \textit{инъективна}.\\
	\textit{Сюрьективные} функции также называют \textbf{\textit{сюрьекциями}}, \textit{инъективные} \textbf{\textit{инъекциями}}, а \textit{биективные} \textbf{\textit{биекциями}}.
\end{definition}

\begin{definition}
	Введём особые символы для часто использующихся множеств. 
	
	$\mathbb{N} = \{1,2,3,4,\dotsc \}$ - множество \textit{натуральных} чисел.
	
	$\mathbb{N}_0 = \{0,1,2,3,4,\dotsc \}$ - множество натуральных чисел и ноль.
	
	$\mathbb{Z} = \{\dotsc, -3, -2, -1, 0,1,2,3,\dotsc \}$ - множество \textit{целых} чисел.
	
	$\mathbb{Q}$ - множество \textit{рациональных} чисел.
	
	$\mathbb{R}$ - множество \textit{вещественных} чисел.
	
	$\mathbb{R\backslash Q}$ - множество \textit{иррациональных} чисел.
	
	$\mathbb{A}$ - множество \textit{алгебраических} чисел (вещественные числа, которые могут быть корнями многочленов с целыми коэффициентами).
\end{definition}

\begin{example}
	$\N \subset \N_0 \subset \Z \subset \Q \subset \A \subset \R.$
\end{example}

\begin{example}
	$f: \N \rightarrow \N, f(x) = 2x$ является \textit{инъективной}.
\end{example}
\begin{example}
	$f: \R \rightarrow \R, f(x) = x^3 - 3x$ является \textit{сюрьективной}.
\end{example}
\begin{example}
	$f: \R_{>0} \rightarrow \R_{>0}, f(x) = \dfrac{1}{x}$ является \textit{биективной}.
\end{example}

\begin{definition}
	$|A| = |B|$~--- множества $A$ и $B$ \textit{\textbf{равномощны}}~--- между ними есть \textit{биекция}.
\end{definition}

\begin{task} 
	Покажите, что для любых $X, Y, Z$ верно: \\
	\newline
	\begin{minipage}[c]{0.5\textwidth}
		\begin{itemize}
			\item $|X\times Y|=|Y \times X|$
			\item $|(X\times Y)\times Z|=|X\times (Y \times Z)|$
		\end{itemize}
	\end{minipage}
	\begin{minipage}[c]{0.5\textwidth}
		\begin{itemize}
			\item если $|X|=|Y|$, то $|X\times Z|=|Y \times Z|$
			\item $|Z^{X\times Y}|=|(Z^Y)^X|$
		\end{itemize}
	\end{minipage}
\end{task}

\begin{example}
	Если $|A| = |\mathbb{N}|$, то $A$ называется \textbf{\textit{счётным}} множеством (его элементы можно пересчитать).
\end{example}

\begin{definition}
	Для конечных множеств $A$ определим $|A|$~--- \textit{\textbf{мощность}} множества~--- количество элементов в нем.
\end{definition}

\begin{example}
	$|\{a,\{b,c\},\{\{d\},e\},\varnothing\}| = |\{0,1,2,3\}| = 4.$
\end{example}

\begin{definition}
	Говорят, что $X$ по мощности меньше или равно $Y$, и пишут $X \preccurlyeq Y$ или $|X| \leqslant |Y|$, если существует инъекция из $X$ в $ Y $. 
\end{definition}

\begin{theorem}[ (Кантора--Шрёдера--Бернштейна)]
	Если $X \preccurlyeq Y$ и $Y \preccurlyeq X$ , то $|X| = |Y|$.
\end{theorem}

\begin{definition}
	$A^* = \{\varnothing\} \cup A \cup A^2 \cup A^3 \dotsc$~--- множество всех конечных последовательностей.
\end{definition}

\begin{example}
	$\{a,b\}^* = \{\varnothing, a,b,aa,ab,ba,bb,aaa,aab,\dotsc\}.$
\end{example}

\begin{definition}
	$\mathcal{P}(A) = 2^A$ - множество всех подмножеств множества $A$.
\end{definition}

\begin{example}
	$2^{\{1,2,3\}} = \{\{\}, \{1\},\{2\},\{3\},\{1,2\},\{1,3\},\{2,3\},\{1,2,3\}\}.$
\end{example}


	
\begin{task} 
	
	\begin{minipage}[c]{0.7\textwidth}
		\begin{itemize}
			\item Докажите, что $|$Чётных положительных$| = |\mathbb{N}|.$
			\item Докажите, что $|$Нечётных положительных$| = |\mathbb{N}|.$
			\item Докажите, что если $|A| = |B| = |\mathbb{N}|$, то $A\cup B = |\mathbb{N}|.$
			\item Докажите, что $|(0,1)| = |\mathbb{R}|.$
			\item Докажите, что $|(0,1)| = |2^{\mathbb{N}}|.$
			\item Докажите, что $|\mathbb{R}| \neq |\mathbb{N}|.$
			\item Докажите, что $|\mathbb{R}^2| = |\mathbb{R}|.$
		\end{itemize}
	\end{minipage}
	\begin{minipage}[c]{0.3\textwidth}
		\begin{itemize}
			\item $|\mathbb{Z}| \stackrel{?}{=} |\mathbb{N}|.$
			\item $|\mathbb{Q}| \stackrel{?}{=} |\mathbb{N}|.$
			\item $|\mathbb{N}^2| \stackrel{?}{=} |\mathbb{N}|.$
			\item $|\mathbb{N}^*| \stackrel{?}{=} |\mathbb{N}|.$
			\item $|\mathbb{A}| \stackrel{?}{=} |\mathbb{N}|.$
			\item $|\mathbb{R}| \stackrel{?}{=} |\mathbb{N}|.$
			\item $|\mathbb{R}^*| \stackrel{?}{=} |\mathbb{R}|.$
		\end{itemize}
	\end{minipage}
\end{task}

\begin{theorem}[ (Кантора)]
	Для любого $A$ верно $|2^A| \neq |A|.$
\end{theorem}

\begin{theorem}[ (Континуум-гипотеза)]
Для любого $A$ если $\N \preccurlyeq A \preccurlyeq \R$, то либо $|A| = |\N|,$ либо $|A| = |\R|.$ 
\end{theorem}


\end{document}