\documentclass{article}

\usepackage[12pt]{extsizes}
\usepackage[T2A]{fontenc}
\usepackage[utf8]{inputenc}
\usepackage[english, russian]{babel}

\usepackage{mathrsfs}
\usepackage[dvipsnames]{xcolor}

\usepackage{amsmath}
\usepackage{amssymb}
\usepackage{amsthm}
\usepackage{indentfirst}
\usepackage{amsfonts}
\usepackage{enumitem}
\usepackage{graphics}
\usepackage{tikz}
\usepackage{tabu}
\usepackage{diagbox}
\usepackage{hyperref}
\usepackage{mathtools}
\usepackage{ucs}
\usepackage{lipsum}
\usepackage{geometry} % Меняем поля страницы
\usepackage{fancyhdr} % Headers and footers
\usepackage[framemethod=TikZ]{mdframed}

\newcommand{\definebox}[3]{%
    \newcounter{#1}
    \newenvironment{#1}[1][]{%
        \stepcounter{#1}%
        \mdfsetup{%
            frametitle={%
            \tikz[baseline=(current bounding box.east),outer sep=0pt]
            \node[anchor=east,rectangle,fill=white]
            {\strut #2~\csname the#1\endcsname\ifstrempty{##1}{}{##1}};}}%
        \mdfsetup{innertopmargin=1pt,linecolor=#3,%
            linewidth=3pt,topline=true,
            frametitleaboveskip=\dimexpr-\ht\strutbox\relax,}%
        \begin{mdframed}[]
            \relax%
            }{
        \end{mdframed}}%
}

\definebox{theorem_boxed}{Теорема}{ForestGreen!24}
\definebox{definition_boxed}{Определение}{blue!24}
\definebox{task_boxed}{Задача}{orange!24}
\definebox{paradox_boxed}{Парадокс}{red!24}

\theoremstyle{plain}
\newtheorem{theorem}{Теорема}
\newtheorem{task}{Задача}
\newtheorem{lemma}{Лемма}
\newtheorem{definition}{Определение}
\newtheorem{statement}{Утверждение}
\newtheorem{corollary}{Следствие}

\theoremstyle{remark}
\newtheorem{remark}{Замечание}
\newtheorem{example}{Пример}
\newcommand{\range}{\mathrm{range}}
\newcommand{\dom}{\mathrm{dom}}
\newcommand{\N}{\mathbb{N}}
\newcommand{\R}{\mathbb{R}}
\newcommand{\E}{\mathbb{E}}
\newcommand{\D}{\mathbb{D}}
\newcommand{\M}{\mathcal{M}}
\newcommand{\Prime}{\mathbb{P}}
\newcommand{\A}{\mathbb{A}}
\newcommand{\Q}{\mathbb{Q}}
\newcommand{\Z}{\mathbb{Z}}
\newcommand{\F}{\mathbb{F}}
\newcommand{\CC}{\mathbb{C}}

\DeclarePairedDelimiter\abs{\lvert}{\rvert}
\DeclarePairedDelimiter\floor{\lfloor}{\rfloor}
\DeclarePairedDelimiter\ceil{\lceil}{\rceil}
\DeclarePairedDelimiter\lr{(}{)}
\DeclarePairedDelimiter\set{\{}{\}}
\DeclarePairedDelimiter\norm{\|}{\|}

\renewcommand{\labelenumi}{(\alph{enumi})}

\newcommand{\smallindent}{
    \geometry{left=1cm}% левое поле
    \geometry{right=1cm}% правое поле
    \geometry{top=1.5cm}% верхнее поле
    \geometry{bottom=1cm}% нижнее поле
}

\newcommand{\header}[3]{
    \pagestyle{fancy} % All pages have headers and footers
    \fancyhead{} % Blank out the default header
    \fancyfoot{} % Blank out the default footer
    \fancyhead[L]{#1}
    \fancyhead[C]{#2}
    \fancyhead[R]{#3}
}

\newcommand{\dividedinto}{
    \,\,\,\vdots\,\,\,
}

\newcommand{\littletaller}{\mathchoice{\vphantom{\big|}}{}{}{}}

\newcommand\restr[2]{{
    \left.\kern-\nulldelimiterspace % automatically resize the bar with \right
    #1 % the function
    \littletaller % pretend it's a little taller at normal size
    \right|_{#2} % this is the delimiter
}}

\DeclareGraphicsExtensions{.pdf,.png,.jpg}

\newenvironment{enumerate_boxed}[1][enumi]{\begin{enumerate}[label*=\protect\fbox{\arabic{#1}}]}{\end{enumerate}}



\smallindent

\header{Математика}{\textit{Теория чисел}}{25 сентября 2022}

%----------------------------------------------------------------------------------------

\begin{document}
    \large

    \begin{center}
        \textbf{Теорема Эйлера}
    \end{center}

    \begin{theorem}[Эйлера]
        Для натуральных взаимно простых $a, m$, верно сравнение \[a^{\phi(m)} \equiv 1 \pmod m\]
    \end{theorem}

    \begin{enumerate_boxed}

        \item Найдите $3$ последние цифры чисел (a) $7^{2000}$; (b) $7^{2003}$.

        \item Докажите, что существует натуральная степень тройки, заканчивающаяся на 00001.
        Найдите явно эту степень.

        \item Найдите последние две цифры в десятичной записи числа $3^{219}$.

        \item Докажите, что для любого натурального числа $a$ верно, что $a^{17} - a$ делится на $510$;

        \item Докажите, что для любого натурального $n$ число $n^{84} - n^4$ делится на $20400$.

        \item Докажите, что если $n$ нечётно, то $2^{n!} - 1$ делится на $n$;

        \item Докажите, что если $n$ чётно, то $2^{n!} - 1$ делится на $n^2 - 1$.

        \item Докажите, что $2^{3^k} + 1$ делится на $3^{k + 1}$.

        \item Докажите, что существует бесконечно много натуральных $n$ таких, что $2^n - 1$ имеет хотя бы 1000 различных простых делителей.

        \item Докажите, что если число $n$ имеет два различных нечетных простых делителя, то для любого $a$, взаимно простого с $n$, верно, что $a^{\varphi(n) / 2} - 1$ делится на $n$.

        \item \textbf{(Усиление теоремы Эйлера)} Если $p_1^{a_1}p_2^{a_2}\ldots p_k^{a_k}$~--- разложение числа $m$ на простые множители и $x$ --- наименьшее общее кратное чисел $\varphi(p_1^{a_1}), \varphi(p_2^{a_2}),\ldots, \varphi(p_k^{a_k})$, то для любого $a$, взаимно простого с $m$, выполняется сравнение $a^x \equiv 1\pmod m$.

        \item Дано число $2^{2021}$.
        Докажите, что можно дописать слева от него несколько цифр так, чтобы получилась степень двойки.

        \item Обозначим через  $L(m)$  длину периода дроби $1/m$.
        Докажите, что если  $(m, 10) = 1$,  то  $L(m)$  является делителем числа $\phi(m)$.

        \item Докажите, что если $(a, p!) = 1$, то $a^{(p-1)!} \equiv 1 \pmod{p!}$.

        \item Докажите, что для каждого $n$ существует число с суммой цифр $n$, делящееся на $n$.

    \end{enumerate_boxed}
\end{document}