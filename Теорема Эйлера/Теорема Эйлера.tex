\documentclass{article}
\usepackage[12pt]{extsizes}
\usepackage[T2A]{fontenc}
\usepackage[utf8]{inputenc}
\usepackage[english, russian]{babel}

\usepackage{amssymb}
\usepackage{amsfonts}
\usepackage{amsmath}
\usepackage{enumitem}
\usepackage{graphics}
\usepackage{graphicx}

\usepackage{lipsum}

\newtheorem{theorem}{Теорема}
\newtheorem{task}{Задача}
\newtheorem{lemma}{Лемма}
\newtheorem{definition}{Определение}
\newtheorem{example}{Пример}
\newtheorem{statement}{Утверждение}
\newtheorem{corollary}{Следствие}


\usepackage{geometry} % Меняем поля страницы
\geometry{left=1cm}% левое поле
\geometry{right=1cm}% правое поле
\geometry{top=1.5cm}% верхнее поле
\geometry{bottom=1cm}% нижнее поле


\usepackage{fancyhdr} % Headers and footers
\pagestyle{fancy} % All pages have headers and footers
\fancyhead{} % Blank out the default header
\fancyfoot{} % Blank out the default footer
\fancyhead[L]{Математика}
\fancyhead[C]{\textit{Теория чисел}}
\fancyhead[R]{25 сентября 2022}% Custom header text


%----------------------------------------------------------------------------------------

%\begin{document}\normalsize
\begin{document}\large
	
\begin{center}
	\textbf{Теорема Эйлера}
\end{center}

\begin{theorem}[Эйлера]
	Для натуральных взаимно простых $a, m$, верно сравнение $$a^{\phi(m)} \equiv 1 \pmod m$$
\end{theorem}

\begin{enumerate}[label*=\protect\fbox{\arabic{enumi}}]

\item Найдите $3$ последние цифры чисел (a) $7^{2000}$; (b) $7^{2003}$.

\item Докажите, что существует натуральная степень тройки, заканчивающаяся на 00001. Найдите явно эту степень.

\item Найдите последние две цифры в десятичной записи числа $3^{219}$.

\item Докажите, что для любого натурального числа $a$ верно, что $a^{17} - a$ делится на $510$;

\item Докажите, что для любого натурального $n$ число $n^{84} - n^4$ делится на $20400$.

\item Докажите, что если $n$ нечётно, то $2^{n!} - 1$ делится на $n$;

\item Докажите, что если $n$ чётно, то $2^{n!} - 1$ делится на $n^2 - 1$. 

\item Докажите, что $2^{3^k} + 1$ делится на $3^{k + 1}$.

\item Докажите, что существует бесконечно много натуральных $n$ таких, что $2^n - 1$ имеет хотя бы 1000 различных простых делителей.

\item Докажите, что если число $n$ имеет два различных нечетных простых делителя, то для любого $a$, взаимно простого с $n$, верно, что $a^{\varphi(n) / 2} - 1$ делится на $n$.

\item \textbf{(Усиление теоремы Эйлера)} Если $p_1^{a_1}p_2^{a_2}\ldots p_k^{a_k}$~--- разложение числа $m$ на простые множители и $x$ --- наименьшее общее кратное чисел $\varphi(p_1^{a_1}), \varphi(p_2^{a_2}),\ldots, \varphi(p_k^{a_k})$, то для любого $a$, взаимно простого с $m$, выполняется сравнение $a^x \equiv 1\pmod m$.

\item Дано число $2^{2021}$. Докажите, что можно дописать слева от него несколько цифр так, чтобы получилась степень двойки.

\item Обозначим через  $L(m)$  длину периода дроби $1/m$. Докажите, что если  $(m, 10) = 1$,  то  $L(m)$  является делителем числа $\phi(m)$.

\item Докажите, что если $(a, p!) = 1$, то $a^{(p-1)!} \equiv 1 \pmod{p!}$.

\item Докажите, что для каждого $n$ существует число с суммой цифр $n$, делящееся на $n$.

\end{enumerate}
\end{document}