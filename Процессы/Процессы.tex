\documentclass{article}
\usepackage[12pt]{extsizes}
\usepackage[T2A]{fontenc}
\usepackage[utf8]{inputenc}
\usepackage[english, russian]{babel}

\usepackage{amssymb}
\usepackage{amsfonts}
\usepackage{amsmath}
\usepackage{enumitem}
\usepackage{graphics}
\usepackage{graphicx}

\usepackage{lipsum}

\newtheorem{theorem}{Теорема}
\newtheorem{task}{Задача}
\newtheorem{lemma}{Лемма}
\newtheorem{definition}{Определение}
\newtheorem{example}{Пример}
\newtheorem{statement}{Утверждение}
\newtheorem{corollary}{Следствие}


\usepackage{geometry} % Меняем поля страницы
\geometry{left=1cm}% левое поле
\geometry{right=1cm}% правое поле
\geometry{top=1.5cm}% верхнее поле
\geometry{bottom=1cm}% нижнее поле


\usepackage{fancyhdr} % Headers and footers
\pagestyle{fancy} % All pages have headers and footers
\fancyhead{} % Blank out the default header
\fancyfoot{} % Blank out the default footer
\fancyhead[L]{Математика}
\fancyhead[C]{\textit{Комбинаторика}}
\fancyhead[R]{16 октября}% Custom header text


%----------------------------------------------------------------------------------------

%\begin{document}\normalsize
\begin{document}\large
	
\begin{center}
	\textbf{Алгоритмы и процессы}
\end{center}

\begin{enumerate}[label*=\protect\fbox{\arabic{enumi}}]
	
\item Плоскость раскрашена в два цвета. Докажите, что найдутся две разноцветные точки на расстоянии не больше 1 миллиметра.

\item Плоскость раскрашена в два цвета. Докажите, что найдутся две разноцветные точки на расстоянии ровно 1 миллиметр.

\item Докажите, что среди сумм вида $1+\dfrac{1}{2} +\dfrac{1}{3}+...+\dfrac{1}{n}$ есть сколь угодно большие.

\item  Два зеркала бесконечной длины образуют угол. Луч света падает на один из них. Докажите, что луч света отразится от зеркал конечное число раз (даже если угол очень маленький).

\item
а) Продлим шахматную доску вправо и влево на миллион клеток. Король стоит на средней клетке нижней горизонтали. Может ли он обойти всю доску, побывав на каждой клетке ровно один раз?
б) Тот же вопрос, если доску продлили вправо и влево до бесконечности?

\item Круг разделен на 2022 сектора, и в каждом написано целое число. В один из секторов ставится фишка. Каждым ходом прочитывается число в секторе, где стоит фишка (пусть прочитано $k$), фишка сдвигается на $|k|$ секторов по часовой стрелке, и там, куда она придет, число увеличивается на 1. Докажите, что со временем все числа станут больше миллиона.

\item На бесконечном шоссе находятся полицейская машина (ездит со скоростью до 100 км/ч) и вор на угнанном мотоцикле (ездит со скоростью до 80 км/ч). Полицейские не знают, в каком месте шоссе находится вор. Как им действовать, чтобы наверняка догнать вора? (Вор не может съехать с шоссе или спрятаться).

\item В таблице две строки и бесконечно много столбцов. Клетки первой строки по порядку занумерованы натуральными числами. В клетках второй строки стоят попарно различные натуральные числа. В каждом столбце верхнее число не равно нижнему. Докажите, что можно выбрать бесконечно много столбцов так, чтобы все числа в выбранных столбцах были различны.

\item На бесконечной шахматной доске находятся ферзь и невидимый король, которому запрещено ходить по диагонали. Они ходят по очереди. Может ли ферзь ходить так, чтобы король рано или поздно наверняка попал под шах?

\item На бесконечной шахматной доске стоят ферзь и невидимый король. Известно, что ферзь дал шах по горизонтали, и король ушел из под шаха. Докажите, что ферзь может ходить так, чтобы король наверняка ещё раз попал под шах.

	
\end{enumerate}
\end{document}