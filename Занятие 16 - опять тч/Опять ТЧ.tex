\documentclass{article}

\usepackage[12pt]{extsizes}
\usepackage[T2A]{fontenc}
\usepackage[utf8]{inputenc}
\usepackage[english, russian]{babel}

\usepackage{mathrsfs}
\usepackage[dvipsnames]{xcolor}

\usepackage{amsmath}
\usepackage{amssymb}
\usepackage{amsthm}
\usepackage{indentfirst}
\usepackage{amsfonts}
\usepackage{enumitem}
\usepackage{graphics}
\usepackage{tikz}
\usepackage{tabu}
\usepackage{diagbox}
\usepackage{hyperref}
\usepackage{mathtools}
\usepackage{ucs}
\usepackage{lipsum}
\usepackage{geometry} % Меняем поля страницы
\usepackage{fancyhdr} % Headers and footers
\newcommand{\range}{\mathrm{range}}
\newcommand{\dom}{\mathrm{dom}}
\newcommand{\N}{\mathbb{N}}
\newcommand{\R}{\mathbb{R}}
\newcommand{\E}{\mathbb{E}}
\newcommand{\D}{\mathbb{D}}
\newcommand{\M}{\mathcal{M}}
\newcommand{\Prime}{\mathbb{P}}
\newcommand{\A}{\mathbb{A}}
\newcommand{\Q}{\mathbb{Q}}
\newcommand{\Z}{\mathbb{Z}}
\newcommand{\F}{\mathbb{F}}
\newcommand{\CC}{\mathbb{C}}

\DeclarePairedDelimiter\abs{\lvert}{\rvert}
\DeclarePairedDelimiter\floor{\lfloor}{\rfloor}
\DeclarePairedDelimiter\ceil{\lceil}{\rceil}
\DeclarePairedDelimiter\lr{(}{)}
\DeclarePairedDelimiter\set{\{}{\}}
\DeclarePairedDelimiter\norm{\|}{\|}

\renewcommand{\labelenumi}{(\alph{enumi})}

\newcommand{\smallindent}{
    \geometry{left=1cm}% левое поле
    \geometry{right=1cm}% правое поле
    \geometry{top=1.5cm}% верхнее поле
    \geometry{bottom=1cm}% нижнее поле
}

\newcommand{\header}[3]{
    \pagestyle{fancy} % All pages have headers and footers
    \fancyhead{} % Blank out the default header
    \fancyfoot{} % Blank out the default footer
    \fancyhead[L]{#1}
    \fancyhead[C]{#2}
    \fancyhead[R]{#3}
}

\newcommand{\dividedinto}{
    \,\,\,\vdots\,\,\,
}

\newcommand{\littletaller}{\mathchoice{\vphantom{\big|}}{}{}{}}

\newcommand\restr[2]{{
    \left.\kern-\nulldelimiterspace % automatically resize the bar with \right
    #1 % the function
    \littletaller % pretend it's a little taller at normal size
    \right|_{#2} % this is the delimiter
}}

\DeclareGraphicsExtensions{.pdf,.png,.jpg}

\newenvironment{enumerate_boxed}[1][enumi]{\begin{enumerate}[label*=\protect\fbox{\arabic{#1}}]}{\end{enumerate}}



\smallindent

\header{Математика}{\textit{Алгебра}}{30 апреля 2024}

\begin{document}
    \large

    \begin{center}
        \textbf{Опять ТЧ}
    \end{center}

    \begin{enumerate_boxed}

        \item Докажите, что для любого натурального числа $n > 2$ число $n!$ можно представить в виде суммы $n$ различных делителей числа $n!$.

        \item На доске записано натуральное число.
        Его последняя цифра запоминается, затем стирается и, умноженная на 17, прибавляется к тому числу, что осталось на доске после стирания.
        Первоначально было записано число $13^{2023}$.
        Может ли после применения нескольких таких операций получиться число $2023^{13}$?

        \item Существуют ли рациональные числа $p, q, r$, такие, что $p + q + r = 0$ и $pqr = 1$?

        \item Найдите все натуральные числа $a, n$, такие, что число
        $\dfrac{(a + 1)^n - a^n}{n}$~--- натуральное.

        \item Двое играют в игру.
        Они по очереди выбирают 7 различных цифр от 1 до 9 (первый — четыре цифры, второй — три).
        Из них составляется по порядку выбора семизначное число $A$ (первая выбранная цифра - первая цифра $A$).
        Первый побеждает, если $A$ - последние 7 цифр десятичной записи седьмой степени некоторого числа.
        Кто из игроков имеет выйгрышную стратегию?

        \item Найдите все натуральные числа $a, b$, такие, что $a^b - 1 \dividedinto b^a$

        \item Найдите все натуральные числа $n$, такие, что $n^4 - n^3 + 3n^2 + 5$~--- точный квадрат.

        \item Найдите все сюрьективные функции $f : \N \rightarrow \N$, такие, что для любых натуральных чисел $m, n$ и любого простого $p$ верно:
        $f(m + n) \dividedinto p \Longleftrightarrow f(m) + f(n) \dividedinto p.$

        \item Докажите, что для любых натуральных чисел $m, n$ существует натуральное число $c$, такое, что $cm, cn$ имеют одинаковый набор (считая повторения) ненулевых цифр.

    \end{enumerate_boxed}
\end{document}