\documentclass{article}

\usepackage[12pt]{extsizes}
\usepackage[T2A]{fontenc}
\usepackage[utf8]{inputenc}
\usepackage[english, russian]{babel}

\usepackage{mathrsfs}
\usepackage[dvipsnames]{xcolor}

\usepackage{amsmath}
\usepackage{amssymb}
\usepackage{amsthm}
\usepackage{indentfirst}
\usepackage{amsfonts}
\usepackage{enumitem}
\usepackage{graphics}
\usepackage{tikz}
\usepackage{tabu}
\usepackage{diagbox}
\usepackage{hyperref}
\usepackage{mathtools}
\usepackage{ucs}
\usepackage{lipsum}
\usepackage{geometry} % Меняем поля страницы
\usepackage{fancyhdr} % Headers and footers
\newcommand{\range}{\mathrm{range}}
\newcommand{\dom}{\mathrm{dom}}
\newcommand{\N}{\mathbb{N}}
\newcommand{\R}{\mathbb{R}}
\newcommand{\E}{\mathbb{E}}
\newcommand{\D}{\mathbb{D}}
\newcommand{\M}{\mathcal{M}}
\newcommand{\Prime}{\mathbb{P}}
\newcommand{\A}{\mathbb{A}}
\newcommand{\Q}{\mathbb{Q}}
\newcommand{\Z}{\mathbb{Z}}
\newcommand{\F}{\mathbb{F}}
\newcommand{\CC}{\mathbb{C}}

\DeclarePairedDelimiter\abs{\lvert}{\rvert}
\DeclarePairedDelimiter\floor{\lfloor}{\rfloor}
\DeclarePairedDelimiter\ceil{\lceil}{\rceil}
\DeclarePairedDelimiter\lr{(}{)}
\DeclarePairedDelimiter\set{\{}{\}}
\DeclarePairedDelimiter\norm{\|}{\|}

\renewcommand{\labelenumi}{(\alph{enumi})}

\newcommand{\smallindent}{
    \geometry{left=1cm}% левое поле
    \geometry{right=1cm}% правое поле
    \geometry{top=1.5cm}% верхнее поле
    \geometry{bottom=1cm}% нижнее поле
}

\newcommand{\header}[3]{
    \pagestyle{fancy} % All pages have headers and footers
    \fancyhead{} % Blank out the default header
    \fancyfoot{} % Blank out the default footer
    \fancyhead[L]{#1}
    \fancyhead[C]{#2}
    \fancyhead[R]{#3}
}

\newcommand{\dividedinto}{
    \,\,\,\vdots\,\,\,
}

\newcommand{\littletaller}{\mathchoice{\vphantom{\big|}}{}{}{}}

\newcommand\restr[2]{{
    \left.\kern-\nulldelimiterspace % automatically resize the bar with \right
    #1 % the function
    \littletaller % pretend it's a little taller at normal size
    \right|_{#2} % this is the delimiter
}}

\DeclareGraphicsExtensions{.pdf,.png,.jpg}

\newenvironment{enumerate_boxed}[1][enumi]{\begin{enumerate}[label*=\protect\fbox{\arabic{#1}}]}{\end{enumerate}}



\smallindent

\header{ЦРОД $\bullet$ Математика}{\textit{Логика}}{Май 2022}


%----------------------------------------------------------------------------------------

%\begin{document}\normalsize
\begin{document}
    \large


    \begin{center}
        \textbf{Булева логика}
    \end{center}

    \begin{enumerate_boxed}

        \item Мистер Бруно уверен, что можно вырезать из шахматной доски $8 \times 8$ ровно $4$ клетки так, чтобы оставшуюся доску можно было разрезать на “доминошки”, то есть прямоугольники $1 \times 2$.
        Прав ли Мистер Бруно?

        \item Мистер Бруно так понравилось вырезать из доски $8 \times 8$ четыре клетки и разбивать оставшуюся часть на доминошки, что теперь он уверен, что как ни вырежи $4$ клетки из шахматной доски $8 \times 8$, всегда оставшуюся фигурку можно разрезать на доминошки.
        Не ошибается ли Мистер Бруно?

        \item Cоставьте утверждения, подходящие под формулу:
        \begin{itemize}
            \item
            $A =$ «На полдник выдали сырки»

            $B =$ «На полдник выдали печенье»

            $C =$ «На полдник не выдали сок или чай»

            $(A\vee B)\wedge \overline{C}$

            \item
            $A =$ «Пираты идут на абордаж»

            $B =$ «На улице град»

            $C =$ «Ниндзя съел суши»

            $D =$ «Роботы не захватили планету Земля»

            $(\overline{A}\vee B)\wedge (C \vee \overline{D})$

        \end{itemize}

        \item Постройте таблицу истинности для выражений:
        \begin{itemize}

            \item $\overline{\overline{\overline{a}}}$

            \item $a \oplus b$

            \item $a \wedge (b \vee \overline{a})$

        \end{itemize}

        \item Упростите выражения:
        \begin{itemize}

            \item $(a \rightarrow b) \wedge (b \rightarrow a)$

            \item $(\overline{a} \wedge b) \vee  (a \wedge b) \vee  (a \wedge \overline{b})$

            \item $(a \rightarrow b) \wedge (\overline{a} \rightarrow \overline{b})$

        \end{itemize}

        \item Докажите с помощью таблиц истинности, что
        \begin{itemize}

            \item $(a \vee b) \wedge c = (a \wedge c) \vee (b \wedge c)$

            \item  $(a \wedge b) \vee c = (a \vee c) \wedge (b \vee c)$

            \item  $\overline{a \wedge b}= \overline{a} \vee \overline{b}$

            \item  $\overline{a \vee b}= \overline{a} \wedge \overline{b}$

        \end{itemize}

        \item Постройте отрицание к утверждениям:
        \begin{itemize}
            \item <<Поле шахматной доски - белое>>;

            \item <<Верблюд синий и весит хотя бы 100 кг>>;

            \item <<Я рыцарь или ты лжец>>.

            \item <<Ёжик не пойдёт сегодня в лес, если будет дождь>>;
        \end{itemize}

        \item Верны ли утверждения:
        \begin{itemize}
            \item <<$2 \times 2 = 5$ или $2 + 2 = 4$>>;

            \item <<На поток приедет К.А. Сухов и лжецы иногда говорят правду>>;

            \item <<Это утверждение ложно>>;
        \end{itemize}

    \end{enumerate_boxed}

\end{document}