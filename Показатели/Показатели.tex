\documentclass{article}
\usepackage[12pt]{extsizes}
\usepackage[T2A]{fontenc}
\usepackage[utf8]{inputenc}
\usepackage[english, russian]{babel}

\usepackage{amssymb}
\usepackage{amsfonts}
\usepackage{amsmath}
\usepackage{enumitem}
\usepackage{graphics}
\usepackage{graphicx}

\usepackage{lipsum}

\newtheorem{theorem}{Теорема}
\newtheorem{task}{Задача}
\newtheorem{lemma}{Лемма}
\newtheorem{definition}{Определение}
\newtheorem{example}{Пример}
\newtheorem{statement}{Утверждение}
\newtheorem{corollary}{Следствие}


\usepackage{geometry} % Меняем поля страницы
\geometry{left=1cm}% левое поле
\geometry{right=1cm}% правое поле
\geometry{top=1.5cm}% верхнее поле
\geometry{bottom=1cm}% нижнее поле


\usepackage{fancyhdr} % Headers and footers
\pagestyle{fancy} % All pages have headers and footers
\fancyhead{} % Blank out the default header
\fancyfoot{} % Blank out the default footer
\fancyhead[L]{Математика}
\fancyhead[C]{\textit{Теория чисел}}
\fancyhead[R]{27 сентября 2023}% Custom header text


%----------------------------------------------------------------------------------------

%\begin{document}\normalsize
\begin{document}\large
	
\begin{center}
	\textbf{Показатели}
\end{center}

\textbf{Определение:} \textit{Показателем} остатка $a$ по модулю $m$ является наименьшее такое число $t$, что $$a^t \equiv 1 \pmod m.$$ Обычно обозначается $ord_m(a)$



\textbf{Свойства:}  

\begin{enumerate}[label*=\protect\fbox{\arabic{enumi}}]
	
\item Покажите, что если $(a,m) = 1,$ то показатель существует

\item Покажите, что если $(a,m) \neq 1,$ то показателя не существует.

\item Пусть $t$~--- показатель $a$ по модулю $m$. 
\begin{enumerate}

\item Докажите, что если $a^{k} \equiv 1 \pmod m$, то $k \,\vdots\, t$.

\item Докажите, что если $a^{t_1} \equiv a^{t_2}  \pmod m$, то $t_1 \equiv t_2  \pmod t$.

\item Докажите, что числа $a^0, a^1, a^2, \dotsc , a^{t-1}$ попарно различны по модулю $m$.

\end{enumerate}

\item Докажите, что показатели взаимно обратных чисел совпадают.

\item Пусть $ord_m(a) = t,  ord_m(b) = d$. 
\begin{enumerate}
	
	\item Докажите, что если $t \,\vdots\, h$, то $ord_m (a^h) = \dfrac{t}{h}$.
	
	\item Докажите, что $ord_m (a^h) = \dfrac{t}{(t,h)}$.
	
	\item Докажите, что если $(t, d) = 1$, то $ord_m (a\cdot b) = t \cdot d$.
	
\end{enumerate}

\end{enumerate}

\textbf{Задачи:}  

\begin{enumerate}[label*=\protect\fbox{\arabic{enumi}}]
	
	\item Найдите $ord_{a^n-1}(a).$
%	n
	
	\item Докажите, что $ \varphi(a^n - 1) $ делится на $ n $ для натуральных $ a $ и $ n $.
	
	\item Рассмотрим все числа вида $10^i - 10^j$ при $0 \leqslant i < j \leqslant 99$. Сколько из них делятся
	на 1001?
%	2C_16^2 + 4*C_33^2 = 16*49 = 784
	
	\item Дано нечётное простое число $p$, а также простые числа $q$ и $r$. Известно, что
	$q^r +1 \,\vdots\, p$. Докажите, что либо $p - 1 \,\vdots\, 2r$, либо $q^2- 1 \,\vdots\, p$.
%	показатель либо 2r либо 2.
	
	\item Сколько делителей от 1 до 200 имеет число $2^{239} - 1$?
%	только 1
	
	\item 
	\begin{enumerate}
		
		\item Докажите, что в разложении на простые сомножители числа $2^q -1$, где $ q $ простое, любое число будет давать остаток 1 по модулю $ q $.
%		по МТФ
		
		\item Выведите из этого, что простых чисел бесконечно много.
%		Надо взять произведение всех простых. Тогда 2^prod - 1 --- простое
		
	\end{enumerate}

	\item Пусть $a > 1, p > 2$ и $ p $ простое. Докажите, что простые нечетные делители $ a^p - 1 $ или делят $ a - 1 $ или сравнимы с 1 по модулю $ 2p $.
%	либо показатель p либо 1 если 1, то первый случай, если p, то по модулю p это простое число равно 
	
	\item Докажите, что любой простой делитель числа $ 2^{2^k} + 1 $ сравним с 1 по модулю $ 2^{k + 1} $.
%	показатель оказывается ровно 2^{k + 1} 

	\item Даны натуральные числа $a, n > 1$. Докажите, что для каждого нечетного простого делителя $ p $ числа $ a^{2^n} + 1 $ число $ p - 1 $ делится на $ 2^{n+1} $.
%	same shit
	
	\item Дано простое число $ p $. Докажите, что $2^{2^p} - 4$ делится на $2^p - 1$.
%	так как 2^p - 2 делится на p по МТФ, а занчит 2^{2^p - 2} - 1 делится на 2^p - 1

	\item Пусть $ p $ и $ q $ простые, $ q > 5 $. Известно, что $ 2^p +3^p $ делится на $ q $.
	Докажите, что $ q > 2p $.
%	если q <= 2p, то q - 1 < 2p, но у q-1 и 2p есть НОД не равный 1. Он либо p либо 2. То есть q = p + 1 или 3, что невзможно
	
	\item Докажите, что при натуральном $ n > 1 $ число $ 2^n - 1 $ не делится на $ n $.
%	рассмотрим наименьшее нечетное простое q. Окажется, что у n и q-1 есть неединичный НОД то есть простое поменьше
	
	\item Найдите все пары простых чисел $ p $ и $ q $ таких, что $ (5^p - 2^p)(5^q - 2^q) \,\vdots\, pq $.
%	(3,13)(13,3),(3,3)
	
\end{enumerate}
\end{document}