\documentclass{article}

\usepackage[12pt]{extsizes}
\usepackage[T2A]{fontenc}
\usepackage[utf8]{inputenc}
\usepackage[english, russian]{babel}

\usepackage{mathrsfs}
\usepackage[dvipsnames]{xcolor}

\usepackage{amsmath}
\usepackage{amssymb}
\usepackage{amsthm}
\usepackage{indentfirst}
\usepackage{amsfonts}
\usepackage{enumitem}
\usepackage{graphics}
\usepackage{tikz}
\usepackage{tabu}
\usepackage{diagbox}
\usepackage{hyperref}
\usepackage{mathtools}
\usepackage{ucs}
\usepackage{lipsum}
\usepackage{geometry} % Меняем поля страницы
\usepackage{fancyhdr} % Headers and footers
\newcommand{\range}{\mathrm{range}}
\newcommand{\dom}{\mathrm{dom}}
\newcommand{\N}{\mathbb{N}}
\newcommand{\R}{\mathbb{R}}
\newcommand{\E}{\mathbb{E}}
\newcommand{\D}{\mathbb{D}}
\newcommand{\M}{\mathcal{M}}
\newcommand{\Prime}{\mathbb{P}}
\newcommand{\A}{\mathbb{A}}
\newcommand{\Q}{\mathbb{Q}}
\newcommand{\Z}{\mathbb{Z}}
\newcommand{\F}{\mathbb{F}}
\newcommand{\CC}{\mathbb{C}}

\DeclarePairedDelimiter\abs{\lvert}{\rvert}
\DeclarePairedDelimiter\floor{\lfloor}{\rfloor}
\DeclarePairedDelimiter\ceil{\lceil}{\rceil}
\DeclarePairedDelimiter\lr{(}{)}
\DeclarePairedDelimiter\set{\{}{\}}
\DeclarePairedDelimiter\norm{\|}{\|}

\renewcommand{\labelenumi}{(\alph{enumi})}

\newcommand{\smallindent}{
    \geometry{left=1cm}% левое поле
    \geometry{right=1cm}% правое поле
    \geometry{top=1.5cm}% верхнее поле
    \geometry{bottom=1cm}% нижнее поле
}

\newcommand{\header}[3]{
    \pagestyle{fancy} % All pages have headers and footers
    \fancyhead{} % Blank out the default header
    \fancyfoot{} % Blank out the default footer
    \fancyhead[L]{#1}
    \fancyhead[C]{#2}
    \fancyhead[R]{#3}
}

\newcommand{\dividedinto}{
    \,\,\,\vdots\,\,\,
}

\newcommand{\littletaller}{\mathchoice{\vphantom{\big|}}{}{}{}}

\newcommand\restr[2]{{
    \left.\kern-\nulldelimiterspace % automatically resize the bar with \right
    #1 % the function
    \littletaller % pretend it's a little taller at normal size
    \right|_{#2} % this is the delimiter
}}

\DeclareGraphicsExtensions{.pdf,.png,.jpg}

\newenvironment{enumerate_boxed}[1][enumi]{\begin{enumerate}[label*=\protect\fbox{\arabic{#1}}]}{\end{enumerate}}


\usepackage[framemethod=TikZ]{mdframed}

\newcommand{\definebox}[3]{%
    \newcounter{#1}
    \newenvironment{#1}[1][]{%
        \stepcounter{#1}%
        \mdfsetup{%
            frametitle={%
            \tikz[baseline=(current bounding box.east),outer sep=0pt]
            \node[anchor=east,rectangle,fill=white]
            {\strut #2~\csname the#1\endcsname\ifstrempty{##1}{}{##1}};}}%
        \mdfsetup{innertopmargin=1pt,linecolor=#3,%
            linewidth=3pt,topline=true,
            frametitleaboveskip=\dimexpr-\ht\strutbox\relax,}%
        \begin{mdframed}[]
            \relax%
            }{
        \end{mdframed}}%
}

\definebox{theorem_boxed}{Теорема}{ForestGreen!24}
\definebox{definition_boxed}{Определение}{blue!24}
\definebox{task_boxed}{Задача}{orange!24}
\definebox{paradox_boxed}{Парадокс}{red!24}

\theoremstyle{plain}
\newtheorem{theorem}{Теорема}
\newtheorem{task}{Задача}
\newtheorem{lemma}{Лемма}
\newtheorem{statement}{Утверждение}
\newtheorem{corollary}{Следствие}

\theoremstyle{remark}
\newtheorem{remark}{Замечание}
\newtheorem{example}{Пример}

\smallindent

\header{Математика}{\textit{Комбинаторика}}{23 июня 2024}

%----------------------------------------------------------------------------------------

\begin{document}
    \large

    \begin{center}
        \textbf{Найдите граф}
    \end{center}

    \begin{enumerate_boxed}
        \item К 3000 году поверхность Марса разделили на регионы, каждый из которых принадлежит какой-то из $100$ стран (регионы, принадлежащие одной стране, могут образовывать несвязную область).
        Назовем страны соседними, если им принадлежат два региона на Марсе, имеющие общую границу.
        Найдите наименьшее возможное число пар соседних стран.


        \item Муравей ползает по поверхности кубика $11 \times 11 \times 11$ вдоль диагоналей квадратиков $1 \times 1$ (поворачивать в центре клетки нельзя).
        Могло ли так оказаться, что он побывал в центре каждого квадратика ровно один раз?


        \item Даны $10$ чисел $a_1,a_2,\dots,a_{10}.$

        \begin{enumerate}
            \item Известно, что среди попарных сумм $a_i + a_j (i \neq j)$ как минимум 26 целых.
            Докажите, что хотя бы одно из чисел $2a_1,2a_2,\ldots,2a_{10}$~---целое.
            \item Известно, что среди попарных сумм $a_i + a_j (i \neq j)$ как минимум 37 целых.
            Докажите, что все числа $2a_1,2a_2,\ldots,2a_{10}$~---целые.
        \end{enumerate}

        \item Назовем лабиринтом шахматную доску $8 \times 8$, где между некоторыми полями вставлены перегородки.
        Если ладья может обойти все поля, не перепрыгивая через перегородки, то лабиринт называется
        хорошим, иначе — плохим.
        Каких лабиринтов больше — хороших или плохих?


        \item На шахматной доске стоит несколько ладей так, что в каждой строке и каждом столбце стоит хотя бы $k$ ладей.
        При каком наименьшем $k$ гарантированно можно выбрать $8$ ладей так, чтобы в каждой строке и каждом столбце стояло по выбранной ладье?


        \item Поле игры <<Сапёр>> — доска $n \times n$, некоторые клетки которой заняты минами.
        На клетках с минами ничего не написано, в каждой клетке без мин написано число клеток с минами, соседних с ней по стороне или углу.
        Какое наибольшее значение может принимать сумма всех чисел, написанных на доске игры <<Сапёр>>?


        \item Для множества $S$ верно, что для любого $k = 2, 3, \dots , n$ существуют $x, y \in S$ такие, что $x - y = F_k$,
        где $F_k$~--- $k$-ое число Фибоначчи.
        Какое наименьшее возможное число элементов может быть в $S$?


    \end{enumerate_boxed}

\end{document}