\documentclass{article}

\usepackage[12pt]{extsizes}
\usepackage[T2A]{fontenc}
\usepackage[utf8]{inputenc}
\usepackage[english, russian]{babel}

\usepackage{mathrsfs}
\usepackage[dvipsnames]{xcolor}

\usepackage{amsmath}
\usepackage{amssymb}
\usepackage{amsthm}
\usepackage{indentfirst}
\usepackage{amsfonts}
\usepackage{enumitem}
\usepackage{graphics}
\usepackage{tikz}
\usepackage{tabu}
\usepackage{diagbox}
\usepackage{hyperref}
\usepackage{mathtools}
\usepackage{ucs}
\usepackage{lipsum}
\usepackage{geometry} % Меняем поля страницы
\usepackage{fancyhdr} % Headers and footers
\newcommand{\range}{\mathrm{range}}
\newcommand{\dom}{\mathrm{dom}}
\newcommand{\N}{\mathbb{N}}
\newcommand{\R}{\mathbb{R}}
\newcommand{\E}{\mathbb{E}}
\newcommand{\D}{\mathbb{D}}
\newcommand{\M}{\mathcal{M}}
\newcommand{\Prime}{\mathbb{P}}
\newcommand{\A}{\mathbb{A}}
\newcommand{\Q}{\mathbb{Q}}
\newcommand{\Z}{\mathbb{Z}}
\newcommand{\F}{\mathbb{F}}
\newcommand{\CC}{\mathbb{C}}

\DeclarePairedDelimiter\abs{\lvert}{\rvert}
\DeclarePairedDelimiter\floor{\lfloor}{\rfloor}
\DeclarePairedDelimiter\ceil{\lceil}{\rceil}
\DeclarePairedDelimiter\lr{(}{)}
\DeclarePairedDelimiter\set{\{}{\}}
\DeclarePairedDelimiter\norm{\|}{\|}

\renewcommand{\labelenumi}{(\alph{enumi})}

\newcommand{\smallindent}{
    \geometry{left=1cm}% левое поле
    \geometry{right=1cm}% правое поле
    \geometry{top=1.5cm}% верхнее поле
    \geometry{bottom=1cm}% нижнее поле
}

\newcommand{\header}[3]{
    \pagestyle{fancy} % All pages have headers and footers
    \fancyhead{} % Blank out the default header
    \fancyfoot{} % Blank out the default footer
    \fancyhead[L]{#1}
    \fancyhead[C]{#2}
    \fancyhead[R]{#3}
}

\newcommand{\dividedinto}{
    \,\,\,\vdots\,\,\,
}

\newcommand{\littletaller}{\mathchoice{\vphantom{\big|}}{}{}{}}

\newcommand\restr[2]{{
    \left.\kern-\nulldelimiterspace % automatically resize the bar with \right
    #1 % the function
    \littletaller % pretend it's a little taller at normal size
    \right|_{#2} % this is the delimiter
}}

\DeclareGraphicsExtensions{.pdf,.png,.jpg}

\newenvironment{enumerate_boxed}[1][enumi]{\begin{enumerate}[label*=\protect\fbox{\arabic{#1}}]}{\end{enumerate}}



\header{\textit{\textbf{XIII Олимпиада Эйлера}}}{}{6 февраля 2023}

%----------------------------------------------------------------------------------------

\begin{document}
    \large

    \begin{center}
        \LARGE\textbf{8 класс}
    \end{center}
    \begin{center}
        \large\textbf{Первый день}
    \end{center}


    \begin{enumerate}[label*=8.{\arabic{enumi}}]
        \setcounter{enumi}{0}
        \item Петя написал на доске десять натуральных чисел, среди которых нет двух равных.
        Известно, что из этих десяти чисел можно выбрать три числа, делящихся на 5.
        Также известно, что из написанных десяти чисел можно выбрать четыре числа, делящихся на 4.
        Может ли сумма всех написанных на доске чисел быть меньше 75?

        \item 10 бегунов стартуют одновременно: пятеро в синих майках с одного конца беговой дорожки, пятеро в красных майках — с другого.
        Их скорости постоянны и различны, причём скорость каждого бегуна больше 9 км/ч, но меньше 12 км/ч.
        Добежав до конца дорожки, каждый бегун сразу бежит назад, а, вернувшись к месту своего старта, заканчивает бег.
        Тренер ставит в блокноте галочку каждый раз, когда встречаются (лицом к лицу или один догоняет другого) двое бегунов в разноцветных майках (больше двух бегунов в одной точке за время бега не встречались).
        Сколько галочек поставит тренер к моменту, когда закончит бег самый быстрый из бегунов?

        \item На боковых сторонах $AB$ и $AC$ равнобедренного треугольника $ABC$ выбраны точки $P$ и $Q$ соответственно так, что $PQ \parallel BC$.
        На биссектрисах треугольников $ABC$ и $APQ$, исходящих из вершин $B$ и $Q$, выбраны точки $X$ и $Y$ соответственно так, что $XY \parallel BC$.
        Докажите, что $PX = CY$.

        \item Дан квадратный трёхчлен $P(x)$, не обязательно с целыми коэффициентами.
        Известно, что при некоторых целых $a$ и $b$ разность $P(a) - P(b)$ является квадратом натурального числа.
        Докажите, что существует более миллиона таких пар целых чисел $(c, d)$, что разность $P(c) - P(d)$ также является квадратом натурального числа.

        \item Между городами страны организованы двусторонние беспосадочные авиарейсы таким образом, что от каждого города до каждого другого можно добраться (возможно, с пересадками).
        Более того, для каждого города $A$ существует город $B$ такой, что любой из остальных городов соединён напрямую с $A$ или с $B$.
        Докажите, что от любого города можно добраться до любого другого не более, чем с двумя пересадками.

    \end{enumerate}

    \newpage

    \begin{center}
        \LARGE\textbf{8 класс}
    \end{center}
    \begin{center}
        \large\textbf{Второй день}
    \end{center}


    \begin{enumerate}[label*=8.{\arabic{enumi}}]
        \setcounter{enumi}{5}
        \item Приведите пример шести различных натуральных чисел таких, что произведение любых двух из них не делится на сумму всех чисел, а произведение любых трех из них — делится.

        \item На острове живут только рыцари, которые всегда говорят правду, и лжецы, которые всегда лгут.
        Однажды все они сели по кругу, и каждый сказал: «Среди двух моих соседей есть лжец!».
        Затем они сели по кругу в другом порядке, и каждый сказал: «Среди двух моих соседей нет рыцаря!».
        Могло ли на острове быть 2023 человека?

        \item В выпуклом четырёхугольнике $ABCD$ биссектриса угла $B$ проходит через середину стороны $AD$, а $\angle C = \angle A+\angle D$.
        Найдите угол $ACD$.

        \item Имеется клетчатая доска размером $2n\times 2n$.
        Петя поставил на неё $(n+1)^2$ фишек.
        Кот может одним взмахом лапы смахнуть на пол любую одну фишку или две фишки, стоящие в соседних по стороне или углу клетках.
        За какое наименьшее количество взмахов кот заведомо сможет смахнуть на пол все поставленные Петей фишки?

        \item На доске написано 100 натуральных чисел, среди которых ровно 33 нечетных.
        Каждую минуту на доску дописывается сумма всех попарных произведений всех чисел, уже находящихся на ней (например, если на доске были записаны числа 1, 2, 3, 3, то следующим ходом было дописано число $1\cdot 2+1\cdot 3+1\cdot 3+2\cdot 3+2\cdot 3+3\cdot 3$). Можно ли утверждать, что рано или поздно на доске появится число, делящееся на $2^{10000000}$?

    \end{enumerate}
\end{document}