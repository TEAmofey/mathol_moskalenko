\documentclass{article}
\usepackage[12pt]{extsizes}
\usepackage[T2A]{fontenc}
\usepackage[utf8]{inputenc}
\usepackage[english, russian]{babel}

\usepackage{amssymb}
\usepackage{amsfonts}
\usepackage{amsmath}
\usepackage{enumitem}
\usepackage{graphics}
\usepackage{graphicx}

\usepackage{lipsum}
\DeclareGraphicsExtensions{.pdf,.png,.jpg}



\usepackage{geometry} % Меняем поля страницы
\geometry{left=1cm}% левое поле
\geometry{right=1cm}% правое поле
\geometry{top=1.5cm}% верхнее поле
\geometry{bottom=1cm}% нижнее поле


\usepackage{fancyhdr} % Headers and footers
\pagestyle{fancy} % All pages have headers and footers
\fancyhead{} % Blank out the default header
\fancyfoot{} % Blank out the default footer
\fancyhead[L]{Математика}
\fancyhead[C]{\textit{Алгебра}}
\fancyhead[R]{12 декабря 2023}% Custom header text

%----------------------------------------------------------------------------------------

%\begin{document}\normalsize
\begin{document}\large
	

\begin{center}
\textbf{Строим примеры}
\end{center}

\begin{enumerate}[label*=\protect\fbox{\arabic{enumi}}]
	
	\item  Существуют ли попарно различные натуральные числа $x$, $y$ и $z$, удовлетворяющие
	уравнению $x^3 + y^3 =z^{2024}$?
	
	\item Два натуральных числа называются похожими, если одно получается из другого зачёркиванием одной цифры (и возможно отбрасыванием впереди стоящих нулей). Докажите, что существует бесконечно много натуральных чисел, не представимых в виде суммы двух похожих.
	
	\item Есть кусок сыра. Разрешается выбрать иррациональное $a > 0$ и разрезать этот кусок в отношении $1 : a$ по весу, затем разрезать в том же отношении любой из имеющихся кусков, и т.д. Можно ли действовать так, что после конечного числа разрезаний весь сыр удастся разложить на две кучки равного веса?
	
	\item Существует ли такое натуральное $n$, что число вида $12345678\underbrace{9\dotsc9}_{n}87654321$, в котором $n$ девяток, делится на 2023?
	
	\item Существует ли $2023^{2024}$ таких различных натуральных чисел, что никакая сумма нескольких из этих чисел не является полным квадратом?
	
	\item Даны натуральные числа $a$ и $b$. Докажите, что существует бесконечно много натуральных $n$ таких, что число $a^n + 1$ не делится на $n^b + 1$.

\end{enumerate}
\end{document}