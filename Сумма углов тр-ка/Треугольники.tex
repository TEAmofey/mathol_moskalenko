\documentclass{article}
\usepackage[12pt]{extsizes}
\usepackage[T2A]{fontenc}
\usepackage[utf8]{inputenc}
\usepackage[english, russian]{babel}

\usepackage{amssymb}
\usepackage{amsfonts}
\usepackage{amsmath}
\usepackage{enumitem}
\usepackage{graphics}

\usepackage{lipsum}



\usepackage{geometry} % Меняем поля страницы
\geometry{left=1cm}% левое поле
\geometry{right=1cm}% правое поле
\geometry{top=1.5cm}% верхнее поле
\geometry{bottom=1cm}% нижнее поле


\usepackage{fancyhdr} % Headers and footers
\pagestyle{fancy} % All pages have headers and footers
\fancyhead{} % Blank out the default header
\fancyfoot{} % Blank out the default footer
\fancyhead[L]{ЦРОД $\bullet$ Математика}
\fancyhead[C]{\textit{Геометрия}}
\fancyhead[R]{Май 2022}% Custom header text


%----------------------------------------------------------------------------------------

%\begin{document}\normalsize
\begin{document}\large


\begin{center}
\textbf{Сумма углов треугольника}
\end{center}



\begin{enumerate}[label*=\protect\fbox{\arabic{enumi}}]

\item Чему равна сумма углов пятиугольника?

\item Чему равна сумма углов $n$-угольника?

\item Внешние углы при вершинах $A$ и $B$ треугольника $ABC$ равны $134^\circ$ и $99^\circ$ соответственно. Чему равна величина внешнего угла при вершине $C$ этого треугольника?

\item В равнобедренном треугольнике один из углов равен $40^\circ$. Чему может быть равна величина наибольшего угла треугольника?

\item В равнобедренном треугольнике один из углов в два раза больше другого. Чему может быть равна величина наименьшего угла этого треугольника?

\item Два угла треугольника равны $10^\circ$ и $70^\circ$ соответственно. Найдите величину угла между высотой и биссектрисой, проведёнными из вершины третьего угла треугольника.

\item На стороне $AB$ равнобедренного треугольника $ABC$ ($AB= AC$) нашлись такие точки $D$ и $E$ (точка $D$ лежит между точками $A$ и $E$), а на стороне $AC$ — такая точка $F$, что $BC=CE=EF=FD=DA$. Найдите величину угла $ABC$.

\item Дан треугольник $ABC$. На продолжении стороны $AC$ за точку $A$ отложен отрезок $AD=AB$, а за точку $C$ — отрезок  $CE = CB$. Выразите углы треугольника  $DBE$, через углы треугольника  $ABC$.

\item Точки $M$ и $N$ лежат на стороне $AC$ треугольника $ABC$, причём $\angle ABM = \angle C$ и $\angle CBN=\angle A$. Докажите, что треугольник $BMN$ равнобедренный.

\item Выразите угол между биссектрисой угла $A$ и биссектрисой внешнего угла $B$ через величину угла $C$.

\item В четырёхугольнике $ABCD$ биссектрисы углов $A$ и $C$ параллельны. Докажите, что углы $B$ и $D$ четырёхугольника равны.

\item Найдите сумму острых углов пятиугольной звезды
\end{enumerate}
\end{document}