\documentclass{article}
\usepackage[12pt]{extsizes}
\usepackage[T2A]{fontenc}
\usepackage[utf8]{inputenc}
\usepackage[english, russian]{babel}

\usepackage{amssymb}
\usepackage{amsfonts}
\usepackage{amsmath}
\usepackage{enumitem}
\usepackage{graphics}

\usepackage{lipsum}



\usepackage{geometry} % Меняем поля страницы
\geometry{left=1cm}% левое поле
\geometry{right=1cm}% правое поле
\geometry{top=1.5cm}% верхнее поле
\geometry{bottom=1cm}% нижнее поле


\usepackage{fancyhdr} % Headers and footers
\pagestyle{fancy} % All pages have headers and footers
\fancyhead{} % Blank out the default header
\fancyfoot{} % Blank out the default footer
\fancyhead[L]{Математика}
\fancyhead[C]{\textit{Геометрия}}
\fancyhead[R]{4 декабря}% Custom header text


%----------------------------------------------------------------------------------------

%\begin{document}\normalsize
\begin{document}\large


\begin{center}
\textbf{Тоерема Чевы}
\end{center}


\textbf{Теорема Фалеса:} Пусть даны две прямые  $a$ и $b$. Их пересекают три параллельные прямые — первая в точках $A_1$ и $A_2$, вторая в точках $B_1$ и $B_2$, третья в точках $C_1$ и $C_2$. Тогда высекаемые отрезки пропорциональны, то есть выполнено
$$\frac{A_1A_2}{B_1B_2} =\frac{A_2A_3}{B_2B_3}.$$
\begin{enumerate}[label*=\protect\fbox{\arabic{enumi}}]

\item Прямая $l$ пересекает стороны $AB, AD$ и диагональ $AC$ параллелограмма $ABCD$ в точках $X, Y, Z$ соответственно. Докажите, что $\dfrac{AB}{AX} + \dfrac{AD}{AY} = \dfrac{AC}{AZ}$.

\item В треугольнике $ABC$ проведены медианы $BB_1$ и $CC_1$ и на стороне $BC$ отмечена точка $X$. На сторонах $AB, AC$ отмечены точки $M$ и $N$ соответственно так, что $MX \parallel CC_1, NX \parallel BB_1$. Докажите, что отрезок $MN$ медианами $BB_1$ и $CC_1$ разбивается на три равные части.

\item На продолжении стороны $AB$ квадрата $ABCD$ за вершину $B$ отложен отрезок $BP = 2AB$. Точка $M$ — середина стороны $CD$, а отрезки $BM$ и $AC$ пересекаются в точке $Q$. В каком отношении прямая $PQ$ делит сторону $BC$?
\end{enumerate}


\textbf{Теорема Чeвы:} На сторонах $AB, BC$ и $CA$ треугольника $ABC$ отмечены точки $C_1, A_1, B_1$ соответственно. Тогда прямые $AA_1, BB_1$ и $CC_1$ пересекаются в одной точке тогда и только тогда, когда
$\dfrac{AB_1}{B_1C}\cdot \dfrac{CA_1}{A_1B} \cdot\dfrac{BC_1}{C_1A} = 1$

\textbf{Определение:} Отрезок, соединяющий вершину треугольника с произвольной точкой на противоположной стороне (или её продолжении), называется \textit{чевианой}.

\textbf{Теорема Менелая:} На сторонах $AB, BC$ и продолжении $CA$ треугольника $ABC$ отмечены точки $C_1, A_1, B_1$ соответственно. Точки $A_1, B_1$ и $C_1$ лежат на одной прямой тогда и только тогда, когда
$\dfrac{AB_1}{B_1C}\cdot \dfrac{CA_1}{A_1B} \cdot\dfrac{BC_1}{C_1A} = 1$

\begin{enumerate}[label*=\protect\fbox{\arabic{enumi}}]
\setcounter{enumi}{3}

\item Докажите обратное следствие в теореме (а) Чевы и (b) Менелая.

\item Чевианы $AA_1, BB_1$ и $CC_1$ треугольника $ABC$ пересекаются в одной точке. Точку $A_1$ отразили симметрично относительно середины отрезка $BC$ и получили точку $A_2$. Точки $B_2$ и $C_2$ определяются аналогично. Докажите, что прямые $AA_2$, $BB_2$ и $CC_2$ тоже пересекаются в одной точке.

\item Дан треугольник $ABC$. На стороне $AB$ отмечена точка $D$, а на стороне $AC$ — точка $E$ так, что $BC \parallel DE$. Докажите, что точка пересечения отрезков $CD$ и $BE$ лежит на медиане, проведенной из вершины $A$.

\item Точка $K$ лежит на стороне $AB$, а точка $M$ — на стороне $AC$ треугольника $ABC$, причем $AK : KB = 3 : 2, AM : MC = 4 : 5$. Прямая, проходящая через точку $K$ параллельно $BC$, пересекает отрезок $BM$ в точке $P$. Найдите отношение $BP : PM$.

\item На чевиане $AA_1$ треугольника $ABC$ выбирается переменная точка $X$. Лучи $BX$ и $CX$ пересекают стороны $AC$ и $AB$ в точках $Y$ и $Z$ соответственно. Докажите, что все построенные таким образом прямые $YZ$ пересекают прямую $BC$ в одной и той же точке, либо все этой прямой параллельны.

\item Из вершины $C$ прямого угла прямоугольного треугольника $ABC$ опущена высота $CK$, и в треугольнике $ACK$ проведена биссектриса $CE$. Прямая, проходящая через точку $B$ параллельно $CE$, пересекает прямую $CK$ в точке $F$. Докажите, что прямая $EF$ делит отрезок $AC$ пополам.

\item В треугольнике $ABC$ на сторонах $AB,AC$ и $BC$ выбраны точки $D,E$ и $F$ соответственно так, что $BF = 2CF, CE = 2AE$ и угол $DEF$~--- прямой. Докажите, что $DE$~--- биссектриса угла $ADF$.

\item Через вершину $A$ и середину медианы $BM$ треугольника $ABC$ провели прямую. В каком отношении она делит сторону $BC$?

\item Пусть $AL$~--- биссектриса треугольника $ABC$, точка $D$~--- ее середина, $E$~--- проекция $D$ на $AB$. Известно, что $AC = 3AE$. Докажите, что треугольник $CEL$ равнобедренный.

\item Точки $M$ и $K$ делят стороны $AB$ и $BC$ треугольника $ABC$ в отношении $2 : 3$ и $4 : 1$, считая от их общей вершины. В каком отношении делится отрезок $MK$ медианой треугольника, проведенной к стороне $AC$?

\item Дан треугольник $ABC$, в котором $BM$~--- медиана. Точка $P$ лежит на стороне $AB$, точка $Q$~--- на стороне $BC$, причем $AP : PB = 2 : 5, BQ : QC = 6$. Отрезок $PQ$ пересекает
медиану $BM$ в точке $R$. Найдите $BR:RM$.

\item В треугольнике $ABC$ проведены биссектрисы $AA_1$ и $CC_1$. Прямые $A_1C_1$ и $AC$ пересекаются в точке $D$. Докажите, что $BD$~--- внешняя биссектриса угла $AB$

\end{enumerate}

\end{document}