\documentclass{article}
\usepackage[12pt]{extsizes}
\usepackage[T2A]{fontenc}
\usepackage[utf8]{inputenc}
\usepackage[english, russian]{babel}

\usepackage{amssymb}
\usepackage{amsfonts}
\usepackage{amsmath}
\usepackage{enumitem}
\usepackage{graphics}
\usepackage{graphicx}

\usepackage{lipsum}

\newtheorem{theorem}{Теорема}
\newtheorem{task}{Задача}
\newtheorem{lemma}{Лемма}
\newtheorem{definition}{Определение}
\newtheorem{example}{Пример}
\newtheorem{statement}{Утверждение}
\newtheorem{corollary}{Следствие}


\usepackage{geometry} % Меняем поля страницы
\geometry{left=1cm}% левое поле
\geometry{right=1cm}% правое поле
\geometry{top=1.5cm}% верхнее поле
\geometry{bottom=1cm}% нижнее поле


\usepackage{fancyhdr} % Headers and footers
\pagestyle{fancy} % All pages have headers and footers
\fancyhead{} % Blank out the default header
\fancyfoot{} % Blank out the default footer
\fancyhead[L]{Математика}
\fancyhead[C]{\textit{Разное}}
\fancyhead[R]{1 декабря}% Custom header text


%----------------------------------------------------------------------------------------

%\begin{document}\normalsize
\begin{document}\large
	
\begin{center}
	\textbf{Разнобой УТЮМа 2}
\end{center}


\begin{enumerate}[label*=\protect\fbox{\arabic{enumi}}]
	
\item Даны различные ненулевые цифры $a, b, c, d$. Известно, что ни одно из чисел $\overline{abcd}$, $\overline{bcda}, \overline{cdab}, \overline{dabc}$ не имеет простых делителей, меньших 10. Чему может быть равна сумма этих четырёх четырёхзначных чисел?

\item На доске написано несколько различных неотрицательных чисел. Оказалось, что про- изведение любых двух выписанных чисел также есть на этой доске. Какое наибольшее количество чисел может быть написано?

\item Два равных отрезка $AB$ и $CD$ пересекаются в точке $P$. Точка $M$~--- середина отрезка $BD$. Оказалось, что точка $M$ равноудалена от точек $A$ и $C$. Докажите, что $AP = CP$.

\item Дан клетчатый квадрат $101\times 101$. Внутри него выбирается квадрат $100\times 100$. Внутри этого квадрата выбирается квадрат $99 \times 99$, и так далее, пока не будет выбран квадрат $1\times 1$. Оказалось, что выбранный квадрат $1\times1$ совпадает с центральной клеткой исходного квадрата $101 \times 101$. Сколько существует таких последовательностей квадратов? Ответ не должен содержать знака многоточия.

\item В стране из 1000 городов некоторые города соединены дорогами, по которым можно двигаться в обе стороны. Известно, что в этой стране нет циклического маршрута. При каком наибольшем $k$ всегда можно выбрать $k$ городов так, чтобы каждый выбранный город был соединен не более чем с двумя из остальных выбранных?

\item Серёжа придумал два положительных не целых числа $a$ и $b$. Затем он подсчитал четыре выражения: $a+b, a-b, a\cdot b, \dfrac{a}{b}$. Докажите, что хотя бы одно из получившихся чисел не целое.

\item Даны $36$ различных чисел (не обязательно целых). Докажите, что их можно расставить в клетках таблицы $6 \times 6$ так, чтобы для любых двух чисел, стоящих в соседних по стороне ячейках, их разность была не равна $1$.

\item На какое наибольшее количество нулей может оканчиваться произведение че- тырёхзначного числа, не содержащего в своей записи нулей, на его сумму цифр?

\item На плоскости отмечено $10$ точек. Докажите, что существует не более $90$ равно- бедренных прямоугольных треугольников с вершинами в этих точках.

\item В остроугольном треугольнике $ABC$ проведены высоты $CF$ и $BE$. На отрезке $BE$ нашлась такая точка $P$, что $BP = AC$. На продолжении отрезка $CF$ за точку $F$ нашлась такая точка $Q$, что $CQ = AB$. Докажите, что $AP \perp AQ$.

\end{enumerate}
\end{document}