\documentclass{article}

\usepackage[12pt]{extsizes}
\usepackage[T2A]{fontenc}
\usepackage[utf8]{inputenc}
\usepackage[english, russian]{babel}

\usepackage{mathrsfs}
\usepackage[dvipsnames]{xcolor}

\usepackage{amsmath}
\usepackage{amssymb}
\usepackage{amsthm}
\usepackage{indentfirst}
\usepackage{amsfonts}
\usepackage{enumitem}
\usepackage{graphics}
\usepackage{tikz}
\usepackage{tabu}
\usepackage{diagbox}
\usepackage{hyperref}
\usepackage{mathtools}
\usepackage{ucs}
\usepackage{lipsum}
\usepackage{geometry} % Меняем поля страницы
\usepackage{fancyhdr} % Headers and footers
\newcommand{\range}{\mathrm{range}}
\newcommand{\dom}{\mathrm{dom}}
\newcommand{\N}{\mathbb{N}}
\newcommand{\R}{\mathbb{R}}
\newcommand{\E}{\mathbb{E}}
\newcommand{\D}{\mathbb{D}}
\newcommand{\M}{\mathcal{M}}
\newcommand{\Prime}{\mathbb{P}}
\newcommand{\A}{\mathbb{A}}
\newcommand{\Q}{\mathbb{Q}}
\newcommand{\Z}{\mathbb{Z}}
\newcommand{\F}{\mathbb{F}}
\newcommand{\CC}{\mathbb{C}}

\DeclarePairedDelimiter\abs{\lvert}{\rvert}
\DeclarePairedDelimiter\floor{\lfloor}{\rfloor}
\DeclarePairedDelimiter\ceil{\lceil}{\rceil}
\DeclarePairedDelimiter\lr{(}{)}
\DeclarePairedDelimiter\set{\{}{\}}
\DeclarePairedDelimiter\norm{\|}{\|}

\renewcommand{\labelenumi}{(\alph{enumi})}

\newcommand{\smallindent}{
    \geometry{left=1cm}% левое поле
    \geometry{right=1cm}% правое поле
    \geometry{top=1.5cm}% верхнее поле
    \geometry{bottom=1cm}% нижнее поле
}

\newcommand{\header}[3]{
    \pagestyle{fancy} % All pages have headers and footers
    \fancyhead{} % Blank out the default header
    \fancyfoot{} % Blank out the default footer
    \fancyhead[L]{#1}
    \fancyhead[C]{#2}
    \fancyhead[R]{#3}
}

\newcommand{\dividedinto}{
    \,\,\,\vdots\,\,\,
}

\newcommand{\littletaller}{\mathchoice{\vphantom{\big|}}{}{}{}}

\newcommand\restr[2]{{
    \left.\kern-\nulldelimiterspace % automatically resize the bar with \right
    #1 % the function
    \littletaller % pretend it's a little taller at normal size
    \right|_{#2} % this is the delimiter
}}

\DeclareGraphicsExtensions{.pdf,.png,.jpg}

\newenvironment{enumerate_boxed}[1][enumi]{\begin{enumerate}[label*=\protect\fbox{\arabic{#1}}]}{\end{enumerate}}



\smallindent

\header{Математика}{\textit{Теория чисел}}{4 сентября 2022}


%----------------------------------------------------------------------------------------

\begin{document}
    \large

    \begin{center}
        \textbf{Китайская теорема об остатках}
    \end{center}

    \begin{enumerate_boxed}

        \item Найдите остаток от деления натурального числа на 115, если известно, что остаток от деления его на 5 равен 3, а остаток от деления на 23 равен 17.

        \item Найдите остаток от деления натурального числа на 105, если известно, что остаток от деления его на 3 равен 1, остаток от деления на 5 равен 2, а остаток от деления на 7 равен 1.

        \item Найдите остаток от деления натурального числа на 30, если известно, что остаток от деления его на 15 равен 7, а остаток от деления на 6 равен 4.

        \item Генерал построил солдат в колонну по 4, но при этом солдат Иванов остался лишним.
        Тогда генерал построил солдат в колонну по 5.
        И снова Иванов остался лишним.
        Когда же и в колонне по 6 Иванов оказался лишним, генерал посулил ему наряд вне очереди, после чего в колонне по 7 Иванов нашел себе место и никого лишнего не осталось.
        Какое наименьшее число солдат могло быть у генерала?

        \item При каких целых $n$ число $n^2 + 3n + 1$ делится на $55$?

        \item Назовем число хорошим, если оно делится на квадрат натурального числа $> 1$.
        При каких $N$ найдется $N$ последовательных хороших чисел?
        (Пример для $N = 3$: $48, 49, 50$).

        \item Докажите, что найдутся $100$ последовательных чисел, каждое из которых не является простым числом или степенью простого числа;

        \item Назовём натуральное число \textit{свободным от квадратов}, если оно не делится ни на один квадрат целого числа.
        Докажите, что существует $1000$ последовательных натуральных чисел, среди которых нет ни одного числа, свободного от квадратов.

        \item Назовём натуральное число \textit{позитивным}, если оно не является степенью натурального числа.
        Докажите, что при любом натуральном $n$ можно указать $n$ последовательных \textit{позитивных} чисел.

        \item Докажите, что найдутся $2022$ последовательных натуральных чисел, каждое из которых имеет не менее трех различных простых делителей.

        \item Докажите, что найдутся гуголплекс $\left(10^{10^{100}}\right) $ последовательных натуральных чисел, каждое из которых имеет не менее гугол $\left(10^{100}\right)$ различных простых делителей.

    \end{enumerate_boxed}
\end{document}