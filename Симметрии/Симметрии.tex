\documentclass{article}

\usepackage[12pt]{extsizes}
\usepackage[T2A]{fontenc}
\usepackage[utf8]{inputenc}
\usepackage[english, russian]{babel}

\usepackage{mathrsfs}
\usepackage[dvipsnames]{xcolor}

\usepackage{amsmath}
\usepackage{amssymb}
\usepackage{amsthm}
\usepackage{indentfirst}
\usepackage{amsfonts}
\usepackage{enumitem}
\usepackage{graphics}
\usepackage{tikz}
\usepackage{tabu}
\usepackage{diagbox}
\usepackage{hyperref}
\usepackage{mathtools}
\usepackage{ucs}
\usepackage{lipsum}
\usepackage{geometry} % Меняем поля страницы
\usepackage{fancyhdr} % Headers and footers
\newcommand{\range}{\mathrm{range}}
\newcommand{\dom}{\mathrm{dom}}
\newcommand{\N}{\mathbb{N}}
\newcommand{\R}{\mathbb{R}}
\newcommand{\E}{\mathbb{E}}
\newcommand{\D}{\mathbb{D}}
\newcommand{\M}{\mathcal{M}}
\newcommand{\Prime}{\mathbb{P}}
\newcommand{\A}{\mathbb{A}}
\newcommand{\Q}{\mathbb{Q}}
\newcommand{\Z}{\mathbb{Z}}
\newcommand{\F}{\mathbb{F}}
\newcommand{\CC}{\mathbb{C}}

\DeclarePairedDelimiter\abs{\lvert}{\rvert}
\DeclarePairedDelimiter\floor{\lfloor}{\rfloor}
\DeclarePairedDelimiter\ceil{\lceil}{\rceil}
\DeclarePairedDelimiter\lr{(}{)}
\DeclarePairedDelimiter\set{\{}{\}}
\DeclarePairedDelimiter\norm{\|}{\|}

\renewcommand{\labelenumi}{(\alph{enumi})}

\newcommand{\smallindent}{
    \geometry{left=1cm}% левое поле
    \geometry{right=1cm}% правое поле
    \geometry{top=1.5cm}% верхнее поле
    \geometry{bottom=1cm}% нижнее поле
}

\newcommand{\header}[3]{
    \pagestyle{fancy} % All pages have headers and footers
    \fancyhead{} % Blank out the default header
    \fancyfoot{} % Blank out the default footer
    \fancyhead[L]{#1}
    \fancyhead[C]{#2}
    \fancyhead[R]{#3}
}

\newcommand{\dividedinto}{
    \,\,\,\vdots\,\,\,
}

\newcommand{\littletaller}{\mathchoice{\vphantom{\big|}}{}{}{}}

\newcommand\restr[2]{{
    \left.\kern-\nulldelimiterspace % automatically resize the bar with \right
    #1 % the function
    \littletaller % pretend it's a little taller at normal size
    \right|_{#2} % this is the delimiter
}}

\DeclareGraphicsExtensions{.pdf,.png,.jpg}

\newenvironment{enumerate_boxed}[1][enumi]{\begin{enumerate}[label*=\protect\fbox{\arabic{#1}}]}{\end{enumerate}}



\smallindent

\header{Математика}{\textit{Геометрия}}{12 сентября 2022}

%----------------------------------------------------------------------------------------

\begin{document}
    \large

    \begin{center}
        \textbf{Симметрии}
    \end{center}

    \begin{enumerate_boxed}

        \item Можно ли так согнуть листок бумаги (листок не обязательно прямоугольной формы), чтобы одним разрезом вырезать в нём квадратную дырку?

        \item Существует ли фигура на плоскости, имеющая среди своих осей симметрии две параллельные не совпадающие прямые?

        \item Верно ли следующее утверждение: <<Если четырёхугольник имеет ось симметрии, то это либо равнобедренная трапеция, либо прямоугольник, либо ромб>>?

        \item Города $A$ и $B$ находятся по одну сторону от реки $l$ (будем считать реку прямой).
        Вася хочет выйти из города $A$, набрать воды в реке $l$ и вернуться в город $B$.
        Покажите как нужно идти Васе, чтобы пройти наименьшее расстояние и достичь своей цели.

        \item На стороне $BC$ треугольника $ABC$ выбрали точки $P$ и $Q$ такие, что $BA=BP$ и $CA=CQ$.
        Точка $I$ — точка пересечения биссектрис треугольника $ABC$.
        Докажите, что треугольник $IPQ$ равнобедренный.

        \item  В треугольнике $ABC$ угол $A$ равен $60^\circ$.
        Серединный перпендикуляр к отрезку $AB$ пересекает прямую $AC$ в точке $B_1$.
        Серединный перпендикуляр к отрезку $AC$ пересекает прямую $AB$ в точке $C_1$.
        Докажите, что расстояния от точки пересечения биссектрис треугольника $ABC$ до прямых $BC$ и $B_{1}C_1$ равны.

        \item Биссектрисы треугольника $ABC$ пересекаются в точке $I$, $\angle ABC=120^\circ$.
        На продолжениях сторон $AB$ и $CB$ за точку $B$ отмечены точки $P$ и $Q$ соответственно так, что $AP=CQ=AC$.
        Докажите, что угол $PIQ$ — прямой.

        \item В равнобедренном треугольнике $ABC$ угол при основании $BC$ равен $80^\circ$.
        На боковых сторонах $AB$ и $AC$ выбраны точки $D$ и $E$ соответственно, причём $\angle BCD=50^\circ$, $\angle CBE=40^\circ$.
        Найдите угол между прямыми $DE$ и $BC$.

        \item В треугольнике $ABC$ угол $A$ равен $60^\circ$.
        На лучах $BA$ и $CA$ отложены отрезки $BX$ и $CY$, равные стороне $BC$.
        Докажите, что прямая $XY$ проходит через точку пересечения биссектрис треугольника $ABC$.

        \item На стороне $AC$ треугольника $ABC$ выбрали точки $K$ и $L$ так, что $L$ — середина отрезка $AK$, $BK$ — биссектриса угла $LBC$ и $BC=2BL$.
        Докажите, что $\angle ACB= \angle ABL$.

        \item Дан треугольник $ABC$. $M$ — середина стороны $BC$, а $P$ — проекция вершины $B$ на серединный перпендикуляр к $AC$.
        Прямая $PM$ пересекает сторону $AB$ в точке $Q$.
        Докажите, что треугольник $QBP$ равнобедренный.


    \end{enumerate_boxed}
\end{document}