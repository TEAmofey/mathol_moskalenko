\documentclass{article}
\usepackage[12pt]{extsizes}
\usepackage[T2A]{fontenc}
\usepackage[utf8]{inputenc}
\usepackage[english, russian]{babel}

\usepackage{amssymb}
\usepackage{amsfonts}
\usepackage{amsmath}
\usepackage{enumitem}
\usepackage{graphics}

\usepackage{lipsum}



\usepackage{geometry} % Меняем поля страницы
\geometry{left=1cm}% левое поле
\geometry{right=1cm}% правое поле
\geometry{top=1.5cm}% верхнее поле
\geometry{bottom=1cm}% нижнее поле


\usepackage{fancyhdr} % Headers and footers
\pagestyle{fancy} % All pages have headers and footers
\fancyhead{} % Blank out the default header
\fancyfoot{} % Blank out the default footer
\fancyhead[L]{Математика}
\fancyhead[C]{\textit{Геометрия}}
\fancyhead[R]{12 сентября}% Custom header text


%----------------------------------------------------------------------------------------

%\begin{document}\normalsize
\begin{document}\large


\begin{center}
\textbf{Симметрии}
\end{center}



\begin{enumerate}[label*=\protect\fbox{\arabic{enumi}}]

\item Можно ли так согнуть листок бумаги (листок не обязательно прямоугольной формы), чтобы одним разрезом вырезать в нём квадратную дырку?

\item Существует ли фигура на плоскости, имеющая среди своих осей симметрии две параллельные не совпадающие прямые?

\item Верно ли следующее утверждение: "Если четырёхугольник имеет ось симметрии, то это либо равнобедренная трапеция, либо прямоугольник, либо ромб"?

\item Города $A$ и $B$ находятся по одну сторону от реки $l$ (будем считать реку прямой). Вася хочет выйти из города $A$, набрать воды в реке $l$ и вернуться в город $B$. Покажите как нужно идти Васе, чтобы пройти наименьшее расстояние и достичь своей цели.

\item На стороне $BC$ треугольника $ABC$ выбрали точки $P$ и $Q$ такие, что $BA=BP$ и $CA=CQ$. Точка $I$ — точка пересечения биссектрис треугольника $ABC$. Докажите, что треугольник $IPQ$ равнобедренный.

\item  В треугольнике $ABC$ угол $A$ равен $60^\circ$. Серединный перпендикуляр к отрезку $AB$ пересекает прямую $AC$ в точке $B_1$. Серединный перпендикуляр к отрезку $AC$ пересекает прямую $AB$ в точке $C_1$. Докажите, что расстояния от точки пересечения биссектрис треугольника $ABC$ до прямых $BC$ и $B_1C_1$ равны.

\item Биссектрисы треугольника $ABC$ пересекаются в точке $I$, $\angle ABC=120^\circ$. На продолжениях сторон $AB$ и $CB$ за точку $B$ отмечены точки $P$ и $Q$ соответственно так, что $AP=CQ=AC$. Докажите, что угол $PIQ$ — прямой.

\item В равнобедренном треугольнике $ABC$ угол при основании $BC$ равен $80^\circ$. На боковых сторонах $AB$ и $AC$ выбраны точки $D$ и $E$ соответственно, причём $\angle BCD=50^\circ$, $\angle CBE=40^\circ$. Найдите угол между прямыми $DE$ и $BC$.

\item В треугольнике $ABC$ угол $A$ равен $60^\circ$. На лучах $BA$ и $CA$ отложены отрезки $BX$ и $CY$, равные стороне $BC$. Докажите, что прямая $XY$ проходит через точку пересечения биссектрис треугольника $ABC$.

\item На стороне $AC$ треугольника $ABC$ выбрали точки $K$ и $L$ так, что $L$ — середина отрезка $AK$, $BK$ — биссектриса угла $LBC$ и $BC=2BL$. Докажите, что $\angle ACB= \angle ABL$.

\item Дан треугольник $ABC$. $M$ — середина стороны $BC$, а $P$ — проекция вершины $B$ на серединный перпендикуляр к $AC$. Прямая $PM$ пересекает сторону $AB$ в точке $Q$. Докажите, что треугольник $QBP$ равнобедренный.





\end{enumerate}
\end{document}