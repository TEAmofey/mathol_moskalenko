\documentclass{article}

\usepackage[12pt]{extsizes}
\usepackage[T2A]{fontenc}
\usepackage[utf8]{inputenc}
\usepackage[english, russian]{babel}

\usepackage{mathrsfs}
\usepackage[dvipsnames]{xcolor}

\usepackage{amsmath}
\usepackage{amssymb}
\usepackage{amsthm}
\usepackage{indentfirst}
\usepackage{amsfonts}
\usepackage{enumitem}
\usepackage{graphics}
\usepackage{tikz}
\usepackage{tabu}
\usepackage{diagbox}
\usepackage{hyperref}
\usepackage{mathtools}
\usepackage{ucs}
\usepackage{lipsum}
\usepackage{geometry} % Меняем поля страницы
\usepackage{fancyhdr} % Headers and footers
\newcommand{\range}{\mathrm{range}}
\newcommand{\dom}{\mathrm{dom}}
\newcommand{\N}{\mathbb{N}}
\newcommand{\R}{\mathbb{R}}
\newcommand{\E}{\mathbb{E}}
\newcommand{\D}{\mathbb{D}}
\newcommand{\M}{\mathcal{M}}
\newcommand{\Prime}{\mathbb{P}}
\newcommand{\A}{\mathbb{A}}
\newcommand{\Q}{\mathbb{Q}}
\newcommand{\Z}{\mathbb{Z}}
\newcommand{\F}{\mathbb{F}}
\newcommand{\CC}{\mathbb{C}}

\DeclarePairedDelimiter\abs{\lvert}{\rvert}
\DeclarePairedDelimiter\floor{\lfloor}{\rfloor}
\DeclarePairedDelimiter\ceil{\lceil}{\rceil}
\DeclarePairedDelimiter\lr{(}{)}
\DeclarePairedDelimiter\set{\{}{\}}
\DeclarePairedDelimiter\norm{\|}{\|}

\renewcommand{\labelenumi}{(\alph{enumi})}

\newcommand{\smallindent}{
    \geometry{left=1cm}% левое поле
    \geometry{right=1cm}% правое поле
    \geometry{top=1.5cm}% верхнее поле
    \geometry{bottom=1cm}% нижнее поле
}

\newcommand{\header}[3]{
    \pagestyle{fancy} % All pages have headers and footers
    \fancyhead{} % Blank out the default header
    \fancyfoot{} % Blank out the default footer
    \fancyhead[L]{#1}
    \fancyhead[C]{#2}
    \fancyhead[R]{#3}
}

\newcommand{\dividedinto}{
    \,\,\,\vdots\,\,\,
}

\newcommand{\littletaller}{\mathchoice{\vphantom{\big|}}{}{}{}}

\newcommand\restr[2]{{
    \left.\kern-\nulldelimiterspace % automatically resize the bar with \right
    #1 % the function
    \littletaller % pretend it's a little taller at normal size
    \right|_{#2} % this is the delimiter
}}

\DeclareGraphicsExtensions{.pdf,.png,.jpg}

\newenvironment{enumerate_boxed}[1][enumi]{\begin{enumerate}[label*=\protect\fbox{\arabic{#1}}]}{\end{enumerate}}



\smallindent

\header{Математика}{\textit{Олимпиадная подготовка}}{25 сентября 2023}

%----------------------------------------------------------------------------------------

\begin{document}
    \large

    \begin{center}
        \textbf{Занятие 1}
    \end{center}


    \begin{enumerate_boxed}


        \item Окружности $S_1$ и $S_2$ касаются внешним образом в точке $F$.
        Прямая $\ell$ касается $S_1$ и $S_2$ в точках $A$ и $B$ соответственно.
        Прямая, параллельная прямой $\ell$, касается $S_2$ в точке $C$ и пересекает $S_1$ в двух точках.
        Докажите, что точки $A, F$ и $C$ лежат на одной прямой.

        \item Петя нашёл сумму всех нечётных делителей некоторого чётного числа (включая 1), а Вася --- сумму всех чётных делителей этого же числа (включая само число).
        Может ли произведение двух найденных чисел быть точным квадратом?

        \item Целые числа $a,x_1,x_2,\dotsc, x_{13}$ таковы, что
        \[a = (1 + x_1) (1 + x_2) \cdots (1 + x_{13}) = (1 - x_1) (1 - x_2) \cdots (1 - x_{13}). \]
        Докажите, что $ax_{1}x_2 \cdots x_{13} = 0.$

        \begin{center}
            \textbf{Домашка}
        \end{center}

        \item В один из дней года оказалось, что каждый житель города сделал не более одного звонка по телефону.
        Докажите, что население города можно разбить не более, чем на три группы так, чтобы жители, входящие в одну группу, не разговаривали в этот день между собой по телефону.


        \item Квадрат $11 \times 11$ разрезали на \texttt{Z}-тетрамино и единичные квадратики. \texttt{Z}-тетрамино можно поворачивать и переворачивать.
        Какое наименьшее число единичных квадратиков может быть в таком разрезании?

        \item Вокруг остроугольного треугольника $ABC$ описана окружность.
        Точка $K$ — середина меньшей дуги $AC$ этой окружности, а точка $L$ — середина меньшей дуги $AK$ этой окружности.
        Отрезки $BK$ и $AC$ пересекаются в точке $P$.
        Найдите угол между прямыми $BC$ и $LP$, если известно, что $BK = BC$.


        \item Внутри выпуклого пятиугольника отметили точку и соединили её со всеми вершинами.
        Какое наибольшее число из десяти проведенных отрезков (пяти сторон и пяти отрезков, соединяющих отмеченную точку с вершинами пятиугольника) может иметь длину 1?

        \item При каких натуральных $n$ клетчатую доску $n \times n$ можно разрезать на прямоугольники $1 \times 1$ и $2 \times 3$ так, что прямоугольников разных размеров будет поровну?

    \end{enumerate_boxed}
\end{document}