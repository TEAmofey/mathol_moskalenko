\documentclass{article}
\usepackage[12pt]{extsizes}
\usepackage[T2A]{fontenc}
\usepackage[utf8]{inputenc}
\usepackage[english, russian]{babel}

\usepackage{amssymb}
\usepackage{amsfonts}
\usepackage{amsmath}
\usepackage{enumitem}
\usepackage{graphics}
\usepackage{graphicx}

\usepackage{lipsum}

\newtheorem{theorem}{Теорема}
\newtheorem{task}{Задача}
\newtheorem{lemma}{Лемма}
\newtheorem{definition}{Определение}
\newtheorem{example}{Пример}
\newtheorem{statement}{Утверждение}
\newtheorem{corollary}{Следствие}


\usepackage{geometry} % Меняем поля страницы
\geometry{left=1cm}% левое поле
\geometry{right=1cm}% правое поле
\geometry{top=1.5cm}% верхнее поле
\geometry{bottom=1cm}% нижнее поле


\usepackage{fancyhdr} % Headers and footers
\pagestyle{fancy} % All pages have headers and footers
\fancyhead{} % Blank out the default header
\fancyfoot{} % Blank out the default footer
\fancyhead[L]{Математика}
\fancyhead[C]{\textit{ТЧ}}
\fancyhead[R]{}% Custom header text


%----------------------------------------------------------------------------------------

%\begin{document}\normalsize
\begin{document}\large
	
\begin{center}
	\textbf{Региональный разнобой}
\end{center}


\begin{enumerate}[label*=\protect\fbox{\arabic{enumi}}]
	
%23.9.4
\item Даны натуральные числа $a, b$ и $c$. Ни одно из них не кратно
другому. Известно, что число $abc + 1$ делится на $ab - b + 1$. Докажите, что $c > b$. 

%23.9.6
\item Для натурального числа n обозначим через $S_n$ наименьшее общее кратное всех чисел $1, 2, \dotsc , n$. Существует ли такое натуральное число $m$, что $S_{m+1} = 4S_m$? 

%23.9.7
\item  На доску записали $99$ чисел, среди которых нет равных. В тетрадку выписали $\dfrac{99 \cdot 98}{2}$
чисел — все разности двух чисел с доски (каждый раз из большего числа вычитали меньшее). Оказалось, что в тетрадке число 1 записано ровно 85 раз. Пусть $d$ — наибольшее число, записанное в тетрадке. Найдите наименьшее возможное значение $d$. 

%22.9.2
\item На доске девять раз (друг под другом) написали некоторое натуральное число $N$. Петя к каждому из 9 чисел приписал слева или справа одну ненулевую цифру; при этом все приписанные цифры различны. Какое наибольшее количество простых чисел могло оказаться среди 9 полученных чисел?

%22.9.10
\item Докажите, что существует натуральное число $b$ такое, что при любом натуральном $n > b$ сумма цифр числа $n!$ не меньше $10^{100}$.

%21.9.3
\item Рассмотрим такие натуральные числа $a, b$ и $c$, что дробь $$k = \frac{ab+c^2}{a + b}$$ является натуральным числом, меньшим $a$ и $b$. Какое наименьшее количество натуральных делителей может быть у числа $a + b$? 

%21.9.6
\item. Десятизначные натуральные числа $a, b, c$ таковы, что $a + b = c$. Какое наибольшее количество из 30 их цифр могут оказаться нечётными?

%20.9.4
\item Пусть $p$ простое число, большее 3. Докажите, что найдётся натуральное число $y$, меньшее $p/2$ и такое, что число $py + 1$ невозможно представить в виде произведения двух целых чисел, каждое из которых больше $y$.

\item


\end{enumerate}
\end{document}