\documentclass{article}
\usepackage[12pt]{extsizes}
\usepackage[T2A]{fontenc}
\usepackage[utf8]{inputenc}
\usepackage[english, russian]{babel}

\usepackage{amssymb}
\usepackage{amsfonts}
\usepackage{amsmath}
\usepackage{enumitem}
\usepackage{graphics}

\usepackage{lipsum}



\usepackage{geometry} % Меняем поля страницы
\geometry{left=1cm}% левое поле
\geometry{right=1cm}% правое поле
\geometry{top=1.5cm}% верхнее поле
\geometry{bottom=1cm}% нижнее поле


\newtheorem{definition}{Опредление}
\usepackage{fancyhdr} % Headers and footers
\pagestyle{fancy} % All pages have headers and footers
\fancyhead{} % Blank out the default header
\fancyfoot{} % Blank out the default footer
\fancyhead[L]{Математика}
\fancyhead[C]{\textit{Геометрия}}
\fancyhead[R]{13 августа 2023}% Custom header text


%----------------------------------------------------------------------------------------

%\begin{document}\normalsize
\begin{document}\large



\begin{center}
\textbf{Изогонали}
\end{center}

\textbf{Определение:} Пусть дан угол $\angle AOB$. Две прямые, проходящие через вершину $O$, называются изогоналями относительно этого угла (или парой изогональных прямых), если они получаются друг из друга отражением относительно биссектрисы этого угла. Например, высота треугольника из вершины $A$ и диаметр описанной окружности, содержащий $A$, являются изогоналями.

Пусть даны угол $\angle AOB$ и луч, выходящий из его вершины и проходящий внутри угла. Тогда во всех точках луча отношение расстояний до сторон угла будет одним и тем же; и наоборот, такое отношение будет задавать этот луч однозначно. Обратные отношения при этом будут соответствовать лучам, содержащимся в изогоналях.


\begin{enumerate}[label*=\protect\fbox{\arabic{enumi}}]

\item Докажите, что касательная к описанной окружности треугольника $ABC$, проведенная в точке $A$, и прямая, проходящая через точку $A$ параллельно $BC$, являются изогоналями относительно угла $A$ треугольника.

\item Внутри угла $ABC$ лежит точка $P$. Точки $P_B$ и $P_C$ симметричны точке $P$ относительно $AB$ и $AC$ соответственно. Докажите, что серединный перпендикуляр к $P_B P_C$ является изогональю для прямой $AP$ относительно угла $BAC$.

\item Внутри угла $BAC$ лежат точки $P$ и $Q$. Докажите, что проекции точек $P$ и $Q$ на стороны угла лежат на одной окружности тогда и только тогда, когда $AP$ и $AQ$ являются изогоналями относительно угла $BAC$.

\item Точка $P$ лежит на описанной окружности треугольника $ABC$. Докажите, что изогональ для $PA$ относительно угла $A$ и изогональ для $PB$ относительно угла $B$ параллельны.

\item \textbf{Лемма об изогоналях} Пусть дан угол $AOB$ и изогонали $OC$ и $OD$ относительно этого угла. Обозначим пересечение прямых $AD$ и $BC$ за $P$, а пересечение прямых $AC$ и $BD$ за $Q$. Тогда прямые $OP$ и $OQ$ также являются изогоналями относительно угла $AOB$.

\item В треугольнике $ABC$ из вершин к противолежащим сторонам проведены отрезки $AA_1$, $BB_1$, $CC_1$, пересекающиеся в одной точке. Докажите, что если углы $C_1A_1B$ и $B_1A_1C$ равны, то $AA_1$ — высота треугольника $ABC$.

\item В треугольнике $ABC$ проведены биссектрисы $AA_1$, $BB_1$, $CC_1$. Отрезок $A_1C_1$ пересекает биссектрису угла $B$ в точке $P$, а отрезок $A_1B_1$ биссектрису угла $C$ в точке $Q$. Докажите, что $\angle PAB = \angle QAC$.

\item На сторонах $ABC$ остроугольного треугольника $ABC$ внешним образом построены квадраты $ABFE$ и $ACGH$. Докажите, что точка $P$ пересечения прямых $CF$ и $BG$ лежит на высоте треугольника $ABC$, проведённой из вершины $A$.

\item a) Касательные к описанной окружности треугольника $ABC$, проведённые в вершинах $B$ и $C$, пересекаются в точке $X$. Докажите, что прямая $AX$ и прямая, содержащая медиану к стороне $BC$, являются изогоналями относительно угла $A$.

b) К описанной окружности треугольника $ABC$ проведены касательные в точках $B$ и $C$. Лучи $CC_1$ и $BB_1$, где $B_1$ и $C_1$ — середины сторон $AC$ и $AB$, пересекают эти касательные в точках $K$ и $L$ соответственно. Докажите, что $\angle BAK=\angle CAL$.

\item Дан вписанный четырёхугольник $ABCD$. Диагонали $AC$ и $BD$ пересекаются в точке $E$, а прямые $AB$ и $CD$ пересекаются в точке $F$. На прямой $EF$ взяли такую точку $P$, что $\angle BPE=\angle EPC$. Докажите, что $\angle APE=\angle DPE$.

\item В выпуклом четырёхугольнике $ABCD$ точки $I$ и $K$ — центры вписанных окружностей треугольников $ABC$ и $ACD$ соответственно, а $J$ и $L$ — центры их вневписанных окружностей, касающихся сторон $BC$ и $CD$ соответственно. Докажите, что прямые $IL$ и $JK$ пересекаются на биссектрисе угла $BCD$.

\textbf{Задачи на вырожденные случаи:}

\item В остроугольном треугольнике \(ABC\) выполнено соотношение \(AB < AC\). На стороне \(BC\) выбраны такие точки \(D\) и \(E\), что \(BD = CE < \frac{BC}{2}\). Точка \(P\) внутри треугольника такова, что \(PD \parallel AE\) и \(\angle PAB = \angle EAC\). Докажите, что \(\angle PBA = \angle PCA\).

\item Вершины \(B\) и \(C\) треугольника \(ABC\) спроецировали на биссектрису внешнего угла \(A\), получили точки \(B_1\) и \(C_1\) соответственно. Докажите, что прямые \(BC_1\) и \(CB_1\) пересекаются на внутренней биссектрисе угла \(A\).

\item В трапеции \(ABCD\) боковая сторона \(CD\) перпендикулярна основаниям, \(O\) — точка пересечения диагоналей. На описанной окружности \(OCD\) взята точка \(S\), диаметрально противоположная точке \(O\). Докажите, что \(\angle BSC = \angle ASD\).

\item На прямой, содержащей высоту остроугольного треугольника \(ABC\), проведённую к стороне \(BC\), выбрали точку \(X\). Точка \(D\) — середина дуги \(BC\) описанной окружности \(ABC\), не содержащей точку \(A\). Прямая, проходящая через центр окружности параллельно \(AD\), пересекает прямую \(XD\) в точке \(N\). Точка \(M\) — середина отрезка \(XD\). Докажите, что \(\angle XAM = \angle NAO\).

\textbf{Задачи посложнее:}

\item Диагонали вписанного четырёхугольника \(ABCD\) пересекаются в точке \(P\). Точки \(E\) и \(F\) — проекции точки \(P\) на стороны \(AB\) и \(CD\) соответственно. Отрезки \(CE\) и \(BF\) пересекаются в точке \(Q\). Докажите, что \(PQ\) и \(EF\) перпендикулярны.

\item а) Внутри треугольника \(ABC\) выбраны такие точки \(X\), \(Y\), \(Z\), что
\(\angle BAZ = \angle CAY\), \(\angle ABZ = \angle CBX\), \(\angle ACY = \angle BCX\).
Докажите, что прямые \(AX\), \(BY\), \(CZ\) пересекаются в одной точке.

б) Обозначим через \(Y'\) и \(Z'\) точки пересечения пар прямых \(AZ\) и \(CX\), \(AY\) и \(BX\). Докажите, что прямые \(YZ\), \(Y'Z'\), \(BC\) пересекаются в одной точке.

\item Точка \(H'\) симметрична основанию высоты \(AH\) треугольника \(ABC\) относительно середины стороны \(BC\). Касательные к описанной окружности треугольника \(ABC\) в точках \(B\) и \(C\) пересекаются в точке \(X\). Прямая, проходящая через \(H'\) и перпендикулярная \(XH'\), пересекает прямые \(AB\) и \(AC\) в точках \(Y\) и \(Z\) соответственно. Докажите, что \(\angle BXY = \angle CXZ\).

\item Вписанная окружность треугольника \(ABC\) касается сторон \(BC\), \(CA\) и \(AB\) в точках \(D\), \(E\) и \(F\) соответственно. Точка \(K\) является проекцией точки \(D\) на прямую \(EF\). Точка \(H\) — ортоцентр треугольника \(ABC\), точка \(A'\) диаметрально противоположна \(A\) в описанной окружности треугольника \(ABC\). Докажите, что \(DK\) — биссектриса угла \(HKA'\).

\end{enumerate}
\end{document}