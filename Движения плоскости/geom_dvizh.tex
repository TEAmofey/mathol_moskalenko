\documentclass{article}
\usepackage[12pt]{extsizes}
\usepackage[T2A]{fontenc}
\usepackage[utf8]{inputenc}
\usepackage[english, russian]{babel}

\usepackage{amssymb}
\usepackage{amsfonts}
\usepackage{amsmath}
\usepackage{enumitem}
\usepackage{graphics}

\usepackage{lipsum}



\usepackage{geometry} % Меняем поля страницы
\geometry{left=1cm}% левое поле
\geometry{right=1cm}% правое поле
\geometry{top=1.5cm}% верхнее поле
\geometry{bottom=1cm}% нижнее поле


\usepackage{fancyhdr} % Headers and footers
\pagestyle{fancy} % All pages have headers and footers
\fancyhead{} % Blank out the default header
\fancyfoot{} % Blank out the default footer
\fancyhead[L]{ЦРОД $\bullet$ Математика}
\fancyhead[C]{\textit{Геометрия}}
\fancyhead[R]{Стратегия 2021}% Custom header text


%----------------------------------------------------------------------------------------

%\begin{document}\normalsize
\begin{document}\large


\begin{center}
\textbf{Движения}
\end{center}



\begin{enumerate}[label*=\protect\fbox{\arabic{enumi}}]

\item Существует ли а) ограниченная, б) неограниченная фигура на плоскости, имеющая среди своих осей симметрии две параллельные не совпадающие прямые?

\item Верно ли следующее утверждение: "Если четырёхугольник имеет ось симметрии, то это либо равнобедренная трапеция, либо прямоугольник, либо ромб"?

\item Города $A$ и $B$ находятся по одну сторону от реки $l$ (будем считать реку прямой). Вася хочет выйти из города $A$, набрать воды в реке $l$ и вернуться в город $B$. Покажите как нужно идти Васе, чтобы пройти наименьшее расстояние и достичь своей цели.

\item Два равносторонних треугольника $ABC$ и $CDE$ имеют общую вершину. Найдите угол между прямыми $AD$ и $BE$.

\item Два квадрата $ABCD$ и $DEFG$ имеют общую вершину. Докажите, что $AE \perp GC$.

\item На сторонах треугольника $ABC$ внешним образом построены правильные треугольники $A_1BC$, $AB_1C$ и $ABC_1$. Докажите, что $AA_1 = BB_1 = CC_1$. 

\item Рассмотрим всевозможные равносторонние треугольники $PKM$, вершина $P$ которых фиксирована, а вершина $K$ лежит в данном квадрате. Найдите геометрическое место вершин $M$.

\item В прямоугольном треугольнике $ABC$ с прямым углом $C$ отмечена точка $N$ на стороне $BC$
так, что $\angle ANC=\angle BNM$, где $M$~--- середина стороны $AB$. Докажите, что точка $N$ делит отрезок $BC$ в отношении $2:1$.

\item Есть четырёхугольник $ABCD$, симметричный относительно своей диагонали $AC$. На его стороне $AB$ построили равносторонний треугольник $AEB$	во внешнюю сторону, а на стороне $BC$ — равносторонний треугольник $BCF$ во внутреннюю сторону. Докажите, что точки $E, F$ и $D$ лежат на одной прямой.

\item Докажите, что прямые, проведенные через середины сторон вписанного четырехугольника перпендикулярно противоположным сторонам, пересекаются в одной точке. 

\item Точки $K$ и $L$ – середины сторон $AB$ и $BC$ правильного шестиугольника $ABCDEF$. Отрезки $KD$ и $LE$ пересекаются в точке $M$. Площадь треугольника $DEM$ равна $12$. Найдите площадь четырёхугольника $KBLM$.

\item В квадрате $ABCD$ отмечена точка $P$ на стороне $BC$ и точка $Q$ на стороне $CD$. Докажите, что равенство $PQ = BP + DQ$ выполнено тогда и только тогда, когда $\angle PAQ = 45^\circ$

\item На сторонах $AB, BC, CD, AD$ квадрата $ABCD$ отметили такие точки $K, L, M, N$ соответственно, что $AK=AN=BL=CM$. Докажите, что $\angle LMC= \angle MKN$

\item На сторонах $BC$ и $CD$ квадрата $ABCD$ выбраны точки $P$ и $Q$ соответственно таким образом, что $\angle PAQ= \angle QAD$. Докажите, что $AP=DQ+BP$.

\item Точка $E$ расположена на диаметре $AB$ окружности радиуса $R$. Точки $C$ и $D$ лежат на окружности в одной полуплоскости относительно $AB$ так, что $\angle DEA=\angle CEB=60^\circ.$ Найдите длину отрезка $CD$.

\item Внутри равнобедренного треугольника $ABC$ с основанием $AC$ отмечена точка $D$ такая, что $\angle ADC=2\angle ABC$. Через точку $D$ проведена прямая — внешняя биссектриса треугольника $ADC$. Докажите, что расстояние от точки $B$ до этой прямой ровно в два раза меньше, чем $AD+BC$.






\end{enumerate}
\end{document}