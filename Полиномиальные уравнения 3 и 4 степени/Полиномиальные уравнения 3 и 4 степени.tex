\documentclass{article}

\usepackage[12pt]{extsizes}
\usepackage[T2A]{fontenc}
\usepackage[utf8]{inputenc}
\usepackage[english, russian]{babel}

\usepackage{mathrsfs}
\usepackage[dvipsnames]{xcolor}

\usepackage{amsmath}
\usepackage{amssymb}
\usepackage{amsthm}
\usepackage{indentfirst}
\usepackage{amsfonts}
\usepackage{enumitem}
\usepackage{graphics}
\usepackage{tikz}
\usepackage{tabu}
\usepackage{diagbox}
\usepackage{hyperref}
\usepackage{mathtools}
\usepackage{ucs}
\usepackage{lipsum}
\usepackage{geometry} % Меняем поля страницы
\usepackage{fancyhdr} % Headers and footers
\newcommand{\range}{\mathrm{range}}
\newcommand{\dom}{\mathrm{dom}}
\newcommand{\N}{\mathbb{N}}
\newcommand{\R}{\mathbb{R}}
\newcommand{\E}{\mathbb{E}}
\newcommand{\D}{\mathbb{D}}
\newcommand{\M}{\mathcal{M}}
\newcommand{\Prime}{\mathbb{P}}
\newcommand{\A}{\mathbb{A}}
\newcommand{\Q}{\mathbb{Q}}
\newcommand{\Z}{\mathbb{Z}}
\newcommand{\F}{\mathbb{F}}
\newcommand{\CC}{\mathbb{C}}

\DeclarePairedDelimiter\abs{\lvert}{\rvert}
\DeclarePairedDelimiter\floor{\lfloor}{\rfloor}
\DeclarePairedDelimiter\ceil{\lceil}{\rceil}
\DeclarePairedDelimiter\lr{(}{)}
\DeclarePairedDelimiter\set{\{}{\}}
\DeclarePairedDelimiter\norm{\|}{\|}

\renewcommand{\labelenumi}{(\alph{enumi})}

\newcommand{\smallindent}{
    \geometry{left=1cm}% левое поле
    \geometry{right=1cm}% правое поле
    \geometry{top=1.5cm}% верхнее поле
    \geometry{bottom=1cm}% нижнее поле
}

\newcommand{\header}[3]{
    \pagestyle{fancy} % All pages have headers and footers
    \fancyhead{} % Blank out the default header
    \fancyfoot{} % Blank out the default footer
    \fancyhead[L]{#1}
    \fancyhead[C]{#2}
    \fancyhead[R]{#3}
}

\newcommand{\dividedinto}{
    \,\,\,\vdots\,\,\,
}

\newcommand{\littletaller}{\mathchoice{\vphantom{\big|}}{}{}{}}

\newcommand\restr[2]{{
    \left.\kern-\nulldelimiterspace % automatically resize the bar with \right
    #1 % the function
    \littletaller % pretend it's a little taller at normal size
    \right|_{#2} % this is the delimiter
}}

\DeclareGraphicsExtensions{.pdf,.png,.jpg}

\newenvironment{enumerate_boxed}[1][enumi]{\begin{enumerate}[label*=\protect\fbox{\arabic{#1}}]}{\end{enumerate}}



\smallindent

\header{ЦРОД $\bullet$ Математика}{\textit{Алгебра}}{3 ноября 2022}

%----------------------------------------------------------------------------------------

\begin{document}
    \large


    \begin{center}
        \textbf{Многочлены третей и четвёртой степени}
    \end{center}

    Для решения уравнений третьей степени вида $x^3 + px + q = 0$ есть формула Кардано:
    \[\boxed{x=\sqrt[3]{-\frac{q^2}{4} + \sqrt{\frac{q^2}{4} +\frac{p^3}{27}}} + \sqrt[3]{-\frac{q^2}{4} - \sqrt{\frac{q^2}{4} +\frac{p^3}{27}}}}\]


    \begin{enumerate}[label*=\protect\fbox{\arabic{enumi}}]

        \item Решите уравнение: $x^3 - 15 x - 126 = 0.$

        \item Решите уравнение: $x^3 - 6x^2 - 6x - 2 = 0.$

        \item Решите уравнение: $x^3 - 6x - 4 = 0.$

        \item Решите уравнение: $x^3 - 6x - 40 = 0.$

        \item Решите уравнение: $x^3 - 6x^2 - 6x - 2 = 0.$


    \end{enumerate}

    А для решения уравнений четвёртой степени вида $x^4 + ax^2+ bx + c = 0$ есть метод Феррари.
    \begin{gather*}
        x^4 + (2\alpha - \beta^2)  x^2 -2\beta\gamma x + \alpha^2- \gamma^2 = 0\\
        (x^2 + \alpha)^2 - (\beta x + \gamma)^2 = 0\\
        (x^2 + \alpha - \beta x - \gamma)(x^2 + \alpha + \beta x + \gamma) = 0\\
    \end{gather*}

    Мы хотим найти $\alpha, \beta, \gamma$ такие, что

    \begin{equation*}
        \begin{cases}
            2\alpha - \beta^2 = a,
            \\
            -2\beta\gamma = b,
            \\
            \alpha^2- \gamma^2 = c.
        \end{cases}
    \end{equation*}

    Тогда корни исходного уравнения четвёртой степени будут корни двух получившихся квадратных трёхчленов.

    \begin{enumerate_boxed}

        \item Решите уравнение: $x^4 - 8x^2 + 16 = 0.$

        \item Решите уравнение: $x^4 - 9x^2 + 2x + 15 = 0.$

        \item Решите уравнение: $x^4 + 4x^3 - 4x^2 - 20x - 5 = 0.$

    \end{enumerate_boxed}

    Для многочленов пятой степени и выше общей формулы нахождения решений не существует.
    Пример:
    \[x^5 - 4x + 2 = 0\]
\end{document}