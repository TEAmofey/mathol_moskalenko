\documentclass{article}
\usepackage[12pt]{extsizes}
\usepackage[T2A]{fontenc}
\usepackage[utf8]{inputenc}
\usepackage[english, russian]{babel}

\usepackage{amssymb}
\usepackage{amsfonts}
\usepackage{amsmath}
\usepackage{enumitem}
\usepackage{graphics}

\usepackage{lipsum}



\usepackage{geometry} % Меняем поля страницы
\geometry{left=1cm}% левое поле
\geometry{right=1cm}% правое поле
\geometry{top=1.5cm}% верхнее поле
\geometry{bottom=1cm}% нижнее поле


\usepackage{fancyhdr} % Headers and footers
\pagestyle{fancy} % All pages have headers and footers
\fancyhead{} % Blank out the default header
\fancyfoot{} % Blank out the default footer
\fancyhead[L]{ЦРОД $\bullet$ Математика}
\fancyhead[C]{\textit{Геометрия}}
\fancyhead[R]{Стратегия 2021}% Custom header text


%----------------------------------------------------------------------------------------

%\begin{document}\normalsize
\begin{document}\large


\begin{center}
\textbf{Вписанные углы}
\end{center}



\begin{enumerate}[label*=\protect\fbox{\arabic{enumi}}]

\item Даны две окружности, пересекающиеся в точках $X$ и $Y$. Прямая, проходящая через $X$, пересекает первую окружность в точке $A$, а вторую — в точке $C$. Другая прямая, проходящая через $Y$, первую окружность пересекает в точке $B$, а вторую — в точке $D$. Докажите, что $AB \parallel CD$.

\item В окружность вписан шестиугольник. Найдите сумму углов при трёх его несоседних вершинах.

\item Окружности с центрами $O_1$ и $O_2$ пересекаются в точках $A$ и $B$. Луч $O_2A$ пересекает первую окружность в точке $C$. Докажите, что точки $O_1$, $O_2$, $B$, $C$ лежат на одной окружности.

\item Докажите, что в равнобедренной трапеции вершины боковой стороны, точка пересечения диагоналей и центр описанной окружности лежат на одной окружности.

\item Дан выпуклый четырёхугольник $ABCD$ Рассмотрим точки пересечения биссектрис его углов $A$ и $B$, $B$ и $C$, $C$ и $D$, $D$ и $A$. Докажите, что эти четыре точки являются вершинами вписанного четырёхугольника.

\item На хорде $AB$ окружности с центром в точке $O$ выбрана точка $C$. Описанная окружность треугольника $AOC$ пересекает исходную окружность в точке $D$. Докажите, что $BC=CD$.

\item Про выпуклый четырёхугольник $ABCD$ известно, что $AB=BC=CD$. Диагонали четырёхугольника пересекаются в точке $M$, $K$ — точка пересечения биссектрис углов $A$ и $D$. Докажите, что точки $A$, $M$, $K$, $D$ лежат на одной окружности.

\item Пусть дан треугольник $ABC$, и в точке $B$ построена касательная к описанной окружности треугольника $ABC$. Рассмотрим произвольную прямую, параллельную этой касательной, и отметим точки $D$ и $E$ пересечения с прямыми $AB$ и $BC$ соответственно. Докажите, что четыре точки $A$, $C$, $D$, $E$ лежат на одной окружности.

\item Диагонали вписанного четырёхугольника $ABCD$ пересекаются в точке $O$. Докажите, что прямая, соединяющая середины дуг $AB$ и $CD$, параллельна биссектрисе угла $AOB$.

\item Четырёхугольник $ABCD$ таков, что в него можно вписать и около него можно описать окружности. Диаметр описанной окружности совпадает с диагональю $AC$. Докажите, что модули разностей длин его противоположных сторон равны.

\item Пусть $I$ — центр вписанной окружности остроугольного треугольника $ABC$, $M$ и $N$ — точки касания вписанной окружности сторон $AB$ и $BC$ соответственно. Через точку $I$ проведена прямая $\l$, параллельная стороне $AC$, и на неё  опущены перпендикуляры $AP$ и $CQ$. Докажите, что точки $M$, $N$, $P$ и $Q$ лежат
на одной окружности.

\end{enumerate}
\end{document}