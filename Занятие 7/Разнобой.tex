\documentclass{article}
\usepackage[12pt]{extsizes}
\usepackage[T2A]{fontenc}
\usepackage[utf8]{inputenc}
\usepackage[english, russian]{babel}

\usepackage{amssymb}
\usepackage{amsfonts}
\usepackage{amsmath}
\usepackage{enumitem}
\usepackage{graphics}
\usepackage{graphicx}

\usepackage{lipsum}
\DeclareGraphicsExtensions{.pdf,.png,.jpg}



\usepackage{geometry} % Меняем поля страницы
\geometry{left=1cm}% левое поле
\geometry{right=1cm}% правое поле
\geometry{top=1.5cm}% верхнее поле
\geometry{bottom=1cm}% нижнее поле


\usepackage{fancyhdr} % Headers and footers
\pagestyle{fancy} % All pages have headers and footers
\fancyhead{} % Blank out the default header
\fancyfoot{} % Blank out the default footer
\fancyhead[L]{Математика}
\fancyhead[C]{\textit{Разное}}
\fancyhead[R]{21 ноября 2023}% Custom header text

%----------------------------------------------------------------------------------------

%\begin{document}\normalsize
\begin{document}\large
	

\begin{center}
\textbf{Разнобой}
\end{center}

\begin{enumerate}[label*=\protect\fbox{\arabic{enumi}}]
	
	\item  Существует ли функция $f: \mathbb{N} \rightarrow \mathbb{N} $, такая что $f(f(f(...f(x))...)$ = $x + 1$ для любого натурального $x$ ($f$ применена $f (x)$ раз)?
	
	\item Внутри непрозрачного единичного куба находится выпуклый многогранник $M$, чей объём больше $1/4$. Вася не знает ни формы, ни положения этого многогранника. Зато он может пронзить куб $k$ прямолинейными лазерными лучами. При каком наименьшем $k$ Вася может гарантировать, что хотя бы один из лучей будет иметь общую точку с $M$?
	



\end{enumerate}
\end{document}