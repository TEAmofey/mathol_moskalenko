\documentclass{article}

\usepackage[12pt]{extsizes}
\usepackage[T2A]{fontenc}
\usepackage[utf8]{inputenc}
\usepackage[english, russian]{babel}

\usepackage{mathrsfs}
\usepackage[dvipsnames]{xcolor}

\usepackage{amsmath}
\usepackage{amssymb}
\usepackage{amsthm}
\usepackage{indentfirst}
\usepackage{amsfonts}
\usepackage{enumitem}
\usepackage{graphics}
\usepackage{tikz}
\usepackage{tabu}
\usepackage{diagbox}
\usepackage{hyperref}
\usepackage{mathtools}
\usepackage{ucs}
\usepackage{lipsum}
\usepackage{geometry} % Меняем поля страницы
\usepackage{fancyhdr} % Headers and footers
\newcommand{\range}{\mathrm{range}}
\newcommand{\dom}{\mathrm{dom}}
\newcommand{\N}{\mathbb{N}}
\newcommand{\R}{\mathbb{R}}
\newcommand{\E}{\mathbb{E}}
\newcommand{\D}{\mathbb{D}}
\newcommand{\M}{\mathcal{M}}
\newcommand{\Prime}{\mathbb{P}}
\newcommand{\A}{\mathbb{A}}
\newcommand{\Q}{\mathbb{Q}}
\newcommand{\Z}{\mathbb{Z}}
\newcommand{\F}{\mathbb{F}}
\newcommand{\CC}{\mathbb{C}}

\DeclarePairedDelimiter\abs{\lvert}{\rvert}
\DeclarePairedDelimiter\floor{\lfloor}{\rfloor}
\DeclarePairedDelimiter\ceil{\lceil}{\rceil}
\DeclarePairedDelimiter\lr{(}{)}
\DeclarePairedDelimiter\set{\{}{\}}
\DeclarePairedDelimiter\norm{\|}{\|}

\renewcommand{\labelenumi}{(\alph{enumi})}

\newcommand{\smallindent}{
    \geometry{left=1cm}% левое поле
    \geometry{right=1cm}% правое поле
    \geometry{top=1.5cm}% верхнее поле
    \geometry{bottom=1cm}% нижнее поле
}

\newcommand{\header}[3]{
    \pagestyle{fancy} % All pages have headers and footers
    \fancyhead{} % Blank out the default header
    \fancyfoot{} % Blank out the default footer
    \fancyhead[L]{#1}
    \fancyhead[C]{#2}
    \fancyhead[R]{#3}
}

\newcommand{\dividedinto}{
    \,\,\,\vdots\,\,\,
}

\newcommand{\littletaller}{\mathchoice{\vphantom{\big|}}{}{}{}}

\newcommand\restr[2]{{
    \left.\kern-\nulldelimiterspace % automatically resize the bar with \right
    #1 % the function
    \littletaller % pretend it's a little taller at normal size
    \right|_{#2} % this is the delimiter
}}

\DeclareGraphicsExtensions{.pdf,.png,.jpg}

\newenvironment{enumerate_boxed}[1][enumi]{\begin{enumerate}[label*=\protect\fbox{\arabic{#1}}]}{\end{enumerate}}



\smallindent

\header{Математика}{\textit{Геометрия}}{19 сентября 2022}

%----------------------------------------------------------------------------------------


\begin{document}
    \large

    \begin{center}
        \textbf{Разнобой по геометрии}
    \end{center}

    \begin{enumerate_boxed}
%повгом изи
        \item На прямоугольную карту положили карту той же местности, но меньшего масштаба.
        Докажите, что можно проткнуть иголкой сразу обе карты так, чтобы точка прокола изображала на обеих картах одну и ту же точку местности.

%компгом изи
        \item Общие внешние касательные к парам окружностей $S_1$ и $S_2$, $S_2$ и $S_3$, $S_3$ и $S_1$ пересекаются в точках $A$, $B$ и $C$ соответственно.
        Докажите, что точки $A$, $B$ и $C$ лежат на одной прямой.

%гомотетия норм
        \item Окружность $\omega$ касается равных сторон $AB$ и $AC$ равнобедренного треугольника $ABC$ и пересекает сторону $BC$ в точках $K$ и $L$.
        Отрезок $AK$ пересекает $\omega$ второй раз в точке $M$.
        Точки $P$ и $Q$ симметричны точке $K$ относительно точек $B$ и $C$ соответственно.
        Докажите, что описанная окружность треугольника $PMQ$ касается окружности $\omega$.

%радоси хард
        \item В треугольнике $ABC$ $AH_1$ и $BH_2$ – высоты; касательная к описанной окружности в точке $A$ пересекает $BC$ в точке $S_1$, а касательная в точке $B$ пересекает $AC$ в точке $S_2$; $T_1$ и $T_2$ – середины отрезков $AS_1$ и $BS_2$.
        Докажите, что $T_{1}T_2$, $AB$ и $H_{1}H_2$ пересекаются в одной точке.

%повгом норм
        \item На диагонали $BD$ вписанного четырёхугольника $ABCD$ выбрана такая точка $K$, что $\angle AKB = \angle ADC$.
        Пусть $I$ и $I'$ – центры вписанных окружностей треугольников $ACD$ и $ABK$ соответственно.
        Отрезки $II'$ и $BD$ пересекаются в точке $X$.
        Докажите, что точки $A, X, I, D$ лежат на одной окружности.

%копмгом норм
        \item Окружность $S$ находится внутри треугольника $ABC$.
        Каждая из окружностей $S_1$, $S_2$ и $S_3$ касается внешним образом окружности $S$ (в точках $A_1$, $B_1$ и $C_1$ соответственно) и двух сторон треугольника $ABC$.
        Докажите, что прямые $AA_1$, $BB_1$ и $CC_1$ пересекаются в одной точке.

%радоси хард
        \item Точка $X$, лежащая вне непересекающихся окружностей $\omega_1$ и $\omega_2$, такова, что отрезки касательных, проведённых из $X$ к $\omega_1$ и $\omega_2$, равны.
        Докажите, что точка пересечения диагоналей четырёхугольника, образованного точками касания, совпадает с точкой пересечения общих внутренних касательных к $\omega_1$ и $\omega_2$.

%гомотетия норм
        \item Окружность проходит через вершины $B$ и $C$ треугольника $ABC$ и пересекает стороны $AB$ и $AC$ в точках $D$ и $E$ соответственно.
        Отрезки $CD$ и $BE$ пересекаются в точке $O$.
        Пусть $M$ и $N$ – центры окружностей, вписанных соответственно в треугольники $ADE$ и $ODE$.
        Докажите, что середина меньшей дуги $DE$ лежат на прямой $MN$.

%радоси хард
        \item На стороне $BC$ треугольника $ABC$ взята точка $A'$.
        Серединный перпендикуляр к отрезку $A'B$ пересекает сторону $AB$ в точке $M$, а серединный перпендикуляр к отрезку $A'C$ пересекает сторону $AC$ в точке $N$.
        Докажите, что точка, симметричная точке $A'$ относительно прямой $MN$, лежит на описанной окружности треугольника $ABC$.

%гомотетия хард
        \item На сторонах $AB$, $BC$ и $CA$ треугольника $ABC$ построены во внешнюю сторону квадраты $ABB_{1}A_2$, $BCC_{1}B_2$ и $CAA_{1}C_2$.
        Докажите, что серединные перпендикуляры к отрезкам $A_{1}A_2$, $B_{1}B_2$ и $C_{1}C_2$, пересекаются в одной точке.

%лемма о трезубце
        \item Биссектрисы треугольника $ABC$ пересекаются в точке $I$, внешние биссектрисы его углов $B$ и $C$ пересекаются в точке $J$.
        Окружность $\omega_b$ с центром в точке $O_b$ проходит через точку $B$ и касается прямой $CI$ в точке $I$.
        Окружность $\omega_c$ с центром в точке $O_c$ проходит через точку $C$ и касается прямой $BI$ в точке $I$.
        Отрезки $O_{b}O_c$ и $IJ$ пересекаются в точке $K$.
        Найдите отношение $\dfrac{IK}{KJ}$.


    \end{enumerate_boxed}
\end{document}