\documentclass{article}
\usepackage[12pt]{extsizes}
\usepackage[T2A]{fontenc}
\usepackage[utf8]{inputenc}
\usepackage[english, russian]{babel}

\usepackage{amssymb}
\usepackage{amsfonts}
\usepackage{amsmath}
\usepackage{enumitem}
\usepackage{graphics}
\usepackage{graphicx}

\usepackage{lipsum}
\DeclareGraphicsExtensions{.pdf,.png,.jpg}



\usepackage{geometry} % Меняем поля страницы
\geometry{left=1cm}% левое поле
\geometry{right=1cm}% правое поле
\geometry{top=1.5cm}% верхнее поле
\geometry{bottom=1cm}% нижнее поле


\usepackage{fancyhdr} % Headers and footers
\pagestyle{fancy} % All pages have headers and footers
\fancyhead{} % Blank out the default header
\fancyfoot{} % Blank out the default footer
\fancyhead[L]{Математика}
\fancyhead[C]{\textit{Алгебра}}
\fancyhead[R]{5 октября 2023}% Custom header text


%----------------------------------------------------------------------------------------

%\begin{document}\normalsize
\begin{document}\large
	

\begin{center}
\textbf{Рекурренты}
\end{center}

\begin{enumerate}[label*=\protect\fbox{\arabic{enumi}}]
	
\item Найдите формулу $n$-го члена для последовательностей, заданных условиями $( n  \geqslant 0)$:
\begin{enumerate}
	\item $a_0 = 0, a_1 = 1, a_{n + 2} = a_{n + 1} + a_n;$
	\item $a_0 = 0, a_1 = 1, a_{n + 2} = 5a_{n + 1} - 6a_n;$ 
	\item $a_0 = 0, a_1 = 1, a_{n + 2} = 2a_{n + 1} + a_n;$ 
	\item $a_0 = 0, a_1 = 1, a_{n + 2} = 2a_{n + 1} - a_n.$ 
\end{enumerate}

\item Сколько существует способов разрезать доску $2 \times 10$ на доминошки?

\item Пусть $ x_1 $ и $ x_2 $ — корни квадратного уравнения $ x^2 - 6x + 1 = 0 $. Докажите, что при любом натуральном $ n $ число $ x^n_1 + x^n_2 $ является целым и не делится на 5.

\item Последовательность задана рекуррентно: $ a_1 = \frac{1}{2} , a_1 + a_2 + . . . + a_n = n^2 a_n $ . Найдите формулу общего члена.

\item Последовательность $ \{a_i\}^{\infty}_{i=0} $ задана рекурентно: $ a_0 = a, a_{n+1} = 2^n - 3a_n $. При каких значениях $ a $ последовательность является монотонно возрастающей?

\item Последовательность $a_0, a_1, a_2, \ldots$ такова, что для всех неотрицательных $m \geq n$ выполняется условие $a_{m+n} + a_{m-n} = \frac{a_{2m} + a_{2n}}{2}$. Найдите $a_{2022}$, если $a_1 = 1$.

\item На какую наибольшую степень двойки может делиться число вида $\left[3 + \sqrt{10}\right]^{2n-1}$?

\item Лягушка прыгает по вершинам треугольника $ABC$, перемещаясь каждый раз на одну из соседних вершин. Сколькими способами она может попасть из $A$ в $A$ ровно за $n$ прыжков?

\item Найдите количество функций $f : \{1, 2, \ldots , n\} \to \{1, 2, 3, 4, 5\}$, удовлетворяющих неравенству $|f(k + 1) - f(k)| \geq 3$ при всех $k \in \{1, \ldots , n - 1\}$.

\item Сколько $n$-разрядных десятичных чисел, которые могут начинаться с нуля, удовлетворяют условиям:
\begin{enumerate}
\item не содержат в своей записи двух стоящих рядом четных цифр;
\item не содержат в своей записи цифры 5 после цифры 2.
\end{enumerate}

\item Сколько существует несамопересекающихся ломаных длины $n$, начинающихся в начале координат $(0, 0)$, каждое звено которых совпадает с одним из векторов $r = (1,0)$, $u = (0,1)$, $d = (0,-1)$?

\item Шеренга солдат называется неправильной, если никакие три подряд стоящих солдата не стоят по росту (ни в порядке возрастания, ни в порядке убывания). Составьте рекуррентное соотношение для количества неправильных шеренг из $n$ солдат разного роста.

\item На клетчатой доске размером $2 \times n$ клеток некоторые клетки закрашиваются в чёрный цвет. Раскраска называется правильной, если среди закрашенных нет двух соседних клеток (соседними называются клетки, имеющие общую сторону). Раскраска, в которой ни одна клетка не закрашена, тоже считается правильной. Пусть $A_n$ — количество правильных раскрасок с четным числом закрашенных клеток, $B_n$ — количество правильных раскрасок с нечетным числом закрашенных клеток. Найдите все возможные значения $A_n - B_n$.

\end{enumerate}
\end{document}