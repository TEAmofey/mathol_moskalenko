\documentclass{article}

\usepackage[12pt]{extsizes}
\usepackage[T2A]{fontenc}
\usepackage[utf8]{inputenc}
\usepackage[english, russian]{babel}

\usepackage{mathrsfs}
\usepackage[dvipsnames]{xcolor}

\usepackage{amsmath}
\usepackage{amssymb}
\usepackage{amsthm}
\usepackage{indentfirst}
\usepackage{amsfonts}
\usepackage{enumitem}
\usepackage{graphics}
\usepackage{tikz}
\usepackage{tabu}
\usepackage{diagbox}
\usepackage{hyperref}
\usepackage{mathtools}
\usepackage{ucs}
\usepackage{lipsum}
\usepackage{geometry} % Меняем поля страницы
\usepackage{fancyhdr} % Headers and footers
\newcommand{\range}{\mathrm{range}}
\newcommand{\dom}{\mathrm{dom}}
\newcommand{\N}{\mathbb{N}}
\newcommand{\R}{\mathbb{R}}
\newcommand{\E}{\mathbb{E}}
\newcommand{\D}{\mathbb{D}}
\newcommand{\M}{\mathcal{M}}
\newcommand{\Prime}{\mathbb{P}}
\newcommand{\A}{\mathbb{A}}
\newcommand{\Q}{\mathbb{Q}}
\newcommand{\Z}{\mathbb{Z}}
\newcommand{\F}{\mathbb{F}}
\newcommand{\CC}{\mathbb{C}}

\DeclarePairedDelimiter\abs{\lvert}{\rvert}
\DeclarePairedDelimiter\floor{\lfloor}{\rfloor}
\DeclarePairedDelimiter\ceil{\lceil}{\rceil}
\DeclarePairedDelimiter\lr{(}{)}
\DeclarePairedDelimiter\set{\{}{\}}
\DeclarePairedDelimiter\norm{\|}{\|}

\renewcommand{\labelenumi}{(\alph{enumi})}

\newcommand{\smallindent}{
    \geometry{left=1cm}% левое поле
    \geometry{right=1cm}% правое поле
    \geometry{top=1.5cm}% верхнее поле
    \geometry{bottom=1cm}% нижнее поле
}

\newcommand{\header}[3]{
    \pagestyle{fancy} % All pages have headers and footers
    \fancyhead{} % Blank out the default header
    \fancyfoot{} % Blank out the default footer
    \fancyhead[L]{#1}
    \fancyhead[C]{#2}
    \fancyhead[R]{#3}
}

\newcommand{\dividedinto}{
    \,\,\,\vdots\,\,\,
}

\newcommand{\littletaller}{\mathchoice{\vphantom{\big|}}{}{}{}}

\newcommand\restr[2]{{
    \left.\kern-\nulldelimiterspace % automatically resize the bar with \right
    #1 % the function
    \littletaller % pretend it's a little taller at normal size
    \right|_{#2} % this is the delimiter
}}

\DeclareGraphicsExtensions{.pdf,.png,.jpg}

\newenvironment{enumerate_boxed}[1][enumi]{\begin{enumerate}[label*=\protect\fbox{\arabic{#1}}]}{\end{enumerate}}



\smallindent

\header{Математический фестиваль Простые Числа}{\textit{Геометрия}}{ЦРОД $\bullet$ 15 марта 2023}

%----------------------------------------------------------------------------------------

\begin{document}
    \large

    \begin{center}
        \textbf{Группа 7--1, Москаленко Тимофей Дмитриевич}
    \end{center}
    \begin{center}
        \textbf{Считаем уголочки}
    \end{center}


    \begin{enumerate_boxed}
        \item Один из углов равнобедренного треугольника в два раза больше другого.
        Найдите этот угол.
        \item Биссектриса прямого угла треугольника образует с его противоположной стороной угол $65^\circ$.
        Найдите меньший угол этого треугольника.
        \item Найдите сумму углов выпуклого пятиугольника.
        \item Два угла пятиугольника прямые, а остальные равны между собой.
        Чему равны эти углы?
        \item Биссектриса угла при основании равнобедренного треугольника образует с его боковой стороной угол $75^\circ$.
        Найдите угол при основании этого треугольника.
        Может ли у задачи быть второе решение?
        \item На стороне квадрата во внешнюю от него область построили равносторонний треугольник.
        Найдите угол, под которым из вершины этого треугольника видна противоположная сторона квадрата.
        \item На двух сторонах квадрата построили равносторонние треугольники так, как это показано на рисунке.
        Лежат ли отмеченные на нём точки $A, B$ и $C$ на одной прямой?
    \end{enumerate_boxed}

    \begin{figure}[h]
        \centering
        \begin{minipage}{0.45\textwidth}
            \centering
            \begin{asy}
                size(8cm);
                include geometry;

                point pF = (0,0), pE = (2, 0);
                point pA = (0,2), pD = (2, 2);
                triangle tB =triangleabc(2,2,2);
                point pB = tB.C;
                triangle tC =triangleabc(2,2,2, angle=-90,(2,2));
                point pC = tC.C;

                fill(tB, rgb(244,216,179));
                draw(tB, rgb(218,129,71));
                fill(tC, rgb(244,216,179));

                draw(tC, rgb(218,129,71));
                dot("$A$", pA, NW);
                dot("$B$", pB, N);
                dot("$C$", pC, E);
                dot(" ", pD, N);
                dot(" ", pE, N);
                dot(" ", pF, N);

                draw(pA -- pD);
                draw(pA -- pF);
                draw(pE -- pF);
                draw(pD -- pE);
            \end{asy}
            \caption{Задача 7}
        \end{minipage}
        \begin{minipage}{0.45\textwidth}
            \centering
            \begin{asy}
                size(8cm);
                include geometry;

                point pF = (0,0), pE = (2, 0);
                point pK = (0,2), pD = (2, 2);
                triangle tA =triangleabc(2,2,2, (0,2));
                point pA = tA.C;
                triangle tB =triangleabc(2,2,2);
                point pB = tB.C;
                triangle tC =triangleabc(2,2,2, angle=-90,(2,2));
                point pC = tC.C;

                fill(tB, rgb(244,216,179));
                draw(tB, rgb(218,129,71));
                fill(tC, rgb(244,216,179));
                draw(tC, rgb(218,129,71));
                fill(tA, rgb(244,216,179));
                draw(tA, rgb(218,129,71));
                draw(pA -- pB);
                draw(pB -- pC);
                markangleradiusfactor *= -2.0;
                markangle(" ", 1, pC, pB, pA, red + linetype(new real[] {4,4}));
                dot("$A$", pA, N);
                dot("$B$", pB, NW);
                dot("$C$", pC, E);
                dot(" ", pD, N);
                dot(" ", pE, N);
                dot(" ", pF, N);
                dot(" ", pK, N);

                draw(pK -- pD);
                draw(pK -- pF);
                draw(pE -- pF);
                draw(pD -- pE);

            \end{asy}
            \caption{Задача 8}
        \end{minipage}\label{fig:figure}
    \end{figure}

    \begin{enumerate_boxed}

        \setcounter{enumi}{7}

        \item На трёх сторонах квадрата построили равносторонние треугольники так, как это показано на рисунке.
        Найдите на нём величину угла $ABC$.

        \item Угол треугольника равен $\alpha$.
        Найдите угол между биссектрисами двух других его углов.

        \item В четырёхугольнике $ABCD$ биссектрисы углов $A$ и $C$ параллельны.
        Докажите, что углы $B$ и $D$ четырёхугольника равны.

        \item На продолжении стороны $AC$ треугольника $ABC$ отметили точки $M$ и $K$ так, что $AM = AB, CK = BC$.
        Найдите угол $MBK$, если угол $ABC$ равен $\beta$.

        \item Найдите сумму углов произвольного $n$-угольника.

        \item Найдите сумму углов произвольной пятиконечной звезды.

        \item Выпуклый шестиугольник таков, что его противоположные углы попарно равны.
        Докажите, что противоположные стороны такого шестиугольника параллельны.

    \end{enumerate_boxed}
\end{document}