\documentclass{article}
\usepackage[12pt]{extsizes}
\usepackage[T2A]{fontenc}
\usepackage[utf8]{inputenc}
\usepackage[english, russian]{babel}

\usepackage{amssymb}
\usepackage{amsfonts}
\usepackage{amsmath}
\usepackage{enumitem}
\usepackage{graphics}
\usepackage{graphicx}

\usepackage{lipsum}

\newtheorem{theorem}{Теорема}
\newtheorem{task}{Задача}
\newtheorem{lemma}{Лемма}
\newtheorem{definition}{Определение}
\newtheorem{example}{Пример}
\newtheorem{statement}{Утверждение}
\newtheorem{corollary}{Следствие}


\usepackage{geometry} % Меняем поля страницы
\geometry{left=1cm}% левое поле
\geometry{right=1cm}% правое поле
\geometry{top=1.5cm}% верхнее поле
\geometry{bottom=1cm}% нижнее поле


\usepackage{fancyhdr} % Headers and footers
\pagestyle{fancy} % All pages have headers and footers
\fancyhead{} % Blank out the default header
\fancyfoot{} % Blank out the default footer
\fancyhead[L]{ЦРОД \textbullet Математика}
\fancyhead[C]{\textit{Геометрия}}
\fancyhead[R]{ЛФМШ 2022}% Custom header text


%----------------------------------------------------------------------------------------

%\begin{document}\normalsize
\begin{document}\large
	
	
	\begin{center}
		\textbf{Лемма о трезубце}
	\end{center}

{\bf Лемма о трезубце.} Середина дуги $AC$ описанной окружности треугольника $ABC$ равноудалена от вершин $A$, $C$ и центров $I$ вписанной и $I_B$ вневписанной окружностей.

\begin{enumerate}[label*=\protect\fbox{\arabic{enumi}}]
	
\item Точки $O$ и $I$ -- центры описанной и вписанной окружностей треугольника $ABC$, $M$ -- середина дуги $AC$ описанной окружности (не содержащей точки $B$). Докажите, что $MI = MO$ тогда и только тогда, когда $\angle ABC = 60^{\circ}$.

\item  Вписанная окружность треугольника $ABC$ с центром в точке $I$ касается сторон $AB$ и $AC$ в точках $D$ и $E$ соответственно. Точка $O$ --- центр описанной окружности треугольника $BIC$. Докажите, что $\angle BDO = \angle CEO$.

\item Отрезок, соединяющий середины «меньших» дуг $AB$ и $AC$ описанной окружности треугольника $ABC$, пересекает стороны $AB$ и $AC$ в точках $P$ и $Q$. Точка $I$ --- центр вписанной окружности треугольника $ABC$. Докажите, что $APIQ$ --- ромб.

\item Вокруг прямоугольного треугольника $ABC$ с прямым углом $C$ описана окружность. На меньших дугах $AC$ и $BC$ взяты их середины -- $K$ и $P$ соответственно. Отрезок $KP$ пересекает катет $AC$ в точке $N$. Центр вписанной окружности треугольника $ABC$ -- точка $I$. Найдите угол $NIC$. 

\item Биссектрисы равнобедренного треугольника $ABC$ с основанием $AC$ пересекаются в точке $I$. Точка $E$ --- середина дуги $BC$ описанной окружности. На плоскости отметили точку $S$ так, что $ABSI$ --- параллелограмм. Докажите, что $\angle BES = \angle ABC$.  

\item Дана равнобокая трапеция $ABCD$ с основаниями $BC$ и $AD$. В треугольники $ABC$ и $ABD$ вписаны окружности с центрами $O_1$ и $O_2$. Докажите, что прямая $O_1O_2$ перпендикулярна $BC$.

\item В остроугольном треугольнике $ABC$ угол при вершине $A$ равен $60^{\circ}$. Точки $I$, $H$, $O$ --- центр вписанной окружности, ортоцентр и центр описанной окружности треугольника $ABC$. Докажите, что $IH = IO$.

\item Окружность, проходящая через вершины $A$ и $B$ треугольника $ABC$, пересекает стороны $AC$ и $BC$ в точках $X$ и $Y$ соответственно. При этом центр вневписанной окружности треугольника $XYC$, касающейся стороны $XY$, лежит на описанной окружности треугольника $ABC$. Докажите, что отрезок $XY$ проходит через центр вписанной окружности треугольника $ABC$.

\item Биссектрисы углов $A$ и $C$ треугольника $ABC$ пересекают описанную окружность этого треугольника в точках $A_0$ и $C_0$ соответственно. Прямая, проходящая через центр вписанной окружности треугольника $ABC$ параллельно стороне $AC$, пересекается с прямой $A_0C_0$ в точке $P$. Докажите, что прямая $PB$ касается описанной окружности треугольника $ABC$.

\end{enumerate}
\end{document}