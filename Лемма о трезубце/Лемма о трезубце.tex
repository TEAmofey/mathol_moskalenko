\documentclass{article}

\usepackage[12pt]{extsizes}
\usepackage[T2A]{fontenc}
\usepackage[utf8]{inputenc}
\usepackage[english, russian]{babel}

\usepackage{mathrsfs}
\usepackage[dvipsnames]{xcolor}

\usepackage{amsmath}
\usepackage{amssymb}
\usepackage{amsthm}
\usepackage{indentfirst}
\usepackage{amsfonts}
\usepackage{enumitem}
\usepackage{graphics}
\usepackage{tikz}
\usepackage{tabu}
\usepackage{diagbox}
\usepackage{hyperref}
\usepackage{mathtools}
\usepackage{ucs}
\usepackage{lipsum}
\usepackage{geometry} % Меняем поля страницы
\usepackage{fancyhdr} % Headers and footers
\newcommand{\range}{\mathrm{range}}
\newcommand{\dom}{\mathrm{dom}}
\newcommand{\N}{\mathbb{N}}
\newcommand{\R}{\mathbb{R}}
\newcommand{\E}{\mathbb{E}}
\newcommand{\D}{\mathbb{D}}
\newcommand{\M}{\mathcal{M}}
\newcommand{\Prime}{\mathbb{P}}
\newcommand{\A}{\mathbb{A}}
\newcommand{\Q}{\mathbb{Q}}
\newcommand{\Z}{\mathbb{Z}}
\newcommand{\F}{\mathbb{F}}
\newcommand{\CC}{\mathbb{C}}

\DeclarePairedDelimiter\abs{\lvert}{\rvert}
\DeclarePairedDelimiter\floor{\lfloor}{\rfloor}
\DeclarePairedDelimiter\ceil{\lceil}{\rceil}
\DeclarePairedDelimiter\lr{(}{)}
\DeclarePairedDelimiter\set{\{}{\}}
\DeclarePairedDelimiter\norm{\|}{\|}

\renewcommand{\labelenumi}{(\alph{enumi})}

\newcommand{\smallindent}{
    \geometry{left=1cm}% левое поле
    \geometry{right=1cm}% правое поле
    \geometry{top=1.5cm}% верхнее поле
    \geometry{bottom=1cm}% нижнее поле
}

\newcommand{\header}[3]{
    \pagestyle{fancy} % All pages have headers and footers
    \fancyhead{} % Blank out the default header
    \fancyfoot{} % Blank out the default footer
    \fancyhead[L]{#1}
    \fancyhead[C]{#2}
    \fancyhead[R]{#3}
}

\newcommand{\dividedinto}{
    \,\,\,\vdots\,\,\,
}

\newcommand{\littletaller}{\mathchoice{\vphantom{\big|}}{}{}{}}

\newcommand\restr[2]{{
    \left.\kern-\nulldelimiterspace % automatically resize the bar with \right
    #1 % the function
    \littletaller % pretend it's a little taller at normal size
    \right|_{#2} % this is the delimiter
}}

\DeclareGraphicsExtensions{.pdf,.png,.jpg}

\newenvironment{enumerate_boxed}[1][enumi]{\begin{enumerate}[label*=\protect\fbox{\arabic{#1}}]}{\end{enumerate}}



\smallindent

\header{ЦРОД $\bullet$ Математика}{\textit{Геометрия}}{ЛФМШ 2022}

%----------------------------------------------------------------------------------------

\begin{document}
    \large

    \begin{center}
        \textbf{Лемма о трезубце}
    \end{center}

    \textbf{Лемма о трезубце.} Середина дуги $AC$ описанной окружности треугольника $ABC$ равноудалена от вершин $A$, $C$ и центров $I$ вписанной и $I_B$ вневписанной окружностей.

    \begin{enumerate_boxed}

        \item Точки $O$ и $I$ -- центры описанной и вписанной окружностей треугольника $ABC$, $M$ -- середина дуги $AC$ описанной окружности (не содержащей точки $B$). Докажите, что $MI = MO$ тогда и только тогда, когда $\angle ABC = 60^{\circ}$.

        \item  Вписанная окружность треугольника $ABC$ с центром в точке $I$ касается сторон $AB$ и $AC$ в точках $D$ и $E$ соответственно.
        Точка $O$ --- центр описанной окружности треугольника $BIC$.
        Докажите, что $\angle BDO = \angle CEO$.

        \item Отрезок, соединяющий середины «меньших» дуг $AB$ и $AC$ описанной окружности треугольника $ABC$, пересекает стороны $AB$ и $AC$ в точках $P$ и $Q$.
        Точка $I$ --- центр вписанной окружности треугольника $ABC$.
        Докажите, что $APIQ$ --- ромб.

        \item Вокруг прямоугольного треугольника $ABC$ с прямым углом $C$ описана окружность.
        На меньших дугах $AC$ и $BC$ взяты их середины -- $K$ и $P$ соответственно.
        Отрезок $KP$ пересекает катет $AC$ в точке $N$.
        Центр вписанной окружности треугольника $ABC$ -- точка $I$.
        Найдите угол $NIC$.

        \item Биссектрисы равнобедренного треугольника $ABC$ с основанием $AC$ пересекаются в точке $I$.
        Точка $E$ --- середина дуги $BC$ описанной окружности.
        На плоскости отметили точку $S$ так, что $ABSI$ --- параллелограмм.
        Докажите, что $\angle BES = \angle ABC$.

        \item Дана равнобокая трапеция $ABCD$ с основаниями $BC$ и $AD$.
        В треугольники $ABC$ и $ABD$ вписаны окружности с центрами $O_1$ и $O_2$.
        Докажите, что прямая $O_{1}O_2$ перпендикулярна $BC$.

        \item В остроугольном треугольнике $ABC$ угол при вершине $A$ равен $60^{\circ}$.
        Точки $I$, $H$, $O$ --- центр вписанной окружности, ортоцентр и центр описанной окружности треугольника $ABC$.
        Докажите, что $IH = IO$.

        \item Окружность, проходящая через вершины $A$ и $B$ треугольника $ABC$, пересекает стороны $AC$ и $BC$ в точках $X$ и $Y$ соответственно.
        При этом центр вневписанной окружности треугольника $XYC$, касающейся стороны $XY$, лежит на описанной окружности треугольника $ABC$.
        Докажите, что отрезок $XY$ проходит через центр вписанной окружности треугольника $ABC$.

        \item Биссектрисы углов $A$ и $C$ треугольника $ABC$ пересекают описанную окружность этого треугольника в точках $A_0$ и $C_0$ соответственно.
        Прямая, проходящая через центр вписанной окружности треугольника $ABC$ параллельно стороне $AC$, пересекается с прямой $A_{0}C_0$ в точке $P$.
        Докажите, что прямая $PB$ касается описанной окружности треугольника $ABC$.

    \end{enumerate_boxed}
\end{document}