\documentclass{article}

\usepackage[12pt]{extsizes}
\usepackage[T2A]{fontenc}
\usepackage[utf8]{inputenc}
\usepackage[english, russian]{babel}

\usepackage{mathrsfs}
\usepackage[dvipsnames]{xcolor}

\usepackage{amsmath}
\usepackage{amssymb}
\usepackage{amsthm}
\usepackage{indentfirst}
\usepackage{amsfonts}
\usepackage{enumitem}
\usepackage{graphics}
\usepackage{tikz}
\usepackage{tabu}
\usepackage{diagbox}
\usepackage{hyperref}
\usepackage{mathtools}
\usepackage{ucs}
\usepackage{lipsum}
\usepackage{geometry} % Меняем поля страницы
\usepackage{fancyhdr} % Headers and footers
\newcommand{\range}{\mathrm{range}}
\newcommand{\dom}{\mathrm{dom}}
\newcommand{\N}{\mathbb{N}}
\newcommand{\R}{\mathbb{R}}
\newcommand{\E}{\mathbb{E}}
\newcommand{\D}{\mathbb{D}}
\newcommand{\M}{\mathcal{M}}
\newcommand{\Prime}{\mathbb{P}}
\newcommand{\A}{\mathbb{A}}
\newcommand{\Q}{\mathbb{Q}}
\newcommand{\Z}{\mathbb{Z}}
\newcommand{\F}{\mathbb{F}}
\newcommand{\CC}{\mathbb{C}}

\DeclarePairedDelimiter\abs{\lvert}{\rvert}
\DeclarePairedDelimiter\floor{\lfloor}{\rfloor}
\DeclarePairedDelimiter\ceil{\lceil}{\rceil}
\DeclarePairedDelimiter\lr{(}{)}
\DeclarePairedDelimiter\set{\{}{\}}
\DeclarePairedDelimiter\norm{\|}{\|}

\renewcommand{\labelenumi}{(\alph{enumi})}

\newcommand{\smallindent}{
    \geometry{left=1cm}% левое поле
    \geometry{right=1cm}% правое поле
    \geometry{top=1.5cm}% верхнее поле
    \geometry{bottom=1cm}% нижнее поле
}

\newcommand{\header}[3]{
    \pagestyle{fancy} % All pages have headers and footers
    \fancyhead{} % Blank out the default header
    \fancyfoot{} % Blank out the default footer
    \fancyhead[L]{#1}
    \fancyhead[C]{#2}
    \fancyhead[R]{#3}
}

\newcommand{\dividedinto}{
    \,\,\,\vdots\,\,\,
}

\newcommand{\littletaller}{\mathchoice{\vphantom{\big|}}{}{}{}}

\newcommand\restr[2]{{
    \left.\kern-\nulldelimiterspace % automatically resize the bar with \right
    #1 % the function
    \littletaller % pretend it's a little taller at normal size
    \right|_{#2} % this is the delimiter
}}

\DeclareGraphicsExtensions{.pdf,.png,.jpg}

\newenvironment{enumerate_boxed}[1][enumi]{\begin{enumerate}[label*=\protect\fbox{\arabic{#1}}]}{\end{enumerate}}



\header{\textit{\textbf{XIV Олимпиада Эйлера}}}{}{20 ноября 2022}

%----------------------------------------------------------------------------------------

\begin{document}
    \large

    \begin{center}
        \LARGE\textbf{8 класс}
    \end{center}
    \begin{center}
        \large\textbf{Первый день}
    \end{center}


    \begin{enumerate}[label*=8.{\arabic{enumi}}]

        \item При каком наибольшем $n$ существует выпуклый $n$-угольник, у которого длины диагоналей принимают не больше двух различных значений?

        \item Числа $1, 2, \dots, 1000$ разбили на два множества по $500$ чисел: красные $k_1, k_2, \dotsc, k_{500}$ и синие $s_1, s_2, \dotsc, s_{500}$.
        Докажите, что количество таких пар $m$ и $n$, у которых разность $k_m-s_n$ дает остаток $7$ при делении на $100$, равно количеству таких пар $m$ и $n$, у которых разность $s_n-k_m$ дает остаток $7$ при делении на $100$.
        Здесь рассматриваются все возможные разности, в том числе и отрицательные.
        Напомним, что остатком от деления целого числа $a$ на $100$ называется разность между числом $a$ и ближайшим числом, не большим $a$ и делящимся на $100$.
        Например, остаток от деления числа $2022$ на $100$ равен $2022-2000 = 22$, а остаток от деления числа $-11$ на $100$ равен $-11-(-100) = 89$.

        \item В треугольнике $ABC$ проведены биссектрисы $BK$ и $CL$.
        На отрезке $BK$ отмечена точка $N$ так, что $LN \parallel AC$.
        Оказалось, что $NK = LN$.
        Найдите величину угла $ABC$.

        \item Учитель придумал ребус, заменив в примере $a+b = c$ на сложение двух натуральных чисел цифры буквами: одинаковые цифры одинаковыми буквами, а разные — разными
        (например, если $a = 23,$ а $b = 528,$ то $c = 551$, и получился, с точностью до выбора букв, ребус АБ $+$ ВАГ $=$ ВВД).
        Оказалось, что по получившемуся ребусу однозначно восстанавливается исходный пример.
        Найдите наименьшее возможное значение суммы $c$.

        \item Можно ли без остатка разрезать клетчатый квадрат размером $8 \times 8$ клеточек на $10$ клетчатых прямоугольников, чтобы все прямоугольники имели различные площади?
        Все разрезы должны проходить по границам клеточек.

    \end{enumerate}
    \newpage

    \begin{center}
        \LARGE\textbf{8 класс}
    \end{center}
    \begin{center}
        \large\textbf{Второй день}
    \end{center}


    \begin{enumerate}[label*=8.{\arabic{enumi}}]
        \setcounter{enumi}{5}
        \item Сумма остатков от деления трёх последовательных натуральных чисел на $2022$~--- простое число.
        Докажите, что одно из чисел делится на $2022$.

        \item Существует ли треугольник, у которого длины не совпадающих между собой медианы и высоты, проведенных из одной его вершины, соответственно равны длинам двух сторон этого треугольника?

        \item Будем называть натуральное число красивым, если в его десятичной записи поровну цифр $0, 1, 2$, а других цифр нет (во избежание недоразумений напомним, что десятичная запись числа не может начинаться с нуля).
        Может ли произведение двух красивых чисел быть красивым?

        \item Петя и Вася написали на доске по $100$ различных натуральных чисел.
        Петя поделил все свои числа на Васины с остатком и выписал все $10000$ получившихся остатков себе в тетрадь.
        Вася поделил все свои числа на Петины с остатком и выписал все $10000$ получившихся остатков себе в тетрадь.
        Оказалось, что наборы выписанных Васей и Петей остатков совпадают.
        Докажите, что тогда и наборы их исходных чисел совпадают.

        \item В вершины правильного $100$-угольника поставили $100$ фишек, на которых написаны номера $1, 2, \dotsc, 100$, именно в таком порядке по часовой стрелке.
        За ход разрешается обменять местами некоторые две фишки, стоящие в соседних вершинах, если номера этих фишек отличаются не более чем на $k$.
        При каком наименьшем $k$ серией таких ходов можно добиться расположения, в котором каждая фишка сдвинута на одну позицию по часовой стрелке по отношению к своему начальному положению?

    \end{enumerate}
\end{document}