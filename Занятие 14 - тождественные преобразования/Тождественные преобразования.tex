\documentclass{article}

\usepackage[12pt]{extsizes}
\usepackage[T2A]{fontenc}
\usepackage[utf8]{inputenc}
\usepackage[english, russian]{babel}

\usepackage{mathrsfs}
\usepackage[dvipsnames]{xcolor}

\usepackage{amsmath}
\usepackage{amssymb}
\usepackage{amsthm}
\usepackage{indentfirst}
\usepackage{amsfonts}
\usepackage{enumitem}
\usepackage{graphics}
\usepackage{tikz}
\usepackage{tabu}
\usepackage{diagbox}
\usepackage{hyperref}
\usepackage{mathtools}
\usepackage{ucs}
\usepackage{lipsum}
\usepackage{geometry} % Меняем поля страницы
\usepackage{fancyhdr} % Headers and footers
\newcommand{\range}{\mathrm{range}}
\newcommand{\dom}{\mathrm{dom}}
\newcommand{\N}{\mathbb{N}}
\newcommand{\R}{\mathbb{R}}
\newcommand{\E}{\mathbb{E}}
\newcommand{\D}{\mathbb{D}}
\newcommand{\M}{\mathcal{M}}
\newcommand{\Prime}{\mathbb{P}}
\newcommand{\A}{\mathbb{A}}
\newcommand{\Q}{\mathbb{Q}}
\newcommand{\Z}{\mathbb{Z}}
\newcommand{\F}{\mathbb{F}}
\newcommand{\CC}{\mathbb{C}}

\DeclarePairedDelimiter\abs{\lvert}{\rvert}
\DeclarePairedDelimiter\floor{\lfloor}{\rfloor}
\DeclarePairedDelimiter\ceil{\lceil}{\rceil}
\DeclarePairedDelimiter\lr{(}{)}
\DeclarePairedDelimiter\set{\{}{\}}
\DeclarePairedDelimiter\norm{\|}{\|}

\renewcommand{\labelenumi}{(\alph{enumi})}

\newcommand{\smallindent}{
    \geometry{left=1cm}% левое поле
    \geometry{right=1cm}% правое поле
    \geometry{top=1.5cm}% верхнее поле
    \geometry{bottom=1cm}% нижнее поле
}

\newcommand{\header}[3]{
    \pagestyle{fancy} % All pages have headers and footers
    \fancyhead{} % Blank out the default header
    \fancyfoot{} % Blank out the default footer
    \fancyhead[L]{#1}
    \fancyhead[C]{#2}
    \fancyhead[R]{#3}
}

\newcommand{\dividedinto}{
    \,\,\,\vdots\,\,\,
}

\newcommand{\littletaller}{\mathchoice{\vphantom{\big|}}{}{}{}}

\newcommand\restr[2]{{
    \left.\kern-\nulldelimiterspace % automatically resize the bar with \right
    #1 % the function
    \littletaller % pretend it's a little taller at normal size
    \right|_{#2} % this is the delimiter
}}

\DeclareGraphicsExtensions{.pdf,.png,.jpg}

\newenvironment{enumerate_boxed}[1][enumi]{\begin{enumerate}[label*=\protect\fbox{\arabic{#1}}]}{\end{enumerate}}



\smallindent

\header{Математика}{\textit{Алгебра}}{16 апреля 2024}

%----------------------------------------------------------------------------------------

%\begin{document}\normalsize
\begin{document}
    \large

    \begin{center}
        \textbf{Тождественные преобразования}
    \end{center}

    \begin{enumerate_boxed}

        \item Чему равняется сумма всех натуральных делителей числа $2^2 \cdot 3^3 \cdot 5^5.$

        \item Для каждого натурального $n \ge 2$ вычислите сумму
        \[
            \dfrac{1}{1} + \dfrac{1}{2} + \dotsc +  \dfrac{1}{n} + \dfrac{1}{1 \cdot 2} + \dfrac{1}{1 \cdot 3} + \dotso +  \dfrac{1}{(n-1) \cdot n} + \dotso + \dfrac{1}{1 \cdot 2 \cdot \dotso \cdot n}.
        \]
        (В знаменателях стоят все возможные произведения нескольких из чисел $1, 2, \dotsc, n$.
        Произведение одного числа равно самому этому числу).

        \item Упростите выражение $(1+3+3^2)\cdot (1+3^3 +3^6)\cdot (1+3^9 +3^{18})\cdot \dotsc \cdot  (1+3^{3^n} +3^{2\cdot 3^n})$

        \item На доске записаны 10 различных чисел.
        Профессор Odd вычислил всевозможные произведения нескольких записанных чисел, взятых в нечетном количестве (по 1, по 3, по 5, по 7, по 9), сложил все эти произведения и полученную сумму записал на листок.
        Аналогично профессор Even вычислил все возможные произведения нескольких чисел, записанных на доске, взятых в четном количестве (по 2, по 4, по 6, по 8, по 10), сложил все эти произведения и полученную сумму записал на свой листок.
        Оказалось, что сумма на листке профессора Odd на 1 больше, чем сумма на листке профессора Even.
        Докажите, что одно из чисел, выписанных на доске, равно 1.

        \item Рассмотрим все произведения некоторого количества чисел из $2, 3, \dotsc, n$ (произведение одного числа равно самому этому числу).
        Найдите сумму всех таких произведений, в которых чётное количество чётных сомножителей.

        \item Про натуральное число $n$ известно, что сумма его натуральных делителей является степенью двойки.
        Докажите, что тогда и количество его делителей является степенью двойки.

    \end{enumerate_boxed}
\end{document}