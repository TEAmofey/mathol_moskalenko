\documentclass{article}
\usepackage[12pt]{extsizes}
\usepackage[T2A]{fontenc}
\usepackage[utf8]{inputenc}
\usepackage[english, russian]{babel}

\usepackage{amssymb}
\usepackage{amsfonts}
\usepackage{amsmath}
\usepackage{enumitem}
\usepackage{graphics}
\usepackage{graphicx}

\usepackage{lipsum}

\newtheorem{theorem}{Теорема}
\newtheorem{task}{Задача}
\newtheorem{lemma}{Лемма}
\newtheorem{definition}{Определение}
\newtheorem{example}{Пример}
\newtheorem{statement}{Утверждение}
\newtheorem{corollary}{Следствие}


\usepackage{geometry} % Меняем поля страницы
\geometry{left=2cm}% левое поле
\geometry{right=2cm}% правое поле
\geometry{top=3cm}% верхнее поле
\geometry{bottom=1cm}% нижнее поле


\usepackage{fancyhdr} % Headers and footers
\pagestyle{fancy} % All pages have headers and footers
\fancyhead{} % Blank out the default header
\fancyfoot{} % Blank out the default footer
\fancyhead[L]{\textit{\textbf{САНКТ-ПЕТЕРБУРГСКАЯ ОЛИМПИАДА ШКОЛЬНИКОВ ПО МАТЕМАТИКЕ}}}

\fancyhead[R]{3.03.2024}% Custom header text


%----------------------------------------------------------------------------------------

%\begin{document}\normalsize
\begin{document}\large
	
\begin{center}
	\LARGE\textbf{9 класс}
\end{center}
\begin{center}
	\large\textbf{II тур}
\end{center}

\begin{enumerate}[label*=\textbf{\arabic{enumi}.}]
\setcounter{enumi}{0}
\item У Димы есть красный и синий фломастеры, одним из них он красит на числовой оси рациональные точки, а другим - иррациональные. Дима покрасил 100 рациональных и 100 иррациональных точек, после чего стер подписи, позволявшие узнать, где находилось начало координат и какой был масштаб. У Сергея есть циркуль, которым он может замерить расстояние между любыми двумя покрашенными точками $A$ и $B$, а потом отметить на оси точку, находящуюся на замеренном расстоянии от любой покрашенной точки $C$ (слева или справа); при этом Дима тут же красит её соответствующим фломастером. Как Сергею узнать, каким цветом Дима красит рациональные точки, а каким -- иррациональные?

\item Силач Бамбула может одновременно нести несколько гирь, если их суммарный вес не превосходит 200 кг, и этих гирь не больше трёх. По дороге на работу он повредил палец и обнаружил, что теперь может нести не более двух гирь (и по-прежнему не более 200 кг). При каком наименьшем $k$ верно утверждение: любой набор из 100 гирь, которые Бамбула раньше мог перенести за 50 заходов, с больным пальцем он сможет перенести не более чем за $k$ заходов?

\item Треугольник $ABC$ вписан в окружность. Из точек $B$ и $C$ одновременно выползают два муравья. Он ползут по дуге $BC$ навстречу друг другу так, что произведение расстояний от них до точки $A$ остается неизменным. Докажите, что во время их движения (до момента встречи) прямая, проходящая через муравьёв, касается некоторой фиксированной окружности.

\item Тренер выстроил в ряд 200 волейболистов и раздал им $m$ мячей (каждый волейболист мог получить сколько угодно мячей). Время от времени один из волейболистов кидает мяч другому (а тот ловит). Через некоторое время оказалось, что из любых двух волейболистов левый кидал правому мяч ровно два раза, а правый левому ровно один раз. При каком наименьшем $m$ это возможно?

\newpage

\item $AH$ - высота остроугольного треугольника $ABC$, вписанного в окружность $S$. На отрезке $BH$ выбрали точки $D$ и $E$. На лучах $AD$ и $AE$ выбрали точки $X$ и $Y$ соответственно так, что середины отрезков $DX$ и $EY$ лежат на окружности $S$. Оказалось, что точки $B, X, Y$ и $C$ лежат на одной окружности. Докажите, что $BD + BE = 2CH.$

\item Натуральное число $n$ назовём бедным, если уравнение
$$x_1x_2\dotsc x_{101} = (n-x_1)(n-x_2)\dotsc (n-x_{101})$$
не имеет решений в натуральных числах $1 \leqslant x_i < n$. Существует ли бедное число, имеющее более 100 000 различных простых делителей?

\item В очень большом Городе строят метро: много станций, некоторые пары которых соединены тоннелями, причем от любой станции можно добраться по тоннелям до любой другой. Все тоннели метро требуется разбить на «линии»: каждая линия состоит из нескольких последовательных тоннелей, все станции в которых различны (в частности, линия не должна быть кольцевой); допускаются и линии, состоящие из одного тоннеля. По закону требуется, чтобы от любой станции до любой другой можно было доехать, сделав не более 100 пересадок с линии на линию. При каком наибольшем $N$ любое связное метро с $N$ станциями можно разбить на линии, соблюдая закон?
\end{enumerate}
\newpage


\begin{center}
	\LARGE\textbf{10 класс}
\end{center}
\begin{center}
	\large\textbf{II тур}
\end{center}

\begin{enumerate}[label*=\textbf{\arabic{enumi}.}]
	\setcounter{enumi}{0}
	\item В клетках доски $2024 \times 2024$ расставлены целые числа так, что в любом прямоугольнике $2 \times 2023$ (вертикальном или горизонтальном) с одной вырезанной угловой клеткой, не выходящем за пределы доски, сумма чисел делится на 13. Докажите, что сумма всех чисел на доске делится на 13.
	
	\item Даны одинаковые на вид 32 настоящие и 32 фальшивые монеты. Все фальшивые монеты весят поровну и меньше настоящих, которые тоже все весят одинаково. Как за шесть взвешиваний на весах с двумя чашами определить тип хотя бы семи монет?
	
	\item На стороне $BC$ остроугольного треугольника $ABC$ отмечена точка $P$. Точка $E$ симметрична точке $B$ относительно прямой $AP$. Отрезок $PE$ пересекает описанную окружность треугольника $ABC$ в точке $D$. Точка $M$~--- середина стороны $AC$. Докажите, что $DE + AC > 2BM$.
	
	\item Рассмотрим всевозможные квадратные трехчлены вида $x^2 + ax + b$, где $a$ и $b$ — натуральные числа, не превосходящие некоторого натурального числа $N$. Докажите, что количество пар таких трехчленов, имеющих общий корень, не превосходит $N^2$
	
	\item На числовой оси отмечено 2000000 точек с целыми координатами. Рассматриваются отрезки длин 97, 100 и 103 с концами в этих точках. Какое наибольшее количество таких отрезков может быть?
	
	\item Дан вписанный шестиугольник $AB_1CA_1BC_!$. Окружность $\omega$ вписана и в треугольник $ABC$, и в треугольник $A_1B_1C_1$ и касается отрезков $AB$ и $A_1B_1$ в точках $D$ и $D_1$ соответственно. Докажите, что если $\angle ACD = \angle BCD_1$, то $\angle A_1C_1D_1 = \angle B_1C_1D$
	
	\item Ребра полного графа на 1000 вершинах покрашены в три цвета. Докажите, что в этом графе имеется несамопересекающийся одноцветный цикл, длина которого нечётна и не меньше 41.
\end{enumerate}
\newpage

\begin{center}
	\LARGE\textbf{11 класс}
\end{center}
\begin{center}
	\large\textbf{II тур}
\end{center}

\begin{enumerate}[label*=\textbf{\arabic{enumi}.}]
	\setcounter{enumi}{0}
	\item Таблица $100 \times 100$ заполнена числами от 1 до 10000 так, как показано на рисунке. Можно ли переставить некоторые числа так, чтобы в каждой клетке по-прежнему стояло одно число, и чтобы во всех прямоугольниках из трех клеток сумма чисел не изменилась?
	
	\item Дана последовательность $a_n$:
	$$1, 2, 2, 3, 3, 3, 4, 4, 4, 4, \dotsc$$
	(одна единица, две двойки, три тройки и т.д.) и еще одна последовательность $b_n$ такая, что $a_{b_n} 
	= b_{a_n}$ для всех натуральных $n$. Известно, что $b_k = 1$ при некотором $k > 100$. Докажите, что $b_m = 1$ при всех $m > k$.
	
	\item В неравнобедренном треугольнике $ABC$ проведена биссектриса $AK$. Диаметр $XY$ его описанной окружности перпендикулярен прямой $AK$ (порядок точек на описанной окружности $B-X-A-Y-C$). Окружность, проходящая через точки $X$ и $Y$, пересекает отрезки $BK$ и $CK$ в точках $T$ и $Z$ соответственно. Докажите, что если $KZ = KT$, то $XT \perp YZ$.
	
	\item Дано 101-значное число $a$ и произвольное натуральное число $b$. Докажите, что найдется такое не более чем 102-значное натуральное число $c$, что любое число вида $\overline{caaaa...ab}$ - составное.
	
	\item На плоскости отмечено 100 точек общего положения (т. е. никакие три не лежат на одной прямой). Докажите, что можно выбрать три отмеченные точки $A, B, C$ так, чтобы для любой точки $D$ из оставшихся 97 отмеченных точек, прямые $AD$ и $CD$ не содержали бы точек, лежащих внутри треугольника $ABC$.
	
	\item 
	
	\item 
	
\end{enumerate}

\end{document}