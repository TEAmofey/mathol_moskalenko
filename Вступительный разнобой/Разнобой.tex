\documentclass{article}

\usepackage[12pt]{extsizes}
\usepackage[T2A]{fontenc}
\usepackage[utf8]{inputenc}
\usepackage[english, russian]{babel}

\usepackage{mathrsfs}
\usepackage[dvipsnames]{xcolor}

\usepackage{amsmath}
\usepackage{amssymb}
\usepackage{amsthm}
\usepackage{indentfirst}
\usepackage{amsfonts}
\usepackage{enumitem}
\usepackage{graphics}
\usepackage{tikz}
\usepackage{tabu}
\usepackage{diagbox}
\usepackage{hyperref}
\usepackage{mathtools}
\usepackage{ucs}
\usepackage{lipsum}
\usepackage{geometry} % Меняем поля страницы
\usepackage{fancyhdr} % Headers and footers
\newcommand{\range}{\mathrm{range}}
\newcommand{\dom}{\mathrm{dom}}
\newcommand{\N}{\mathbb{N}}
\newcommand{\R}{\mathbb{R}}
\newcommand{\E}{\mathbb{E}}
\newcommand{\D}{\mathbb{D}}
\newcommand{\M}{\mathcal{M}}
\newcommand{\Prime}{\mathbb{P}}
\newcommand{\A}{\mathbb{A}}
\newcommand{\Q}{\mathbb{Q}}
\newcommand{\Z}{\mathbb{Z}}
\newcommand{\F}{\mathbb{F}}
\newcommand{\CC}{\mathbb{C}}

\DeclarePairedDelimiter\abs{\lvert}{\rvert}
\DeclarePairedDelimiter\floor{\lfloor}{\rfloor}
\DeclarePairedDelimiter\ceil{\lceil}{\rceil}
\DeclarePairedDelimiter\lr{(}{)}
\DeclarePairedDelimiter\set{\{}{\}}
\DeclarePairedDelimiter\norm{\|}{\|}

\renewcommand{\labelenumi}{(\alph{enumi})}

\newcommand{\smallindent}{
    \geometry{left=1cm}% левое поле
    \geometry{right=1cm}% правое поле
    \geometry{top=1.5cm}% верхнее поле
    \geometry{bottom=1cm}% нижнее поле
}

\newcommand{\header}[3]{
    \pagestyle{fancy} % All pages have headers and footers
    \fancyhead{} % Blank out the default header
    \fancyfoot{} % Blank out the default footer
    \fancyhead[L]{#1}
    \fancyhead[C]{#2}
    \fancyhead[R]{#3}
}

\newcommand{\dividedinto}{
    \,\,\,\vdots\,\,\,
}

\newcommand{\littletaller}{\mathchoice{\vphantom{\big|}}{}{}{}}

\newcommand\restr[2]{{
    \left.\kern-\nulldelimiterspace % automatically resize the bar with \right
    #1 % the function
    \littletaller % pretend it's a little taller at normal size
    \right|_{#2} % this is the delimiter
}}

\DeclareGraphicsExtensions{.pdf,.png,.jpg}

\newenvironment{enumerate_boxed}[1][enumi]{\begin{enumerate}[label*=\protect\fbox{\arabic{#1}}]}{\end{enumerate}}



\smallindent

\header{Математика}{\textit{Разное}}{7 Августа 2022}

%----------------------------------------------------------------------------------------

%\begin{document}\normalsize
\begin{document}
    \large

    \begin{center}
        \textbf{Разнобой}
    \end{center}


    \begin{enumerate_boxed}

        \item Серёжа написал на доске три натуральных числа, а затем вычислил их попарные НОДы и НОКи.
        Могла ли сумма шести полученных чисел оказаться равной 2023?

        \item Докажите, что $2016 \cdot 2018 \cdot 2020 \cdot 2022+16$~--- точный квадрат

        \item Шестиугольник $ABCDEF$ правильный, $K$ и $M$ -- середины отрезков $BD$ и $EF$.
        Докажите, что треугольник $AMK$ равносторонний.

        \item В очереди стояло 2022 человека.
        Касса сломалась, и все перешли в соседнюю только что открытую кассу.
        Сколькими способами они могут выстроиться в новую очередь так, чтобы человек, стоявший на месте с номером $k$ изменил свой номер не более, чем на $k$?

        \item Найдите все натуральные $n$ такие, что $3^n + 5^n$ делится на $3^{n-1} + 5^{n-1}$.

        \item Докажите, что диагонали четырехугольника перпендикулярны тогда и только тогда, когда суммы квадратов противоположных сторон равны.

        \item Шах разбил свой квадратный одноэтажный дворец на $64$ одинаковые квадратные комнаты, разделил комнаты на семь квартир (проделав двери в некоторых перегородках между комнатами) и в каждой квартире поселил по жене.
        Жёны могут ходить по всем комнатам своей квартиры, не заходя к другим.
        Какое наименьшее число дверей пришлось проделать во внутренних стенах?

        \item Докажите, что $1^2 + 2^2 + \dots + n^2 =\dfrac{n (n+1)(2n + 1)}{6}$

        \item Диагонали трапеции взаимно перпендикулярны.
        Докажите, что произведение длин оснований трапеции равно сумме произведений длин отрезков одной диагонали и длин отрезков другой диагонали, на которые они делятся точкой пересечения.

        \item В компании у каждых двух людей ровно пять общих знакомых.
        Докажите, что количество пар знакомых делится на 3.

        \item Докажите, что $2021^2+2021^2 \cdot 2022^2+2022^2$~--- точный квадрат.

        \item Вещественные числа $a$, $b$ и $c$ таковы, что $a+b+c=0$.
        Докажите, что $ab+bc+ac \le 0$

        \item Докажите, что $1^3 + 2^3 + \dots + n^3 =(1 + 2 + \dots + n) ^ 2$

    \end{enumerate_boxed}
\end{document}