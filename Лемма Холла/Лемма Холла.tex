\documentclass{article}

\usepackage[12pt]{extsizes}
\usepackage[T2A]{fontenc}
\usepackage[utf8]{inputenc}
\usepackage[english, russian]{babel}

\usepackage{mathrsfs}
\usepackage[dvipsnames]{xcolor}

\usepackage{amsmath}
\usepackage{amssymb}
\usepackage{amsthm}
\usepackage{indentfirst}
\usepackage{amsfonts}
\usepackage{enumitem}
\usepackage{graphics}
\usepackage{tikz}
\usepackage{tabu}
\usepackage{diagbox}
\usepackage{hyperref}
\usepackage{mathtools}
\usepackage{ucs}
\usepackage{lipsum}
\usepackage{geometry} % Меняем поля страницы
\usepackage{fancyhdr} % Headers and footers
\newcommand{\range}{\mathrm{range}}
\newcommand{\dom}{\mathrm{dom}}
\newcommand{\N}{\mathbb{N}}
\newcommand{\R}{\mathbb{R}}
\newcommand{\E}{\mathbb{E}}
\newcommand{\D}{\mathbb{D}}
\newcommand{\M}{\mathcal{M}}
\newcommand{\Prime}{\mathbb{P}}
\newcommand{\A}{\mathbb{A}}
\newcommand{\Q}{\mathbb{Q}}
\newcommand{\Z}{\mathbb{Z}}
\newcommand{\F}{\mathbb{F}}
\newcommand{\CC}{\mathbb{C}}

\DeclarePairedDelimiter\abs{\lvert}{\rvert}
\DeclarePairedDelimiter\floor{\lfloor}{\rfloor}
\DeclarePairedDelimiter\ceil{\lceil}{\rceil}
\DeclarePairedDelimiter\lr{(}{)}
\DeclarePairedDelimiter\set{\{}{\}}
\DeclarePairedDelimiter\norm{\|}{\|}

\renewcommand{\labelenumi}{(\alph{enumi})}

\newcommand{\smallindent}{
    \geometry{left=1cm}% левое поле
    \geometry{right=1cm}% правое поле
    \geometry{top=1.5cm}% верхнее поле
    \geometry{bottom=1cm}% нижнее поле
}

\newcommand{\header}[3]{
    \pagestyle{fancy} % All pages have headers and footers
    \fancyhead{} % Blank out the default header
    \fancyfoot{} % Blank out the default footer
    \fancyhead[L]{#1}
    \fancyhead[C]{#2}
    \fancyhead[R]{#3}
}

\newcommand{\dividedinto}{
    \,\,\,\vdots\,\,\,
}

\newcommand{\littletaller}{\mathchoice{\vphantom{\big|}}{}{}{}}

\newcommand\restr[2]{{
    \left.\kern-\nulldelimiterspace % automatically resize the bar with \right
    #1 % the function
    \littletaller % pretend it's a little taller at normal size
    \right|_{#2} % this is the delimiter
}}

\DeclareGraphicsExtensions{.pdf,.png,.jpg}

\newenvironment{enumerate_boxed}[1][enumi]{\begin{enumerate}[label*=\protect\fbox{\arabic{#1}}]}{\end{enumerate}}


\usepackage[framemethod=TikZ]{mdframed}

\newcommand{\definebox}[3]{%
    \newcounter{#1}
    \newenvironment{#1}[1][]{%
        \stepcounter{#1}%
        \mdfsetup{%
            frametitle={%
            \tikz[baseline=(current bounding box.east),outer sep=0pt]
            \node[anchor=east,rectangle,fill=white]
            {\strut #2~\csname the#1\endcsname\ifstrempty{##1}{}{##1}};}}%
        \mdfsetup{innertopmargin=1pt,linecolor=#3,%
            linewidth=3pt,topline=true,
            frametitleaboveskip=\dimexpr-\ht\strutbox\relax,}%
        \begin{mdframed}[]
            \relax%
            }{
        \end{mdframed}}%
}

\definebox{theorem_boxed}{Теорема}{ForestGreen!24}
\definebox{definition_boxed}{Определение}{blue!24}
\definebox{task_boxed}{Задача}{orange!24}
\definebox{paradox_boxed}{Парадокс}{red!24}

\theoremstyle{plain}
\newtheorem{theorem}{Теорема}
\newtheorem{task}{Задача}
\newtheorem{lemma}{Лемма}
\newtheorem{definition}{Определение}
\newtheorem{statement}{Утверждение}
\newtheorem{corollary}{Следствие}

\theoremstyle{remark}
\newtheorem{remark}{Замечание}
\newtheorem{example}{Пример}

\smallindent

\header{Математика}{\textit{Комбинаторика}}{23 июня 2024}

%----------------------------------------------------------------------------------------

\begin{document}
    \large

    \begin{center}
        \textbf{Лемма Холла}
    \end{center}

    \begin{enumerate_boxed}

        \item \textbf{(Лемма Холла)} Есть $n$ юношей и несколько девушек.
        Тогда все юноши могут выбрать по невесте из числа своих знакомых тогда и только тогда, когда выполнено условие разнообразия: любому набору из $k$ юношей в совокупности знакомы не менее $k$ девушек.


        \item \textbf{(Феминистическая лемма Холла)} В компании из $n$ юношей и $n$ девушек каждые $k$ юношей знакомы не менее чем с $k$ девушками.
        Докажите, что каждые $k$ девушек знакомы не менее, чем с $k$ юношами.


        \item \textbf{(Арабская лемма Холла)} Среди $n$ юношей и нескольких девушек некоторые юноши знакомы с некоторыми девушками.
        Каждый юноша хочет жениться на $m$ знакомых девушках.
        Докажите, что они могут это сделать тогда и только тогда, когда для любого набора из $k$ юношей количество знакомых им в совокупности девушек не меньше
        $km$.


        \item \textbf{(Физматовская лемма Холла)} Докажите, что если
        любые $k\, (1 \leqslant k \leqslant n)$ юношей знакомы в совокупности не менее чем с $k - d$ девушками, то $n - d$ юношей могут выбрать себе невест из числа знакомых.


        \item \textbf{(Деревенская лемма Холла)} В нескольких деревнях
        есть женихи и невесты, тех и других поровну.
        В каждой деревне общее число женихов и невест не больше половины общего их числа.
        Докажите, что можно всех переженить так, чтобы в каждой паре жених и невеста были из разных деревень.


        \item \textbf{(Пуританская лемма Холла)} Есть $n$ юношей и $n$ девушек.
        Каждый юноша знает хотя бы одну девушку.
        Докажите, что можно некоторых юношей поженить на знакомых девушках так, чтобы женатые юноши не знали незамужних девушек.


        \item \textbf{(Викторианская лемма Холла)} Есть $n$ юношей и
        $2n - 1$ девушка.
        Докажите, что можно переженить всех юношей так, чтобы каждому юноше либо нравилась его жена, либо не нравилась никакая чужая жена.


        \item \textbf{(Кёнигсбергская лемма Холла)} В компании юношей и девушек не менее $n$ человек.
        Оказалось, что среди них нельзя составить $n + 1$ брак так, чтобы каждый юноша был знаком со своей женой.
        Докажите, что тогда можно выбрать $n$ людей и запереть их в крепости так, что из оставшихся нельзя было составить ни одной супружеской пары.


        \item \textbf{(Французская лемма Холла)} Есть $n$ юношей и $n$ девушек.
        Известно, что для любых $k$ юношей количество их знакомых девушек плюс количество девушек, знакомых хотя бы с двумя из них, не меньше чем $2k$.
        Докажите,что каждый юноша может выбрать себе жену и любовницу (это разные девушки) так, чтобы и у любой девушки оказался один муж и один любовник.


        \item \textbf{(Вообще без комментариев)} В классе учатся $n$
        мальчиков и $n$ девочек.
        Каждый мальчик составил рейтинг девочек в порядке убывания: какая нравится ему больше всего, какая на втором месте и т.д. (никакие две девочки не нравятся никакому мальчику в одинаковой степени).
        На день святого Валентина каждому мальчику подарили по девочке.
        Обсуждая полученные подарки, мальчики заметили, что при любом другом распределении девочек хотя бы одному из них досталась бы меньше нравящаяся ему девочка.
        Докажите, что хотя бы один из мальчиков получил ту девочку, которая нравилась ему больше всего.

    \end{enumerate_boxed}

\end{document}