\documentclass{article}
\usepackage[12pt]{extsizes}
\usepackage[T2A]{fontenc}
\usepackage[utf8]{inputenc}
\usepackage[english, russian]{babel}

\usepackage{amssymb}
\usepackage{amsfonts}
\usepackage{amsmath}
\usepackage{enumitem}
\usepackage{graphics}
\usepackage{graphicx}

\usepackage{lipsum}

\newtheorem{theorem}{Теорема}
\newtheorem{task}{Задача}
\newtheorem{lemma}{Лемма}
\newtheorem{definition}{Определение}
\newtheorem{example}{Пример}
\newtheorem{statement}{Утверждение}
\newtheorem{corollary}{Следствие}


\usepackage{geometry} % Меняем поля страницы
%\geometry{left=1cm}% левое поле
%\geometry{right=1cm}% правое поле
\geometry{top=3cm}% верхнее поле
%\geometry{bottom=1cm}% нижнее поле


\usepackage{fancyhdr} % Headers and footers
\pagestyle{fancy} % All pages have headers and footers
\fancyhead{} % Blank out the default header
\fancyfoot{} % Blank out the default footer
\fancyhead[C]{\textit{\textbf{САНКТ-ПЕТЕРБУРГСКАЯОЛИМПИАДА ШКОЛЬНИКОВ ПО МАТЕМАТИКЕ}}}

\fancyhead[R]{19 февраля 2023}% Custom header text


%----------------------------------------------------------------------------------------

%\begin{document}\normalsize
\begin{document}\large
	
\begin{center}
	\LARGE\textbf{8 класс}
\end{center}
\begin{center}
	\large\textbf{II тур}
\end{center}

\begin{enumerate}[label*=\textbf{\arabic{enumi}.}]
\setcounter{enumi}{0}
\item На основании $AD$ трапеции $ABCD$ нашлась такая точка $E$, что $AE = EC$ и $BE = ED$. Докажите, что точка пересечения диагоналей трапеции лежит на биссектрисе угла $BEC$.

\item Найдите все такие натуральные $n$, что десятичная запись числа $n^2$ начинается с числа $n$.

\item По кругу стоят 300 стаканов, некоторые из них --- с водой, а остальные с березовым соком, причём нет трех стаканов подряд с одинаковой жидкостью. За ход можно отпить из одного стакана. Можно ли за 200 проб гарантированно определить содержимое всех стаканов?

\item Можно ли так расположить на плоскости четыре квадрата попарно разных размеров, чтобы любые два квадрата имели общую вершину, а никакие три общих вершин не имели?

\item У Саши есть 80 кусков пластилина. За один ход он может склеить три любых куска в один, после чего разделить этот кусок на три одинаковых по весу куска.
Докажите, что Саша может добиться того, чтобы через несколько ходов среди имеющихся у него кусков пластилина не нашлось бы трех кусков попарно разного веса.

\item Дано натуральное число $m$. Сумма нескольких натуральных чисел равна $m^2$. Докажите, что количество чисел, которые являются делителем хотя бы одного из них, меньше чем $2m$.

\item Существует ли выпуклый многоугольник, который можно разрезать непересекающимися (во внутренних точках) диагоналями на треугольники равной площади хотя бы тремя разными способами?

\end{enumerate}
\end{document}