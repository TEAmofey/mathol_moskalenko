\documentclass{article}

\usepackage[12pt]{extsizes}
\usepackage[T2A]{fontenc}
\usepackage[utf8]{inputenc}
\usepackage[english, russian]{babel}

\usepackage{mathrsfs}
\usepackage[dvipsnames]{xcolor}

\usepackage{amsmath}
\usepackage{amssymb}
\usepackage{amsthm}
\usepackage{indentfirst}
\usepackage{amsfonts}
\usepackage{enumitem}
\usepackage{graphics}
\usepackage{tikz}
\usepackage{tabu}
\usepackage{diagbox}
\usepackage{hyperref}
\usepackage{mathtools}
\usepackage{ucs}
\usepackage{lipsum}
\usepackage{geometry} % Меняем поля страницы
\usepackage{fancyhdr} % Headers and footers
\newcommand{\range}{\mathrm{range}}
\newcommand{\dom}{\mathrm{dom}}
\newcommand{\N}{\mathbb{N}}
\newcommand{\R}{\mathbb{R}}
\newcommand{\E}{\mathbb{E}}
\newcommand{\D}{\mathbb{D}}
\newcommand{\M}{\mathcal{M}}
\newcommand{\Prime}{\mathbb{P}}
\newcommand{\A}{\mathbb{A}}
\newcommand{\Q}{\mathbb{Q}}
\newcommand{\Z}{\mathbb{Z}}
\newcommand{\F}{\mathbb{F}}
\newcommand{\CC}{\mathbb{C}}

\DeclarePairedDelimiter\abs{\lvert}{\rvert}
\DeclarePairedDelimiter\floor{\lfloor}{\rfloor}
\DeclarePairedDelimiter\ceil{\lceil}{\rceil}
\DeclarePairedDelimiter\lr{(}{)}
\DeclarePairedDelimiter\set{\{}{\}}
\DeclarePairedDelimiter\norm{\|}{\|}

\renewcommand{\labelenumi}{(\alph{enumi})}

\newcommand{\smallindent}{
    \geometry{left=1cm}% левое поле
    \geometry{right=1cm}% правое поле
    \geometry{top=1.5cm}% верхнее поле
    \geometry{bottom=1cm}% нижнее поле
}

\newcommand{\header}[3]{
    \pagestyle{fancy} % All pages have headers and footers
    \fancyhead{} % Blank out the default header
    \fancyfoot{} % Blank out the default footer
    \fancyhead[L]{#1}
    \fancyhead[C]{#2}
    \fancyhead[R]{#3}
}

\newcommand{\dividedinto}{
    \,\,\,\vdots\,\,\,
}

\newcommand{\littletaller}{\mathchoice{\vphantom{\big|}}{}{}{}}

\newcommand\restr[2]{{
    \left.\kern-\nulldelimiterspace % automatically resize the bar with \right
    #1 % the function
    \littletaller % pretend it's a little taller at normal size
    \right|_{#2} % this is the delimiter
}}

\DeclareGraphicsExtensions{.pdf,.png,.jpg}

\newenvironment{enumerate_boxed}[1][enumi]{\begin{enumerate}[label*=\protect\fbox{\arabic{#1}}]}{\end{enumerate}}


\usepackage[framemethod=TikZ]{mdframed}

\newcommand{\definebox}[3]{%
    \newcounter{#1}
    \newenvironment{#1}[1][]{%
        \stepcounter{#1}%
        \mdfsetup{%
            frametitle={%
            \tikz[baseline=(current bounding box.east),outer sep=0pt]
            \node[anchor=east,rectangle,fill=white]
            {\strut #2~\csname the#1\endcsname\ifstrempty{##1}{}{##1}};}}%
        \mdfsetup{innertopmargin=1pt,linecolor=#3,%
            linewidth=3pt,topline=true,
            frametitleaboveskip=\dimexpr-\ht\strutbox\relax,}%
        \begin{mdframed}[]
            \relax%
            }{
        \end{mdframed}}%
}

\definebox{theorem_boxed}{Теорема}{ForestGreen!24}
\definebox{definition_boxed}{Определение}{blue!24}
\definebox{task_boxed}{Задача}{orange!24}
\definebox{paradox_boxed}{Парадокс}{red!24}

\theoremstyle{plain}
\newtheorem{theorem}{Теорема}
\newtheorem{task}{Задача}
\newtheorem{lemma}{Лемма}
\newtheorem{definition}{Определение}
\newtheorem{statement}{Утверждение}
\newtheorem{corollary}{Следствие}

\theoremstyle{remark}
\newtheorem{remark}{Замечание}
\newtheorem{example}{Пример}

\smallindent

\header{Математика}{\textit{Геометрия}}{22 июля 2024}

%----------------------------------------------------------------------------------------

\begin{document}
    \large

    \begin{center}
        \textbf{Площади}
    \end{center}

    \begin{enumerate_boxed}
        \item Найдите множество таких точек $P$ внутри треугольника $ABC$, что $A$ треугольники $ABP$ и $CAP$
        \begin{enumerate}
            \item равновелики;
            \item $S_{ABP} : S_{CAP} = 2:1$.
        \end{enumerate}
        \item \textbf{Лемма об отношении площадей треугольников с общей стороной:}
        $S_{ABC'} : S_{ABC} = PC' : PC$, где $P$~--- точка пересечения прямых $AB$ и $CC'$
        \begin{enumerate}
            \item Докажите эту лемму.
            \item Что можно сказать о площадях $ABC$ и $ABC'$, если точка $P$ не существует (прямые $AB$ и $CC'$ параллельны)?
        \end{enumerate}
        \item Найдите множество точек $P$, таких что $S_{PCD} = S_{PAB}$, Где $AB$ и $CD$~--- произвольные отрезки.
        \item Каждая вершина треугольника симметрично отражается относительно следующей по часовой стрелке вершины.
        Полученные три точки задают новый треугольник.
        Во сколько раз его площадь больше площади исходного треугольника?

        \item На сторонах треугольника $ABC$ площади $S$ выбраны точки $A_1, B_1, C_1$ так, что \[BA_1: A_{1}C = 3:1, AC_1:C_{1}B = CB_1:B_{1}A=1:5.
        \]
        Найдите площадь треугольника $A_{1}B_{1}C_1$.
        \item В условиях задачи 5 найдите отношения, в которых отрезки $AA_1$ и $B_{1}C_1$ делят друг друга.
        \item В параллелограмме $ABCD$ точки $E$ и $F$ лежат соответственно на сторонах $AB$ и $BC$, $M$~--- точка пересечения прямых $AF$ и $DE$, причем $AE = 2BE$, а $BF = 3CF$.
        Найдите отношения $AM:MF$ и $DM:ME$.
        \item Точка $F$ лежит на продолжении стороны $BC$ параллелограмма $ABCD$ за точку $C$.
        Отрезок $AF$ пересекает диагональ $BD$ в точке $E$ и сторону $CD$ в точке $G$.
        Известно, что $AE=2, GF=3$.
        Найти отношение площадей треугольников $BAE$ и $EDG$.

        \item Вершины параллелограмма площади $S$ соединены с серединами противоположных сторон, как показано на рисунке.
        Найдите площадь закрашенной фигуры.
        \begin{center}
            \begin{asy}
                import geometry;
                import patterns;

                size(5cm,0);

                point pA = (-1,-1);
                point pB = (-2,1);
                point pC = (1,1);
                point pD = (2,-1);


                point pK = intersectionpoint(pA -- (pB + pC) / 2, pB -- (pC + pD) / 2);
                point pL = intersectionpoint(pC -- (pD + pA) / 2, pB -- (pC + pD) / 2);
                point pM = intersectionpoint(pC -- (pD + pA) / 2, pD -- (pA + pB) / 2);
                point pN = intersectionpoint(pA -- (pB + pC) / 2, pD -- (pA + pB) / 2);

                draw(pA -- pB -- pC -- pD -- cycle);
                draw(pA -- (pB + pC) / 2);
                draw(pB -- (pC + pD) / 2);
                draw(pC -- (pD + pA) / 2);
                draw(pD -- (pA + pB) / 2);

                filldraw(pK -- pL -- pM -- pN -- cycle);
            \end{asy}
        \end{center}

        \item Докажите, что если площадь четырехугольника вдвое меньше площади описанного около него параллелограмма,то одна из его диагоналей параллельна стороне параллелограмма.

        \item Точки $A_1, B_1, C_1$ делят стороны $BC, CA, AB$ треугольника $ABC$ площади $1$ в одинаковом отношении $1:2$, считая от вершин $B, C, A$ соответственно.
        Найдите площадь треугольника, ограниченного чевианами $AA_1, BB_1, CC_1$.
        \item Каждая сторона треугольника поделена на три равные части и точки деления соединены с противоположной к этой стороне вершиной.
        Найдите отношение площади шестиугольника, ограниченного проведенными отрезками, к площади треугольника.
        \item Где нужно взять точки на двух сторонах треугольника так, чтобы из четырех частей, на которые разбивается треугольник чевианами, проведенными из выбранных точек, по крайней мере три были равновелики?
        Площадь и подобие
        \item Из точки $M$ на основании $AB$ треугольника $ABC$ проведены прямые параллельно двум другим сторонам.
        Площадь отсекаемого ими параллелограмма составляет $5/18$ площади треугольника.
        Найдите отношение $AM/MB$.
        \item Прямые, проходящие через точку внутри данного треугольника и параллельные его сторонам, разбивают его на три параллелограмма и три треугольника.
        \begin{enumerate}
            \item Докажите, что площади $s_1, s_2$ и $s_3$ треугольников, отсекаемых от данного треугольника этими прямыми и площадь $S$ данного треугольника связаны соотношением $\sqrt{s_1} + \sqrt{s_2} + \sqrt{s_3} = \sqrt{S}$.
            \item Выразите $S$ через площади $S_1, S_2, S_3$ параллелограммов.
        \end{enumerate}
        \item Через точку $M$ внутри параллелограмма $ABCD$ проведены прямые $\ell$ и $m$, параллельные его сторонам.
        \begin{enumerate}
            \item Найдите геометрическое место таких точек $M$, что образовавшиеся параллелограммы с диагоналями $MA$ и $MC$ равновелики.
            \item Соединим точки пересечения $\ell$ и $m$
            со сторонами $AB$ и $AD$ с вершинами $D$ и $B$, соответственно.
            Докажите, что получившиеся прямые и прямая $MC$ пересекаются в одной точке.
            (Это теорема о трех параллелограммах.)
        \end{enumerate}

        \item Пусть $ABCD$~--- прямоугольник и на сторонах $AB$ и $AD$ отложены равные отрезки $BE = DF$.
        Докажите, что прямая $CP$, где $P = BF \cap DE$, делит пополам угол $C$.

        \item Через вершины $A$ и $C$ треугольника $BC$ проведены прямые параллельно сторонам $BC$ и $AC$ соответственно; на них отложены равные отрезки $AD$ и $CE$, $F$~--- точка пересечения прямых $AE$ и $CD$.
        Докажите, что $BF$~--- биссектриса угла $B$, если отрезки отложены по одну сторону от $AC$, и внешняя биссектриса, если по разные стороны,
    \end{enumerate_boxed}

\end{document}