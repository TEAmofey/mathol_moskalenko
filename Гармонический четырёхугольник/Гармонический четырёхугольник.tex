\documentclass{article}

\usepackage[12pt]{extsizes}
\usepackage[T2A]{fontenc}
\usepackage[utf8]{inputenc}
\usepackage[english, russian]{babel}

\usepackage{mathrsfs}
\usepackage[dvipsnames]{xcolor}

\usepackage{amsmath}
\usepackage{amssymb}
\usepackage{amsthm}
\usepackage{indentfirst}
\usepackage{amsfonts}
\usepackage{enumitem}
\usepackage{graphics}
\usepackage{tikz}
\usepackage{tabu}
\usepackage{diagbox}
\usepackage{hyperref}
\usepackage{mathtools}
\usepackage{ucs}
\usepackage{lipsum}
\usepackage{geometry} % Меняем поля страницы
\usepackage{fancyhdr} % Headers and footers
\newcommand{\range}{\mathrm{range}}
\newcommand{\dom}{\mathrm{dom}}
\newcommand{\N}{\mathbb{N}}
\newcommand{\R}{\mathbb{R}}
\newcommand{\E}{\mathbb{E}}
\newcommand{\D}{\mathbb{D}}
\newcommand{\M}{\mathcal{M}}
\newcommand{\Prime}{\mathbb{P}}
\newcommand{\A}{\mathbb{A}}
\newcommand{\Q}{\mathbb{Q}}
\newcommand{\Z}{\mathbb{Z}}
\newcommand{\F}{\mathbb{F}}
\newcommand{\CC}{\mathbb{C}}

\DeclarePairedDelimiter\abs{\lvert}{\rvert}
\DeclarePairedDelimiter\floor{\lfloor}{\rfloor}
\DeclarePairedDelimiter\ceil{\lceil}{\rceil}
\DeclarePairedDelimiter\lr{(}{)}
\DeclarePairedDelimiter\set{\{}{\}}
\DeclarePairedDelimiter\norm{\|}{\|}

\renewcommand{\labelenumi}{(\alph{enumi})}

\newcommand{\smallindent}{
    \geometry{left=1cm}% левое поле
    \geometry{right=1cm}% правое поле
    \geometry{top=1.5cm}% верхнее поле
    \geometry{bottom=1cm}% нижнее поле
}

\newcommand{\header}[3]{
    \pagestyle{fancy} % All pages have headers and footers
    \fancyhead{} % Blank out the default header
    \fancyfoot{} % Blank out the default footer
    \fancyhead[L]{#1}
    \fancyhead[C]{#2}
    \fancyhead[R]{#3}
}

\newcommand{\dividedinto}{
    \,\,\,\vdots\,\,\,
}

\newcommand{\littletaller}{\mathchoice{\vphantom{\big|}}{}{}{}}

\newcommand\restr[2]{{
    \left.\kern-\nulldelimiterspace % automatically resize the bar with \right
    #1 % the function
    \littletaller % pretend it's a little taller at normal size
    \right|_{#2} % this is the delimiter
}}

\DeclareGraphicsExtensions{.pdf,.png,.jpg}

\newenvironment{enumerate_boxed}[1][enumi]{\begin{enumerate}[label*=\protect\fbox{\arabic{#1}}]}{\end{enumerate}}



\smallindent

\header{Математика}{\textit{Геометрия}}{13 марта 2024}

%----------------------------------------------------------------------------------------

%\begin{document}\normalsize
\begin{document}
    \large

    \begin{center}
        \textbf{Гармонический четырёхугольник}
    \end{center}

    \begin{definition}
        Вписанный четырехугольник называется гармоническим, если произведения длин
        его противоположных сторон равны.
    \end{definition}

    \begin{enumerate_boxed}

        \item
        \begin{enumerate}
            \item $ABCD$ — гармонический четырехугольник, $M$ — точка пересечения его диагоналей.
            Докажите, что

            $\dfrac{|AM|}{|MC|}=\dfrac{|AB|^2}{|BC|^2}=\dfrac{|AD|^2}{|DC|^2}.$
            \item Докажите, что каждая диагональ гармонического четырехугольника является симедианой
            треугольников, на которые разбивает четырехугольник другая диагональ.

            \item Диагональ $BD$ вписанного четырехугольника $ABCD$ является симедианой треугольника
            $ABC$.
            Докажите, что четырехугольник $ABCD$ гармонический.
        \end{enumerate}

        \item Гармонический четырёхугольник $ABCD$ вписан в окружность с центром $O$; точка $M$ — середина диагонали $BD$.
        Докажите, что точки $M, O, A, C$ лежат на одной окружности или прямой.

        \item Через точку $A$, лежащую вне окружности, проведены касательные $AB$ и $AC$ к этой окружности, а также прямая, пересекающая окружность в точках $X$ и $Y$.
        Докажите, что точки $A, B, C,$ и середина отрезка $XY$ лежат на одной окружности.

        \item Касательные к описанной окружности треугольника $ABC$ (у которого $\angle BAC \neq 90^\circ$), восстановленные в вершинах $B$ и $C$, пересекаются в точке $S$; точка $M$ — середина $BC$.
        Докажите, что прямые $AM$ и $AS$ симметричны относительно биссектрисы угла $BAC$.

        \item Пусть $N$ — середина диагонали $AC$ вписанного четырехугольника $ABCD$.
        Докажите, что$ABCD$ — гармонический тогда и только тогда, когда $\angle BNC = \angle DNC$.

        \item Пусть $ABCD$ — вписанный четырехугольник, в котором биссектрисы углов $A$ и $C$ пересекаются на диагонали $BD$.
        Докажите, что биссектрисы углов $B$ и $D$ пересекаются на диагонали $AC$.


        \item В окружности $S$ проведены две параллельные хорды $AB$ и $CD$.
        Прямая, проведенная через $C$ и середину $AB$, вторично пересекает $S$ в точке $E$.
        Точка $K$ — середина отрезка $DE$.
        Докажите, что $\angle AKE = \angle BKE.$

        \item Две окружности $\omega_1$ и $\omega_2$ пересекаются в точках $P$ и $Q$.
        Касательные к окружности $\omega_1$ в точках $P$ и $Q$ пересекаются в точке $S$.
        На окружности $\omega_1$ вне окружности $\omega_2$ отмечена точка $A$.
        Прямые $AP$ и $AQ$ второй раз пересекают окружность $\omega_2$ в точках $B$ и $C$.
        Докажите, что прямая $AS$ делит отрезок $BC$ пополам.

        \item Биссектриса угла $BAC$ пересекает отрезок $BC$ и описанную окружность $\omega$ неравнобедренного треугольника $ABC$ в точках $D$ и $E$ соответственно.
        На окружности $\omega$ отмечена такая точка $S$, что $\angle DSE = 90^\circ$.
        Докажите, что прямая $AS$ содержит симедиану треугольника $ABC$.

        \item На плоскости зафиксирована окружность $\omega$ и точка $A$ вне неё.
        Через точку $A$
        проведена касательная $AT$ (где $T \in \omega$) и произвольная секущая $XY$ (точки $X, Y$
        лежат на $\omega$). Докажите, что окружность, проходящая через точки $T$ и $X$, касающаяся прямой $TY$, проходит через фиксированную точку, отличную от точки $T$.

        \item В угол $BAC$ вписана окружность $\omega$, касающаяся сторон угла в точках $B, C$.
        Хорда $CD$ окружности $\omega$ параллельна прямой $AB$.
        Прямая $AD$ второй раз пересекает окружность $\omega$ в точке $E$.
        Докажите, что прямая $CE$ делит отрезок $AB$ пополам.

        \item Из точки $P$ к окружности $\omega$ проведены отрезки касательных $PA, PB$, точка $C$ диаметрально противоположна точке $B$.
        Докажите, что прямая $CP$ делит пополам перпендикуляр, опущенный из точки $A$ на прямую $BC$.

        \item Две неравные окружности $\omega_1$ и $\omega_2$ касаются внутренним образом окружности $\omega$ в точках $A$ и $B$.
        Пусть $C$ и $D$ точки пересечения окружностей $\omega_1$ и $\omega_2$.
        Прямая $CD$ пересекает $\omega$ в точках $E$ и $F$.
        Докажите, что касательные к $\omega$, проведенные в точках $E$ и $F$, пересекаются на прямой $AB$.

        \item Четырехугольник $ABCD$ вписан в окружность $\omega$ с центром $O$.
        Биссектриса угла $ABD$ пересекает отрезок $AD$ в точке $K$ и окружность $\omega$ второй раз в точке $M$.
        Биссектриса угла $CBD$ пересекает отрезок $CD$ в точке $L$ и окружность $\omega$ второй раз в точке $N$.
        Известно, что прямые $KL$ и $MN$ параллельны.
        Докажите, что описанная окружность треугольника $MON$ проходит через середину отрезка $BD$.

        \item Вписанная окружность $\omega$ треугольника $ABC$ касается стороны $BC$ в точке $D$.
        Прямая $AD$ пересекает $\omega$ в точке $L \neq D$.
        Точка $K$ – центр вневписанной окружности треугольника $ABC$, касающейся стороны $BC$.
        Точки $M$ и $N$ – середины отрезков $BC$ и $KM$ соответственно.
        Докажите, что точки $B, C, N$ и $L$ лежат на одной окружности.

    \end{enumerate_boxed}
\end{document}