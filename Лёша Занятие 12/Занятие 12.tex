\documentclass{article}
\usepackage[12pt]{extsizes}
\usepackage[T2A]{fontenc}
\usepackage[utf8]{inputenc}
\usepackage[english, russian]{babel}

\usepackage{amssymb}
\usepackage{amsfonts}
\usepackage{amsmath}
\usepackage{enumitem}
\usepackage{graphics}
\usepackage{graphicx}

\usepackage{lipsum}

\newtheorem{theorem}{Теорема}
\newtheorem{task}{Задача}
\newtheorem{lemma}{Лемма}
\newtheorem{definition}{Определение}
\newtheorem{example}{Пример}
\newtheorem{statement}{Утверждение}
\newtheorem{corollary}{Следствие}


\usepackage{geometry} % Меняем поля страницы
\geometry{left=1cm}% левое поле
\geometry{right=1cm}% правое поле
\geometry{top=1.5cm}% верхнее поле
\geometry{bottom=1cm}% нижнее поле


\usepackage{fancyhdr} % Headers and footers
\pagestyle{fancy} % All pages have headers and footers
\fancyhead{} % Blank out the default header
\fancyfoot{} % Blank out the default footer
\fancyhead[L]{Математика}
\fancyhead[C]{\textit{Разнобой}}
\fancyhead[R]{11 декабря 2023}% Custom header text


%----------------------------------------------------------------------------------------

%\begin{document}\normalsize
\begin{document}\large
	
\begin{center}
	\textbf{Остатки при делении 2}
\end{center}


\begin{enumerate}[label*=\protect\fbox{\arabic{enumi}}]

\item Может ли сумма трёх последовательных натуральных чисел быть простым числом?

\item Какое из чисел больше: $2^{30}$ или $3^{20}$?


\item Делится ли число 32561698 на 12? Решите эту задачу:
\begin{enumerate}
\item с помощью признака делимости на 4;
\item с помощью признака делимости на 3.
\end{enumerate}

\item Даша и Таня по очереди выписывают на доску цифры шестизначного числа. Сначала Даша выписывает первую цифру, затем Таня — вторую, и так далее. Таня хочет, чтобы полученное в результате число делилось на три, а Даша хочет ей помешать. Кто из них может добиться желаемого результата независимо от ходов соперника?

\item Замените звездочки в записи числа 72*4* цифрами так, чтобы это число делилось на 45. Укажите все возможные варианты!

\item 
\begin{enumerate}
	\item Докажите, что произведение двух последовательных чётных чисел всегда делится на 8.
	
	\item Может ли произведение четырех последовательных натуральных чисел оканчиваться на 116?
	
\end{enumerate}


\item Может ли натуральное число, записываемое с помощью 10 нулей, 10 единиц и 10 двоек, быть квадратом некоторого другого натурального числа?

\item Сколько существует натуральных чисел меньших 1000 и таких, что произведение всех цифр числа равно 6?

\end{enumerate}


\end{document}