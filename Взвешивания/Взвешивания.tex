\documentclass{article}

\usepackage[12pt]{extsizes}
\usepackage[T2A]{fontenc}
\usepackage[utf8]{inputenc}
\usepackage[english, russian]{babel}

\usepackage{mathrsfs}
\usepackage[dvipsnames]{xcolor}

\usepackage{amsmath}
\usepackage{amssymb}
\usepackage{amsthm}
\usepackage{indentfirst}
\usepackage{amsfonts}
\usepackage{enumitem}
\usepackage{graphics}
\usepackage{tikz}
\usepackage{tabu}
\usepackage{diagbox}
\usepackage{hyperref}
\usepackage{mathtools}
\usepackage{ucs}
\usepackage{lipsum}
\usepackage{geometry} % Меняем поля страницы
\usepackage{fancyhdr} % Headers and footers
\newcommand{\range}{\mathrm{range}}
\newcommand{\dom}{\mathrm{dom}}
\newcommand{\N}{\mathbb{N}}
\newcommand{\R}{\mathbb{R}}
\newcommand{\E}{\mathbb{E}}
\newcommand{\D}{\mathbb{D}}
\newcommand{\M}{\mathcal{M}}
\newcommand{\Prime}{\mathbb{P}}
\newcommand{\A}{\mathbb{A}}
\newcommand{\Q}{\mathbb{Q}}
\newcommand{\Z}{\mathbb{Z}}
\newcommand{\F}{\mathbb{F}}
\newcommand{\CC}{\mathbb{C}}

\DeclarePairedDelimiter\abs{\lvert}{\rvert}
\DeclarePairedDelimiter\floor{\lfloor}{\rfloor}
\DeclarePairedDelimiter\ceil{\lceil}{\rceil}
\DeclarePairedDelimiter\lr{(}{)}
\DeclarePairedDelimiter\set{\{}{\}}
\DeclarePairedDelimiter\norm{\|}{\|}

\renewcommand{\labelenumi}{(\alph{enumi})}

\newcommand{\smallindent}{
    \geometry{left=1cm}% левое поле
    \geometry{right=1cm}% правое поле
    \geometry{top=1.5cm}% верхнее поле
    \geometry{bottom=1cm}% нижнее поле
}

\newcommand{\header}[3]{
    \pagestyle{fancy} % All pages have headers and footers
    \fancyhead{} % Blank out the default header
    \fancyfoot{} % Blank out the default footer
    \fancyhead[L]{#1}
    \fancyhead[C]{#2}
    \fancyhead[R]{#3}
}

\newcommand{\dividedinto}{
    \,\,\,\vdots\,\,\,
}

\newcommand{\littletaller}{\mathchoice{\vphantom{\big|}}{}{}{}}

\newcommand\restr[2]{{
    \left.\kern-\nulldelimiterspace % automatically resize the bar with \right
    #1 % the function
    \littletaller % pretend it's a little taller at normal size
    \right|_{#2} % this is the delimiter
}}

\DeclareGraphicsExtensions{.pdf,.png,.jpg}

\newenvironment{enumerate_boxed}[1][enumi]{\begin{enumerate}[label*=\protect\fbox{\arabic{#1}}]}{\end{enumerate}}



\smallindent

\header{Математика}{\textit{Комбинаторика}}{19 марта 2024}

%----------------------------------------------------------------------------------------

%\begin{document}\normalsize
\begin{document}
    \large


    \begin{center}
        \textbf{Взвешивания}
    \end{center}

    \begin{enumerate_boxed}

        \item Имеется $11$ мешков монет.
        В $10$ из них монеты настоящие, а в одном — все монеты фальшивые.
        Все настоящие монеты весят по $11$ г, все фальшивые монеты — по $10$ г. Имеются весы, (a) показывающие точный вес груза (b) с помощью которых можно определить, какой из двух грузов тяжелее.
        За минимальное число взвешиваний определить, в каком мешке фальшивые монеты?

        \item Среди $n$ монет одна фальшивая, причем фальшивая монета отличается по массе от настоящих.
        За какое наименьшее число взвешиваний на правильных чашечных весах без гирь можно определить, легче или тяжелее настоящей фальшивая монета?
        (Находить фальшивую монету не нужно.)

        \item У Пети хранилось $9$ внешне одинаковых золотых монет.
        Однажды эксперт Вася решил проверить их на подлинность, после чего сообщил Пете, что среди них ровно $5$ фальшивых.
        Петя может задать Васе любой вопрос, на который тот может ответить “Да” или “Нет”.
        Какое наименьшее количество вопросов потребуется Пете, чтобы опознать все фальшивые монеты?

        \item Есть $100$ гирек различных весов.
        За одну операцию можно найти суммарный вес любых двух выбранных гирек.
        За какое наименьшее число
        операций удастся узнать вес каждой из гирек?

        \item У царя Гиерона есть 11 металлических слитков, неразличимых на вид; царь знает, что их веса (в некотором порядке) равны $1, 2, \ldots , 11$ кг.
        Ещё у него есть мешок, который порвётся, если в него положить больше 11 кг.
        Архимед узнал веса всех слитков и хочет доказать Гиерону, что первый слиток имеет вес $1$ кг.
        За один шаг он может загрузить несколько слитков в мешок и продемонстрировать Гиерону, что мешок не порвался (рвать мешок нельзя!).
        За какое наименьшее число загрузок мешка Архимед может добиться требуемого?

        \item Пусть $n$ — натуральное число, большее $1$.
        У Кости есть прибор, устроенный так, что если в него положить $2n+1$ различных по весу монет, то он укажет, какая из монет — средняя по весу среди положенных.
        Барон Мюнхгаузен дал Косте $4n+1$ различных по весу монет и про одну из них сказал, что она является средней по весу.
        Как Косте, использовав прибор не более $n+2$ раз, выяснить, прав ли барон?

        \item Среди $49$ одинаковых на вид монет — $25$ настоящих и $24$ фальшивых.
        Для определения фальшивых монет имеется тестер.
        В него можно положить любое количество монет, и если среди этих монет больше половины — фальшивые, тестер подает сигнал.
        Как за пять тестов найти две фальшивых монеты?


    \end{enumerate_boxed}

\end{document}