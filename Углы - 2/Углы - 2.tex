\documentclass{article}
\usepackage[12pt]{extsizes}
\usepackage[T2A]{fontenc}
\usepackage[utf8]{inputenc}
\usepackage[english, russian]{babel}

\usepackage{amssymb}
\usepackage{amsfonts}
\usepackage{amsmath}
\usepackage{enumitem}
\usepackage{graphics}

\usepackage{lipsum}



\usepackage{geometry} % Меняем поля страницы
\geometry{left=1cm}% левое поле
\geometry{right=1cm}% правое поле
\geometry{top=1.5cm}% верхнее поле
\geometry{bottom=1cm}% нижнее поле


\usepackage{fancyhdr} % Headers and footers
\pagestyle{fancy} % All pages have headers and footers
\fancyhead{} % Blank out the default header
\fancyfoot{} % Blank out the default footer
\fancyhead[L]{ЦРОД $\bullet$ Математика}
\fancyhead[C]{\textit{Геометрия}}
\fancyhead[R]{Май 2022}% Custom header text


%----------------------------------------------------------------------------------------

%\begin{document}\normalsize
\begin{document}\large


\begin{center}
\textbf{Вписанные углы}
\end{center}



\begin{enumerate}[label*=\protect\fbox{\arabic{enumi}}]

\item Даны два угла $\angle ABC = 90^\circ$ и $\angle ADC = 90^\circ$. Докажите, что $A, B, C, D$ лежат на одной окружности.

\item Дан треугольник $ABC$. $I$ - центр вписанной окружности. Докажите (и запомните), что $\angle AIB = 90^\circ + \frac{\angle A}{2}$

\item Дан треугольник $ABC$. $H$ - ортоцентр (точка пересечения высот). Докажите (и запомните), что $\angle AHB = 180^\circ - \angle C$

\item Дан треугольник $ABC$. $BH_1$, $CH_2$ - высоты треугольника. Докажите, что $C,B,H_1, H_2$ лежат на одной окружности.

\item В условии предыдущей задачи пусть $H = BH_1 \cap CH_2$. Докажите, что $A,H,H_1, H_2$ лежат на одной окружности.

\item Рассмотрим вписанный четырёхугольник $ABCD$. Пусть дуга $\breve{AB} = \alpha$, дуга $\breve{CD} = \beta$. $O$ - точка пересечения диагоналей. Докажите, что $\angle AOB = \frac{\alpha +  \beta}{2}$.

\item Дана точка $O$ и окружность $\omega$, так что $O \notin \omega$. Через $O$ провели 2 прямые, которые пересекают $\omega$ в точках $A, B$ и $C, D$. Докажите, что $OA\cdot OB = OC\cdot OD$

\item На гипотенузе $AB$ прямоугольного треугольника $ABC$ во внешнюю сторону треугольника построен квадрат с центром в точке $O$. Докажите, что $CO$ — биссектриса угла $ACB$.

\item В остроугольном треугольнике $ABC$ на высоте, проведённой из вершины $C$, выбрана точка $X$. Пусть $A_1$ и $B_1$ — основания перпендикуляров из точки $X$ на стороны $AC$ и $BC$ соответственно. Докажите, что точки $A$, $B$, $B_1$, $A_1$ лежат на одной окружности.

\item Пусть $AA_1$ , $BB_1$, $CC_1$ — высоты остроугольного треугольника $ABC$. Докажите, что основания перпендикуляров из точки $A_1$ на прямые $AB$, $AC$, $BB_1$, $CC_1$ лежат на одной прямой.

\item Дан выпуклый шестиугольник $ABCDEF$. Известно, что $\angle FAE = \angle BDC$, а четырёхугольники $ABDF$ и $ACDE$ являются вписанными. Докажите, что прямые $BF$ и $CE$ параллельны.

\item Дан остроугольный треугольник $ABC$, в котором $AB < AC$. Пусть $M$ и $N$ — середины сторон $AB$ и $AC$ соответственно, а $D$ — основание высоты, проведённой из $A$. На
отрезке $MN$ нашлась точка K такая, что $BK = CK$. Луч $KD$ пересекает окружность Ω, описанную около треугольника $ABC$, в точке $Q$. Докажите, что точки $C, N, K$ и $Q$
лежат на одной окружности.



\end{enumerate}
\end{document}