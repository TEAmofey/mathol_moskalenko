\documentclass{article}
\usepackage[12pt]{extsizes}
\usepackage[T2A]{fontenc}
\usepackage[utf8]{inputenc}
\usepackage[english, russian]{babel}

\usepackage{amssymb}
\usepackage{amsfonts}
\usepackage{amsmath}
\usepackage{enumitem}
\usepackage{graphics}
\usepackage{graphicx}

\usepackage{lipsum}

\newtheorem{theorem}{Теорема}
\newtheorem{task}{Задача}
\newtheorem{lemma}{Лемма}
\newtheorem{definition}{Определение}
\newtheorem{example}{Пример}
\newtheorem{statement}{Утверждение}
\newtheorem{corollary}{Следствие}


\usepackage{geometry} % Меняем поля страницы
%\geometry{left=1cm}% левое поле
%\geometry{right=1cm}% правое поле
\geometry{top=3cm}% верхнее поле
%\geometry{bottom=1cm}% нижнее поле


\usepackage{fancyhdr} % Headers and footers
\pagestyle{fancy} % All pages have headers and footers
\fancyhead{} % Blank out the default header
\fancyfoot{} % Blank out the default footer
\fancyhead[L]{\textit{\textbf{XIV Олимпиада Эйлера}}}
\fancyhead[C]{}
\fancyhead[R]{20 ноября}% Custom header text


%----------------------------------------------------------------------------------------

%\begin{document}\normalsize
\begin{document}\large
	
\begin{center}
	\LARGE\textbf{8 класс}
\end{center}
\begin{center}
	\large\textbf{Первый день}
\end{center}


\begin{enumerate}[label*=8.{\arabic{enumi}}]
	
\item При каком наибольшем $n$ существует выпуклый $n$-угольник, у которого длины диагоналей принимают не больше двух различных значений?

\item Числа $1, 2, ..., 1000$ разбили на два множества по $500$ чисел: красные $k_1, k_2, \dotsc, k_{500}$ и синие $s_1, s_2, \dotsc, s_{500}$. Докажите, что количество таких пар $m$ и $n$, у которых разность $k_m-s_n$ дает остаток $7$ при делении на $100$, равно количеству таких пар $m$ и $n$, у которых разность $s_n-k_m$ дает остаток $7$ при делении на $100$. Здесь рассматриваются все возможные разности, в том числе и отрицательные.
Напомним, что остатком от деления целого числа $a$ на $100$ называется разность между числом $a$ и ближайшим числом, не большим $a$ и делящимся на $100$. Например, остаток от деления числа $2022$ на $100$ равен $2022-2000 = 22$, а остаток от деления числа $-11$ на $100$ равен $-11-(-100) = 89$.

\item В треугольнике $ABC$ проведены биссектрисы $BK$ и $CL$. На отрезке $BK$ отмечена точка $N$ так, что $LN \parallel AC$. Оказалось, что $NK = LN$. Найдите величину угла $ABC$.

\item Учитель придумал ребус, заменив в примере $a+b = c$ на сложение двух натуральных чисел цифры буквами: одинаковые цифры одинаковыми буквами, а разные — разными
(например, если $a = 23, а b = 528,$ то $c = 551$, и получился, с точностью до выбора букв, ребус АБ $+$ ВАГ $=$ ВВД). 
Оказалось, что по получившемуся ребусу однозначно восстанавливается исходный пример. Найдите наименьшее возможное значение суммы $c$.
	
\item Можно ли без остатка разрезать клетчатый квадрат размером $8\times 8$ клеточек на $10$ клетчатых прямоугольников, чтобы все прямоугольники имели различные площади? Все разрезы должны проходить по границам клеточек.

\end{enumerate}
\end{document}