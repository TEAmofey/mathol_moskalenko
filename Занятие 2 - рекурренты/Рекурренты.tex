\documentclass{article}

\usepackage[12pt]{extsizes}
\usepackage[T2A]{fontenc}
\usepackage[utf8]{inputenc}
\usepackage[english, russian]{babel}

\usepackage{mathrsfs}
\usepackage[dvipsnames]{xcolor}

\usepackage{amsmath}
\usepackage{amssymb}
\usepackage{amsthm}
\usepackage{indentfirst}
\usepackage{amsfonts}
\usepackage{enumitem}
\usepackage{graphics}
\usepackage{tikz}
\usepackage{tabu}
\usepackage{diagbox}
\usepackage{hyperref}
\usepackage{mathtools}
\usepackage{ucs}
\usepackage{lipsum}
\usepackage{geometry} % Меняем поля страницы
\usepackage{fancyhdr} % Headers and footers
\newcommand{\range}{\mathrm{range}}
\newcommand{\dom}{\mathrm{dom}}
\newcommand{\N}{\mathbb{N}}
\newcommand{\R}{\mathbb{R}}
\newcommand{\E}{\mathbb{E}}
\newcommand{\D}{\mathbb{D}}
\newcommand{\M}{\mathcal{M}}
\newcommand{\Prime}{\mathbb{P}}
\newcommand{\A}{\mathbb{A}}
\newcommand{\Q}{\mathbb{Q}}
\newcommand{\Z}{\mathbb{Z}}
\newcommand{\F}{\mathbb{F}}
\newcommand{\CC}{\mathbb{C}}

\DeclarePairedDelimiter\abs{\lvert}{\rvert}
\DeclarePairedDelimiter\floor{\lfloor}{\rfloor}
\DeclarePairedDelimiter\ceil{\lceil}{\rceil}
\DeclarePairedDelimiter\lr{(}{)}
\DeclarePairedDelimiter\set{\{}{\}}
\DeclarePairedDelimiter\norm{\|}{\|}

\renewcommand{\labelenumi}{(\alph{enumi})}

\newcommand{\smallindent}{
    \geometry{left=1cm}% левое поле
    \geometry{right=1cm}% правое поле
    \geometry{top=1.5cm}% верхнее поле
    \geometry{bottom=1cm}% нижнее поле
}

\newcommand{\header}[3]{
    \pagestyle{fancy} % All pages have headers and footers
    \fancyhead{} % Blank out the default header
    \fancyfoot{} % Blank out the default footer
    \fancyhead[L]{#1}
    \fancyhead[C]{#2}
    \fancyhead[R]{#3}
}

\newcommand{\dividedinto}{
    \,\,\,\vdots\,\,\,
}

\newcommand{\littletaller}{\mathchoice{\vphantom{\big|}}{}{}{}}

\newcommand\restr[2]{{
    \left.\kern-\nulldelimiterspace % automatically resize the bar with \right
    #1 % the function
    \littletaller % pretend it's a little taller at normal size
    \right|_{#2} % this is the delimiter
}}

\DeclareGraphicsExtensions{.pdf,.png,.jpg}

\newenvironment{enumerate_boxed}[1][enumi]{\begin{enumerate}[label*=\protect\fbox{\arabic{#1}}]}{\end{enumerate}}



\smallindent

\header{Математика}{\textit{Олимпиадная подготовка}}{3 октября 2023}

%----------------------------------------------------------------------------------------

\begin{document}\large

\begin{center}
	\textbf{Рекурренты}
\end{center}

\begin{enumerate_boxed}
	
\item Найдите формулу $n$-го члена для последовательностей, заданных условиями $( n  \geqslant 0)$:
\begin{enumerate}
	\item $a_0 = 0, a_1 = 1, a_{n + 2} = a_{n + 1} + a_n;$
	\item $a_0 = 0, a_1 = 1, a_{n + 2} = 5a_{n + 1} - 6a_n;$ 
	\item $a_0 = 0, a_1 = 1, a_{n + 2} = 2a_{n + 1} + a_n;$ 
	\item $a_0 = 0, a_1 = 1, a_{n + 2} = 2a_{n + 1} - a_n.$ 
\end{enumerate}

	
\item Сколько существует способов разрезать доску $2 \times 10$ на доминошки?

\item Пусть $x_1$ и $x_2$ — корни квадратного уравнения $x^2 - 6x + 1 = 0$.
Докажите, что при любом натуральном $n$ число $x^n_1 + x^n_2$ является целым и не делится на 5.

\item Последовательность задана рекуррентно: $ a_1 = \frac{1}{2} , a_1 + a_2 + \dotsc + a_n = n^2 a_n $ .
Найдите формулу общего члена.

\item Последовательность $ \{a_i\}^{\infty}_{i=0} $ задана рекурентно: $ a_0 = a, a_{n+1} = 2^n - 3a_n $.
При каких значениях $ a $ последовательность является монотонно возрастающей?

\item Последовательность $a_0, a_1, a_2, \ldots$ такова, что для всех неотрицательных $m \geq n$ выполняется условие $a_{m+n} + a_{m-n} = \frac{a_{2m} + a_{2n}}{2}$.
Найдите $a_{2022}$, если $a_1 = 1$.

\item Рассмотрим все возможные наборы чисел из множества $\{1, 2, 3, \dotsc , n\}$, не содержащие двух соседних чисел.
Докажите, что сумма квадратов произведений чисел в этих наборах равна $(n + 1)! - 1$.

\end{enumerate_boxed}
\end{document}