\documentclass{article}
\usepackage[12pt]{extsizes}
\usepackage[T2A]{fontenc}
\usepackage[utf8]{inputenc}
\usepackage[english, russian]{babel}

\usepackage{amssymb}
\usepackage{amsfonts}
\usepackage{amsmath}
\usepackage{enumitem}
\usepackage{graphics}
\usepackage{graphicx}

\usepackage{lipsum}

\newtheorem{theorem}{Теорема}
\newtheorem{task}{Задача}
\newtheorem{lemma}{Лемма}
\newtheorem{definition}{Определение}
\newtheorem{example}{Пример}
\newtheorem{statement}{Утверждение}
\newtheorem{corollary}{Следствие}


\usepackage{geometry} % Меняем поля страницы
\geometry{left=1cm}% левое поле
\geometry{right=1cm}% правое поле
\geometry{top=1.5cm}% верхнее поле
\geometry{bottom=1cm}% нижнее поле


\usepackage{fancyhdr} % Headers and footers
\pagestyle{fancy} % All pages have headers and footers
\fancyhead{} % Blank out the default header
\fancyfoot{} % Blank out the default footer
\fancyhead[L]{Математика}
\fancyhead[C]{\textit{Разнобой}}
\fancyhead[R]{23 октября 2023}% Custom header text


%----------------------------------------------------------------------------------------

%\begin{document}\normalsize
\begin{document}\large
	
\begin{center}
	\textbf{Разнобой}
\end{center}


\begin{enumerate}[label*=\protect\fbox{\arabic{enumi}}]
	
\item Сколько треугольников изображено на картинке ниже?

\item Можно ли из семи прямоугольников $1 \times1, 1 \times2, 1 \times3, ..., 1 \times6, 1 \times7$ сложить какой-нибудь прямоугольник, обе стороны которого больше 1?

\item От шахматной доски $8 \times 8$ отрезали 
\begin{enumerate}
	\item угловую клетку (например, a1)
	
	\item две соседние угловые клетки (a1 и a8) 
	
	\item две противоположные угловые клетки (a1 и h8). 
\end{enumerate}

Можно ли оставшуюся часть разрезать на доминошки?

\item В кружке художественного свиста у каждого ровно один друг и ровно один враг. Докажите, что в кружке четное число людей.

\item Докажите, что число 

\begin{enumerate}
	\item $4^{101} + 6^{101}$
	
	\item $9^{101} + 1$ 
\end{enumerate}

Делится на 10

\item Сформулируйте и докажите:

\begin{enumerate}
	\item Признак делимости на 2
	
	\item Признак делимости на 3
	
	\item Признак делимости на 5
	
	\item Признак делимости на 10
	
	\item Признак делимости на 6
	
	\item Признак делимости на 9
	
	\item Признак делимости на 4
	
	\item Признак делимости на 8
\end{enumerate}


\end{enumerate}


\end{document}