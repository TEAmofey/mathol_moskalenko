\documentclass{article}
\usepackage[12pt]{extsizes}
\usepackage[T2A]{fontenc}
\usepackage[utf8]{inputenc}
\usepackage[english, russian]{babel}

\usepackage{amssymb}
\usepackage{amsfonts}
\usepackage{amsmath}
\usepackage{enumitem}
\usepackage{graphics}
\usepackage{graphicx}

\usepackage{lipsum}

\newtheorem{theorem}{Теорема}
\newtheorem{task}{Задача}
\newtheorem{lemma}{Лемма}
\newtheorem{definition}{Определение}
\newtheorem{example}{Пример}
\newtheorem{statement}{Утверждение}
\newtheorem{corollary}{Следствие}


\usepackage{geometry} % Меняем поля страницы
%\geometry{left=1cm}% левое поле
%\geometry{right=1cm}% правое поле
\geometry{top=3cm}% верхнее поле
%\geometry{bottom=1cm}% нижнее поле


\usepackage{fancyhdr} % Headers and footers
\pagestyle{fancy} % All pages have headers and footers
\fancyhead{} % Blank out the default header
\fancyfoot{} % Blank out the default footer
\fancyhead[L]{\textit{\textbf{XIV Олимпиада Эйлера}}}
\fancyhead[C]{}
\fancyhead[R]{21 ноября}% Custom header text


%----------------------------------------------------------------------------------------

%\begin{document}\normalsize
\begin{document}\large
	
\begin{center}
	\LARGE\textbf{8 класс}
\end{center}
\begin{center}
	\large\textbf{Второй день}
\end{center}


\begin{enumerate}[label*=8.{\arabic{enumi}}]
\setcounter{enumi}{5}
\item Сумма остатков от деления трёх последовательных натуральных чисел на $2022$~--- простое число. Докажите, что одно из чисел делится на $2022$.
\item Существует ли треугольник, у которого длины не совпадающих между собой медианы и высоты, проведенных из одной его вершины, соответственно равны длинам двух сторон этого треугольника?
\item Будем называть натуральное число красивым, если в его десятичной записи поровну цифр $0, 1, 2$, а других цифр нет (во избежание недоразумений напомним, что десятичная запись числа не может начинаться с нуля). Может ли произведение двух красивых чисел быть красивым?
\item Петя и Вася написали на доске по $100$ различных натуральных чисел. Петя поделил все свои числа на Васины с остатком и выписал все $10000$ получившихся остатков себе в тетрадь. Вася поделил все свои числа на Петины с остатком и выписал все $10000$ получившихся остатков себе в тетрадь. Оказалось, что наборы выписанных Васей и Петей остатков совпадают. Докажите, что тогда и наборы их исходных чисел совпадают.
\item В вершины правильного $100$-угольника поставили $100$ фишек, на которых написаны номера $1, 2, \dotsc, 100$, именно в таком порядке по часовой стрелке. За ход разрешается обменять местами некоторые две фишки, стоящие в соседних вершинах, если номера этих фишек отличаются не более чем на $k$. При каком наименьшем $k$ серией таких ходов можно добиться расположения, в котором каждая фишка сдвинута на одну позицию по часовой стрелке по отношению к своему начальному положению?

\end{enumerate}
\end{document}