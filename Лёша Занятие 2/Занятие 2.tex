\documentclass{article}
\usepackage[12pt]{extsizes}
\usepackage[T2A]{fontenc}
\usepackage[utf8]{inputenc}
\usepackage[english, russian]{babel}

\usepackage{amssymb}
\usepackage{amsfonts}
\usepackage{amsmath}
\usepackage{enumitem}
\usepackage{graphics}
\usepackage{graphicx}

\usepackage{lipsum}

\newtheorem{theorem}{Теорема}
\newtheorem{task}{Задача}
\newtheorem{lemma}{Лемма}
\newtheorem{definition}{Определение}
\newtheorem{example}{Пример}
\newtheorem{statement}{Утверждение}
\newtheorem{corollary}{Следствие}


\usepackage{geometry} % Меняем поля страницы
\geometry{left=1cm}% левое поле
\geometry{right=1cm}% правое поле
\geometry{top=1.5cm}% верхнее поле
\geometry{bottom=1cm}% нижнее поле


\usepackage{fancyhdr} % Headers and footers
\pagestyle{fancy} % All pages have headers and footers
\fancyhead{} % Blank out the default header
\fancyfoot{} % Blank out the default footer
\fancyhead[L]{Математика}
\fancyhead[C]{\textit{Логика}}
\fancyhead[R]{2 октября 2023}% Custom header text


%----------------------------------------------------------------------------------------

%\begin{document}\normalsize
\begin{document}\large
	
\begin{center}
	\textbf{Логика}
\end{center}


\begin{enumerate}[label*=\protect\fbox{\arabic{enumi}}]
	
\item Мистер Саша уверен, что можно вырезать из шахматной доски $8 \times 8$ ровно $4$ клетки так, чтобы оставшуюся доску можно было разрезать на “доминошки”, то есть прямоугольники $1 \times 2$. Прав ли Мистер Саша?

\item Мистер Саша так понравилось вырезать из доски $8 \times 8$ четыре клетки и разбивать оставшуюся часть на доминошки, что теперь он уверен, что как ни вырежи $4$ клетки из шахматной доски $8 \times 8$, всегда оставшуюся фигурку можно разрезать на доминошки. Не ошибается ли Мистер Саша?


\item Постройте отрицание к утверждениям:
\begin{itemize}
	\item "Поле шахматной доски - белое";
	
	\item "Это или синее или белое";
	
	\item "Я рыцарь или ты лжец";
	
	\item "Верблюд синий и весит хотя бы 100 кг".
	
\end{itemize}

\item Сколько квадратов изображено на картинке ниже?

\begin{table}[h]
	\centering
	\begin{tabular}{|c|c|c|c|c|} \hline
		\, & \, & \,& \, & \, \\\hline
		 &  &  &  & \\\hline
		 &  &  &  & \\\hline
		 &  &  &  & \\\hline
	 \end{tabular}
\end{table}


\item Найдите сумму a) $1+2+3+...+50;$ b) $1+2+3+...+51.$


\item Получив двойку по географии, Вася решил порвать географическую карту в клочья. Каждый попавший ему в руки клочок он рвет на четыре части. Может ли он когда-нибудь получить ровно a) 2022 клочок? b) 2023 клочка?

\item На прямой расположено пять точек $A, B, C, D, E$ (именно в таком порядке!). Известно, что $AB = 19$ см., $CE = 97$ см., $AC = BD$. Найдите длину отрезка $DE$.

\item На каждой клетке доски $5 \times 5$ сидит один дрессированный лягушонок. По команде <<Ква!>> каждый лягушонок перепрыгивает на одну из соседних (по стороне) клеток. Докажите, что после команды <<Ква!>> какие-то два лягушонка окажутся на одной клетке.

\item В кружке художественного свиста у каждого ровно один друг и ровно один враг. Докажите, что в кружке четное число людей.

\item От шахматной доски $8 \times 8$ отрезали a) угловую клетку (например, a1) b) две соседние угловые клетки (a1 и a8) c) две противоположные угловые клетки (a1 и h8). Можно ли оставшуюся часть разрезать на фигурки вида

\end{enumerate}


\end{document}