\documentclass{article}

\usepackage[12pt]{extsizes}
\usepackage[T2A]{fontenc}
\usepackage[utf8]{inputenc}
\usepackage[english, russian]{babel}

\usepackage{mathrsfs}
\usepackage[dvipsnames]{xcolor}

\usepackage{amsmath}
\usepackage{amssymb}
\usepackage{amsthm}
\usepackage{indentfirst}
\usepackage{amsfonts}
\usepackage{enumitem}
\usepackage{graphics}
\usepackage{tikz}
\usepackage{tabu}
\usepackage{diagbox}
\usepackage{hyperref}
\usepackage{mathtools}
\usepackage{ucs}
\usepackage{lipsum}
\usepackage{geometry} % Меняем поля страницы
\usepackage{fancyhdr} % Headers and footers
\newcommand{\range}{\mathrm{range}}
\newcommand{\dom}{\mathrm{dom}}
\newcommand{\N}{\mathbb{N}}
\newcommand{\R}{\mathbb{R}}
\newcommand{\E}{\mathbb{E}}
\newcommand{\D}{\mathbb{D}}
\newcommand{\M}{\mathcal{M}}
\newcommand{\Prime}{\mathbb{P}}
\newcommand{\A}{\mathbb{A}}
\newcommand{\Q}{\mathbb{Q}}
\newcommand{\Z}{\mathbb{Z}}
\newcommand{\F}{\mathbb{F}}
\newcommand{\CC}{\mathbb{C}}

\DeclarePairedDelimiter\abs{\lvert}{\rvert}
\DeclarePairedDelimiter\floor{\lfloor}{\rfloor}
\DeclarePairedDelimiter\ceil{\lceil}{\rceil}
\DeclarePairedDelimiter\lr{(}{)}
\DeclarePairedDelimiter\set{\{}{\}}
\DeclarePairedDelimiter\norm{\|}{\|}

\renewcommand{\labelenumi}{(\alph{enumi})}

\newcommand{\smallindent}{
    \geometry{left=1cm}% левое поле
    \geometry{right=1cm}% правое поле
    \geometry{top=1.5cm}% верхнее поле
    \geometry{bottom=1cm}% нижнее поле
}

\newcommand{\header}[3]{
    \pagestyle{fancy} % All pages have headers and footers
    \fancyhead{} % Blank out the default header
    \fancyfoot{} % Blank out the default footer
    \fancyhead[L]{#1}
    \fancyhead[C]{#2}
    \fancyhead[R]{#3}
}

\newcommand{\dividedinto}{
    \,\,\,\vdots\,\,\,
}

\newcommand{\littletaller}{\mathchoice{\vphantom{\big|}}{}{}{}}

\newcommand\restr[2]{{
    \left.\kern-\nulldelimiterspace % automatically resize the bar with \right
    #1 % the function
    \littletaller % pretend it's a little taller at normal size
    \right|_{#2} % this is the delimiter
}}

\DeclareGraphicsExtensions{.pdf,.png,.jpg}

\newenvironment{enumerate_boxed}[1][enumi]{\begin{enumerate}[label*=\protect\fbox{\arabic{#1}}]}{\end{enumerate}}



\smallindent

\header{ЦРОД $\bullet$ Математика}{\textit{Геометрия}}{Май 2022}

%----------------------------------------------------------------------------------------

\begin{document}
    \large

    \begin{center}
        \textbf{Треугольники, высоты, окружности}
    \end{center}

    Рассмотрим треугольник $ABC$. $AH_a, BH_b, CH_c$~--- высоты этого треугольника. $H$~--- ортоцентр (точка пересечения высот). $O$~--- центр описанной окружности. $M_a, M_b, M_c$~--- середины сторон $BC, AC, AB$ соответственно.

    Дальше в задачах мы будем ссылаться на эти обозначения


    \begin{enumerate}[label*=\protect\fbox{\arabic{enumi}}]

        \item Докажите, что $\angle ABH = \angle CBO$.

        \item Докажите, что $\angle ABH = \angle H_{c}H_{a}H$.

        \item Докажите, что $H_{a}A$~--- биссектриса $ \angle H_{c}H_{a}H_b$.

        \item Докажите, что $H$~--- центр вписанной окружности треугольника $H_{a}H_{b}H_c$.

        \item Докажите, что $O$~--- ортоцентр треугольника $M_{a}M_{b}M_c$.


        \item Точку $O$ отразили относительно сторон треугольника $ABC$ и получили точки $A'$, $B'$ и $C'$.
        Докажите, что треугольник $A'B'C'$ равен исходному, причём точка $O$ для него является ортоцентром.

        \item Докажите, что  $AH = 2OM_a$

        \item Докажите, что отражение  $H$ относительно стороны $BC$ лежит на описанной окружности треугольника $ABC$.

        \item Докажите, что отражение  $H$ относительно точки $M_a$ лежит на описанной окружности треугольника $ABC$.

        \item Докажите, что точка из предыдущей задачи диаметрально противоположна точке $A$

        \item Докажите, что четырёхугольник  $M_{a}M_{b}M_{c}H_a$~--- равнобедренная трапеция.

        \item Докажите, что четырёхугольник  $H_{b}M_{b}M_{c}H_c$~--- вписан.

        \item Докажите что 6 точек  $M_a, M_b, M_c, H_a, H_b, H_c$ лежат на одной окружности.

        \item Описанная окружность треугольника $BHC$ пересекает отрезки $AB$ и $AC$ в точках $Y$ и $X$ соответственно.
        Докажите, что $XY$=$2H_{b}H_c$

        \item Пусть $O_1$ и $O_2$ — центры описанных окружностей треугольников $H_{b}HH_c$ и $BHC$ соответственно.
        Докажите, что $O_{1}O_2 \parallel AM_a$.

        \item Дана равнобокая трапеция $ABCD$ с основаниями $BC$ и $AD$.
        Окружность $\omega$ проходит через вершины $B$ и $C$ и вторично пересекает сторону $AB$ и диагональ $BD$ в точках $X$ и $Y$ соответственно.
        Касательная, проведенная к окружности $\omega$ в точке $C$, пересекает луч $AD$ в точке $Z$.
        Докажите, что точки $X, Y$ и $Z$ лежат на одной прямой.

    \end{enumerate}
\end{document}