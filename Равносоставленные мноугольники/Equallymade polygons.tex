\documentclass{article}

\usepackage[12pt]{extsizes}
\usepackage[T2A]{fontenc}
\usepackage[utf8]{inputenc}
\usepackage[english, russian]{babel}

\usepackage{mathrsfs}
\usepackage[dvipsnames]{xcolor}

\usepackage{amsmath}
\usepackage{amssymb}
\usepackage{amsthm}
\usepackage{indentfirst}
\usepackage{amsfonts}
\usepackage{enumitem}
\usepackage{graphics}
\usepackage{tikz}
\usepackage{tabu}
\usepackage{diagbox}
\usepackage{hyperref}
\usepackage{mathtools}
\usepackage{ucs}
\usepackage{lipsum}
\usepackage{geometry} % Меняем поля страницы
\usepackage{fancyhdr} % Headers and footers
\usepackage[framemethod=TikZ]{mdframed}

\newcommand{\definebox}[3]{%
    \newcounter{#1}
    \newenvironment{#1}[1][]{%
        \stepcounter{#1}%
        \mdfsetup{%
            frametitle={%
            \tikz[baseline=(current bounding box.east),outer sep=0pt]
            \node[anchor=east,rectangle,fill=white]
            {\strut #2~\csname the#1\endcsname\ifstrempty{##1}{}{##1}};}}%
        \mdfsetup{innertopmargin=1pt,linecolor=#3,%
            linewidth=3pt,topline=true,
            frametitleaboveskip=\dimexpr-\ht\strutbox\relax,}%
        \begin{mdframed}[]
            \relax%
            }{
        \end{mdframed}}%
}

\definebox{theorem_boxed}{Теорема}{ForestGreen!24}
\definebox{definition_boxed}{Определение}{blue!24}
\definebox{task_boxed}{Задача}{orange!24}
\definebox{paradox_boxed}{Парадокс}{red!24}

\theoremstyle{plain}
\newtheorem{theorem}{Теорема}
\newtheorem{task}{Задача}
\newtheorem{lemma}{Лемма}
\newtheorem{statement}{Утверждение}
\newtheorem{corollary}{Следствие}

\theoremstyle{remark}
\newtheorem{remark}{Замечание}
\newtheorem{example}{Пример}
\newcommand{\range}{\mathrm{range}}
\newcommand{\dom}{\mathrm{dom}}
\newcommand{\N}{\mathbb{N}}
\newcommand{\R}{\mathbb{R}}
\newcommand{\E}{\mathbb{E}}
\newcommand{\D}{\mathbb{D}}
\newcommand{\M}{\mathcal{M}}
\newcommand{\Prime}{\mathbb{P}}
\newcommand{\A}{\mathbb{A}}
\newcommand{\Q}{\mathbb{Q}}
\newcommand{\Z}{\mathbb{Z}}
\newcommand{\F}{\mathbb{F}}
\newcommand{\CC}{\mathbb{C}}

\DeclarePairedDelimiter\abs{\lvert}{\rvert}
\DeclarePairedDelimiter\floor{\lfloor}{\rfloor}
\DeclarePairedDelimiter\ceil{\lceil}{\rceil}
\DeclarePairedDelimiter\lr{(}{)}
\DeclarePairedDelimiter\set{\{}{\}}
\DeclarePairedDelimiter\norm{\|}{\|}

\renewcommand{\labelenumi}{(\alph{enumi})}

\newcommand{\smallindent}{
    \geometry{left=1cm}% левое поле
    \geometry{right=1cm}% правое поле
    \geometry{top=1.5cm}% верхнее поле
    \geometry{bottom=1cm}% нижнее поле
}

\newcommand{\header}[3]{
    \pagestyle{fancy} % All pages have headers and footers
    \fancyhead{} % Blank out the default header
    \fancyfoot{} % Blank out the default footer
    \fancyhead[L]{#1}
    \fancyhead[C]{#2}
    \fancyhead[R]{#3}
}

\newcommand{\dividedinto}{
    \,\,\,\vdots\,\,\,
}

\newcommand{\littletaller}{\mathchoice{\vphantom{\big|}}{}{}{}}

\newcommand\restr[2]{{
    \left.\kern-\nulldelimiterspace % automatically resize the bar with \right
    #1 % the function
    \littletaller % pretend it's a little taller at normal size
    \right|_{#2} % this is the delimiter
}}

\DeclareGraphicsExtensions{.pdf,.png,.jpg}

\newenvironment{enumerate_boxed}[1][enumi]{\begin{enumerate}[label*=\protect\fbox{\arabic{#1}}]}{\end{enumerate}}



\smallindent

\header{Математика}{\textit{Геометрия}}{1 июня 2023}

%----------------------------------------------------------------------------------------

\begin{document}
    \large

    \begin{center}
        \textbf{Равносоставленные многоугольники}
    \end{center}

    \begin{definition}
        Две фигуры \textbf{равновелики}, если у них одинаковые площади.
    \end{definition}

    \begin{definition}
        Два многоугольника \textbf{равносоставлены}, если один из них можно разрезать на части, из которых можно сложить другой (без наложений, используя все части).
    \end{definition}

    \begin{theorem}[Бойяи–Гервина]
        Любые два равновеликих многоугольника равносоставлены.
    \end{theorem}

    \begin{enumerate}[label*=\protect\fbox{\arabic{enumi}}]


        \item Докажите что произвольный треугольник равносоставлен какому-то прямоугольнику

        \item Докажите \textit{транзитивность} равносоставленности (если фигуры $A$ и $B$ равносоставлены, и фигуры $B$ и $C$ равносоставлены, то фигуры $A$ и $C$ равносоставлены).

        \item Докажите, что два равновеликих параллелограмма с общим основанием равносоставлены.

        \item Докажите, что любые два равновеликих прямоугольника равносоставлены.

        \item Докажите, что любой треугольник равносоставлен некоторому прямоугольнику со стороной 1.

        \begin{center}
            \begin{asy}
                size(10cm, 0);
                include geometry;

                point pA = (0,0), pB = (4, 0);
                point pD = (2,3), pC = (6, 3);
                point pE = (-1,3), pF = (-5, 3);

                dot("$A$", pA, S);
                dot("$B$", pB, S);
                dot("$C$", pC, N);
                dot("$D$", pD, N);
                dot("$E$", pE, N);
                dot("$F$", pF, N);

                draw(pF -- pE);
                draw(pA -- pB);
                draw(pA -- pF);
                draw(pA -- pD);
                draw(pB -- pE);
                draw(pD -- pC);
                draw(pB -- pC);
                draw(pE -- pD, linetype(new real[] {3,6}));
            \end{asy}
        \end{center}

        \item Докажите \textbf{теорему Бойяи–Гервина}.

    \end{enumerate}

    \begin{definition}
        Фигуры называются \textbf{равнодополняемыми}, если их можно получить, отрезая от равных фигур одну или несколько равных частей.
    \end{definition}


    \begin{enumerate}[label*=\protect\fbox{\arabic{enumi}}]

        \setcounter{enumi}{6}
        \item Докажите, что равнодополняемые фигуры равновелики.

        \item Докажите, что параллелограмм равнодополняем некоторому прямоугольнику.

        \item Докажите, что равновеликие многоугольники равнодополняемы.

        \item Перекроите прямоугольник $1 \times 3$ в квадрат.

        \item Перекроите квадрат в правильный треугольник, разрезав его не более, чем на 10 частей.

        \item Перекроите прямоугольник $3\times 4$ в квадрат, разрезав его всего на 3 части.

        \item Перекроите прямоугольник $1 \times 3$ в квадрат, разрезав его не более чем на 6 частей.

        \item Перекроите квадрат в 3 равных квадрата, разрезав его не более чем на а) 10 частей; б) 7 частей.

        \item Докажите, что правильный пятиугольник можно разрезать на 4 части, из которых без просветов и наложений можно сложить прямоугольник.

    \end{enumerate}
\end{document}