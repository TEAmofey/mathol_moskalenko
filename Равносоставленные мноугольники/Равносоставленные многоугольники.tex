\documentclass{article}
\usepackage[12pt]{extsizes}
\usepackage[T2A]{fontenc}
\usepackage[utf8]{inputenc}
\usepackage[english, russian]{babel}

\usepackage{amssymb}
\usepackage{amsfonts}
\usepackage{amsmath}
\usepackage{enumitem}
\usepackage{graphics}
\usepackage{graphicx}
\usepackage{asymptote}

\usepackage{lipsum}

\newtheorem{theorem}{Теорема}
\newtheorem{task}{Задача}
\newtheorem{lemma}{Лемма}
\newtheorem{definition}{Определение}
\newtheorem{example}{Пример}
\newtheorem{statement}{Утверждение}
\newtheorem{corollary}{Следствие}


\usepackage{geometry} % Меняем поля страницы
\geometry{left=1cm}% левое поле
\geometry{right=1cm}% правое поле
\geometry{top=1.5cm}% верхнее поле
\geometry{bottom=1cm}% нижнее поле


\usepackage{fancyhdr} % Headers and footers
\pagestyle{fancy} % All pages have headers and footers
\fancyhead{} % Blank out the default header
\fancyfoot{} % Blank out the default footer
\fancyhead[L]{Математика}
\fancyhead[C]{\textit{Геометрия}}
\fancyhead[R]{1 июня 2023}% Custom header text


%----------------------------------------------------------------------------------------

%\begin{document}\normalsize
\begin{document}\large
	
\begin{center}
	\textbf{Равносоставленные многоугольники}
\end{center}

\begin{definition}
	Две фигуры \textbf{равновелики}, если у них одинаковые площади.
\end{definition}

\begin{definition}
	Два многоугольника \textbf{равносоставлены}, если один из них можно разрезать на части, из которых можно сложить другой (без наложений, используя все части).
\end{definition}

\begin{theorem}[Бойяи–Гервина]
	Любые два равновеликих многоугольника равносоставлены.
\end{theorem}

\begin{enumerate}[label*=\protect\fbox{\arabic{enumi}}]
	
	
\item Докажите что произвольный треугольник равносоставлен какому-то прямоугольнику

\item Докажите \textit{транзитивность} равносоставленности (если фигуры $A$ и $B$ равносоставлены, и фигуры $B$ и $C$ равносоставлены, то фигуры $A$ и $C$ равносоставлены).

\item Докажите, что два равновеликих параллелограмма с общим основанием равносоставлены.

\item Докажите, что любые два равновеликих прямоугольника равносоставлены.

\item Докажите, что любой треугольник равносоставлен некоторому прямоугольнику со стороной 1.
%
%\begin{figure}[h]
%	\begin{asy} 
%	size(10cm, 0);
%	include geometry;
%	
%	point pA = (0,0), pB = (4, 0);
%	point pD = (2,3), pC = (6, 3);
%	point pE = (-1,3), pF = (-5, 3);
%	
%	dot("$A$", pA, S);
%	dot("$B$", pB, S);
%	dot("$C$", pC, N);
%	dot("$D$", pD, N);
%	dot("$E$", pE, N);
%	dot("$F$", pF, N);
%	
%	draw(pF -- pE);
%	draw(pA -- pB);
%	draw(pA -- pF);
%	draw(pA -- pD);
%	draw(pB -- pE);
%	draw(pD -- pC);
%	draw(pB -- pC);
%	draw(pE -- pD, linetype(new real[] {3,6}));
%	\end{asy}
%\end{figure}

\item Докажите \textbf{теорему Бойяи–Гервина}.

\end{enumerate}

\begin{definition}
	Фигуры называются \textbf{равнодополняемыми}, если их можно получить, отрезая от равных фигур одну или несколько равных частей.
\end{definition}


\begin{enumerate}[label*=\protect\fbox{\arabic{enumi}}]

\setcounter{enumi}{6}
\item Докажите, что равнодополняемые фигуры равновелики.

\item Докажите, что параллелограмм равнодополняем некоторому прямоугольнику.

\item Докажите, что равновеликие многоугольники равнодополняемы.

\item Перекроите прямоугольник $1 \times 3$ в квадрат.

\item Перекроите квадрат в правильный треугольник, разрезав его не более, чем на 10 частей.

\item Перекроите прямоугольник $3\times 4$ в квадрат, разрезав его всего на 3 части.

\item Перекроите прямоугольник $1 \times 3$ в квадрат, разрезав его не более чем на 6 частей.

\item Перекроите квадрат в 3 равных квадрата, разрезав его не более чем на а) 10 частей; б) 7 частей.

\item Докажите, что правильный пятиугольник можно разрезать на 4 части, из которых без просветов и наложений можно сложить прямоугольник.

\end{enumerate}
\end{document}