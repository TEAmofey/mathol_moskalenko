\documentclass{article}

\usepackage[12pt]{extsizes}
\usepackage[T2A]{fontenc}
\usepackage[utf8]{inputenc}
\usepackage[english, russian]{babel}

\usepackage{mathrsfs}
\usepackage[dvipsnames]{xcolor}

\usepackage{amsmath}
\usepackage{amssymb}
\usepackage{amsthm}
\usepackage{indentfirst}
\usepackage{amsfonts}
\usepackage{enumitem}
\usepackage{graphics}
\usepackage{tikz}
\usepackage{tabu}
\usepackage{diagbox}
\usepackage{hyperref}
\usepackage{mathtools}
\usepackage{ucs}
\usepackage{lipsum}
\usepackage{geometry} % Меняем поля страницы
\usepackage{fancyhdr} % Headers and footers
\newcommand{\range}{\mathrm{range}}
\newcommand{\dom}{\mathrm{dom}}
\newcommand{\N}{\mathbb{N}}
\newcommand{\R}{\mathbb{R}}
\newcommand{\E}{\mathbb{E}}
\newcommand{\D}{\mathbb{D}}
\newcommand{\M}{\mathcal{M}}
\newcommand{\Prime}{\mathbb{P}}
\newcommand{\A}{\mathbb{A}}
\newcommand{\Q}{\mathbb{Q}}
\newcommand{\Z}{\mathbb{Z}}
\newcommand{\F}{\mathbb{F}}
\newcommand{\CC}{\mathbb{C}}

\DeclarePairedDelimiter\abs{\lvert}{\rvert}
\DeclarePairedDelimiter\floor{\lfloor}{\rfloor}
\DeclarePairedDelimiter\ceil{\lceil}{\rceil}
\DeclarePairedDelimiter\lr{(}{)}
\DeclarePairedDelimiter\set{\{}{\}}
\DeclarePairedDelimiter\norm{\|}{\|}

\renewcommand{\labelenumi}{(\alph{enumi})}

\newcommand{\smallindent}{
    \geometry{left=1cm}% левое поле
    \geometry{right=1cm}% правое поле
    \geometry{top=1.5cm}% верхнее поле
    \geometry{bottom=1cm}% нижнее поле
}

\newcommand{\header}[3]{
    \pagestyle{fancy} % All pages have headers and footers
    \fancyhead{} % Blank out the default header
    \fancyfoot{} % Blank out the default footer
    \fancyhead[L]{#1}
    \fancyhead[C]{#2}
    \fancyhead[R]{#3}
}

\newcommand{\dividedinto}{
    \,\,\,\vdots\,\,\,
}

\newcommand{\littletaller}{\mathchoice{\vphantom{\big|}}{}{}{}}

\newcommand\restr[2]{{
    \left.\kern-\nulldelimiterspace % automatically resize the bar with \right
    #1 % the function
    \littletaller % pretend it's a little taller at normal size
    \right|_{#2} % this is the delimiter
}}

\DeclareGraphicsExtensions{.pdf,.png,.jpg}

\newenvironment{enumerate_boxed}[1][enumi]{\begin{enumerate}[label*=\protect\fbox{\arabic{#1}}]}{\end{enumerate}}



\smallindent

\header{Математика}{\textit{Геометрия}}{22 августа 2023}

%----------------------------------------------------------------------------------------

\begin{document}
    \large

    \begin{center}
        \textbf{Изогональное сопряжение 1}
    \end{center}

    \textbf{Определение:} Пусть дан треугольник $ABC$ и точка $P$.
    Тогда изогонали к прямым $AP$, $BP$, $CP$ относительно соответствующих углов треугольника пересекаются в одной точке или параллельны.
    Если они пересекаются, то точка их пересечения $Q$ называется изогонально сопряжённой точке $P$ относительно треугольника $ABC$.
    Если $P$ лежала не на стороне и не на описанной окружности треугольника, то изогональное сопряжение является взаимно-однозначным соответствием.

    Если точку $P$ отразить относительно сторон треугольника, то изогонально сопряжённая ей точка $Q$ будет центром окружности, проходящей через эти три отражения.


    \begin{enumerate_boxed}

        \item Точка $T$ такова, что все стороны треугольника $ABC$ видны из неё под углами $120^\circ$.
        Докажите, что основания перпендикуляров, опущенных из изогонально сопряжённой ей точки, являются вершинами равностороннего треугольника.

        \item Касательные к описанной окружности треугольника $ABC$ в точках $B$ и $C$ пересекаются в точке $P$.
        Точка $Q$ такова, что четырёхугольник $ABQC$ является параллелограммом.
        Докажите, что точки $P$ и $Q$ изогонально сопряжены.

        \item Про выпуклый четырёхугольник $ABCD$ известно, что $\angle A=\angle C\neq 90^\circ$.
        Докажите, что основания перпендикуляров, опущенных из точки $D$ на прямые $AB$, $BC$, $AC$, и середина отрезка $AC$ лежат на одной окружности.

        \item  В треугольнике $ABC$ проведена высота $AK$.
        Точка $K'$ симметрична точке $K$ относительно середины стороны $BC$.
        Касательные в точках $B$ и $C$ к описанной окружности треугольника $ABC$ пересекаются в точке $X$.
        Докажите, что точка $K'$ и основания перпендикуляров, опущенных из точки $X$ на прямые $AB$, $BC$ и $CA$, лежат на одной окружности.

        \item В трапеции $ABCD$ боковая сторона $CD$ перпендикулярна основаниям, $O$ — точка пересечения диагоналей.
        На описанной окружности треугольника $OCD$ взята точка $S$, диаметрально противоположная точке $O$.
        Докажите, что $\angle BSC=\angle ASD$.

        \item Точки $P$ и $Q$ изогонально сопряжены относительно треугольника $ABC$.
        Точка $QA$ симметрична точке $Q$ относительно прямой $BC$.
        Тогда точки $A$ и $QA$ изогонально сопряжены относительно треугольника $BPC$.

        \item Стороны треугольника $ABC$ видны из точки $T$ под углами $120^\circ$.
        Докажите, что прямые, симметричные прямым $AT$, $BT$ и $CT$ относительно прямых $BC$, $CA$ и $AB$ соответственно, пересекаются в одной точке.

        \item Точка $M$ — середина основания $AB$ равнобедренного треугольника $ABC$.
        Точка $P$ внутри треугольника такова, что $\angle CAP=\angle ABP$.
        Докажите, что $\angle APM+\angle CPB=180^\circ$.

        \item Внутри выпуклого четырёхугольника $ABCD$ выбрана точка $P$.
        Точки $Q_1$ и $Q_2$ расположены внутри $ABCD$ и таковы, что
        \[\angle Q_{1}BC=\angle ABP, \angle Q_{1}CB=\angle DCP, \angle Q_{2}AD=\angle BAP, \angle Q_{2}DA=\angle CDP.\] Докажите, что $Q_{1}Q_2\parallel AB$ тогда и только тогда, когда $Q_{1}Q_2\parallel CD$.

        \item В треугольнике $ABC$ провели высоты $AA_0$, $BB_0$, $CC_0$.
        Точка $M$ — произвольная точка, $A_1$ — точка, симметричная $M$ относительно $BC$, аналогично определим точки $B_1$, $C_1$.
        Докажите, что прямые $A_{0}A_1$, $B_{0}B_1$, $C_{0}C_1$ пересекаются в одной точке или параллельны.

        \item Про параллелограмм \(ABCD\) известно, что \(\angle DAC=90^\circ\).
        Пусть \(H\) — основание перпендикуляра, опущенного из \(A\) на \(DC\), \(P\) — такая точка на прямой \(AC\), что прямая \(PD\) касается описанной окружности треугольника \(ABD\).
        Докажите, что \(\angle PBA=\angle DBH\).

        \item Точки \(I\) и \(I_a\) являются центрами вписанной и вневписанной (напротив вершины \(A\)) окружностей треугольника \(ABC\).
        Точка \(A'\) диаметрально противоположна точке \(A\) на описанной окружности треугольника \(ABC\).
        Докажите, что \(\angle BA'I+\angle CA'I_a=180^\circ\).

        \item В треугольнике \(ABC\) выполнено неравенство \(AB<BC\).
        Биссектриса угла \(C\) пересекает прямую, параллельную \(AC\) и проходящую через точку \(B\), в точке \(P\).
        Касательная к описанной окружности треугольника \(ABC\), проведённая в точке \(B\), пересекает ту же биссектрису в точке \(R\).
        Точка \(R'\) симметрична точке \(R\) относительно \(AB\).
        Докажите, что \(\angle R'PB=\angle RPA\).

        \item Пусть \(P\) — точка внутри треугольника \(ABC\) такая, что
        \[\angle APB-\angle ACB=\angle APC-\angle ABC.\]
        Докажите, что биссектрисы углов \(ABP\) и \(ACP\) пересекаются на прямой \(AP\).

    \end{enumerate_boxed}
\end{document}