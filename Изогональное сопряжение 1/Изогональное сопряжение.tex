\documentclass{article}
\usepackage[12pt]{extsizes}
\usepackage[T2A]{fontenc}
\usepackage[utf8]{inputenc}
\usepackage[english, russian]{babel}

\usepackage{amssymb}
\usepackage{amsfonts}
\usepackage{amsmath}
\usepackage{enumitem}
\usepackage{graphics}

\usepackage{lipsum}



\usepackage{geometry} % Меняем поля страницы
\geometry{left=1cm}% левое поле
\geometry{right=1cm}% правое поле
\geometry{top=1.5cm}% верхнее поле
\geometry{bottom=1cm}% нижнее поле


\newtheorem{definition}{Опредление}
\usepackage{fancyhdr} % Headers and footers
\pagestyle{fancy} % All pages have headers and footers
\fancyhead{} % Blank out the default header
\fancyfoot{} % Blank out the default footer
\fancyhead[L]{Математика}
\fancyhead[C]{\textit{Геометрия}}
\fancyhead[R]{22 августа 2023}% Custom header text


%----------------------------------------------------------------------------------------

%\begin{document}\normalsize
\begin{document}\large



\begin{center}
\textbf{Изогональное сопряжение 1}
\end{center}

\textbf{Определение:} Пусть дан треугольник $ABC$ и точка $P$. Тогда изогонали к прямым $AP$, $BP$, $CP$ относительно соответствующих углов треугольника пересекаются в одной точке или параллельны. Если они пересекаются, то точка их пересечения $Q$ называется изогонально сопряжённой точке $P$ относительно треугольника $ABC$. Если $P$ лежала не на стороне и не на описанной окружности треугольника, то изогональное сопряжение является взаимно-однозначным соответствием.

Если точку $P$ отразить относительно сторон треугольника, то изогонально сопряжённая ей точка $Q$ будет центром окружности, проходящей через эти три отражения.


\begin{enumerate}[label*=\protect\fbox{\arabic{enumi}}]

\item Точка $T$ такова, что все стороны треугольника $ABC$ видны из неё под углами $120^\circ$. Докажите, что основания перпендикуляров, опущенных из изогонально сопряжённой ей точки, являются вершинами равностороннего треугольника.

\item Касательные к описанной окружности треугольника $ABC$ в точках $B$ и $C$ пересекаются в точке $P$. Точка $Q$ такова, что четырёхугольник $ABQC$ является параллелограммом. Докажите, что точки $P$ и $Q$ изогонально сопряжены.

\item Про выпуклый четырёхугольник $ABCD$ известно, что $\angle A=\angle C\neq 90^\circ$. Докажите, что основания перпендикуляров, опущенных из точки $D$ на прямые $AB$, $BC$, $AC$, и середина отрезка $AC$ лежат на одной окружности.

\item  В треугольнике $ABC$ проведена высота $AK$. Точка $K'$ симметрична точке $K$ относительно середины стороны $BC$. Касательные в точках $B$ и $C$ к описанной окружности треугольника $ABC$ пересекаются в точке $X$. Докажите, что точка $K'$ и основания перпендикуляров, опущенных из точки $X$ на прямые $AB$, $BC$ и $CA$, лежат на одной окружности.

\item В трапеции $ABCD$ боковая сторона $CD$ перпендикулярна основаниям, $O$ — точка пересечения диагоналей. На описанной окружности треугольника $OCD$ взята точка $S$, диаметрально противоположная точке $O$. Докажите, что $\angle BSC=\angle ASD$.


\item Точки $P$ и $Q$ изогонально сопряжены относительно треугольника $ABC$. Точка $QA$ симметрична точке $Q$ относительно прямой $BC$. Тогда точки $A$ и $QA$ изогонально сопряжены относительно треугольника $BPC$.

\item Стороны треугольника $ABC$ видны из точки $T$ под углами $120^\circ$. Докажите, что прямые, симметричные прямым $AT$, $BT$ и $CT$ относительно прямых $BC$, $CA$ и $AB$ соответственно, пересекаются в одной точке.

\item Точка $M$ — середина основания $AB$ равнобедренного треугольника $ABC$. Точка $P$ внутри треугольника такова, что $\angle CAP=\angle ABP$. Докажите, что $\angle APM+\angle CPB=180^\circ$.


\item Внутри выпуклого четырёхугольника $ABCD$ выбрана точка $P$. Точки $Q_1$ и $Q_2$ расположены внутри $ABCD$ и таковы, что
$$\angle Q_1BC=\angle ABP, \angle Q_1CB=\angle DCP, \angle Q_2AD=\angle BAP, \angle Q_2DA=\angle CDP.$$ Докажите, что $Q_1Q_2\parallel AB$ тогда и только тогда, когда $Q_1Q_2\parallel CD$.


\item В треугольнике $ABC$ провели высоты $AA_0$, $BB_0$, $CC_0$. Точка $M$ — произвольная точка, $A_1$ — точка, симметричная $M$ относительно $BC$, аналогично определим точки $B_1$, $C_1$. Докажите, что прямые $A_0A_1$, $B_0B_1$, $C_0C_1$ пересекаются в одной точке или параллельны.

\item Про параллелограмм \(ABCD\) известно, что \(\angle DAC=90^\circ\). Пусть \(H\) — основание перпендикуляра, опущенного из \(A\) на \(DC\), \(P\) — такая точка на прямой \(AC\), что прямая \(PD\) касается описанной окружности треугольника \(ABD\). Докажите, что \(\angle PBA=\angle DBH\).

\item Точки \(I\) и \(I_a\) являются центрами вписанной и вневписанной (напротив вершины \(A\)) окружностей треугольника \(ABC\). Точка \(A'\) диаметрально противоположна точке \(A\) на описанной окружности треугольника \(ABC\). Докажите, что \(\angle BA'I+\angle CA'I_a=180^\circ\).

\item В треугольнике \(ABC\) выполнено неравенство \(AB<BC\). Биссектриса угла \(C\) пересекает прямую, параллельную \(AC\) и проходящую через точку \(B\), в точке \(P\). Касательная к описанной окружности треугольника \(ABC\), проведённая в точке \(B\), пересекает ту же биссектрису в точке \(R\). Точка \(R'\) симметрична точке \(R\) относительно \(AB\). Докажите, что \(\angle R'PB=\angle RPA\).

\item Пусть \(P\) — точка внутри треугольника \(ABC\) такая, что
$$\angle APB-\angle ACB=\angle APC-\angle ABC.$$
Докажите, что биссектрисы углов \(ABP\) и \(ACP\) пересекаются на прямой \(AP\).

\end{enumerate}
\end{document}