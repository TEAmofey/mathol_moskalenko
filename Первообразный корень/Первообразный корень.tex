\documentclass{article}
\usepackage[12pt]{extsizes}
\usepackage[T2A]{fontenc}
\usepackage[utf8]{inputenc}
\usepackage[english, russian]{babel}

\usepackage{amssymb}
\usepackage{amsfonts}
\usepackage{amsmath}
\usepackage{enumitem}
\usepackage{graphics}
\usepackage{graphicx}

\usepackage{lipsum}

\newtheorem{theorem}{Теорема}
\newtheorem{task}{Задача}
\newtheorem{lemma}{Лемма}
\newtheorem{definition}{Определение}
\newtheorem{example}{Пример}
\newtheorem{statement}{Утверждение}
\newtheorem{corollary}{Следствие}


\usepackage{geometry} % Меняем поля страницы
\geometry{left=1cm}% левое поле
\geometry{right=1cm}% правое поле
\geometry{top=1.5cm}% верхнее поле
\geometry{bottom=1cm}% нижнее поле


\usepackage{fancyhdr} % Headers and footers
\pagestyle{fancy} % All pages have headers and footers
\fancyhead{} % Blank out the default header
\fancyfoot{} % Blank out the default footer
\fancyhead[L]{Математика}
\fancyhead[C]{\textit{Теория чисел}}
\fancyhead[R]{17 декабря 2023}% Custom header text


%----------------------------------------------------------------------------------------

%\begin{document}\normalsize
\begin{document}\large
	
\begin{center}
	\textbf{Первообразный корень}
\end{center}

\begin{enumerate}[label*=\protect\fbox{\arabic{enumi}}]
	
	\item \textbf{Повторение.} Пусть $1 = d_1 < d_2 < \ldots < d_k = n$ — все делители числа $n$. Докажите, что $\varphi(d_1) + \ldots + \varphi(d_k) = n$.
	
	Обозначим через $\psi(t)$ количество остатков от деления на $p$, чей показатель равен $t$.
	
	\item Пусть $p$ — простое число. $1 = d_1 < d_2 < \ldots < d_k = p - 1$ — все делители числа $p - 1$. Докажите, что $\psi(d_1) + \ldots + \psi(d_k) = p - 1$.
	
	\item Пусть показатель остатка $a$ по модулю $p$ равен $d$.
	
	\begin{enumerate}
		\item Докажите, что $1, a, \ldots, a^{d-1}$ — это все корни многочлена $x^d - 1$.
		\item Пусть показатель остатка $b$ по модулю $p$ также равен $d$. Докажите, что $b \equiv a^s \pmod{p}$.
		\item В условиях предыдущего пункта докажите, что $\text{НОД}(d, s) = 1$.
	\end{enumerate}
	
	\item Выведите из предыдущей задачи, что $\psi(d) \leq \varphi(d)$ для любого делителя $d \mid (p - 1)$.
	
	\item Докажите, что $\psi(d) = \varphi(d)$ для любого делителя $d \mid (p - 1)$.
	
	Из задачи 5 следует, что $\psi(p - 1) = \varphi(p - 1) > 0$. Значит, существует остаток $g$ (и не один, а целых $\varphi(p - 1)$ штук) такой, что показатель $g$ равен $p - 1$. Такой остаток называется первообразным корнем по модулю $p$.
	
	\item Для каких простых $p$ первообразный корень может быть квадратичным вычетом?
	
	\item Докажите, что любой ненулевой остаток $a$ от деления на $p$ представим в виде $a \equiv g^t \pmod{p}$ для некоторой степени $t$.
	
	\item Сколько решений имеет уравнение
	\begin{enumerate}
		\item $x^5 \equiv 1 \pmod{101}$?
		\item $x^{70} \equiv 1 \pmod{101}$?
		\item $x^4 + x^3 + x^2 + x + 1 \equiv 0 \pmod{101}$?
	\end{enumerate}
	
	\item Найдите остаток $1^{10} + 2^{10} + \ldots + 100^{10}$ от деления на $101$.
	
	\item Пусть $p$ — простое. Можно ли расставить по кругу числа $1, 2, \ldots, p - 1$ так, чтобы для любых трех подряд идущих чисел $a, b, c$ (именно в таком порядке) число $b^2 - ac$ делилось бы на $p$?
	
	\item Можно ли разбить числа от $1$ до $2016$ на группы по $7$ так, чтобы сумма чисел в каждой семёрке делилась на $2017$?
	
\end{enumerate}
\end{document}