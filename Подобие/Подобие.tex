\documentclass{article}

\usepackage[12pt]{extsizes}
\usepackage[T2A]{fontenc}
\usepackage[utf8]{inputenc}
\usepackage[english, russian]{babel}

\usepackage{mathrsfs}
\usepackage[dvipsnames]{xcolor}

\usepackage{amsmath}
\usepackage{amssymb}
\usepackage{amsthm}
\usepackage{indentfirst}
\usepackage{amsfonts}
\usepackage{enumitem}
\usepackage{graphics}
\usepackage{tikz}
\usepackage{tabu}
\usepackage{diagbox}
\usepackage{hyperref}
\usepackage{mathtools}
\usepackage{ucs}
\usepackage{lipsum}
\usepackage{geometry} % Меняем поля страницы
\usepackage{fancyhdr} % Headers and footers
\newcommand{\range}{\mathrm{range}}
\newcommand{\dom}{\mathrm{dom}}
\newcommand{\N}{\mathbb{N}}
\newcommand{\R}{\mathbb{R}}
\newcommand{\E}{\mathbb{E}}
\newcommand{\D}{\mathbb{D}}
\newcommand{\M}{\mathcal{M}}
\newcommand{\Prime}{\mathbb{P}}
\newcommand{\A}{\mathbb{A}}
\newcommand{\Q}{\mathbb{Q}}
\newcommand{\Z}{\mathbb{Z}}
\newcommand{\F}{\mathbb{F}}
\newcommand{\CC}{\mathbb{C}}

\DeclarePairedDelimiter\abs{\lvert}{\rvert}
\DeclarePairedDelimiter\floor{\lfloor}{\rfloor}
\DeclarePairedDelimiter\ceil{\lceil}{\rceil}
\DeclarePairedDelimiter\lr{(}{)}
\DeclarePairedDelimiter\set{\{}{\}}
\DeclarePairedDelimiter\norm{\|}{\|}

\renewcommand{\labelenumi}{(\alph{enumi})}

\newcommand{\smallindent}{
    \geometry{left=1cm}% левое поле
    \geometry{right=1cm}% правое поле
    \geometry{top=1.5cm}% верхнее поле
    \geometry{bottom=1cm}% нижнее поле
}

\newcommand{\header}[3]{
    \pagestyle{fancy} % All pages have headers and footers
    \fancyhead{} % Blank out the default header
    \fancyfoot{} % Blank out the default footer
    \fancyhead[L]{#1}
    \fancyhead[C]{#2}
    \fancyhead[R]{#3}
}

\newcommand{\dividedinto}{
    \,\,\,\vdots\,\,\,
}

\newcommand{\littletaller}{\mathchoice{\vphantom{\big|}}{}{}{}}

\newcommand\restr[2]{{
    \left.\kern-\nulldelimiterspace % automatically resize the bar with \right
    #1 % the function
    \littletaller % pretend it's a little taller at normal size
    \right|_{#2} % this is the delimiter
}}

\DeclareGraphicsExtensions{.pdf,.png,.jpg}

\newenvironment{enumerate_boxed}[1][enumi]{\begin{enumerate}[label*=\protect\fbox{\arabic{#1}}]}{\end{enumerate}}



\smallindent

\header{Математика}{\textit{Геометрия}}{15 декабря 2022}

%----------------------------------------------------------------------------------------

\begin{document}
    \large

    \begin{center}
        \textbf{Подобие}
    \end{center}

    \begin{enumerate_boxed}

        \item В прямоугольный треугольник с катетами, равными 6 и 8, вписан квадрат, имеющий с треугольником общий прямой угол.
        Найдите сторону этого квадрата.


        \item Середина основания трапеции соединена с вершинами другого основания.
        Эти прямые пересекают диагонали трапеции в точках $P$ и $Q$.
        Докажите, что прямая $PQ$ параллельна основаниям трапеции, и её отрезок, заключенный между боковыми сторонами, делится точками $P$ и $Q$ на три равные части.

        \item На стороне $AB$ треугольника $ABC$ взята точка $K$, а на стороне $BC$ – точки $M$ и $N$ так, что $AB = 4AK, CM = BN, MN = 2BN$.
        Найдите отношения $AO:ON$ и $KO:OM$, где $O$ – точка пересечения прямых $AN$ и $KM$.

        \item $AL$ – биссектриса треугольника $ABC$, причём $AL=LB$.
        На луче $AL$ отложен отрезок $AK$, равный $CL$.
        Докажите, что $AK=CK$.

        \item В выпуклом четырёхугольнике $ABCD$, углы $BAD$ и $BCD$ равны, а биссектриса угла $ABC$ проходит через середину отрезка $CD$.
        Известно, что $CD = 3AD$.
        Найдите отношение $AB : BC$.

        \item На диагонали $BD$ параллелограмма $ABCD$ взята точка $K$.
        Прямая $AK$ пересекает прямые $BC$ и $CD$ в точках $L$ и $M$ соответственно.
        Докажите, что $AK^2 =KL \cdot KM$.

        \item Дан треугольник $ABC$, в котором $\angle A = 2\angle B$.
        Докажите, что $BC^2 = CA^2 + CA \cdot AB$.

        \item В треугольнике $ABC$ точка $D$ лежит на стороне $AC$, углы $ABD$ и $BCD$ равны, $AB = CD$, $AE$ биссектриса угла $A$.
        Докажите, что $ED \parallel AB$.

        \item Точка $M$ середина стороны $BC$ треугольника $ABC$.
        Точка $D$ на стороне $AC$ такова, что $AD = BD$.
        Точка $E$ лежит на прямой $AM$ так, что прямые $DE$ и $AB$ параллельны.
        Докажите, что $\angle DBE =  \angle ACB$.

        \item Продолжения боковых сторон $AB$ и $CD$ трапеции $ABCD$ пересекаются в точке $E$.
        Найдите стороны треугольника $AED$, если
        $AB = 3, BC = 10, CD = 4, AD = 12$.

        \item На боковой стороне $CD$ трапеции $ABCD$ нашлась такая точка $M$, что треугольник $ABM$ равносторонний.
        Докажите, что на прямой $AB$ есть точка $N$, для которой треугольник $CDN$ равносторонний.

    \end{enumerate_boxed}

\end{document}