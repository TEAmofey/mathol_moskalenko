\documentclass{article}

\usepackage[12pt]{extsizes}
\usepackage[T2A]{fontenc}
\usepackage[utf8]{inputenc}
\usepackage[english, russian]{babel}

\usepackage{mathrsfs}
\usepackage[dvipsnames]{xcolor}

\usepackage{amsmath}
\usepackage{amssymb}
\usepackage{amsthm}
\usepackage{indentfirst}
\usepackage{amsfonts}
\usepackage{enumitem}
\usepackage{graphics}
\usepackage{tikz}
\usepackage{tabu}
\usepackage{diagbox}
\usepackage{hyperref}
\usepackage{mathtools}
\usepackage{ucs}
\usepackage{lipsum}
\usepackage{geometry} % Меняем поля страницы
\usepackage{fancyhdr} % Headers and footers
\usepackage[framemethod=TikZ]{mdframed}

\newcommand{\definebox}[3]{%
    \newcounter{#1}
    \newenvironment{#1}[1][]{%
        \stepcounter{#1}%
        \mdfsetup{%
            frametitle={%
            \tikz[baseline=(current bounding box.east),outer sep=0pt]
            \node[anchor=east,rectangle,fill=white]
            {\strut #2~\csname the#1\endcsname\ifstrempty{##1}{}{##1}};}}%
        \mdfsetup{innertopmargin=1pt,linecolor=#3,%
            linewidth=3pt,topline=true,
            frametitleaboveskip=\dimexpr-\ht\strutbox\relax,}%
        \begin{mdframed}[]
            \relax%
            }{
        \end{mdframed}}%
}

\definebox{theorem_boxed}{Теорема}{ForestGreen!24}
\definebox{definition_boxed}{Определение}{blue!24}
\definebox{task_boxed}{Задача}{orange!24}
\definebox{paradox_boxed}{Парадокс}{red!24}

\theoremstyle{plain}
\newtheorem{theorem}{Теорема}
\newtheorem{task}{Задача}
\newtheorem{lemma}{Лемма}
\newtheorem{definition}{Определение}
\newtheorem{statement}{Утверждение}
\newtheorem{corollary}{Следствие}

\theoremstyle{remark}
\newtheorem{remark}{Замечание}
\newtheorem{example}{Пример}
\newcommand{\range}{\mathrm{range}}
\newcommand{\dom}{\mathrm{dom}}
\newcommand{\N}{\mathbb{N}}
\newcommand{\R}{\mathbb{R}}
\newcommand{\E}{\mathbb{E}}
\newcommand{\D}{\mathbb{D}}
\newcommand{\M}{\mathcal{M}}
\newcommand{\Prime}{\mathbb{P}}
\newcommand{\A}{\mathbb{A}}
\newcommand{\Q}{\mathbb{Q}}
\newcommand{\Z}{\mathbb{Z}}
\newcommand{\F}{\mathbb{F}}
\newcommand{\CC}{\mathbb{C}}

\DeclarePairedDelimiter\abs{\lvert}{\rvert}
\DeclarePairedDelimiter\floor{\lfloor}{\rfloor}
\DeclarePairedDelimiter\ceil{\lceil}{\rceil}
\DeclarePairedDelimiter\lr{(}{)}
\DeclarePairedDelimiter\set{\{}{\}}
\DeclarePairedDelimiter\norm{\|}{\|}

\renewcommand{\labelenumi}{(\alph{enumi})}

\newcommand{\smallindent}{
    \geometry{left=1cm}% левое поле
    \geometry{right=1cm}% правое поле
    \geometry{top=1.5cm}% верхнее поле
    \geometry{bottom=1cm}% нижнее поле
}

\newcommand{\header}[3]{
    \pagestyle{fancy} % All pages have headers and footers
    \fancyhead{} % Blank out the default header
    \fancyfoot{} % Blank out the default footer
    \fancyhead[L]{#1}
    \fancyhead[C]{#2}
    \fancyhead[R]{#3}
}

\newcommand{\dividedinto}{
    \,\,\,\vdots\,\,\,
}

\newcommand{\littletaller}{\mathchoice{\vphantom{\big|}}{}{}{}}

\newcommand\restr[2]{{
    \left.\kern-\nulldelimiterspace % automatically resize the bar with \right
    #1 % the function
    \littletaller % pretend it's a little taller at normal size
    \right|_{#2} % this is the delimiter
}}

\DeclareGraphicsExtensions{.pdf,.png,.jpg}

\newenvironment{enumerate_boxed}[1][enumi]{\begin{enumerate}[label*=\protect\fbox{\arabic{#1}}]}{\end{enumerate}}



\smallindent

\header{ЦРОД $\bullet$ Математика}{\textit{Алгебра и Теория Чисел}}{ЛФМШ 2023}

%----------------------------------------------------------------------------------------

\begin{document}
    \large

    \begin{center}
        \textbf{Телескопические суммы}
    \end{center}

    Слагаемые разных знаков могут компенсировать друг друга или давать более простую разность.
    То же верно и для сомножителей.

    \begin{enumerate_boxed}

        \setcounter{enumi}{-3}

        \item $(1\cdot 3+3\cdot 5+5\cdot 7+ \dotsc +99\cdot 101) - (2^2+4^2+6^2+ \dotsc +100^2).$
    \end{enumerate_boxed}

    Взаимно уничтожающиеся слагаемые можно получить, заменив каждое слагаемое на разность.

    \begin{enumerate_boxed}

        \setcounter{enumi}{-2}

        \item $\dfrac{1}{1 \cdot 2} + \dfrac{1}{2 \cdot 3} + \dotsc +  \dfrac{1}{(n-1) \cdot n}.$


    \end{enumerate_boxed}

    Разность может отличаться от члена суммы в несколько раз или содержать дополнительное слагаемое.
    Но и общий множитель, и лишние слагаемые часто удаётся учесть или сосчитать отдельно.

    \begin{enumerate_boxed}

        \setcounter{enumi}{-1}

        \item $\dfrac{1}{1 \cdot 4} + \dfrac{1}{4 \cdot 7} + \dotsc +  \dfrac{1}{97 \cdot 100}.$

    \end{enumerate_boxed}

    Если мы «подозреваем» формулу $F(n)$ для суммы $n$ первых членов, это немедленно дает нам представление $n$-го члена как разности $F(n)-F(n-1)$.
    Для подозрений есть основания: сумма линейных выражений (арифметической прогресии) всегда квадратична, квадратичных – кубична, и т.д. Коэффициенты можно подобрать.


    \begin{enumerate_boxed}

        \setcounter{enumi}{0}

        \item $(1\cdot 3+3\cdot 5+5\cdot 7+\dotsc+99\cdot 101) - (1^2+3^2+5^2+ \dotsc +99^2).$
        \item $\left(1+ \dfrac{1}{2}\right) \cdot  \left(1+ \dfrac{1}{3}\right) \cdot  \left(1+ \dfrac{1}{4}\right) \cdot  \dotsc \cdot  \left(1+ \dfrac{1}{100}\right).$

        \item $1!\cdot 1 + 2!\cdot 2 + 3!\cdot 3 + \dotsc +100!\cdot 100.$

        \item $\dfrac{2^2}{1 \cdot 3} + \dfrac{4^2}{3 \cdot 5} + \dotsc +  \dfrac{100^2}{99 \cdot 101}.$

        \item $3\cdot 1\cdot 2 + 3\cdot 2\cdot 3 + 3\cdot 3\cdot 4 + \dotsc +3\cdot 99\cdot 100.$

        \item $\dfrac{1}{1^2 \cdot 3^2} + \dfrac{2}{3^2 \cdot 5^2} + \dotsc +  \dfrac{50}{99^2 \cdot 101^2}.$

        Найдите формулу, зависящую от $n$

        \item $1^2 + 2^2 + 3^2 + \dotsc +n^2$

        \item $1^3 + 2^3 + 3^3 + \dotsc +n^3$

        \item $1^4 + 2^4 + 3^4 + \dotsc +n^4$

        \item Для каждого натурального $n \ge 2$ вычислите сумму
        \[\dfrac{1}{1} + \dfrac{1}{2} + \dotso +  \dfrac{1}{n} + \dfrac{1}{1 \cdot 2} + \dfrac{1}{1 \cdot 3} + \dotso + \dfrac{1}{(n-1) \cdot n} + \dotso + \dfrac{1}{1 \cdot 2 \cdot \dotso \cdot n}.\]

    \end{enumerate_boxed}

\end{document}