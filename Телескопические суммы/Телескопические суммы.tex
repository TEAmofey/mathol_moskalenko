\documentclass{article}
\usepackage[12pt]{extsizes}
\usepackage[T2A]{fontenc}
\usepackage[utf8]{inputenc}
\usepackage[english, russian]{babel}

\usepackage{amssymb}
\usepackage{amsfonts}
\usepackage{amsmath}
\usepackage{enumitem}
\usepackage{graphics}
\usepackage{graphicx}

\usepackage{lipsum}

\newcommand{\definebox}[3]{%
	\newcounter{#1}
	\newenvironment{#1}[1][]{%
		\stepcounter{#1}%
		\mdfsetup{%
			frametitle={%
				\tikz[baseline=(current bounding box.east),outer sep=0pt]
				\node[anchor=east,rectangle,fill=white]
				{\strut #2~\csname the#1\endcsname\ifstrempty{##1}{}{##1}};}}%
		\mdfsetup{innertopmargin=1pt,linecolor=#3,%
			linewidth=3pt,topline=true,
			frametitleaboveskip=\dimexpr-\ht\strutbox\relax,}%
		\begin{mdframed}[]\relax%
		}{\end{mdframed}}%
}

\definebox{definition}{Определение}{blue!24}
\definebox{task}{Задача}{orange!24}

\usepackage{geometry} % Меняем поля страницы
\geometry{left=1cm}% левое поле
\geometry{right=1cm}% правое поле
\geometry{top=1.5cm}% верхнее поле
\geometry{bottom=1cm}% нижнее поле


\usepackage{fancyhdr} % Headers and footers
\pagestyle{fancy} % All pages have headers and footers
\fancyhead{} % Blank out the default header
\fancyfoot{} % Blank out the default footer
\fancyhead[L]{ЦРОД $\bullet$ Математика}
\fancyhead[C]{\textit{Алгебра и Теория Чисел}}
\fancyhead[R]{ЛФМШ 2023}% Custom header text



%----------------------------------------------------------------------------------------

%\begin{document}\normalsize
\begin{document}\large
	
	
	\begin{center}
		\textbf{Телескопические суммы}
	\end{center}

Слагаемые разных знаков могут компенсировать друг друга или давать более простую разность. То же верно и для сомножителей.

\begin{enumerate}[label*=\protect\fbox{\arabic{enumi}}]

\setcounter{enumi}{-3}

\item $(1\cdot 3+3\cdot 5+5\cdot 7+...+99\cdot 101) - (2^2+4^2+6^2+ \dotsc +100^2).$
\end{enumerate}

Взаимно уничтожающиеся слагаемые можно получить, заменив каждое слагаемое на разность.

\begin{enumerate}[label*=\protect\fbox{\arabic{enumi}}]
	
\setcounter{enumi}{-2}

\item $\dfrac{1}{1 \cdot 2} + \dfrac{1}{2 \cdot 3} + \dotsc +  \dfrac{1}{(n-1) \cdot n}.$
	


\end{enumerate}

Разность может отличаться от члена суммы в несколько раз или содержать дополнительное слагаемое. Но и общий множитель, и лишние слагаемые часто удаётся учесть или сосчитать отдельно.

\begin{enumerate}[label*=\protect\fbox{\arabic{enumi}}]
	
	\setcounter{enumi}{-1}
	
	\item $\dfrac{1}{1 \cdot 4} + \dfrac{1}{4 \cdot 7} + \dotsc +  \dfrac{1}{97 \cdot 100}.$
	
\end{enumerate}

Если мы «подозреваем» формулу $F(n)$ для суммы $n$ первых членов, это немедленно дает нам представление $n$-го члена как разности $F(n)-F(n-1)$. Для подозрений есть основания: сумма линейных выражений (арифметической прогресии) всегда квадратична, квадратичных – кубична, и т.д. Коэффициенты можно подобрать.


\begin{enumerate}[label*=\protect\fbox{\arabic{enumi}}]
	
	\setcounter{enumi}{0}
	
	\item $(1\cdot 3+3\cdot 5+5\cdot 7+...+99\cdot 101) - (1^2+3^2+5^2+ \dotsc +99^2).$
	\item $\left(1+ \dfrac{1}{2}\right) \cdot  \left(1+ \dfrac{1}{3}\right) \cdot  \left(1+ \dfrac{1}{4}\right) \cdot  \dotsc \cdot  \left(1+ \dfrac{1}{100}\right).$
	
	\item $1!\cdot 1 + 2!\cdot 2 + 3!\cdot 3 + ... +100!\cdot 100.$
	
	\item $\dfrac{2^2}{1 \cdot 3} + \dfrac{4^2}{3 \cdot 5} + \dotsc +  \dfrac{100^2}{99 \cdot 101}.$
	
	\item $3\cdot 1\cdot 2 + 3\cdot 2\cdot 3 + 3\cdot 3\cdot 4 + ... +3\cdot 99\cdot 100.$ 
	
	\item $\dfrac{1}{1^2 \cdot 3^2} + \dfrac{2}{3^2 \cdot 5^2} + \dotsc +  \dfrac{50}{99^2 \cdot 101^2}.$
	
	Найдите формулу зависящую от $n$ 
	\item $1^2 + 2^2 + 3^2 + \dotsc +n^2$
	\item $1^3 + 2^3 + 3^3 + \dotsc +n^3$
	\item $1^4 + 2^4 + 3^4 + \dotsc +n^4$
	
	\item Для каждого натурального $n \ge 2$ вычислите сумму $$\dfrac{1}{1} + \dfrac{1}{2} + \dotso +  \dfrac{1}{n} + \dfrac{1}{1 \cdot 2} + \dfrac{1}{1 \cdot 3} + \dotso +  \dfrac{1}{(n-1) \cdot n} + \dotso + \dfrac{1}{1 \cdot 2 \cdot \dotso \cdot n}$$.
	
\end{enumerate}

\end{document}