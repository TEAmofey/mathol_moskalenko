\documentclass{article}

\usepackage[12pt]{extsizes}
\usepackage[T2A]{fontenc}
\usepackage[utf8]{inputenc}
\usepackage[english, russian]{babel}

\usepackage{mathrsfs}
\usepackage[dvipsnames]{xcolor}

\usepackage{amsmath}
\usepackage{amssymb}
\usepackage{amsthm}
\usepackage{indentfirst}
\usepackage{amsfonts}
\usepackage{enumitem}
\usepackage{graphics}
\usepackage{tikz}
\usepackage{tabu}
\usepackage{diagbox}
\usepackage{hyperref}
\usepackage{mathtools}
\usepackage{ucs}
\usepackage{lipsum}
\usepackage{geometry} % Меняем поля страницы
\usepackage{fancyhdr} % Headers and footers
\newcommand{\range}{\mathrm{range}}
\newcommand{\dom}{\mathrm{dom}}
\newcommand{\N}{\mathbb{N}}
\newcommand{\R}{\mathbb{R}}
\newcommand{\E}{\mathbb{E}}
\newcommand{\D}{\mathbb{D}}
\newcommand{\M}{\mathcal{M}}
\newcommand{\Prime}{\mathbb{P}}
\newcommand{\A}{\mathbb{A}}
\newcommand{\Q}{\mathbb{Q}}
\newcommand{\Z}{\mathbb{Z}}
\newcommand{\F}{\mathbb{F}}
\newcommand{\CC}{\mathbb{C}}

\DeclarePairedDelimiter\abs{\lvert}{\rvert}
\DeclarePairedDelimiter\floor{\lfloor}{\rfloor}
\DeclarePairedDelimiter\ceil{\lceil}{\rceil}
\DeclarePairedDelimiter\lr{(}{)}
\DeclarePairedDelimiter\set{\{}{\}}
\DeclarePairedDelimiter\norm{\|}{\|}

\renewcommand{\labelenumi}{(\alph{enumi})}

\newcommand{\smallindent}{
    \geometry{left=1cm}% левое поле
    \geometry{right=1cm}% правое поле
    \geometry{top=1.5cm}% верхнее поле
    \geometry{bottom=1cm}% нижнее поле
}

\newcommand{\header}[3]{
    \pagestyle{fancy} % All pages have headers and footers
    \fancyhead{} % Blank out the default header
    \fancyfoot{} % Blank out the default footer
    \fancyhead[L]{#1}
    \fancyhead[C]{#2}
    \fancyhead[R]{#3}
}

\newcommand{\dividedinto}{
    \,\,\,\vdots\,\,\,
}

\newcommand{\littletaller}{\mathchoice{\vphantom{\big|}}{}{}{}}

\newcommand\restr[2]{{
    \left.\kern-\nulldelimiterspace % automatically resize the bar with \right
    #1 % the function
    \littletaller % pretend it's a little taller at normal size
    \right|_{#2} % this is the delimiter
}}

\DeclareGraphicsExtensions{.pdf,.png,.jpg}

\newenvironment{enumerate_boxed}[1][enumi]{\begin{enumerate}[label*=\protect\fbox{\arabic{#1}}]}{\end{enumerate}}



\smallindent

\header{Математика}{\textit{Алгебра}}{2024}

%----------------------------------------------------------------------------------------

%\begin{document}\normalsize
\begin{document}
    \large

    \begin{center}
        \textbf{Региональный разнобой}
    \end{center}


    \begin{enumerate_boxed}

%23.9.10
        \item Найдите наибольшее число $m$ такое, что для любых положительных чисел $a, b$ и $c$, сумма которых равна 1, выполнено неравенство
        \[\sqrt{\frac{ab}{c + ab}} + \sqrt{\frac{bc}{a + bc}} + \sqrt{\frac{ca}{b + ca}} \geqslant m\]

%22.9.1
        \item Петя написал на доске десять натуральных чисел, среди которых нет двух равных.
        Известно, что из этих десяти чисел можно выбрать три числа, делящихся на 5.
        Также известно, что из написанных десяти чисел можно выбрать четыре числа, делящихся на 4.
        Может ли сумма всех написанных на доске чисел быть меньше 75?

%22.9.3
        \item  Дан квадратный трёхчлен $P(x)$, не обязательно с целыми коэффициентами.
        Известно, что при некоторых целых $a$ и $b$ разность $P(a) - P(b)$ является квадратом натурального числа.
        Докажите, что существует более миллиона таких пар целых чисел $(c, d)$, что разность $P(c) - P(d)$ также является квадратом натурального числа.

%22.9.6
        \item Последовательность чисел $a_1, a_2, \dotsc , a_{2022}$ такова, что $a_n - a_k \geqslant n^3 - k^3$ для любых $n$ и $k$ таких, что $1 \leqslant n \leqslant 2022$ и $1 \leqslant k \leqslant 2022$.
        При этом $a_{1011} = 0$.
        Какие значения может принимать $a_{2022}$?

%21.9.2
        \item  Ненулевые числа $x$ и $y$ удовлетворяют неравенствам $x^2 - x > y^2$ и $y^2 - y > x^2$.
        Какой знак может иметь произведение $xy$?

%21.9.10
        \item Витя записал в тетрадь $n$ различных натуральных чисел.
        Для каждой пары чисел из тетради он выписал на доску их наименьшее общее кратное.
        Могло ли при каком-то $n > 100$ случиться так, что \[\frac{n(n - 1)}{2}\] чисел на доске являются (в некотором порядке) последовательными членами непостоянной арифметической прогрессии?

%20.9.3
        \item На доске написаны $n$ различных целых чисел, любые два из них отличаются хотя бы на 10.
        Сумма квадратов трёх наибольших из них меньше трёх миллионов.
        Сумма квадратов трёх наименьших из них также меньше трёх миллионов.
        При каком наибольшем $n$ это возможно?

%20.9.6
        \item Петя и Миша стартуют по круговой дорожке из одной точки в направлении против часовой стрелки.
        Оба бегут с постоянными скоростями, скорость Миши на 2\% больше скорости Пети.
        Петя всё время бежит против часовой стрелки, а Миша может менять направление бега в любой момент, непосредственно перед которым он пробежал полкруга или больше в одном направлении.
        Покажите, что пока Петя бежит первый круг, Миша может трижды, не считая момента старта, поравняться (встретиться или догнать) с ним.

        \item Докажите, что для любых положительных $x_1, x_2, \dotsc , x_9$
        \[\frac{x_1 - x_3}{x_{1}x_3 + 2x_{2}x_3 + x_2^2} + \frac{x_2 - x_4}{x_{2}x_3 + 2x_{3}x_4 + x_3^2} + \dotsc + \frac{x_8 - x_1}{x_{8}x_1 + 2x_{9}x_1 + x_9^2} + \frac{x_9 - x_2}{x_{9}x_2 + 2x_{1}x_2 + x_1^2} \geqslant 0\]

    \end{enumerate_boxed}
\end{document}