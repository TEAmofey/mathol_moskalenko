\documentclass{article}
\usepackage[12pt]{extsizes}
\usepackage[T2A]{fontenc}
\usepackage[utf8]{inputenc}
\usepackage[english, russian]{babel}

\usepackage{amssymb}
\usepackage{amsfonts}
\usepackage{amsmath}
\usepackage{enumitem}
\usepackage{graphics}
\usepackage{graphicx}

\usepackage{lipsum}

\newtheorem{theorem}{Теорема}
\newtheorem{task}{Задача}
\newtheorem{lemma}{Лемма}
\newtheorem{definition}{Определение}
\newtheorem{example}{Пример}
\newtheorem{statement}{Утверждение}
\newtheorem{corollary}{Следствие}


\usepackage{geometry} % Меняем поля страницы
\geometry{left=1cm}% левое поле
\geometry{right=1cm}% правое поле
\geometry{top=1.5cm}% верхнее поле
\geometry{bottom=1cm}% нижнее поле


\usepackage{fancyhdr} % Headers and footers
\pagestyle{fancy} % All pages have headers and footers
\fancyhead{} % Blank out the default header
\fancyfoot{} % Blank out the default footer
\fancyhead[L]{Математика}
\fancyhead[C]{\textit{Методы}}
\fancyhead[R]{26 марта 2023}% Custom header text


%----------------------------------------------------------------------------------------

%\begin{document}\normalsize
\begin{document}\large
	
\begin{center}
	\textbf{Теорема Хелли}
\end{center}

\begin{enumerate}[label*=\protect\fbox{\arabic{enumi}}]
	
\item На плоскости дано произвольное множество точек. Любые три из них можно накрыть кругом радиуса 1. Тогда 1 и все множество можно накрыть кругом радиуса 1.

\item Для произвольного выпуклого семиугольника все выпуклые пятиугольники с вершинами в вершинах семиугольника имеют общую точку.

\item Дана система из 100 линейных неравенств. Если любые три из них имеют общее решение, то и вся система имеет решение.

\item В выпуклом многоугольнике $M$ для любых трёх сторон существует точка $X$ внутри $M$ такая, что основания высот из $X$ на выбранные три стороны лежат на этих сторонах. Докажите, что внутри $M$ существует точка $Y$, что из неё все перпендикуляры падают на стороны $M$.

\item Докажите, что внутри любого выпуклого семиугольника есть точка, не принадлежащая ни одному из четырехугольников, образованных четверками его соседних вершин.

\item На плоскости даны несколько параллельных отрезков. Известно, что для любых трех отрезков найдется прямая, их пересекающая. Тогда существует прямая, пересекающая все эти отрезки.

\item На координатной плоскости дано несколько вертикаль- ных отрезков. Если для любых трех отрезков существует парабола y = $x^2 + px + q$, которая их пересекает, то найдется такая парабола, пересекающая сразу все отрезки.

\item На плоскости дано конечное семейство прямых. Известно, что любые три прямые можно пересечь кругом радиуса $r$. Тогда все прямые семейства можно пересечь кругом радиуса $r$.

\item На плоскости лежат несколько прямоугольников со сторонами, параллельными осям координат (не обязательно одинаковых), каждые два из которых пересекаются. Тогда все прямоугольники имеют общую точку.

\item Внутри ограниченной выпуклой фигуры всегда найдется точка, обладающая следующим свойством: любая прямая, проходящая через эту точку, делит площадь фигуры на части, отношение которых не превосходит 2.

\end{enumerate}
\end{document}