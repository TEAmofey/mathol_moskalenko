\documentclass{article}
\usepackage[12pt]{extsizes}
\usepackage[T2A]{fontenc}
\usepackage[utf8]{inputenc}
\usepackage[english, russian]{babel}

\usepackage{amssymb}
\usepackage{amsfonts}
\usepackage{amsmath}
\usepackage{enumitem}
\usepackage{graphics}

\usepackage{lipsum}



\usepackage{geometry} % Меняем поля страницы
\geometry{left=1cm}% левое поле
\geometry{right=1cm}% правое поле
\geometry{top=1.5cm}% верхнее поле
\geometry{bottom=1cm}% нижнее поле


\usepackage{fancyhdr} % Headers and footers
\pagestyle{fancy} % All pages have headers and footers
\fancyhead{} % Blank out the default header
\fancyfoot{} % Blank out the default footer
\fancyhead[L]{ЦРОД $\bullet$ Математика}
\fancyhead[C]{\textit{Геометрия}}
\fancyhead[R]{Май 2022}% Custom header text


%----------------------------------------------------------------------------------------

%\begin{document}\normalsize
\begin{document}\large


\begin{center}
\textbf{Треугольники, высоты, окружности}
\end{center}

Рассмотрим треугольник $ABC$. $AH_a, BH_b, CH_c$~--- высоты этого треугольника. $H$~--- ортоцентр (точка пересечения высот). $O$~--- центр описанной окружности. $M_a, M_b, M_c$~--- середины сторон $BC, AC, AB$ соответственно.

Дальше в задачах мы будем ссылаться на эти обозначения


\begin{enumerate}[label*=\protect\fbox{\arabic{enumi}}]

\item Докажите, что $\angle ABH = \angle CBO$.

\item Докажите, что $\angle ABH = \angle H_cH_aH$.

\item Докажите, что $H_aA$~--- биссектриса $ \angle H_cH_aH_b$.

\item Докажите, что $H$~--- центр вписанной окружности треугольника $H_aH_bH_c$.

\item Докажите, что $O$~--- ортоцентр треугольника $M_aM_bM_c$.


\item Точку $O$ отразили относительно сторон треугольника $ABC$ и получили точки $A'$, $B'$ и $C'$. Докажите, что треугольник $A'B'C'$ равен исходному, причём точка $O$ для него является ортоцентром.

\item Докажите, что  $AH = 2OM_a$

\item Докажите, что отражение  $H$ относительно стороны $BC$ лежит на описанной окружности треугольника $ABC$.

\item Докажите, что отражение  $H$ относительно точки $M_a$ лежит на описанной окружности треугольника $ABC$.

\item Докажите, что точка из предыдущей задачи диаметрально противоположна точке $A$

\item Докажите, что четырёхугольник  $M_aM_bM_cH_a$~--- равнобедренная трапеция.

\item Докажите, что четырёхугольник  $H_bM_bM_cH_c$~--- вписан.

\item Докажите что 6 точек  $M_a, M_b, M_c, H_a, H_b, H_c$ лежат на одной окружности.

\item Описанная окружность треугольника $BHC$ пересекает отрезки $AB$ и $AC$ в точках $Y$ и $X$ соответственно. Докажите, что $XY$=$2H_bH_c$

\item Пусть $O_1$ и $O_2$ — центры описанных окружностей треугольников $H_bHH_c$ и $BHC$ соответственно. Докажите, что $O_1O_2 \parallel AM_a$.

\item Дана равнобокая трапеция $ABCD$ с основаниями $BC$ и $AD$.
Окружность $\omega$ проходит через вершины $B$ и $C$ и вторично пересекает сторону $AB$ и диагональ $BD$ в точках $X$ и $Y$ соответственно. Касательная, проведенная к окружности $\omega$ в точке $C$,
пересекает луч $AD$ в точке $Z$. Докажите, что точки $X, Y$ и $Z$
лежат на одной прямой. 


\end{enumerate}
\end{document}