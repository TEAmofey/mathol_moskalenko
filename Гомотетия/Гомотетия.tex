\documentclass{article}

\usepackage[12pt]{extsizes}
\usepackage[T2A]{fontenc}
\usepackage[utf8]{inputenc}
\usepackage[english, russian]{babel}

\usepackage{mathrsfs}
\usepackage[dvipsnames]{xcolor}

\usepackage{amsmath}
\usepackage{amssymb}
\usepackage{amsthm}
\usepackage{indentfirst}
\usepackage{amsfonts}
\usepackage{enumitem}
\usepackage{graphics}
\usepackage{tikz}
\usepackage{tabu}
\usepackage{diagbox}
\usepackage{hyperref}
\usepackage{mathtools}
\usepackage{ucs}
\usepackage{lipsum}
\usepackage{geometry} % Меняем поля страницы
\usepackage{fancyhdr} % Headers and footers
\newcommand{\range}{\mathrm{range}}
\newcommand{\dom}{\mathrm{dom}}
\newcommand{\N}{\mathbb{N}}
\newcommand{\R}{\mathbb{R}}
\newcommand{\E}{\mathbb{E}}
\newcommand{\D}{\mathbb{D}}
\newcommand{\M}{\mathcal{M}}
\newcommand{\Prime}{\mathbb{P}}
\newcommand{\A}{\mathbb{A}}
\newcommand{\Q}{\mathbb{Q}}
\newcommand{\Z}{\mathbb{Z}}
\newcommand{\F}{\mathbb{F}}
\newcommand{\CC}{\mathbb{C}}

\DeclarePairedDelimiter\abs{\lvert}{\rvert}
\DeclarePairedDelimiter\floor{\lfloor}{\rfloor}
\DeclarePairedDelimiter\ceil{\lceil}{\rceil}
\DeclarePairedDelimiter\lr{(}{)}
\DeclarePairedDelimiter\set{\{}{\}}
\DeclarePairedDelimiter\norm{\|}{\|}

\renewcommand{\labelenumi}{(\alph{enumi})}

\newcommand{\smallindent}{
    \geometry{left=1cm}% левое поле
    \geometry{right=1cm}% правое поле
    \geometry{top=1.5cm}% верхнее поле
    \geometry{bottom=1cm}% нижнее поле
}

\newcommand{\header}[3]{
    \pagestyle{fancy} % All pages have headers and footers
    \fancyhead{} % Blank out the default header
    \fancyfoot{} % Blank out the default footer
    \fancyhead[L]{#1}
    \fancyhead[C]{#2}
    \fancyhead[R]{#3}
}

\newcommand{\dividedinto}{
    \,\,\,\vdots\,\,\,
}

\newcommand{\littletaller}{\mathchoice{\vphantom{\big|}}{}{}{}}

\newcommand\restr[2]{{
    \left.\kern-\nulldelimiterspace % automatically resize the bar with \right
    #1 % the function
    \littletaller % pretend it's a little taller at normal size
    \right|_{#2} % this is the delimiter
}}

\DeclareGraphicsExtensions{.pdf,.png,.jpg}

\newenvironment{enumerate_boxed}[1][enumi]{\begin{enumerate}[label*=\protect\fbox{\arabic{#1}}]}{\end{enumerate}}



\smallindent

\header{Математика}{\textit{Геометрия}}{23 февраля 2023}


%----------------------------------------------------------------------------------------

%\begin{document}\normalsize
\begin{document}
    \large

    \begin{center}
        \textbf{Гомотетия}
    \end{center}

    \textbf{Определение 1} Пусть заданы точка $O$ и ненулевое число $k$.
    Тогда гомотетия с центром $O$ и коэффициентом $k$ — это преобразование плоскости, переводящее каждую точку плоскости $A$ в $A'$ так, что $OA' = k\cdot OA$ и $O, A, A'$ лежат на одной прямой.
    Гомотетию с такими параметрами обозначают $H^k_O$.

    \textbf{Основные свойства гомотетии}:
    \begin{itemize}
        \item каждая фигура переходит в фигуру, подобную изначальной, и коэффициент подобия равен $|k|$;

        \item прямая переходит в прямую, параллельную исходной прямой;

        \item окружность переходит в окружность, и радиус увеличивается в $|k|$ раз.
    \end{itemize}

    \begin{enumerate_boxed}

        \item Две окружности радиусов $14$ и $35$ касаются внутренним образом в точке $A$.
        Через точку $A$ проведена прямая, пересекающая меньшую окружность в точке $B$, а большую — в точке $C$.
        Найдите длину отрезка $BC$, если $AB=12$.

        \item Между двумя параллельными прямыми расположили окружность радиуса $12$, касающуюся обеих прямых, и равнобедренный треугольник, основание которого лежит на одной прямой, а вершина — на другой.
        Известно, что треугольник и окружность имеют ровно одну общую точку, и что эта точка лежит на вписанной окружности треугольника.
        Найдите радиус вписанной окружности треугольника.

        \item Внутри угла расположены три окружности $S_1, S_2, S_3$, каждая из которых касается двух сторон угла, причем окружность $S_2$ касается внешним образом окружностей $S_1$ и $S_3$.
        Известно, что радиус окружности $S_1$ равен $1$, а радиус окружности $S_3$ равен $9$.
        Чему равен радиус окружности $S_2$?

        \item Внутри квадрата $ABCD$ взята точка $M$.
        Докажите, что точки пересечения медиан треугольников  $ABM, BCM, CDM$ и $DAM$ образуют квадрат.

        \item Докажите, что точки, симметричные произвольной точке относительно середин сторон квадрата, являются вершинами некоторого квадрата.

        \item Докажите \textit{лемму Архимеда}

        \textit{Лемма Архимеда}: Пусть в окружности $\omega$ проведена хорда $AB$, и ещё одна окружность касается $\omega$ в точке $C$ и отрезка $AB$ в точке $D$.
        Тогда прямая $CD$ проходит через середину дуги $AB$, не содержащей точки $C$.

        \item На каждом из оснований $AD$ и $BC$ трапеции $ABCD$ построены вне трапеции равносторонние треугольники.
        Докажите, что отрезок, соединяющий третьи вершины этих треугольников, проходит через точку пересечения диагоналей трапеции.

        \item Две окружности касаются внутренним образом в точке A. Секущая пересекает окружности в точках $M, N, P$ и $Q$ (точки расположены на секущей в указанном порядке).
        Докажите, что $\angle MAP=\angle NAQ$.

        \item На плоскости проведены параллельные прямые $l_1$ и $l_2$.
        Окружности $\omega_1$ и $\omega_2$ лежат между этими прямыми.
        Окружность $\omega_1$ касается $l_1$ в точке $A$, окружность $\omega_2$ касается $l_2$ в точке B, окружности $\omega_1$ и $\omega_2$ касаются друг друга в точке $C$.
        Докажите, что точки $A, B, C$ лежат на одной прямой.

        \item В окружности $\omega$ проведена хорда $AB$.
        Найдите геометрической место точек пересечения медиан треугольников $ABC$, где $C\in \omega$.

        \item Внутри треугольника $ABC$ выбрана точка $X$.
        Докажите, что прямые, проходящие через середины сторон $AB, AC, BC$ параллельно прямым $CX, BX, AX$ соответственно, пересекаются в одной точке.

        \item Пусть $M$ и $P$ — точки касания вписанной и вневписанной окружностей треугольника $ABC$ со стороной $BC, MN$ — диаметр вписанной окружности.
        Докажите, что точки $A, N$ и $P$ лежат на одной прямой.

        \item На сторонах $AB$ и $AC$ треугольника $ABC$ нашлись точки $M$ и $N$ такие, что $MC=AC$ и $NB=AB$.
        Точка $P$ симметрична точке $A$ относительно прямой $BC$.
        Докажите, что $PA$ — биссектриса угла $MPN$.

        \item Дана трапеция $ABCD (BC \parallel AD$ и $AD>BC)$, в которой на основаниях выбраны точки $K$ и $L$ так, что прямые $AB, CD$ и $KL$ пересекаются в одной точке.
        На отрезке $KL$ выбраны такие точки $P$ и $Q$, что $\angle AQD=\angle ABC$ и $\angle BPC=\angle BAD$.
        Докажите, что четырёхугольник $ABPQ$ вписанный.

        \item Вписанная окружность треугольника $ABC$ касается сторон $AB, AC, BC$ в точках $C_1, B_1, A_1$ соответственно.
        Точки $A_2, B_2, C_2$ — середины дуг $BC, AC, AB$ описанной окружности треугольника $ABC$.
        Докажите, что прямые $A_{1}A_2, B_{1}B_2, C_{1}C_2$ пересекаются в одной точке.

        \item Середины сторон выпуклого шестиугольника образуют шестиугольник, противоположные стороны которого параллельны.
        Докажите, что большие диагонали исходного шестиугольника пересекаются в одной точке.

        \item Высоты остроугольного треугольника $ABC$ пересекаются в точке $H$.
        На отрезках $BH$ и $CH$ отмечены точки $B_1$ и $C_1$ соответственно, так что $BC \parallel B_{1}C_1$.
        Оказалось, что центр окружности $\omega$, описанной около треугольника $B_{1}HC_1$ лежит на прямой $BC$.
        Докажите, что окружность $\varGamma$, описанная около треугольника $ABC$, касается $\omega$.

    \end{enumerate_boxed}
\end{document}