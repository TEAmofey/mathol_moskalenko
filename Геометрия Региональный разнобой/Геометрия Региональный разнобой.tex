\documentclass{article}

\usepackage[12pt]{extsizes}
\usepackage[T2A]{fontenc}
\usepackage[utf8]{inputenc}
\usepackage[english, russian]{babel}

\usepackage{mathrsfs}
\usepackage[dvipsnames]{xcolor}

\usepackage{amsmath}
\usepackage{amssymb}
\usepackage{amsthm}
\usepackage{indentfirst}
\usepackage{amsfonts}
\usepackage{enumitem}
\usepackage{graphics}
\usepackage{tikz}
\usepackage{tabu}
\usepackage{diagbox}
\usepackage{hyperref}
\usepackage{mathtools}
\usepackage{ucs}
\usepackage{lipsum}
\usepackage{geometry} % Меняем поля страницы
\usepackage{fancyhdr} % Headers and footers
\newcommand{\range}{\mathrm{range}}
\newcommand{\dom}{\mathrm{dom}}
\newcommand{\N}{\mathbb{N}}
\newcommand{\R}{\mathbb{R}}
\newcommand{\E}{\mathbb{E}}
\newcommand{\D}{\mathbb{D}}
\newcommand{\M}{\mathcal{M}}
\newcommand{\Prime}{\mathbb{P}}
\newcommand{\A}{\mathbb{A}}
\newcommand{\Q}{\mathbb{Q}}
\newcommand{\Z}{\mathbb{Z}}
\newcommand{\F}{\mathbb{F}}
\newcommand{\CC}{\mathbb{C}}

\DeclarePairedDelimiter\abs{\lvert}{\rvert}
\DeclarePairedDelimiter\floor{\lfloor}{\rfloor}
\DeclarePairedDelimiter\ceil{\lceil}{\rceil}
\DeclarePairedDelimiter\lr{(}{)}
\DeclarePairedDelimiter\set{\{}{\}}
\DeclarePairedDelimiter\norm{\|}{\|}

\renewcommand{\labelenumi}{(\alph{enumi})}

\newcommand{\smallindent}{
    \geometry{left=1cm}% левое поле
    \geometry{right=1cm}% правое поле
    \geometry{top=1.5cm}% верхнее поле
    \geometry{bottom=1cm}% нижнее поле
}

\newcommand{\header}[3]{
    \pagestyle{fancy} % All pages have headers and footers
    \fancyhead{} % Blank out the default header
    \fancyfoot{} % Blank out the default footer
    \fancyhead[L]{#1}
    \fancyhead[C]{#2}
    \fancyhead[R]{#3}
}

\newcommand{\dividedinto}{
    \,\,\,\vdots\,\,\,
}

\newcommand{\littletaller}{\mathchoice{\vphantom{\big|}}{}{}{}}

\newcommand\restr[2]{{
    \left.\kern-\nulldelimiterspace % automatically resize the bar with \right
    #1 % the function
    \littletaller % pretend it's a little taller at normal size
    \right|_{#2} % this is the delimiter
}}

\DeclareGraphicsExtensions{.pdf,.png,.jpg}

\newenvironment{enumerate_boxed}[1][enumi]{\begin{enumerate}[label*=\protect\fbox{\arabic{#1}}]}{\end{enumerate}}



\smallindent

\header{Математика}{\textit{Геометрия}}{}

%----------------------------------------------------------------------------------------

%\begin{document}\normalsize
\begin{document}
    \large

    \begin{center}
        \textbf{Региональный разнобой}
    \end{center}


    \begin{enumerate_boxed}

%23.9.2
        \item Дан бумажный треугольник, длины сторон которого равны 5 см,
        12 см и 13 см.
        Можно ли разрезать его на несколько (больше одного) многоугольников, у каждого из которых площадь (измеренная в см$^2$) численно равна периметру (измеренному в см)?

%23.9.5
        \item Четырёхугольник $ABCD$ вписан в окружность $\gamma$.
        Оказалось, что окружности, построенные на отрезках $AB$ и $CD$ как на диаметрах, касаются друг друга внешним образом в точке $S$.
        Пусть точки $M$ и $N$ — середины отрезков $AB$ и $CD$ соответственно.
        Докажите, что перпендикуляр $\ell$ к прямой $MN$, восставленный в точке $M$, пересекает прямую $CS$ в точке, лежащей на $\gamma$.

%23.9.8
        \item Дан остроугольный треугольник $ABC$, в котором $AB < BC$.
        Пусть $M$ и $N$ — середины сторон $AB$ и $AC$ соответственно, а $H$ — основание высоты, опущенной из вершины $B$.
        Вписанная окружность касается стороны $AC$ в точке $K$.
        Прямая, проходящая через $K$ и параллельная $MH$, пересекает отрезок $MN$ в точке $P$.
        Докажите, что в четырехугольник $AMPK$ можно вписать окружность.

%22.9.5
        \item  Пусть $CE$ — биссектриса в остроугольном треугольнике $ABC$.
        На внешней биссектрисе угла $ACB$ отмечена точка $D$, а на стороне $BC$ — точка $F$, причём $\angle BAD = 90^\circ = \angle DEF$.
        Докажите, что центр окружности, описанной около треугольника $CEF$, лежит на прямой $BD$.

%22.9.8
        \item  В трапеции $ABCD$ диагональ $BD$ равна основанию $AD$.
        Диагонали $AC$ и $BD$ пересекаются в точке $E$.
        Точка $F$ на отрезке $AD$ выбрана так, что $EF \parallel CD$.
        Докажите, что $BE = DF$.

%22.9.9
        \item На плоскости отмечены $N$ точек.
        Любые три из них образуют треугольник, величины углов которого в градусах выражаются натуральными числами.
        При каком наибольшем $N$ это возможно?

%22.9.4
        \item Окружности $\Omega$ и $\omega$ касаются друг друга внутренним образом в точке $A$.
        Проведем в большей окружности $\Omega$ хорду $CD$, касающуюся $\omega$ в точке $B$ (хорда $AB$ не является диаметром $\omega$). Точка $M$ — середина отрезка $AB$.
        Докажите, что окружность, описанная около треугольника $CMD$, проходит через центр $\omega$.

%22.9.8
        \item Дана трапеция $ABCD$ с основаниями $AD$ и $BC$.
        Оказалось, что точка пересечения медиан треугольника $ABD$ лежит на биссектрисе угла $BCD$.
        Докажите, что точка пересечения медиан треугольника $ABC$ лежит на биссектрисе угла $ADC$.

%20.9.5
        \item Четырёхугольник $ABCD$ описан около окружности $\omega$.
        Докажите, что диаметр окружности $\omega$ не превосходит длины отрезка, соединяющего середины сторон $BC$ и $AD$.

%20.9.8
        \item В остроугольном треугольнике $ABC$ проведена биссектриса $BL$.
        Окружность, описанная около треугольника $ABL$, пересекает сторону $BC$ в точке $D$.
        Оказалось, что точка $S$, симметричная точке $C$ относительно прямой $DL$, лежит на стороне $AB$ и не совпадает с её концами.
        Какие значения может принимать $\angle ABC$?

    \end{enumerate_boxed}
\end{document}