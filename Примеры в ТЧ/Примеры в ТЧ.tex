\documentclass{article}

\usepackage[12pt]{extsizes}
\usepackage[T2A]{fontenc}
\usepackage[utf8]{inputenc}
\usepackage[english, russian]{babel}

\usepackage{mathrsfs}
\usepackage[dvipsnames]{xcolor}

\usepackage{amsmath}
\usepackage{amssymb}
\usepackage{amsthm}
\usepackage{indentfirst}
\usepackage{amsfonts}
\usepackage{enumitem}
\usepackage{graphics}
\usepackage{tikz}
\usepackage{tabu}
\usepackage{diagbox}
\usepackage{hyperref}
\usepackage{mathtools}
\usepackage{ucs}
\usepackage{lipsum}
\usepackage{geometry} % Меняем поля страницы
\usepackage{fancyhdr} % Headers and footers
\newcommand{\range}{\mathrm{range}}
\newcommand{\dom}{\mathrm{dom}}
\newcommand{\N}{\mathbb{N}}
\newcommand{\R}{\mathbb{R}}
\newcommand{\E}{\mathbb{E}}
\newcommand{\D}{\mathbb{D}}
\newcommand{\M}{\mathcal{M}}
\newcommand{\Prime}{\mathbb{P}}
\newcommand{\A}{\mathbb{A}}
\newcommand{\Q}{\mathbb{Q}}
\newcommand{\Z}{\mathbb{Z}}
\newcommand{\F}{\mathbb{F}}
\newcommand{\CC}{\mathbb{C}}

\DeclarePairedDelimiter\abs{\lvert}{\rvert}
\DeclarePairedDelimiter\floor{\lfloor}{\rfloor}
\DeclarePairedDelimiter\ceil{\lceil}{\rceil}
\DeclarePairedDelimiter\lr{(}{)}
\DeclarePairedDelimiter\set{\{}{\}}
\DeclarePairedDelimiter\norm{\|}{\|}

\renewcommand{\labelenumi}{(\alph{enumi})}

\newcommand{\smallindent}{
    \geometry{left=1cm}% левое поле
    \geometry{right=1cm}% правое поле
    \geometry{top=1.5cm}% верхнее поле
    \geometry{bottom=1cm}% нижнее поле
}

\newcommand{\header}[3]{
    \pagestyle{fancy} % All pages have headers and footers
    \fancyhead{} % Blank out the default header
    \fancyfoot{} % Blank out the default footer
    \fancyhead[L]{#1}
    \fancyhead[C]{#2}
    \fancyhead[R]{#3}
}

\newcommand{\dividedinto}{
    \,\,\,\vdots\,\,\,
}

\newcommand{\littletaller}{\mathchoice{\vphantom{\big|}}{}{}{}}

\newcommand\restr[2]{{
    \left.\kern-\nulldelimiterspace % automatically resize the bar with \right
    #1 % the function
    \littletaller % pretend it's a little taller at normal size
    \right|_{#2} % this is the delimiter
}}

\DeclareGraphicsExtensions{.pdf,.png,.jpg}

\newenvironment{enumerate_boxed}[1][enumi]{\begin{enumerate}[label*=\protect\fbox{\arabic{#1}}]}{\end{enumerate}}



\smallindent

\header{Математика}{\textit{Теория чисел}}{27 февраля 2024}

%----------------------------------------------------------------------------------------

\begin{document}
    \large

    \begin{center}
        \textbf{Примеры в ТЧ}
    \end{center}


    \begin{enumerate_boxed}

%23.9.6
        \item Для натурального числа $n$ обозначим через $S_n$ наименьшее общее кратное всех чисел $1, 2, \dotsc , n$.
        Существует ли такое натуральное число $m$, что $S_{m+1} = 4S_m$?

        \item Докажите, что найдется такое натуральное число $n > 1$, что произведение некоторых $n$ последовательных натуральных чисел равно произведению некоторых $n+100$ последовательных натуральных чисел.

        \item Существует ли арифметическая прогрессия $a_1, a_2, \dotsc, a_{1000}$ с ненулевой разностью такая, что каждый её член имеет вид $\frac{1}{n}$, для некоторого натурального $n$?

        \item Существуют ли три попарно различных ненулевых целых числа, сумма которых равна нулю, а сумма тринадцатых степеней которых является квадратом некоторого натурального числа?

        \item Докажите, что существует бесконечно много таких троек натуральных чисел $a, b, c$, что \[a^{21} + b^{23} = c^{22}\]

        \item Существуют ли такие натуральные $a, b, c > 10^5$, что $a^2 - 1$ делится на $b$, $b^2 - 1$ делится на $c$, $c^2 - 1$ делится на $a$?

        \item Существует ли возрастающая арифметическая прогрессия длины 2023, все члены которой являются точными степенями, выше десятой?


    \end{enumerate_boxed}
\end{document}