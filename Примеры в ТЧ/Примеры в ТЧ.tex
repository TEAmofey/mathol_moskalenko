\documentclass{article}
\usepackage[12pt]{extsizes}
\usepackage[T2A]{fontenc}
\usepackage[utf8]{inputenc}
\usepackage[english, russian]{babel}

\usepackage{amssymb}
\usepackage{amsfonts}
\usepackage{amsmath}
\usepackage{enumitem}
\usepackage{graphics}
\usepackage{graphicx}

\usepackage{lipsum}

\newtheorem{theorem}{Теорема}
\newtheorem{task}{Задача}
\newtheorem{lemma}{Лемма}
\newtheorem{definition}{Определение}
\newtheorem{example}{Пример}
\newtheorem{statement}{Утверждение}
\newtheorem{corollary}{Следствие}


\usepackage{geometry} % Меняем поля страницы
\geometry{left=1cm}% левое поле
\geometry{right=1cm}% правое поле
\geometry{top=1.5cm}% верхнее поле
\geometry{bottom=1cm}% нижнее поле


\usepackage{fancyhdr} % Headers and footers
\pagestyle{fancy} % All pages have headers and footers
\fancyhead{} % Blank out the default header
\fancyfoot{} % Blank out the default footer
\fancyhead[L]{Математика}
\fancyhead[C]{\textit{Теория чисел}}
\fancyhead[R]{27 февраля 2024}% Custom header text


%----------------------------------------------------------------------------------------

%\begin{document}\normalsize
\begin{document}\large
	
\begin{center}
	\textbf{Примеры в ТЧ}
\end{center}


\begin{enumerate}[label*=\protect\fbox{\arabic{enumi}}]
	
%23.9.6
\item Для натурального числа $n$ обозначим через $S_n$ наименьшее общее кратное всех чисел $1, 2, \dotsc , n$. Существует ли такое натуральное число $m$, что $S_{m+1} = 4S_m$? 

\item Докажите, что найдется такое натуральное число $n > 1$, что произведение некоторых $n$ последовательных натуральных чисел равно произведению некоторых $n+100$ последовательных натуральных чисел.

\item Существует ли арифметическая прогрессия $a_1, a_2, \dotsc, a_{1000}$ с ненулевой разностью такая, что каждый её член имеет вид $\frac{1}{n}$, для некоторого натурального $n$?

\item Существуют ли три попарно различных ненулевых целых числа, сумма которых равна нулю, а сумма тринадцатых степеней которых является квадратом некоторого натурального числа?

\item Докажите, что существует бесконечно много таких троек натуральных чисел $a, b, c$, что $$a^{21} + b^{23} = c^{22}$$

\item Существуют ли такие натуральные $a, b, c > 10^5$, что $a^2 - 1$ делится на $b$, $b^2 - 1$ делится на $c$, $c^2 - 1$ делится на $a$,?

\item Существует ли возрастающая арифметическая прогрессия длины 2023, все члены которой являются точными степенями, выше десятой?


\end{enumerate}
\end{document}