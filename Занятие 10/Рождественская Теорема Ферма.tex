\documentclass{article}
\usepackage[12pt]{extsizes}
\usepackage[T2A]{fontenc}
\usepackage[utf8]{inputenc}
\usepackage[english, russian]{babel}

\usepackage{amssymb}
\usepackage{amsfonts}
\usepackage{amsmath}
\usepackage{enumitem}
\usepackage{graphics}
\usepackage{graphicx}

\usepackage{lipsum}
\DeclareGraphicsExtensions{.pdf,.png,.jpg}



\usepackage{geometry} % Меняем поля страницы
\geometry{left=1cm}% левое поле
\geometry{right=1cm}% правое поле
\geometry{top=1.5cm}% верхнее поле
\geometry{bottom=1cm}% нижнее поле


\usepackage{fancyhdr} % Headers and footers
\pagestyle{fancy} % All pages have headers and footers
\fancyhead{} % Blank out the default header
\fancyfoot{} % Blank out the default footer
\fancyhead[L]{Математика}
\fancyhead[C]{\textit{Алгебра}}
\fancyhead[R]{12 декабря 2023}% Custom header text

%----------------------------------------------------------------------------------------

%\begin{document}\normalsize
\begin{document}\large
	

\begin{center}
\textbf{Рождественская теорема Ферма}
\end{center}

\textbf{Рождественская теорема Ферма.} Натуральное число представимо в виде суммы двух квадратов целых чисел тогда и только тогда, когда в его разложение на простые множители любое простое число вида $4k + 3$ входит в четной степени.

\begin{enumerate}[label*=\protect\fbox{\arabic{enumi}}]
	
	\item Два числа представляются в виде суммы двух квадратов. Докажите, что их произведение представляется в виде суммы двух квадратов.
	
	\item Докажите, что для любого простого $p = 4k + 1$ существует $x$ такое, что $x^2 + 1$ делится на $p$
	
	\item Докажите, что $p = 4k + 1$ представляется в виде суммы двух квадратов.
	
	\textbf{Доказательство Акселя Туэ:}
	\begin{enumerate}
		\item Для любого целого s существуют две различные пары $(x,y)$ и $(x',y')$ чисел из множества $\{0,1,\dotsc [\sqrt{p}]\}$, такие что $x-sy \equiv x'-sy' \pmod p$. 
		\item Для любого $s$ существует ненулевая пара $(x, y)$ чисел из множества $\{0,1,...[\sqrt{p}]\}$, такая что $x \equiv \pm sy \pmod p$. 
		\item Примените пункт (b) для $s$ такого, что что $s^2 \equiv -1 \pmod p$, и закончите доказательство.
	\end{enumerate}
	
	\item Докажите, что если $x^2+y^2$ делится на $p=4k+3$,то $x$ и $y$ делятся на $p$.
	\begin{enumerate}
		\item Рассмотрим пары обратных остатков для чисел $2, 3, \dotsc , p - 2$. Назовем пару $(x, y)$ далекой, если $x < \frac{p}{2} < y$. Докажите, что количество далеких пар четно.
		\item Используя пункт (a) докажите, что сравнение $x^2 \equiv -1 \pmod p$ не имеет решений. Завершите доказательство.
	\end{enumerate}
	
	\item Докажите \textbf{рождественскую теорему Ферма}
	
	\item  Докажите, что $p = 4k + 1$ представляется в виде суммы двух квадратов единственным
	способом.
	
	\item Докажите, что уравнение $x^2 + y^2 = z^5 + z$ имеет бесконечно много целых решений, в
	которых $x, y$ и $z$ попарно взаимно просты.
	\item Пусть $n$ — нечетное целое число, большее 1. Докажите, что уравнение $x^n + 2^{n-1} = y^2$
	не разрешимо в нечетных натуральных числах.
	\item Докажите, что уравнение $x^3 - x^2 + 8 = y^2$ не имеет решений в целых числах.

\end{enumerate}
\end{document}