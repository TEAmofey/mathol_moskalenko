\documentclass{article}

\usepackage[12pt]{extsizes}
\usepackage[T2A]{fontenc}
\usepackage[utf8]{inputenc}
\usepackage[english, russian]{babel}

\usepackage{mathrsfs}
\usepackage[dvipsnames]{xcolor}

\usepackage{amsmath}
\usepackage{amssymb}
\usepackage{amsthm}
\usepackage{indentfirst}
\usepackage{amsfonts}
\usepackage{enumitem}
\usepackage{graphics}
\usepackage{tikz}
\usepackage{tabu}
\usepackage{diagbox}
\usepackage{hyperref}
\usepackage{mathtools}
\usepackage{ucs}
\usepackage{lipsum}
\usepackage{geometry} % Меняем поля страницы
\usepackage{fancyhdr} % Headers and footers
\newcommand{\range}{\mathrm{range}}
\newcommand{\dom}{\mathrm{dom}}
\newcommand{\N}{\mathbb{N}}
\newcommand{\R}{\mathbb{R}}
\newcommand{\E}{\mathbb{E}}
\newcommand{\D}{\mathbb{D}}
\newcommand{\M}{\mathcal{M}}
\newcommand{\Prime}{\mathbb{P}}
\newcommand{\A}{\mathbb{A}}
\newcommand{\Q}{\mathbb{Q}}
\newcommand{\Z}{\mathbb{Z}}
\newcommand{\F}{\mathbb{F}}
\newcommand{\CC}{\mathbb{C}}

\DeclarePairedDelimiter\abs{\lvert}{\rvert}
\DeclarePairedDelimiter\floor{\lfloor}{\rfloor}
\DeclarePairedDelimiter\ceil{\lceil}{\rceil}
\DeclarePairedDelimiter\lr{(}{)}
\DeclarePairedDelimiter\set{\{}{\}}
\DeclarePairedDelimiter\norm{\|}{\|}

\renewcommand{\labelenumi}{(\alph{enumi})}

\newcommand{\smallindent}{
    \geometry{left=1cm}% левое поле
    \geometry{right=1cm}% правое поле
    \geometry{top=1.5cm}% верхнее поле
    \geometry{bottom=1cm}% нижнее поле
}

\newcommand{\header}[3]{
    \pagestyle{fancy} % All pages have headers and footers
    \fancyhead{} % Blank out the default header
    \fancyfoot{} % Blank out the default footer
    \fancyhead[L]{#1}
    \fancyhead[C]{#2}
    \fancyhead[R]{#3}
}

\newcommand{\dividedinto}{
    \,\,\,\vdots\,\,\,
}

\newcommand{\littletaller}{\mathchoice{\vphantom{\big|}}{}{}{}}

\newcommand\restr[2]{{
    \left.\kern-\nulldelimiterspace % automatically resize the bar with \right
    #1 % the function
    \littletaller % pretend it's a little taller at normal size
    \right|_{#2} % this is the delimiter
}}

\DeclareGraphicsExtensions{.pdf,.png,.jpg}

\newenvironment{enumerate_boxed}[1][enumi]{\begin{enumerate}[label*=\protect\fbox{\arabic{#1}}]}{\end{enumerate}}



\header{\textit{\textbf{XIII Олимпиада Эйлера}}}{}{25 декабря 2022}

%----------------------------------------------------------------------------------------

%\begin{document}\normalsize
\begin{document}
    \large

    \begin{center}
        \LARGE\textbf{8 класс}
    \end{center}
    \begin{center}
        \large\textbf{Первый день}
    \end{center}


    \begin{enumerate}[label*=8.{\arabic{enumi}}]

        \item Натуральное число, большее 1000000, даёт одинаковые остатки при делении на 40 и на 125.
        Какая цифра может стоять у этого числа в разряде сотен?

        \item Числа $x$ и $y$, не равные 0, удовлетворяют неравенствам $x^2-x > y^2$ и $y^2-y > x^2$.
        Какой знак может иметь произведение $xy$ (укажите все возможности)?

        \item В группе из 79 школьников у каждого не более 39 знакомых, причем у любого мальчика есть знакомая девочка, а у любой девочки~--- знакомый мальчик.
        Может ли оказаться, что все девочки из этой группы имеют в ней поровну знакомых мальчиков, а все мальчики~--- поровну знакомых девочек?
        Все знакомства~--- взаимные.

        \item Петя и Вася играют в игру.
        Вася кладёт в ряд 150 монет: некоторые «орлом» вверх, некоторые — «решкой».
        Петя своим ходом может показать на любые три лежащие подряд монеты, после чего Вася обязан перевернуть какие-то две монеты из этих трёх по своему выбору.
        Петя хочет, чтобы как можно больше монет лежали «решкой» вверх, а Вася хочет ему помешать.
        При каком наибольшем $k$ Петя сможет независимо от действий Васи добиться того, чтобы хотя бы $k$ монет лежали «решкой» вверх?
        \item $CL$~--- биссектриса треугольника $ABC$. $CLBK$~--- параллелограмм.
        Прямая $AK$ пересекает отрезок $CL$ в точке $P$.
        Оказалось, что точка $P$ равноудалена от диагоналей параллелограмма $CLBK$ Докажите, что $AK \geqslant CL$.


    \end{enumerate}

    \newpage

    \begin{center}
        \LARGE\textbf{8 класс}
    \end{center}
    \begin{center}
        \large\textbf{Второй день}
    \end{center}


    \begin{enumerate}[label*=8.{\arabic{enumi}}]
        \setcounter{enumi}{5}
        \item У уголка из трёх клеток центральной назовём клетку, соседнюю по стороне с двумя другими.
        Существует ли клетчатая фигура, которую можно разбить на уголки из трех клеток тремя способами так, чтобы каждая ее клетка в одном из разбиений была центральной в своем уголке?
        \item Точка $M$~--- середина стороны $AC$ равностороннего треугольника $ABC$.
        Точки $P$ и $R$ на отрезках $AM$ и $BC$ соответственно выбраны так, что $AP = BR$.
        Найдите сумму углов $ARM$, $PBM$ и $BMR$.
        \item Сначала Саша прямолинейными разрезами, каждый из которых соединяет две точки на сторонах квадрата, делит квадрат со стороной 2 на 2020 частей.
        Затем Дима вырезает из каждой части по кругу.
        Докажите, что Дима всегда может добиться того, чтобы сумма радиусов этих кругов была не меньше 1.
        \item Дано натуральное число $n$, большее 2.
        Докажите, что если число $n!+n^3+1$ — простое, то число $n^2+2$ представляется в виде суммы двух простых чисел.
        \item В квадратной таблице $2021 \times 2021$ стоят натуральные числа.
        Можно выбрать любой столбец или любую строку в таблице и выполнить одно из следующих действий:

        1) Прибавить к каждому выбранному числу 1.

        2) Разделить каждое из выбранных чисел на какое-нибудь натуральное число.

        Можно ли за несколько таких действий добиться того, чтобы каждое число в таблице было равно 1?

    \end{enumerate}
\end{document}