\documentclass{article}
\usepackage[12pt]{extsizes}
\usepackage[T2A]{fontenc}
\usepackage[utf8]{inputenc}
\usepackage[english, russian]{babel}

\usepackage{amssymb}
\usepackage{amsfonts}
\usepackage{amsmath}
\usepackage{enumitem}
\usepackage{graphics}
\usepackage{graphicx}

\usepackage{lipsum}
\DeclareGraphicsExtensions{.pdf,.png,.jpg}



\usepackage{geometry} % Меняем поля страницы
\geometry{left=1cm}% левое поле
\geometry{right=1cm}% правое поле
\geometry{top=1.5cm}% верхнее поле
\geometry{bottom=1cm}% нижнее поле


\usepackage{fancyhdr} % Headers and footers
\pagestyle{fancy} % All pages have headers and footers
\fancyhead{} % Blank out the default header
\fancyfoot{} % Blank out the default footer
\fancyhead[L]{Математика}
\fancyhead[C]{\textit{Разное}}
\fancyhead[R]{14 ноября 2023}% Custom header text

%----------------------------------------------------------------------------------------

%\begin{document}\normalsize
\begin{document}\large
	

\begin{center}
\textbf{Теорема Виета}
\end{center}

 \textbf{Теорема Виета:}

\text{Пусть многочлен } $ a_nx^n + a_{n-1}x^{n-1} + \ldots + a_1x + a_0$ имеет корни $x_1, x_2, \ldots, x_n.$ 
\text{Тогда:}
\begin{eqnarray*}
	x_1 + x_2 + \ldots + x_n &=& -\frac{a_{n-1}}{a_n},\\
	x_1x_2 + x_1x_3 + \ldots + x_{n-1}x_n &=& \frac{a_{n-2}}{a_n}, \\
	&\ldots& \\
	\sum_{1\leq i_1 < i_2 < \ldots < i_k \leq n} x_{i_1}x_{i_2}\ldots x_{i_k} &=& (-1)^k\frac{a_{n-k}}{a_n}, \\
	&\ldots& \\
	x_1x_2\ldots x_n &=& (-1)^n\frac{a_0}{a_n}.
\end{eqnarray*}

\begin{enumerate}[label*=\protect\fbox{\arabic{enumi}}]
	
	\item 
	\text{Пусть } $x_1, x_2, x_3$ \text{ — корни уравнения } $x^3 - 2x^2 + x + 1 = 0.$
	Составьте кубическое уравнение, корнями которого являются числа $\frac{1}{x_1^2}, \frac{1}{x_2^2}, \frac{1}{x_3^2}$.
	
	\item У многочлена с целыми коэффициентами $x^3 + px + q$ имеется три различных корня. Докажите, что сумма кубов этих корней есть целое число, кратное трём.
	
	\item Известно, что $a + b + c = d$, и что $$\frac{1}{a} + \frac{1}{b} + \frac{1}{c} = \frac{1}{d}.$$
	Докажите, что по меньшей мере одно из чисел $a, b, c$ равно $d$.
	
	\item Даны действительные числа $a_1 \leqslant a_2 \leqslant a_3$  и  $b_1 \leqslant b_2 \leqslant b_3,$
	такие что $$a_1 + a_2 + a_3 = b_1 + b_2 + b_3,$$  $$a_1a_2 + a_2a_3 + a_1a_3 = b_1b_2 + b_2b_3 + b_1b_3.$$
	Докажите, что если $a_1 \leq b_1$, то $a_3 \leq b_3$.
	
	\item На доске написано несколько приведённых многочленов 37-й степени, все коэффициенты которых неотрицательны. Разрешается выбрать любые два выписанных многочлена $f$ и $g$ и заменить их на такие два приведённых многочлена 37-й степени $f_1$ и $g_1$, что $f + g = f_1 + g_1$ или $fg = f_1g_1$. Докажите, что после применения любого конечного числа таких операций не может оказаться, что каждый многочлен на доске имеет 37 различных положительных корней.

	\item Натуральные числа $a, b, c, d, e, f$ таковы, что число $S = a + b + c + d + e + f$ делит
    числа $abc + def$ и  $ab + bc + ca - de - ef - df.$ Докажите, что $S$ составное.




\end{enumerate}
\end{document}