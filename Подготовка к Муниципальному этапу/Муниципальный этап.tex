\documentclass{article}

\usepackage[12pt]{extsizes}
\usepackage[T2A]{fontenc}
\usepackage[utf8]{inputenc}
\usepackage[english, russian]{babel}

\usepackage{mathrsfs}
\usepackage[dvipsnames]{xcolor}

\usepackage{amsmath}
\usepackage{amssymb}
\usepackage{amsthm}
\usepackage{indentfirst}
\usepackage{amsfonts}
\usepackage{enumitem}
\usepackage{graphics}
\usepackage{tikz}
\usepackage{tabu}
\usepackage{diagbox}
\usepackage{hyperref}
\usepackage{mathtools}
\usepackage{ucs}
\usepackage{lipsum}
\usepackage{geometry} % Меняем поля страницы
\usepackage{fancyhdr} % Headers and footers
\newcommand{\range}{\mathrm{range}}
\newcommand{\dom}{\mathrm{dom}}
\newcommand{\N}{\mathbb{N}}
\newcommand{\R}{\mathbb{R}}
\newcommand{\E}{\mathbb{E}}
\newcommand{\D}{\mathbb{D}}
\newcommand{\M}{\mathcal{M}}
\newcommand{\Prime}{\mathbb{P}}
\newcommand{\A}{\mathbb{A}}
\newcommand{\Q}{\mathbb{Q}}
\newcommand{\Z}{\mathbb{Z}}
\newcommand{\F}{\mathbb{F}}
\newcommand{\CC}{\mathbb{C}}

\DeclarePairedDelimiter\abs{\lvert}{\rvert}
\DeclarePairedDelimiter\floor{\lfloor}{\rfloor}
\DeclarePairedDelimiter\ceil{\lceil}{\rceil}
\DeclarePairedDelimiter\lr{(}{)}
\DeclarePairedDelimiter\set{\{}{\}}
\DeclarePairedDelimiter\norm{\|}{\|}

\renewcommand{\labelenumi}{(\alph{enumi})}

\newcommand{\smallindent}{
    \geometry{left=1cm}% левое поле
    \geometry{right=1cm}% правое поле
    \geometry{top=1.5cm}% верхнее поле
    \geometry{bottom=1cm}% нижнее поле
}

\newcommand{\header}[3]{
    \pagestyle{fancy} % All pages have headers and footers
    \fancyhead{} % Blank out the default header
    \fancyfoot{} % Blank out the default footer
    \fancyhead[L]{#1}
    \fancyhead[C]{#2}
    \fancyhead[R]{#3}
}

\newcommand{\dividedinto}{
    \,\,\,\vdots\,\,\,
}

\newcommand{\littletaller}{\mathchoice{\vphantom{\big|}}{}{}{}}

\newcommand\restr[2]{{
    \left.\kern-\nulldelimiterspace % automatically resize the bar with \right
    #1 % the function
    \littletaller % pretend it's a little taller at normal size
    \right|_{#2} % this is the delimiter
}}

\DeclareGraphicsExtensions{.pdf,.png,.jpg}

\newenvironment{enumerate_boxed}[1][enumi]{\begin{enumerate}[label*=\protect\fbox{\arabic{#1}}]}{\end{enumerate}}



\smallindent

\header{Математика}{\textit{Разное}}{14 сентября 2022}

%----------------------------------------------------------------------------------------

\begin{document}
    \large

    \begin{center}
        \textbf{Муниципальный этап}
    \end{center}


    \begin{enumerate_boxed}
        \item Найти наибольшее значение, которое может принять произведение натуральных чисел, сумма которых равна 2023.

        \item Можно ли расставить натуральные числа в клетки таблицы размером $7 \times 7$ так, чтобы в любом квадрате $2 \times 2$ и любом квадрате $3 \times 3$ сумма чисел была нечётна?

%9 класс
        \item Все трехзначные числа записаны в ряд: $100, 101, 102, \dotso 1998, 1999$.
        Сколько раз в этом ряду после двойки идет ноль?

        \item У натурального числа $N$ выписали все его делители, затем у каждого из этих делителей подсчитали сумму цифр.
        Оказалось, что среди этих сумм нашлись все числа от $1$ до $9$.
        Найдите наименьшее значение $N$.

        \item Сумма квадратов $n$ простых чисел, каждое из которых больше $5$, делится на $6$.
        Докажите что и $n$ делится на 6.

        \item На доску в каком-то порядке Тимофей написал числа 1, 4, 9, 16, 25, 36, 49, 64, 81.
        Затем Алиса разделила числа на три группы: первые три числа, последние три числа, средние три числа и запомнила сумму в каждой группе.
        После этого Алиса разделила числа на три новые группы: у которых позиция даёт остаток 0 при делении на 3, остаток 1, остаток 2 и снова запомнила сумму в каждой группе.
        Могло ли получиться так, что 6 чисел, которые запомнила Алиса совпадают?

        \item В десятичной записи некоторого натурального числа переставили цифры и получили число в три раза меньшее.
        Доказать, что исходное число делится на 27.

        \item В классе учатся 30 человек: отличники, троечники и двоечники.
        Отличники на все вопросы отвечают правильно, двоечники всегда ошибаются, а троечники на заданные им вопросы строго по очереди то отвечают верно, то ошибаются.
        Всем ученикам было задано по три вопроса: <<Ты отличник?>>, <<Ты троечник?>>, <<Ты двоечник?>>.
        Ответили <<Да>> на первый вопрос – $19$ учащихся, на второй – $12$, на третий – $9$.
        Сколько троечников учится в этом классе?

        \item Найти все натуральные числа, оканчивающиеся на 2023, которые после зачеркивания последних четырех цифр уменьшаются в целое число раз.

        \item Имеются два сосуда, в первом из них 10 л воды, второй сосуд пустой.
        Последовательно проводятся переливания из первого сосуда во второй, из второго в первый и т.д., причем доля отливаемой воды составляет последовательно $\frac{1}{2}$, $\frac{1}{3}$, $\frac{1}{4}$ и т.д. от количества воды в сосуде, из которого вода отливается.
        Сколько воды будет в сосудах после $2019$ переливаний?

        \item На доске записано 30 чисел: $1, 2, \dotsc, 30$.
        За одну операцию разрешается стереть с доски любые два числа $a, b$, а вместо них записать числа $a + 2b$ и $b + 2a$.
        Может ли получиться так, что в результате нескольких операций на доске будут записаны 30 одинаковых чисел?

        \item При каких натуральных $n$ выражение $n^2 - 4n + 11$ является квадратом натурального числа?

        \item Внутри острого угла расположен выпуклый четырёхугольник $ABCD$.
        Оказалось, что для каждой из двух прямых, содержащих стороны угла, выполняется условие: сумма расстояний от вершин $A$ и $C$ до этой прямой равна сумме расстояний от вершин $B$ и $D$ до этой же прямой.
        Докажите, что $ABCD$ – параллелограмм.

        \item В квадрате $7 \times 7$ клеток размещено $16$ плиток размером $1 \times 3$ и одна плитка размером $1 \times 1$.
        Докажите, что плитка размером $1 \times 1$ либо лежит в центре, либо примыкает к границам квадрата.

        \item На острове Лжецов и Рыцарей расстановку по кругу называют правильной, если каждый, стоящий в кругу, может сказать, что среди двух его соседей есть представитель его племени.
        Однажды 2022 аборигенов образовали правильную расстановку по кругу.
        К ним подошел лжец и сказал: «Теперь мы вместе тоже можем образовать правильную расстановку по кругу».
        Сколько рыцарей могло быть в исходной расстановке?

        \item На столе лежат 2023 монет.
        Двое играют в следующую игру: ходят по очереди; за ход первый может взять со стола любое нечетное число монет от 1 до 99, второй любое четное число монет от 2 до 100.
        Проигрывает тот, кто не сможет сделать ход.
        Кто выиграет при правильной игре?

        \item В правильном десятиугольнике проведены все диагонали.
        Возле каждой вершины и возле каждой точки пересечения диагоналей поставлено число +1 (рассматриваются только сами диагонали, а не их продолжения).
        Разрешается одновременно изменить все знаки у чисел, стоящих на одной стороне или на одной диагонали.
        Можно ли с помощью нескольких таких операций изменить все знаки на противоположные?

        \item Алиса и Базилио играют в следующую игру:
        из мешка, первоначально содержащего 1331 монету, они по очереди берут монеты, причем первый ход делает Алиса и берет 1 монету, а далее при каждом следующем ходе игрок берет (по своему усмотрению) либо столько же монет, сколько взял другой игрок последним ходом, либо на одну больше.
        Проигрывает тот, кто не может сделать очередной ход по правилам.
        Кто из игроков может обеспечить себе выигрыш независимо от ходов другого?

        \item Две биссектрисы треугольника пересекаются под углом $60^\circ$.
        Докажите, что один из углов этого треугольника равен $60^\circ$.

        \item Разложите на множители $x^4 + 2021x^2 + 2020x + 2021$.

        \item Найдите $x^3 + y^3$, если известно, что $x + y = 5$ и $x + y + x^{2}y + xy^2 = 24$.

        \item Разложите на множители $x^5 + x^4 + 1$.

        \item Решить в натуральных числах $n$ и $m$ уравнение $(n+1)!(m+1)!=(n+m)!$

        \item На окружности расставлено 100 точек.
        Играют двое, они ходят по очереди.
        За один ход разрешается соединить любые две из этих 100 точек отрезком, не пересекающим отрезков, проведенных ранее.
        Проигрывает тот, кто не может сделать ход.
        Кто выигрывает при правильной игре?

        \item Натуральные числа $a, b, c$ и $d$ удовлетворяют равенству $a^2 + b^2 + c^2 = d^2$.
        Доказать, что число $abc$ делится на 4.

    \end{enumerate_boxed}
\end{document}