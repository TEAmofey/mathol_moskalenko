\documentclass{article}
\usepackage[12pt]{extsizes}
\usepackage[T2A]{fontenc}
\usepackage[utf8]{inputenc}
\usepackage[english, russian]{babel}

\usepackage{amssymb}
\usepackage{amsfonts}
\usepackage{amsmath}
\usepackage{enumitem}
\usepackage{graphics}
\usepackage{graphicx}

\usepackage{lipsum}

\newtheorem{theorem}{Теорема}
\newtheorem{task}{Задача}
\newtheorem{lemma}{Лемма}
\newtheorem{definition}{Определение}
\newtheorem{example}{Пример}
\newtheorem{statement}{Утверждение}
\newtheorem{corollary}{Следствие}


\usepackage{geometry} % Меняем поля страницы
%\geometry{left=1cm}% левое поле
%\geometry{right=1cm}% правое поле
\geometry{top=3cm}% верхнее поле
%\geometry{bottom=1cm}% нижнее поле


\usepackage{fancyhdr} % Headers and footers
\pagestyle{fancy} % All pages have headers and footers
\fancyhead{} % Blank out the default header
\fancyfoot{} % Blank out the default footer
\fancyhead[L]{\textit{\textbf{Олимпиада Эйлера}}}
\fancyhead[C]{}
\fancyhead[R]{7 февраля 2023}% Custom header text


%----------------------------------------------------------------------------------------

%\begin{document}\normalsize
\begin{document}\large
	
\begin{center}
	\LARGE\textbf{8 класс}
\end{center}
\begin{center}
	\large\textbf{Второй день}
\end{center}


\begin{enumerate}[label*=8.{\arabic{enumi}}]
\setcounter{enumi}{5}
\item Приведите пример шести различных натуральных чисел таких, что произведение любых двух из них не делится на сумму всех чисел, а произведение любых трех из них — делится.

\item На острове живут только рыцари, которые всегда говорят правду, и лжецы, которые всегда лгут. Однажды все они сели по кругу, и каждый сказал: «Среди двух моих соседей есть лжец!». Затем они сели по кругу в другом порядке, и каждый сказал: «Среди двух моих соседей нет рыцаря!». Могло ли на острове быть 2023 человека?

\item В выпуклом четырёхугольнике $ABCD$ биссектриса угла $B$ проходит через середину стороны $AD$, а $\angle C = \angle A+\angle D$. Найдите угол $ACD$.

\item Имеется клетчатая доска размером $2n\times 2n$. Петя поставил на неё $(n+1)^2$ фишек. Кот может одним взмахом лапы смахнуть на пол любую одну фишку или две фишки, стоящие в соседних по стороне или углу клетках. За какое наименьшее количество взмахов кот заведомо сможет смахнуть на пол все поставленные Петей фишки?

\item На доске написано 100 натуральных чисел, среди которых ровно 33 нечетных. Каждую минуту на доску дописывается сумма всех попарных произведений всех чисел, уже находящихся на ней (например, если на доске были записаны числа 1, 2, 3, 3, то следующим ходом было дописано число $1\cdot 2+1\cdot 3+1\cdot 3+2\cdot 3+2\cdot 3+3\cdot 3$). Можно ли утверждать, что рано или поздно на доске появится число, делящееся на $2^{10000000}$?

\end{enumerate}
\end{document}