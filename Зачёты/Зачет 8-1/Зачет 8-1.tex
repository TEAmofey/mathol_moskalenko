\documentclass{article}
\usepackage[12pt]{extsizes}
\usepackage[T2A]{fontenc}
\usepackage[utf8]{inputenc}
\usepackage[english, russian]{babel}

\usepackage{amssymb}
\usepackage{amsfonts}
\usepackage{amsmath}
\usepackage{enumitem}
\usepackage{graphics}

\usepackage{lipsum}



\usepackage{geometry} % Меняем поля страницы
\geometry{left=1cm}% левое поле
\geometry{right=1cm}% правое поле
\geometry{top=1.5cm}% верхнее поле
\geometry{bottom=1cm}% нижнее поле


\usepackage{fancyhdr} % Headers and footers
\pagestyle{fancy} % All pages have headers and footers
\fancyhead{} % Blank out the default header
\fancyfoot{} % Blank out the default footer
\fancyhead[L]{ЦРОД $\bullet$ Математика}
\fancyhead[C]{\textit{Зачёт}}
\fancyhead[R]{Май 2022}% Custom header text


%----------------------------------------------------------------------------------------

%\begin{document}\normalsize
\begin{document}\large


\begin{center}
\textbf{Группа 8-1}
\end{center}

\begin{center}
	\textbf{Ортоцентр}
\end{center}

\begin{enumerate}[label*=\protect\fbox{\arabic{enumi}}]

\item Докажите, что $\angle ABH = \angle CBO$.

\item Докажите, что $\angle ABH = \angle H_cH_aH$.

\item Докажите, что $H_aA$~--- биссектриса $ \angle H_cH_aH_b$.

\item Докажите, что $O$~--- ортоцентр треугольника $M_aM_bM_c$.

\end{enumerate}


\begin{center}
	\textbf{Углы}
\end{center}

\begin{enumerate}[label*=\protect\fbox{\arabic{enumi}}]
	
	\item Даны две окружности, пересекающиеся в точках $X$ и $Y$. Прямая, проходящая через $X$, пересекает первую окружность в точке $A$, а вторую — в точке $C$. Другая прямая, проходящая через $Y$, первую окружность пересекает в точке $B$, а вторую — в точке $D$. Докажите, что $AB \parallel CD$.
	
	\item В окружность вписан шестиугольник. Найдите сумму углов при трёх его несоседних вершинах.
	
	\item Окружности с центрами $O_1$ и $O_2$ пересекаются в точках $A$ и $B$. Луч $O_2A$ пересекает первую окружность в точке $C$. Докажите, что точки $O_1$, $O_2$, $B$, $C$ лежат на одной окружности.
	
	\item Докажите, что в равнобедренной трапеции вершины боковой стороны, точка пересечения диагоналей и центр описанной окружности лежат на одной окружности.
	
\end{enumerate}


\begin{center}
	\textbf{Углы-2}
\end{center}

\begin{enumerate}[label*=\protect\fbox{\arabic{enumi}}]
	
	\item Даны два угла $\angle ABC = 90^\circ$ и $\angle ADC = 90^\circ$. Докажите, что $A, B, C, D$ лежат на одной окружности.
	
	\item Дан треугольник $ABC$. $I$ - центр вписанной окружности. Докажите (и запомните), что $\angle AIB = 90^\circ + \frac{\angle A}{2}$
	
	\item Дан треугольник $ABC$. $H$ - ортоцентр (точка пересечения высот). Докажите (и запомните), что $\angle AHB = 180^\circ - \angle C$
	
	\item Дан треугольник $ABC$. $BH_1$, $CH_2$ - высоты треугольника. Докажите, что $C,B,H_1, H_2$ лежат на одной окружности.
	
\end{enumerate}


\end{document}