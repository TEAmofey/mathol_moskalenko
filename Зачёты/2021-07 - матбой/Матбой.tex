\documentclass{article}

\usepackage[12pt]{extsizes}
\usepackage[T2A]{fontenc}
\usepackage[utf8]{inputenc}
\usepackage[english, russian]{babel}

\usepackage{mathrsfs}
\usepackage[dvipsnames]{xcolor}

\usepackage{amsmath}
\usepackage{amssymb}
\usepackage{amsthm}
\usepackage{indentfirst}
\usepackage{amsfonts}
\usepackage{enumitem}
\usepackage{graphics}
\usepackage{tikz}
\usepackage{tabu}
\usepackage{diagbox}
\usepackage{hyperref}
\usepackage{mathtools}
\usepackage{ucs}
\usepackage{lipsum}
\usepackage{geometry} % Меняем поля страницы
\usepackage{fancyhdr} % Headers and footers
\newcommand{\range}{\mathrm{range}}
\newcommand{\dom}{\mathrm{dom}}
\newcommand{\N}{\mathbb{N}}
\newcommand{\R}{\mathbb{R}}
\newcommand{\E}{\mathbb{E}}
\newcommand{\D}{\mathbb{D}}
\newcommand{\M}{\mathcal{M}}
\newcommand{\Prime}{\mathbb{P}}
\newcommand{\A}{\mathbb{A}}
\newcommand{\Q}{\mathbb{Q}}
\newcommand{\Z}{\mathbb{Z}}
\newcommand{\F}{\mathbb{F}}
\newcommand{\CC}{\mathbb{C}}

\DeclarePairedDelimiter\abs{\lvert}{\rvert}
\DeclarePairedDelimiter\floor{\lfloor}{\rfloor}
\DeclarePairedDelimiter\ceil{\lceil}{\rceil}
\DeclarePairedDelimiter\lr{(}{)}
\DeclarePairedDelimiter\set{\{}{\}}
\DeclarePairedDelimiter\norm{\|}{\|}

\renewcommand{\labelenumi}{(\alph{enumi})}

\newcommand{\smallindent}{
    \geometry{left=1cm}% левое поле
    \geometry{right=1cm}% правое поле
    \geometry{top=1.5cm}% верхнее поле
    \geometry{bottom=1cm}% нижнее поле
}

\newcommand{\header}[3]{
    \pagestyle{fancy} % All pages have headers and footers
    \fancyhead{} % Blank out the default header
    \fancyfoot{} % Blank out the default footer
    \fancyhead[L]{#1}
    \fancyhead[C]{#2}
    \fancyhead[R]{#3}
}

\newcommand{\dividedinto}{
    \,\,\,\vdots\,\,\,
}

\newcommand{\littletaller}{\mathchoice{\vphantom{\big|}}{}{}{}}

\newcommand\restr[2]{{
    \left.\kern-\nulldelimiterspace % automatically resize the bar with \right
    #1 % the function
    \littletaller % pretend it's a little taller at normal size
    \right|_{#2} % this is the delimiter
}}

\DeclareGraphicsExtensions{.pdf,.png,.jpg}

\newenvironment{enumerate_boxed}[1][enumi]{\begin{enumerate}[label*=\protect\fbox{\arabic{#1}}]}{\end{enumerate}}



\header{ЦРОД $\bullet$ Математика}{\textit{Битва}}{Стратегия 2021}

%----------------------------------------------------------------------------------------

\begin{document}
    \large

    \begin{center}
        \textbf{Математический бой за славу и честь!}
    \end{center}

    \begin{enumerate}[label*=\arabic{enumi}.]

        \item Существует ли такое натуральное число $N$, что и у числа $N$, и у числа $N^2 - 1$ сумма цифр равна $2021$

        \item Существует ли такая непериодическая функция $f$, что при любом действительном $x$ выполнено равенство \[f (x + 1) = f (x + 1)f (x) + 1\]

        \item По кругу сидит $2021$ человек, каждый из которых либо рыцарь, который всегда говорит правду, либо лжец, который всегда лжет.
        Каждый из них сказал: «Если моего соседа справа спросить, кем является мой сосед слева, то он ответит — лжецом.». Сколько лжецов за этим столом?

        \item Найдите все квадратные трехчлены $x^2 + mx + n$ такие, что числа $m$ и $n$ (не обязательно различные) являются их корнями?

        \item Докажите, что не может одновременно выполняться для положительных $a,b,c$
        \begin{gather*}
            a(1-b)>1/4\\
            b(1-c)>1/4\\
            c(1-a)>1/4\\
        \end{gather*}

        \item Точка $H$ — ортоцентр остроугольного треугольника $ABC$, в котором $AB > AC$.
        Точка $E$ симметрична $C$ относительно высоты $AH$. $F$ — точка пересечения прямых $EH$ и $AC$.
        Докажите, что центр описанной окружности треугольника $AEF$ лежит на прямой $AB$.

        \item Из клетчатого квадрата $2021 \times 2021$ вырезали угловой квадрат $6 \times 6$.
        Можно ли о«ставшуюся фигуру разрезать на прямоугольники $1 \times 5$?

        \item На первое занятие по олимпиадной математике пришло 7 человек, которые сели по кругу.
        При этом, если выбрать любых 6 человек, то возможна рассадка, при которой все будут сидеть рядом со своим знакомым.
        Верно ли, что и вся группа из 7 человек может быть так рассажена?

        \item В трапеции $ABCD$ с основаниями $AB$ и $CD$ выполнено равенство $AB = BD + CD$.
        Пусть $M$ — середина диагонали $AC$.
        Докажите, что $\angle BMD = 90^\circ$.

    \end{enumerate}
\end{document}