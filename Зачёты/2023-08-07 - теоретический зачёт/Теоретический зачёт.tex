\documentclass{article}

\usepackage[12pt]{extsizes}
\usepackage[T2A]{fontenc}
\usepackage[utf8]{inputenc}
\usepackage[english, russian]{babel}

\usepackage{mathrsfs}
\usepackage[dvipsnames]{xcolor}

\usepackage{amsmath}
\usepackage{amssymb}
\usepackage{amsthm}
\usepackage{indentfirst}
\usepackage{amsfonts}
\usepackage{enumitem}
\usepackage{graphics}
\usepackage{tikz}
\usepackage{tabu}
\usepackage{diagbox}
\usepackage{hyperref}
\usepackage{mathtools}
\usepackage{ucs}
\usepackage{lipsum}
\usepackage{geometry} % Меняем поля страницы
\usepackage{fancyhdr} % Headers and footers
\newcommand{\range}{\mathrm{range}}
\newcommand{\dom}{\mathrm{dom}}
\newcommand{\N}{\mathbb{N}}
\newcommand{\R}{\mathbb{R}}
\newcommand{\E}{\mathbb{E}}
\newcommand{\D}{\mathbb{D}}
\newcommand{\M}{\mathcal{M}}
\newcommand{\Prime}{\mathbb{P}}
\newcommand{\A}{\mathbb{A}}
\newcommand{\Q}{\mathbb{Q}}
\newcommand{\Z}{\mathbb{Z}}
\newcommand{\F}{\mathbb{F}}
\newcommand{\CC}{\mathbb{C}}

\DeclarePairedDelimiter\abs{\lvert}{\rvert}
\DeclarePairedDelimiter\floor{\lfloor}{\rfloor}
\DeclarePairedDelimiter\ceil{\lceil}{\rceil}
\DeclarePairedDelimiter\lr{(}{)}
\DeclarePairedDelimiter\set{\{}{\}}
\DeclarePairedDelimiter\norm{\|}{\|}

\renewcommand{\labelenumi}{(\alph{enumi})}

\newcommand{\smallindent}{
    \geometry{left=1cm}% левое поле
    \geometry{right=1cm}% правое поле
    \geometry{top=1.5cm}% верхнее поле
    \geometry{bottom=1cm}% нижнее поле
}

\newcommand{\header}[3]{
    \pagestyle{fancy} % All pages have headers and footers
    \fancyhead{} % Blank out the default header
    \fancyfoot{} % Blank out the default footer
    \fancyhead[L]{#1}
    \fancyhead[C]{#2}
    \fancyhead[R]{#3}
}

\newcommand{\dividedinto}{
    \,\,\,\vdots\,\,\,
}

\newcommand{\littletaller}{\mathchoice{\vphantom{\big|}}{}{}{}}

\newcommand\restr[2]{{
    \left.\kern-\nulldelimiterspace % automatically resize the bar with \right
    #1 % the function
    \littletaller % pretend it's a little taller at normal size
    \right|_{#2} % this is the delimiter
}}

\DeclareGraphicsExtensions{.pdf,.png,.jpg}

\newenvironment{enumerate_boxed}[1][enumi]{\begin{enumerate}[label*=\protect\fbox{\arabic{#1}}]}{\end{enumerate}}



\smallindent

\header{Математика}{\textit{Зачёт}}{7 августа 2023}

%----------------------------------------------------------------------------------------

\begin{document}
    \large

    \begin{center}
        \textbf{Теоретический зачёт}
    \end{center}

    \begin{enumerate_boxed}

        \item \textbf{Принцип крайнего}

        \textbf{Задача:} Семь грибников собрали вместе 59 грибов, причем каждый собрал разное количество.
        Докажите, что какие-то три грибника собрали вместе не менее 33 грибов.

        \item \textbf{Теорема Безу}

        Формулировка + доказательство.

        \textbf{Задача:} Докажите, что если значения двух многочленов степени не выше $n$ совпадают в $n+1$ точке, то и сами многочлены равны.

        \item \textbf{Процессы}

        \textbf{Задача:} Круг разделен на 2022 сектора, и в каждом написано целое число.
        В один из секторов ставится фишка.
        Каждым ходом прочитывается число в секторе, где стоит фишка (пусть прочитано $k$), фишка сдвигается на $|k|$ секторов по часовой стрелке, и там, куда она придет, число увеличивается на 1.
        Докажите, что со временем все числа станут больше миллиона.

        \item \textbf{Информация 1}

        Идея о том, как работают оценки в задачах на весы.


        \textbf{Задача:} Есть 4 гири разных масс, за наименьшее число взвешиваний на чашечных весах упорядочите их по массе.


        \item \textbf{Уравнения в целых числах    }

        Основные методы решения.

        \textbf{Задача:} Решите в натуральных числах уравнение   $1! + 2! + \dotsc + n! = m^2$.

        \textbf{Задача:} Найдите все целые $a, b, c, d$, для которых $a^2 +b^2 +c^2 +d^2 = 2abcd$.


        \item \textbf{Минимакс}

        Доказательство неравенства ломанной.

        \textbf{Задача:} Из точки $M$, лежащей на стороне $AB$ остроугольного треугольника $ABC$, опущены перпендикуляры $MP$ и $MQ$ на стороны $BC$ и $AC$.
        При каком положении точки $M$ длина отрезка $PQ$ минимальна?

        \item \textbf{Теорема Чевы, Менелая, Фалеса}

        Формулировка и доказательство прямой и обратной теоремы Фалеса.

        Формулировка и доказательство прямой и обратной теоремы Чевы.

        Формулировка и доказательство прямой и обратной теоремы Менелая.

        \textbf{Задача:} Пусть $AL$~--- биссектриса треугольника $ABC$, точка $D$~--- ее середина, $E$~--- проекция $D$ на $AB$.
        Известно, что $AC = 3AE$.
        Докажите, что треугольник $CEL$ равнобедренный.

        \item \textbf{Подобие}

        Что?
        У нас был листик на подобие? \ldots Ну ладно.
        Пускай будет 3 теоремы про подобные треугольники.

        \textbf{Задача:} В треугольнике $ABC$ точка $D$ лежит на стороне $AC$, углы $ABD$ и $BCD$ равны, $AB = CD$, $AE$ биссектриса угла $A$.
        Докажите, что $ED \parallel AB$.

        \item \textbf{Конструктивы}

        \textbf{Задача:} Существуют ли на плоскости три такие точки $A, B$ и $C$, что для любой точки $X$ длина хотя бы одного из отрезков $XA, XB$ и $XC$ иррациональна?

        \item \textbf{Рациональность}

        \textbf{Задача:} Докажите, что если выражение  $\dfrac{x}{x^2 + x + 1}$ принимает рациональное значение, то и выражение  $\dfrac{x^2}{x^4 + x^2 + 1}$ также рационально.

        \item \textbf{Гомотетия}

        Два определение и доказательство основных свойств.

        \textbf{Задача:} Вписанная окружность треугольника $ABC$ касается сторон $AB, AC, BC$ в точках $C_1, B_1, A_1$ соответственно.
        Точки $A_2, B_2, C_2$ — середины дуг $BC, AC, AB$ описанной окружности треугольника $ABC$.
        Докажите, что прямые $A_{1}A_2, B_{1}B_2, C_{1}C_2$ пересекаются в одной точке.

        \item \textbf{Радоси}

        Определение степени точки и радикальной оси.
        Доказательство, что это прямая.
        Теорема о радикальном центре.
        Задача о медиане.

        \textbf{Задача:} На сторонах треугольника $ABC$ взято по две точки так, что шесть отрезков, соединяющих вершину с точкой на противолежащей стороне, равны.
        Докажите, что середины этих отрезков лежат на одной окружности.

        \item \textbf{Теорема Хелли}

        Формулировка + доказательство.

        \textbf{Задача:} Внутри ограниченной выпуклой фигуры всегда найдется точка, обладающая следующим свойством: любая прямая, проходящая через эту точку, делит площадь фигуры на части, отношение которых не превосходит 2.

        \item \textbf{Степени вхождения простых}

        Определение степени вхождения простых в рациональное число.
        Доказательство корректности.
        Формула Лежандра.

        \textbf{Задача:} Даны различные натуральные числа $a_1 , a_2 , \dotsc , a_n$.
        Положим
        \[b_i = (a_i - a_1)(a_i - a_2)\dotsc(a_i - a_{i-1})(a_i - a_{i+1})\dotsc(a_i - a_n).\]
        Докажите, что наименьшее общее кратное $[b_1 , b_2 , \dotsc , b_n ]$ делится на $(n - 1)!$

        \item \textbf{Информация - 2}

        \textbf{Задача:} Есть $2n$ монеток попарно различного веса.
        За одно действие можно сравнить любые две монетки.
        Какое минимальное количество взвешиваний потребуется, чтобы найти самую тяжёлую и самую лёгкую монетки?

        \item \textbf{Транснеравенства}

        Формулировка, доказательство.

        \textbf{Задача:} Докажите неравенство \[\frac{a}{b(b+c)} + \frac{b}{c(c+a)} + \frac{c}{a(a+b)} \geqslant \frac{1}{a+b} + \frac{1}{b+c} + \frac{1}{c+a}.\]

        \item \textbf{КБШ}

        Формулировка, доказательство.

        Формулировка, доказательство важнейшей формы КБШ.

        \textbf{Задача:} Докажите, что при всех положительных $a$, $b$, $c$, $d$ выполнено
        \[
        \frac{a}{b+2c+3d} + \frac{b}{c+2d+3a} + \frac{c}{d+2a+3b} +
        \frac{d}{a+2b+3c} \geqslant \frac{2}{3}.
        \]

        \item \textbf{Функции}

        \textbf{Задача:} Найдите все такие функции $f(x)$, что \[f(x) + f\left(\frac{x - 1}{x}\right) = 2x \,\,\, (x \neq  0).\]

        \item \textbf{Случайные события}

        Условная вероятность, определение.

        Формула полной вероятности.

        \textbf{Задача:} Карточка <<спортлото>> содержит 36 чисел.
        Игрок может выбрать 6, а выигрышных номеров в тираже определяется тоже 6.
        Какова вероятность того, что верно будет угадано ровно 3 числа?

        \item \textbf{Формула Байеса}

        Формула Байеса, формулировка доказательство.

        \textbf{Задача:} 100 пассажиров садятся по очереди в 100-местный самолёт.
        Первый занимает случайное место.
        Второй садится на своё место, если оно свободно, а если нет, то садится на случайное место из оставшихся.
        Остальные делают то же самое.
        С какой вероятностью последний пассажир окажется на своём месте?


        \item \textbf{Равносоставленные Многоульники}

        Определение.
        Формулировка основной теоремы, идея доказательства.

        \item \textbf{Теорема Паскаля}

        Формулировка.

        \textbf{Задача:} Точка $M$ — середина гипотенузы $AC$ прямоугольного треугольника $ABC$.
        Точки $D$ и $E$ на отрезках $AM$ и $AB$ соответственно выбраны так, что $BC \parallel DE$.
        Окружность $\omega$, описанная около треугольника $ABC$, пересекает описанную окружность треугольника $CDE$ в точках $C$ и $F$, а прямую $DE$ в точках $X$ и $Y$.
        Касательная к окружности $\omega$ в точке $B$ пересекает луч $FC$ в точке $Z$.
        Докажите, что описанные окружности треугольников $XMY$ и $BFZ$ касаются.

        \item \textbf{Комбинаторика}

        \textbf{Задача:}  Сколькими способами можно на доске $30 \times 30$ расставить сорок одинаковых ладей так, чтобы каждая била ровно одну другую?

        \textbf{Задача:} Игральный кубик имеет 6 граней с цифрами 1, 2, 3, 4, 5, 6.
        Сколько различных игральных кубиков существует, если считать различными два кубика, которые нельзя спутать, как ни переворачивай?

        \item \textbf{Телескопические суммы}

        Доказательство теоремы про полиномиальную формулу.

        \textbf{Задача:} $\dfrac{1}{1^2 \cdot 3^2} + \dfrac{2}{3^2 \cdot 5^2} + \dotsc +  \dfrac{50}{99^2 \cdot 101^2}.$

        \item \textbf{Инверсия}

        Определение.
        Доказательство Инволюции.
        Образ окружности, не проходящей через центр инверсии.
        Образ прямой не проходящей через центр инверсии.

        \textbf{Задача:} Пусть $AH$ — высота остроугольного треугольника $ABC$, а точки $K$ и $L$ — проекции $H$ на стороны $AB$ и $AC$.
        Описанная окружность $\Omega$ треугольника $ABC$ пересекает прямую $KL$ в точках $P$ и $Q$, а прямую $AH$ — в точках $A$ и $T$.
        Докажите, что точка $H$ является центром вписанной окружности треугольника $PQT$.

        \item \textbf{Теория множеств}

        Определение инъективности, сюръективности, биективности.
        Доказательство теоремы Кантора.

        \textbf{Задача:}   $|\mathbb{R}^*| \stackrel{?}{=} |\mathbb{R}|.$

    \end{enumerate_boxed}

\end{document}