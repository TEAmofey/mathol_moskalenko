\documentclass{article}
\usepackage[12pt]{extsizes}
\usepackage[T2A]{fontenc}
\usepackage[utf8]{inputenc}
\usepackage[english, russian]{babel}

\usepackage{amssymb}
\usepackage{amsfonts}
\usepackage{amsmath}
\usepackage{enumitem}
\usepackage{graphics}

\usepackage{lipsum}



\usepackage{geometry} % Меняем поля страницы


\geometry{top=1.5cm}% верхнее поле
\geometry{bottom=1cm}% нижнее поле
\geometry{left=1cm}% левое поле
\geometry{right=1cm}% правое поле


\usepackage{fancyhdr} % Headers and footers
\pagestyle{fancy} % All pages have headers and footers
\fancyhead{} % Blank out the default header
\fancyfoot{} % Blank out the default footer
\fancyhead[L]{ЦРОД $\bullet$ Математика}
\fancyhead[C]{\textit{Битва}}
\fancyhead[R]{ЛФМШ 2022}% Custom header text


%----------------------------------------------------------------------------------------

%\begin{document}\normalsize
\begin{document}\large
	

\begin{center}
\textbf{Математический бой за славу и честь!}
\end{center}

\begin{enumerate}

%Т
\item На доске написаны числа $1, 2, 3, \dotso, 10$. За одну операцию можно стереть с доски два числа, и записать на доску их произведение, увеличенное на $91$. Может ли через 9 таких операций на доске оказаться число вида $10^n$?

%А
\item Найдите все квадратные трехчлены $x^2 + mx + n$ такие, что числа $m$ и $n$ (не обязательно различные) являются их корнями?

%К
\item Глеб пригласил на День Рождения 2345 своих знакомых. Оказалось, что в каком бы порядке друзья ни приходили на праздник, каждый новый пришедший будет знать не менее четверти из уже присутствующих (включая Глеба). Докажите, что у кого-то из гостей на вечеринке не более двух незнакомых.

%Г
\item В выпуклом четырехугольнике $ABCD$ углы $A$ и $C$ прямые. На продолжении стороны $AD$ за точку $D$ дана такая точка $E$, что $\angle ABE = \angle ADC$. Точка $K$ симметрична точке $C$ относительно точки $A$. Докажите, что $\angle ADB = \angle AKE$.

%К
\item Квадрат $104 \times 104$ разрезали на тетрамино “Т”. Докажите, что найдутся 6 прямых (линий сетки), которые разрезают одинаковое количество Т-шек?

%А
\item Тоня загадывает натуральное число, а Аня хочет привести это число к единице. Аня может делать следующие 4 операции: $\bullet$ $2x$ $\bullet$ $ \frac{x}{2}$ $\bullet$ $\frac{x-1}{3}$ $\bullet$ $3x+1$

В процессе могут получаться нецелые числа. Сможет ли Аня привести загаданное число к 1 или Тоня сможет придумать хитрое число?

%К
\item В ЦРОДе присутствовали $2021$ человек, некоторые из них являются друзьями. Назовём популярностью человека размер наибольшей группы людей, в которую он входит, такой, что любые двое из этой группы дружат между собой. Если у человека нет друзей в ЦРОДе, его популярность равна единице. Какое наибольшее число различных популярностей может быть у присутствующих в центре?

%Г
\item Точка $H$ — ортоцентр остроугольного треугольника $ABC$, в котором $AB > AC$. Точка $E$ симметрична $C$ относительно высоты $AH$. $F$ — точка пересечения прямых $EH$ и $AC$. Докажите, что центр описанной окружности треугольника $AEF$ лежит на прямой $AB$.
%К
\item За ЧГК вы получили 30 коинов, однако Денис Раймундо сказал, что среди них есть ровно 5 фальшивых. Вы можете попросить Дениса посмотреть на любую подгруппу из 30 коинов и сказать сколько из них поддельных. Вам нужно найти 5 \textit{настоящих} коинов и при этом задать Денису наименьшее количество вопросов. Как это сделать?

%К
\item Из клетчатого квадрата $2022 \times 2022$ вырезали угловой квадрат $3 \times 3$. Можно ли оставшуюся фигуру разрезать на прямоугольники $1 \times 5$?
%\item В гостинице $n$ номеров. Ключи от номеров хранятся в 10 коробках, в каждой коробке 2022 ключа. Администраторы могут потерять какие-то коробки, поэтому хозяин гостиницы организовал хранение ключей так, что даже если любые три коробки потеряются, от каждого номера всё равно можно будет найти хотя бы один ключ. Какое наибольшее количество номеров может быть в этой гостинице?

\end{enumerate}
\end{document}