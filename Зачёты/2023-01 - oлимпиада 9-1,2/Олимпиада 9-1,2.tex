\documentclass{article}

\usepackage[12pt]{extsizes}
\usepackage[T2A]{fontenc}
\usepackage[utf8]{inputenc}
\usepackage[english, russian]{babel}

\usepackage{mathrsfs}
\usepackage[dvipsnames]{xcolor}

\usepackage{amsmath}
\usepackage{amssymb}
\usepackage{amsthm}
\usepackage{indentfirst}
\usepackage{amsfonts}
\usepackage{enumitem}
\usepackage{graphics}
\usepackage{tikz}
\usepackage{tabu}
\usepackage{diagbox}
\usepackage{hyperref}
\usepackage{mathtools}
\usepackage{ucs}
\usepackage{lipsum}
\usepackage{geometry} % Меняем поля страницы
\usepackage{fancyhdr} % Headers and footers
\newcommand{\range}{\mathrm{range}}
\newcommand{\dom}{\mathrm{dom}}
\newcommand{\N}{\mathbb{N}}
\newcommand{\R}{\mathbb{R}}
\newcommand{\E}{\mathbb{E}}
\newcommand{\D}{\mathbb{D}}
\newcommand{\M}{\mathcal{M}}
\newcommand{\Prime}{\mathbb{P}}
\newcommand{\A}{\mathbb{A}}
\newcommand{\Q}{\mathbb{Q}}
\newcommand{\Z}{\mathbb{Z}}
\newcommand{\F}{\mathbb{F}}
\newcommand{\CC}{\mathbb{C}}

\DeclarePairedDelimiter\abs{\lvert}{\rvert}
\DeclarePairedDelimiter\floor{\lfloor}{\rfloor}
\DeclarePairedDelimiter\ceil{\lceil}{\rceil}
\DeclarePairedDelimiter\lr{(}{)}
\DeclarePairedDelimiter\set{\{}{\}}
\DeclarePairedDelimiter\norm{\|}{\|}

\renewcommand{\labelenumi}{(\alph{enumi})}

\newcommand{\smallindent}{
    \geometry{left=1cm}% левое поле
    \geometry{right=1cm}% правое поле
    \geometry{top=1.5cm}% верхнее поле
    \geometry{bottom=1cm}% нижнее поле
}

\newcommand{\header}[3]{
    \pagestyle{fancy} % All pages have headers and footers
    \fancyhead{} % Blank out the default header
    \fancyfoot{} % Blank out the default footer
    \fancyhead[L]{#1}
    \fancyhead[C]{#2}
    \fancyhead[R]{#3}
}

\newcommand{\dividedinto}{
    \,\,\,\vdots\,\,\,
}

\newcommand{\littletaller}{\mathchoice{\vphantom{\big|}}{}{}{}}

\newcommand\restr[2]{{
    \left.\kern-\nulldelimiterspace % automatically resize the bar with \right
    #1 % the function
    \littletaller % pretend it's a little taller at normal size
    \right|_{#2} % this is the delimiter
}}

\DeclareGraphicsExtensions{.pdf,.png,.jpg}

\newenvironment{enumerate_boxed}[1][enumi]{\begin{enumerate}[label*=\protect\fbox{\arabic{#1}}]}{\end{enumerate}}



\header{Сириус $\bullet$ Математика}{\textit{\textbf{Олимпиада}}}{24 января 2023}

%----------------------------------------------------------------------------------------

\begin{document}
    \large

    \begin{center}
        \LARGE\textbf{9-1}
    \end{center}
    \begin{center}
        \large\textbf{Второй тур}
    \end{center}


    \begin{enumerate}[label*=9.{\arabic{enumi}}]
        \setcounter{enumi}{4}
        \item Класс из 32 учеников на зимние каникулы поехал в лагерь в Комарово, где было 5 комнат.
        После смены выяснилось, что любые 2 ребёнка, жившие в одной комнате, стали ненавидеть друг друга.
        На весенние каникулы класс поехал в лагерь в Рощино, где было 6 комнат.
        Докажите, что ребят не удастся расселить так, чтобы дети, живущие в одной комнате пока что не испытывали ненависть друг к другу.

        \item Дан приведённый квадратный трёхчлен $f(x)$.
        Оказалось, что трёхчлены $f(x) + 2022x - 2023$ и  $f(x) - 2023x + 2022$ не имеют корней.
        Докажите, что $2023\cdot f(x) - 2022$ тоже не имеет корней.

        \item В выпуклом четырёхугольнике $ABCD$ верно тождество $AB + DC = AD + BC$.
        Оказалось, что радиусы вписанных окружностей треугольников $ABC$ и $ACD$ равны.
        Найдите угол между прямыми $AC$ и $BD$.

        \item Тая, Таня и Маша пошли гулять.
        Маша искала интересные факты о натуральном числе $m$ и попросила девочек помочь.
        Тая заметила, что некоторое натуральное число $n>m$ можно представить в виде суммы 2023 целых неотрицательных степеней числа $m$.
        На что Таня ответила, что то же самое число $n$ можно представить и в виде суммы 2023 целых неотрицательных степеней числа $m + 1$.
        При каком наибольшем $m$ такое могло произойти?
    \end{enumerate}
\end{document}