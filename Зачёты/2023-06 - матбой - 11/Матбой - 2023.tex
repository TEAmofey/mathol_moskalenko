\documentclass{article}

\usepackage[12pt]{extsizes}
\usepackage[T2A]{fontenc}
\usepackage[utf8]{inputenc}
\usepackage[english, russian]{babel}

\usepackage{mathrsfs}
\usepackage[dvipsnames]{xcolor}

\usepackage{amsmath}
\usepackage{amssymb}
\usepackage{amsthm}
\usepackage{indentfirst}
\usepackage{amsfonts}
\usepackage{enumitem}
\usepackage{graphics}
\usepackage{tikz}
\usepackage{tabu}
\usepackage{diagbox}
\usepackage{hyperref}
\usepackage{mathtools}
\usepackage{ucs}
\usepackage{lipsum}
\usepackage{geometry} % Меняем поля страницы
\usepackage{fancyhdr} % Headers and footers
\newcommand{\range}{\mathrm{range}}
\newcommand{\dom}{\mathrm{dom}}
\newcommand{\N}{\mathbb{N}}
\newcommand{\R}{\mathbb{R}}
\newcommand{\E}{\mathbb{E}}
\newcommand{\D}{\mathbb{D}}
\newcommand{\M}{\mathcal{M}}
\newcommand{\Prime}{\mathbb{P}}
\newcommand{\A}{\mathbb{A}}
\newcommand{\Q}{\mathbb{Q}}
\newcommand{\Z}{\mathbb{Z}}
\newcommand{\F}{\mathbb{F}}
\newcommand{\CC}{\mathbb{C}}

\DeclarePairedDelimiter\abs{\lvert}{\rvert}
\DeclarePairedDelimiter\floor{\lfloor}{\rfloor}
\DeclarePairedDelimiter\ceil{\lceil}{\rceil}
\DeclarePairedDelimiter\lr{(}{)}
\DeclarePairedDelimiter\set{\{}{\}}
\DeclarePairedDelimiter\norm{\|}{\|}

\renewcommand{\labelenumi}{(\alph{enumi})}

\newcommand{\smallindent}{
    \geometry{left=1cm}% левое поле
    \geometry{right=1cm}% правое поле
    \geometry{top=1.5cm}% верхнее поле
    \geometry{bottom=1cm}% нижнее поле
}

\newcommand{\header}[3]{
    \pagestyle{fancy} % All pages have headers and footers
    \fancyhead{} % Blank out the default header
    \fancyfoot{} % Blank out the default footer
    \fancyhead[L]{#1}
    \fancyhead[C]{#2}
    \fancyhead[R]{#3}
}

\newcommand{\dividedinto}{
    \,\,\,\vdots\,\,\,
}

\newcommand{\littletaller}{\mathchoice{\vphantom{\big|}}{}{}{}}

\newcommand\restr[2]{{
    \left.\kern-\nulldelimiterspace % automatically resize the bar with \right
    #1 % the function
    \littletaller % pretend it's a little taller at normal size
    \right|_{#2} % this is the delimiter
}}

\DeclareGraphicsExtensions{.pdf,.png,.jpg}

\newenvironment{enumerate_boxed}[1][enumi]{\begin{enumerate}[label*=\protect\fbox{\arabic{#1}}]}{\end{enumerate}}



\header{ЦРОД $\bullet$ Математика}{\textit{Не балуемся}}{ЛФМШ 2023}

%----------------------------------------------------------------------------------------

\begin{document}
    \large


    \begin{center}
        \textbf{Математический бой }
    \end{center}

    \begin{enumerate}[label*=\arabic{enumi}.]

        \item Биссектрисы неравнобедренного треугольника $ABC$ пересекаются в точке $I$.
        Точки $I_A$, $I_B$ и $I_C$ симметричны $I$ относительно прямых $BC$, $AC$ и $AB$ соответственно.
        Докажите, что центры описанных окружностей треугольников $AI_{A}I$, $BI_{B}I$, $CI_{C}I$ лежат на одной прямой.

        \item Решите уравнение в целых числах $3x^2 - y^2 = 3^{x+y}.$


        \item В команде на ЧГК есть семь человек.
        Но как мы знаем, в ЧГК можно играть только вшестером.
        Поэтому каждый вопрос они меняют одного игрока на другого и все вместе пересаживаются.
        Известно, что любых шестерых участников можно посадить по кругу так, чтобы каждый сидел рядом с двумя своими друзьями.
        Докажите, что можно так посадить всех семерых, что каждый будет сидеть рядом со своими друзьями.

        \item Найдите значение выражения

        \[\frac{(3^4+4)\cdot(7^4+4)\cdot\dotsc\cdot(2019^4+4)\cdot(2023^4+4)}{(1^4+4)\cdot(5^4+4)\cdot\dotsc\cdot(2017^4+4)\cdot(2021^4+4)}\]

        \item Хорда $CD$ окружности с центром $O$ перпендикулярна ее диаметру $AB$, а хорда $AE$ делит пополам радиус $OC$.
        Докажите, что хорда $DE$ делит пополам хорду $BC.$

        \item Алиса, Вадим и Настя записали на доске 3 числа $a, b$ и $c$.
        Затем пришел Матвей и стёр с доски все числа, заменив их на попарные произведения.
        Из-за этого в классе начался сущий кошмар\ldots До наших дней дошла информация от очевидцев, что на доске в итоге было записано два последовательных числа и два числа отличались на 1024.
        Восстановите числа, которые загадали изначально.

        \item Пока Никита разбирался в теории чисел, он заметил, что число \[\underbrace{1\dotsc1}_{p}\underbrace{2\dotsc2}_{p}\underbrace{3\dotsc3}_{p}\dotsc \underbrace{9\dotsc9}_{p} - 123456789\] делится на $p$ при любом простом $p$.
        Докажите это.

        \item Сколькими способами можно переставить буквы в слове «СУПЕРИЗБАЛОВАННАЯ», чтобы никакие две гласные не стояли рядом?

    \end{enumerate}
\end{document}