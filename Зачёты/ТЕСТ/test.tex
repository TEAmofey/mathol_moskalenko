\documentclass{article}
\usepackage[14pt]{extsizes}
\usepackage[T2A]{fontenc}
\usepackage[utf8]{inputenc}
\usepackage[english, russian]{babel}

\usepackage{amssymb}
\usepackage{amsfonts}
\usepackage{amsmath}
\usepackage{enumitem}
\usepackage{graphics}

\usepackage{lipsum}



\usepackage{geometry} % Меняем поля страницы
\geometry{top=1.5cm}% верхнее поле
\geometry{bottom=1cm}% нижнее поле


\usepackage{fancyhdr} % Headers and footers
\pagestyle{fancy} % All pages have headers and footers
\fancyhead{} % Blank out the default header
\fancyfoot{} % Blank out the default footer
\fancyhead[L]{ЦРОД $\bullet$ Математика}
\fancyhead[C]{\textit{Тест}}
\fancyhead[R]{Стратегия 2021}% Custom header text


%----------------------------------------------------------------------------------------

%\begin{document}\normalsize
\begin{document}\large
	
	
\begin{center}
	\textbf{Тест}
\end{center}

Вам предлагается 5 задач по 5 темам, которые мы успели пройти. Время выполнения работы 90 минут. Писать нужно полные решения. Ответ без решения оценивается 0 баллов. Успехов!

\begin{enumerate}

\item Две команды сыграли друг с другом серию из $12$ футбольных матчей. Оказалось, что каждая из команд набрала при этом по $17$ очков (за победу дается 3 очка, за ничью 1 очко, за поражение 0 очков). Сколько из матчей закончились ничьей?
%2 ничьи

\item На сторонах $AB$ и $AC$ треугольника $ABC$ внешним образом построены правильные треугольники $A_1BC$, и $AB_1C$. Найдите , что угол между прямыми $AA_1$ и $BB_1$.
%60

\item Найдите остаток при делении $x + y + z$ на 19, если $20x + 6 - 18z \equiv 29 - y$ (mod 19)
%4

\item Вася расставил по кругу 40 красных, 30 синих и 20 зелёных фишек так, что фишки каждого цвета идут подряд. За ход он может поменять местами стоящие рядом синюю и красную фишки, или стоящие рядом красную и зелёную фишки. Можно ли за несколько таких операций добиться того, чтобы любые две фишки, которые стоят рядом были разных цветов?
%Нельзя

\item Найдите сумму всех коэффициентов многочлена $$(2x^2 - 99x + 98)^{123} + (10x^4 - 22x + 11)^{321}$$ после раскрытия скобок и приведения подобных членов.


\end{enumerate}
\end{document}