\documentclass{article}
\usepackage[12pt]{extsizes}
\usepackage[T2A]{fontenc}
\usepackage[utf8]{inputenc}
\usepackage[english, russian]{babel}

\usepackage{amssymb}
\usepackage{amsfonts}
\usepackage{amsmath}
\usepackage{enumitem}
\usepackage{graphics}

\usepackage{lipsum}



\usepackage{geometry} % Меняем поля страницы


%----------------------------------------------------------------------------------------

%\begin{document}\normalsize
\begin{document}\large
	

\begin{center}
	\textbf{ПPАВИЛА МАТЕМАТИЧЕСКОГО БОЯ}
\end{center}
\begin{center}
	\textbf{Общие положения}
\end{center}

Математический бой - соревнование двух команд в решении математических задач. Сначала команды получают условия задач и определенное время на их решение. При решении задач команда может использовать любую литературу, но не имеет права общаться по поводу решения этих задач ни с кем, кроме членов жюри. По истечении отведенного времени начинается собственно бой, когда команды рассказывают друг другу решения задач в соответствии с данными правилами.

Если одна из команд рассказывает решение, то другая выступает в качестве оппонента, то есть, ищет в нем ошибки (недочеты). Выступления оппонента и докладчика оцениваются жюри в баллах. Если команды, обсудив предложенное решение, все-таки не решили задачу до конца или не обнаружили допущенные ошибки, то часть баллов (или даже все) может забpать себе жюри боя. Победителем боя объявляется команда, которая в итоге наберет большее количество баллов. Если по окончании боя результаты команд отличаются не более чем на три балла, то принято считать, что бой закончился вничью. Если по каким-то причинам бой не может закончиться вничью, то жюри объявляет это командам до боя и оглашает процедуру определения победителя.

\begin{center}
\textbf{Общая схема боя}
\end{center}

Бой состоит из нескольких pаундов. В начале каждого раунда (если не происходит отказа от вызова - см. пункт "Окончание боя") одна из команд вызывает дpугую на одну из задач, pешение котоpой еще не pассказывалось (например: "Мы вызываем команду сопеpников на задачу номеp 8"). После этого, вызванная команда сообщает, пpинимает ли она вызов, то есть, согласна ли она рассказывать решение этой задачи (на решение о принятии вызова отводится не более одной минуты). Если команда принимает вызов, то она выставляет докладчика, котоpый должен pассказать pешение, а вызвавшая команда выставляет оппонента, обязанность котоpого - искать ошибки в представленном решении. Если вызов не принят, то команда, которая вызывала, обязана выставить докладчика, а команда, отклонившая вызов, выставляет оппонента. В этом случае говорят, что происходит проверка корректности вызова.

\begin{center}
\textbf{Конкурс капитанов}
\end{center}

Кто будет делать первый вызов, определяет команда, победившая в конкурсе капитанов. Он проводится в начале боя. Капитанам предлагается задача. Капитан, первым сообщивший жюри о своем желании отвечать, получает такое право. Если он рассказывает правильное решение, то он победил, а если неправильное - победил его соперник. При этом что понимается под "правильным решением": просто верный ответ, ответ с объяснением и т. п. - жюри уточняет перед началом конкурса капитанов.

На решение задачи конкурса капитанов жюри отводит определенное время. Если за это время ни один из капитанов не высказал желания отвечать, жюри может заменить задачу или выявить победителя жребием. Вместо задачи жюри может предложить капитанам сыграть в какую-либо игру. Возможны и другие схемы проведения конкурса капитанов. Жюри боя заранее определяет способ проведения конкурса капитанов и сообщает о нем командам перед началом боя.

Команда имеет право выставить на конкурс капитанов любого члена команды.
\newpage
\begin{center}
\textbf{Ход раунда}
\end{center}

Доклад. В начале раунда докладчик рассказывает решение у доски. Доклад должен содержать ответы на все поставленные в задаче вопросы и доказательство правильности и полноты полученных ответов. В частности, докладчик обязан доказать каждое сформулированное им промежуточное утверждение либо сослаться на него, как на общеизвестное. Докладчик должен стремиться к ясности изложения, в частности, он обязан повторить по просьбе оппонента или жюри любую часть своего доклада. Время на доклад ограничено 15 минутами, по истечении которых доклад может быть продолжен только с разрешения жюри.

Докладчик может иметь при себе бумагу с чертежами и (с отдельного разрешения жюри) вычислениями, но не имеет права брать с собой текст решения.

Докладчик имеет право: 
- до начала выступления вынести на доску всю необходимую ему информацию; 
- не отвечать на вопросы оппонента, заданные до начала обсуждения; 
- просить оппонента уточнить свой вопрос (в частности, докладчик может предложить свою версию вопроса: "Правильно ли я понимаю, что вы спросили о том-то и том-то?"); 
- отказаться отвечать на вопрос, сказав, что: а) он не имеет ответа на этот вопрос; б) он уже ответил на этот вопрос (объяснив, когда и как); в) вопрос некорректен или выходит за рамки научной дискуссии по поставленной задаче. В случае несогласия оппонента с основаниями б) и в) арбитром в споре выступает жюри.

Докладчик не обязан: 
- излагать способ получения ответа, если он может доказать его правильность и полноту; 
- сравнивать свой метод решения с другими возможными методами, в том числе с точки зрения краткости, красоты и пригодности для решения других задач.

Оппонирование. Пока доклад не окончен, оппонент может задавать вопросы только с согласия докладчика, но имеет право попросить повторить часть решения. Он может разрешить докладчику не доказывать какие-либо очевидные факты (со своей точки зрения). После окончания доклада оппонент имеет право задавать вопросы докладчику. Если в течение минуты оппонент не задал ни одного вопроса, то считается, что вопросов у него нет. Если докладчик не начинает отвечать на вопрос в течение минуты, то считается, что у него нет ответа.

В качестве вопроса оппонент может: 
- потребовать повторить любую часть доклада; 
- попросить уточнения любого из высказываний докладчика, в том числе: а) попросить дать определение любого термина ("Что Вы понимаете под ..."); б) переформулировать утверждение докладчика своими словами и попросить подтверждения ("Правильно ли я понимаю, что Вы утверждаете следующее: ... "); 
- попросить доказать сформулированное докладчиком утверждение, если оно не является очевидным или общеизвестным (в спорных случаях, вопрос об известности или очевидности решает жюри; во всяком случае, известными считаются факты, включенные в общеобразовательную программу по математике); 
- после ответа на вопрос выразить удовлетворенность или мотивированную неудовлетворенность ответом.

Если оппонент считает, что докладчик тянет время, придумывая решение у доски, или что существенная часть доклада не является изложением решения обсуждаемой задачи, он имеет право (но не ранее, чем через 10 минут после начала доклада) попросить докладчика предъявить ответ (если таковой в задаче подразумевается) или план дальнейших рассуждений.

Докладчик и оппонент обязаны: 
- высказываться в вежливой и корректной форме, обращаясь к друг другу на "Вы"; 
- критикуя высказывания друг друга не "переходить на личности"; 
- повторять и уточнять свои вопросы и ответы по просьбе друг друга или жюри.

По итогам доклада и ответов на вопросы оппонент имеет право дать свою оценку докладу и обсуждению в одной из следующих форм: а) признать решение правильным; б) признать решение (ответ) в основном правильным, но имеющим недостатки и (или) пробелы с обязательным их указанием; в) признать решение (ответ) неправильным, указав ошибки в обоснованиях ключевых утверждений доклада, или приведя контрпример, или указав существенные пробелы в обоснованиях или плане решения. Если оппонент согласился с решением, то он и его команда в этом раунде больше не участвуют.

Если оппонент имеет контрпример, опровергающий решение докладчика в целом, и этот контрпример сам является решением задачи (такое бывает, например, в случаях, когда вопрос задачи звучит как "Можно ли ...?", "Верно ли, что ...?" и т. п.), то оппонент имеет право заявить: "Я с решением не согласен, у меня есть контрпример", но сам контрпример пока докладчику не предъявлять (жюри имеет право потребовать предъявления контрпримера в письменном виде, чтобы убедиться в корректности заявления оппонента). В этом случае, если докладчик не изменит своего решения в течение минуты или после взятого командой перерыва, оппонент получает право предъявить докладчику упомянутый контрпример, причем докладчик и его команда уже не имеют права изменять решение или ответ.

Аналогично, если решение требует перебора случаев, оппонент имеет право заявить "Я с решением не согласен, рассмотрены не все случаи", не указывая докладчику, какой именно случай не рассмотрен. Дальнейшие действия докладчика, жюри и оппонента такие же, как в ситуации с контр примером.

Участие жюри в обсуждении. После окончания диалога докладчика и оппонента жюри задает свои вопросы. При необходимости, оно имеет право вмешаться и ранее, во время диалога докладчика и оппонента.

Выступающие и команда. Докладчик и оппонент могут обращаться к своим капитанам с просьбой о замене или перерыве для консультации. Другое общение между командой и докладчиком (оппонентом) допускается только во время полуминутного перерыва, который любая команда может взять в любой момент (при этом соперники также могут пользоваться этим временем). Каждая команда может взять в течение одного боя не более шести полуминутных перерывов (см. пункт "Количество выходов к доске").

Перемена ролей. Перемена ролей в раунде может произойти только в том случае, если вызов в этом раунде был принят. Если оппонент доказал, что у докладчика нет решения (так ли это, решает жюри, см. пункт "Начисление баллов") то оппонент получает право (но не обязан) рассказать свое решение. Если оппонент взялся рассказывать свое решение, то происходит полная перемена ролей, то есть, бывший докладчик становится оппонентом. Если же оппонент не доказал, что у докладчика нет решения, но выявил в предложенном решении некоторые конкретные недостатки, то он получает право (но не обязан) устранить все (или некоторые) из этих недостатков ("залатать дыры"). Такое же право оппонент получает, если он доказал, что у докладчика решения нет, но собственное решение рассказывать отказался. Если оппонент взялся "латать дыры", то происходит частичная перемена ролей: оппонент формулирует, что именно он собирается делать (например: разбирать случай, не разобранный докладчиком; доказывать утверждение, недоказанное докладчиком; и т. п.), а бывший докладчик ему оппонирует.

Обратной перемены ролей не происходит ни в каком случае!

Корректность вызова. Если вызов принят, то вопрос о его корректности не ставится, то есть, принятый вызов всегда считается корректным!

Если вызов не принят, то возможны два случая: 
а) вызывавшая команда также отказалась отвечать, и тогда, вызов "автоматически" признается некорректным; 
б) вызывавшая команда выставила докладчика, тогда корректность вызова зависит от дальнейшего хода раунда, а именно, вызов признается некорректным, если оппоненту удается доказать, что задача не решена. В случае, если оппонент признал задачу решенной, вызов "автоматически" признается корректным!

\begin{center}
\textbf{Количество выходов к доске}
\end{center}

Каждый член команды имеет право выйти к доске в качестве докладчика или оппонента не более двух раз за бой. Команда имеет право не более трех раз за бой заменять докладчика или оппонента, причем в каждом таком случае выход засчитывается обоим членам команды. При каждой замене, время, отведенное команде на перерывы, уменьшается на одну минуту (эту минуту можно как использовать непосредственно перед заменой, так и не использовать. В последнем случае команда соперников тоже не имеет права ее использовать).

\begin{center}
\textbf{Порядок вызовов. Окончание боя}
\end{center}

В случае, если вызов был признан некорректным, команда должна в следующем раунде повторить вызов. Во всех остальных случаях команды вызывают друг друга поочередно.

В любой момент боя та команда, которая должна вызывать, может отказаться делать это (обычно это происходит, когда у команды больше нет решенных задач, а делать вызов, который может оказаться некорректным, она не рискует). Тогда, другая команда получает право (но не обязана) рассказать решения оставшихся задач. При этом команда, отказавшаяся делать вызов, может выставлять оппонентов и получать баллы только за оппониpование, но рассказывать решения она уже не имеет права (то есть после отказа от вызова не происходит ни полной, ни частичной перемены ролей). Бой заканчивается, когда все задачи обсуждены или когда одна из команд отказалась от вызова, а другая команда отказалась рассказывать решения оставшихся задач.

\begin{center}
\textbf{Начисление баллов}
\end{center}

Каждая задача оценивается в 12 баллов, которые по итогам раунда распределяются между докладчиком, оппонентом и жюри. Если докладчик рассказал правильное и полное решение, все 12 баллов достаются ему. Если оппонент сумел найти в решении более или менее существенные ошибки, жюри прежде всего решает вопрос о том, удалось ли оппоненту доказать, что задача докладчиком не решена. Если это оппоненту не удалось, то он, тем не менее, может получить баллы за оппонирование в зависимости от серьезности указанных недочетов и от того, насколько докладчику (или оппоненту, если произошла частичная перемена ролей) удалось их исправить. Как правило, оппонент получает половину "стоимости" не "залатанных" докладчиком "дыр" в решении (принцип "половины"), но, если докладчик сумел изложить полное решение только после существенных наводящих вопросов оппонента и (или) жюри ("грязь" в решении), то жюри может отобрать у докладчика не более двух баллов и передать их оппоненту или оставить себе. Если же произошла частичная перемена ролей, то бывший оппонент получает дополнительно баллы за доказательство сформулированных им предварительно утверждений, а бывший докладчик - за их оппонирование (при этом "стоимость" рассматриваемых утверждений определяет жюри, а распределение баллов происходит так же, как при оппонировании полного решения - с учетом принципа "половины" и "грязи" в рассуждениях). Остальные баллы распределяются между докладчиком и жюри, и раунд заканчивается. Если же оппонент сумел доказать, что решения у докладчика нет, он получает баллы за оппонирование (с учетом принципа "половины") и, если вызов был принят, право рассказать свое решение (см. пункт "Перемена ролей"). Если при этом происходит полная или частичная перемена ролей, то начисление баллов происходит по схеме, изложенной выше.

Если ошибки или пробелы в докладе указаны самим докладчиком и не устранены его командой, то оппонент получает за них баллы так, как если бы он нашел эти недостатки сам. В частности, если, получив отказ от вызова, капитан вызывающей команды сразу признается, что у его команды нет решения, команда соперников получает 6 баллов за оппониpование (которое в этом случае состоит из одной фразы: "У Вас нет решения"), а вызов признается некорректным. Докладчик и оппонент в этом случае не назначаются и выходы к доске не засчитываются.

\begin{center}
\textbf{Капитан}
\end{center}

Во время боя только капитан может от имени команды обращаться к жюри и соперникам: сообщать о вызове или отказе, просить перерыв и т. д. Он имеет право в любой момент прекратить доклад или оппонирование представителя своей команды. Если капитан у доски, он оставляет за себя заместителя, исполняющего в это время обязанности капитана. Имена капитана и заместителя сообщаются жюри до начала боя.

Во время решения задач главная обязанность капитана - координировать действия членов команды так, чтобы имеющимися силами решить как можно больше задач. Для этого капитан с учетом пожеланий членов команды распределяет между ними задачи для решения, следит, чтобы каждая задача кем-то решалась, организует проверку найденных решений. Капитан заранее выясняет, кто будет докладчиком или оппонентом по той или иной задаче, определяет тактику команды на предстоящем бое.

Капитаны команд имеют право попросить жюри о предоставлении в ходе боя перерывов на 5-10 минут (примерно через каждые полтора часа). Перерыв может предоставляться только между раундами. При этом команда, которая должна сделать вызов, делает его в письменной форме (без оглашения) непосредственно перед началом перерыва и сдает жюри, которое оглашает этот вызов сразу после окончания перерыва.

\begin{center}
\textbf{Жюри}
\end{center}

Жюри является верховным толкователем правил боя. В случаях, не предусмотренных правилами, оно принимает решение по своему усмотрению. Решения жюри являются обязательными для команд.

Во время решения командами задач всякое существенное разъяснение условий задач, данное одной из команд, должно быть в кратчайшее время сообщено всем остальным командам.

Жюри может снять вопрос оппонента (например, если он не по существу), прекратить доклад или оппониpование, если они затягиваются. Если жюри не может быстро разобраться в решении, оно может с согласия обоих капитанов выделить своего представителя, который продолжит обсуждение задачи совместно с докладчиком и оппонентом в другом помещении. При этом бой продолжается по другим задачам, а очки по этой задаче начисляются позже.

Жюри ведет на доске протокол боя. Если одна из команд не согласна с принятым жюри решением по задаче, она имеет право немедленно потребовать перерыв на несколько минут для разбора ситуации с участием председателя жюри. После начала следующего раунда счет предыдущего раунда изменен быть не может.

Жюри следит за порядком. Оно может оштрафовать команду за шум, некорректное поведение, общение со своим представителем, находящимся у доски.

Жюри обязано мотивировать все свои решения, не вытекающие непосредственно из правил боя. 


\end{document}