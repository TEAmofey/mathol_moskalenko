\documentclass{article}
\usepackage[12pt]{extsizes}
\usepackage[T2A]{fontenc}
\usepackage[utf8]{inputenc}
\usepackage[english, russian]{babel}

\usepackage{amssymb}
\usepackage{amsfonts}
\usepackage{amsmath}
\usepackage{enumitem}
\usepackage{graphics}

\usepackage{lipsum}



\usepackage{geometry} % Меняем поля страницы


\geometry{top=1.5cm}% верхнее поле
\geometry{bottom=1cm}% нижнее поле


\usepackage{fancyhdr} % Headers and footers
\pagestyle{fancy} % All pages have headers and footers
\fancyhead{} % Blank out the default header
\fancyfoot{} % Blank out the default footer
\fancyhead[L]{ЦРОД $\bullet$ Математика}
\fancyhead[C]{\textit{Битва}}
\fancyhead[R]{Стратегия 2021}% Custom header text


%----------------------------------------------------------------------------------------

%\begin{document}\normalsize
\begin{document}\large
	

\begin{center}
\textbf{Математический бой за славу и честь!}
\end{center}

\begin{enumerate}

\item Существует ли такое натуральное число $N$, что и у числа $N$, и у числа $N^2 - 1$ сумма цифр равна $2021$

\item Существует ли такая непериодическая функция $f$, что при любом действительном $x$ выполнено равенство $$f (x + 1) = f (x + 1)f (x) + 1$$

\item По кругу сидит $2021$ человек, каждый из которых либо рыцарь, который всегда говорит правду, либо лжец, который всегда лжет. Каждый из них сказал: «Если моего соседа справа спросить, кем является мой сосед слева, то он ответит — лжецом.». Сколько лжецов за этим столом?

\item Найдите все квадратные трехчлены $x^2 + mx + n$ такие, что числа $m$ и $n$ (не обязательно различные) являются их корнями?

\item Докажите, что не может одновременно выполняться для положительных $a,b,c$
$$a(1-b)>1/4$$
$$b(1-c)>1/4 $$
$$c(1-a)>1/4$$

\item Точка $H$ — ортоцентр остроугольного треугольника $ABC$, в котором $AB > AC$. Точка $E$ симметрична $C$ относительно высоты $AH$. $F$ — точка пересечения прямых $EH$ и $AC$. Докажите, что центр описанной окружности треугольника $AEF$ лежит на прямой $AB$.

\item Из клетчатого квадрата $2021 \times 2021$ вырезали угловой квадрат $6 \times 6$. Можно ли оставшуюся фигуру разрезать на прямоугольники $1 \times 5$?

\item На первое занятие по олимпиадной математике пришло 7 человек, которые сели по кругу. При этом, если выбрать любых 6 человек, то возможна рассадка, при которой все будут сидеть рядом со своим знакомым. Верно ли, что и вся группа из 7 человек может быть так рассажена? 

\item В трапеции $ABCD$ с основаниями $AB$ и $CD$ выполнено равенство $AB = BD + CD$. Пусть $M$ — середина диагонали $AC$. Докажите, что $\angle BMD = 90^\circ$.



\end{enumerate}
\end{document}