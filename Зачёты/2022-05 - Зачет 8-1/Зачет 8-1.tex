\documentclass{article}

\usepackage[12pt]{extsizes}
\usepackage[T2A]{fontenc}
\usepackage[utf8]{inputenc}
\usepackage[english, russian]{babel}

\usepackage{mathrsfs}
\usepackage[dvipsnames]{xcolor}

\usepackage{amsmath}
\usepackage{amssymb}
\usepackage{amsthm}
\usepackage{indentfirst}
\usepackage{amsfonts}
\usepackage{enumitem}
\usepackage{graphics}
\usepackage{tikz}
\usepackage{tabu}
\usepackage{diagbox}
\usepackage{hyperref}
\usepackage{mathtools}
\usepackage{ucs}
\usepackage{lipsum}
\usepackage{geometry} % Меняем поля страницы
\usepackage{fancyhdr} % Headers and footers
\newcommand{\range}{\mathrm{range}}
\newcommand{\dom}{\mathrm{dom}}
\newcommand{\N}{\mathbb{N}}
\newcommand{\R}{\mathbb{R}}
\newcommand{\E}{\mathbb{E}}
\newcommand{\D}{\mathbb{D}}
\newcommand{\M}{\mathcal{M}}
\newcommand{\Prime}{\mathbb{P}}
\newcommand{\A}{\mathbb{A}}
\newcommand{\Q}{\mathbb{Q}}
\newcommand{\Z}{\mathbb{Z}}
\newcommand{\F}{\mathbb{F}}
\newcommand{\CC}{\mathbb{C}}

\DeclarePairedDelimiter\abs{\lvert}{\rvert}
\DeclarePairedDelimiter\floor{\lfloor}{\rfloor}
\DeclarePairedDelimiter\ceil{\lceil}{\rceil}
\DeclarePairedDelimiter\lr{(}{)}
\DeclarePairedDelimiter\set{\{}{\}}
\DeclarePairedDelimiter\norm{\|}{\|}

\renewcommand{\labelenumi}{(\alph{enumi})}

\newcommand{\smallindent}{
    \geometry{left=1cm}% левое поле
    \geometry{right=1cm}% правое поле
    \geometry{top=1.5cm}% верхнее поле
    \geometry{bottom=1cm}% нижнее поле
}

\newcommand{\header}[3]{
    \pagestyle{fancy} % All pages have headers and footers
    \fancyhead{} % Blank out the default header
    \fancyfoot{} % Blank out the default footer
    \fancyhead[L]{#1}
    \fancyhead[C]{#2}
    \fancyhead[R]{#3}
}

\newcommand{\dividedinto}{
    \,\,\,\vdots\,\,\,
}

\newcommand{\littletaller}{\mathchoice{\vphantom{\big|}}{}{}{}}

\newcommand\restr[2]{{
    \left.\kern-\nulldelimiterspace % automatically resize the bar with \right
    #1 % the function
    \littletaller % pretend it's a little taller at normal size
    \right|_{#2} % this is the delimiter
}}

\DeclareGraphicsExtensions{.pdf,.png,.jpg}

\newenvironment{enumerate_boxed}[1][enumi]{\begin{enumerate}[label*=\protect\fbox{\arabic{#1}}]}{\end{enumerate}}



\smallindent

\header{ЦРОД $\bullet$ Математика}{\textit{Зачёт}}{Май 2022}

%----------------------------------------------------------------------------------------

%\begin{document}\normalsize
\begin{document}
    \large

    \begin{center}
        \textbf{Группа 8--1}
    \end{center}

    \begin{center}
        \textbf{Ортоцентр}
    \end{center}

    \begin{enumerate_boxed}

        \item Докажите, что $\angle ABH = \angle CBO$.

        \item Докажите, что $\angle ABH = \angle H_{c}H_{a}H$.

        \item Докажите, что $H_{a}A$~--- биссектриса $ \angle H_{c}H_{a}H_b$.

        \item Докажите, что $O$~--- ортоцентр треугольника $M_{a}M_{b}M_c$.

    \end{enumerate_boxed}


    \begin{center}
        \textbf{Углы}
    \end{center}

    \begin{enumerate_boxed}

        \item Даны две окружности, пересекающиеся в точках $X$ и $Y$.
        Прямая, проходящая через $X$, пересекает первую окружность в точке $A$, а вторую — в точке $C$.
        Другая прямая, проходящая через $Y$, первую окружность пересекает в точке $B$, а вторую — в точке $D$.
        Докажите, что $AB \parallel CD$.

        \item В окружность вписан шестиугольник.
        Найдите сумму углов при трёх его несоседних вершинах.

        \item Окружности с центрами $O_1$ и $O_2$ пересекаются в точках $A$ и $B$.
        Луч $O_{2}A$ пересекает первую окружность в точке $C$.
        Докажите, что точки $O_1$, $O_2$, $B$, $C$ лежат на одной окружности.

        \item Докажите, что в равнобедренной трапеции вершины боковой стороны, точка пересечения диагоналей и центр описанной окружности лежат на одной окружности.

    \end{enumerate_boxed}


    \begin{center}
        \textbf{Углы-2}
    \end{center}

    \begin{enumerate_boxed}

        \item Даны два угла $\angle ABC = 90^\circ$ и $\angle ADC = 90^\circ$.
        Докажите, что $A, B, C, D$ лежат на одной окружности.

        \item Дан треугольник $ABC$. $I$ - центр вписанной окружности.
        Докажите (и запомните), что $\angle AIB = 90^\circ + \frac{\angle A}{2}$

        \item Дан треугольник $ABC$. $H$ - ортоцентр (точка пересечения высот).
        Докажите (и запомните), что $\angle AHB = 180^\circ - \angle C$

        \item Дан треугольник $ABC$. $BH_1$, $CH_2$ - высоты треугольника.
        Докажите, что $C,B,H_1, H_2$ лежат на одной окружности.

    \end{enumerate_boxed}
\end{document}