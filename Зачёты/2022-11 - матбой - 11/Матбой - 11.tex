\documentclass{article}

\usepackage[12pt]{extsizes}
\usepackage[T2A]{fontenc}
\usepackage[utf8]{inputenc}
\usepackage[english, russian]{babel}

\usepackage{mathrsfs}
\usepackage[dvipsnames]{xcolor}

\usepackage{amsmath}
\usepackage{amssymb}
\usepackage{amsthm}
\usepackage{indentfirst}
\usepackage{amsfonts}
\usepackage{enumitem}
\usepackage{graphics}
\usepackage{tikz}
\usepackage{tabu}
\usepackage{diagbox}
\usepackage{hyperref}
\usepackage{mathtools}
\usepackage{ucs}
\usepackage{lipsum}
\usepackage{geometry} % Меняем поля страницы
\usepackage{fancyhdr} % Headers and footers
\newcommand{\range}{\mathrm{range}}
\newcommand{\dom}{\mathrm{dom}}
\newcommand{\N}{\mathbb{N}}
\newcommand{\R}{\mathbb{R}}
\newcommand{\E}{\mathbb{E}}
\newcommand{\D}{\mathbb{D}}
\newcommand{\M}{\mathcal{M}}
\newcommand{\Prime}{\mathbb{P}}
\newcommand{\A}{\mathbb{A}}
\newcommand{\Q}{\mathbb{Q}}
\newcommand{\Z}{\mathbb{Z}}
\newcommand{\F}{\mathbb{F}}
\newcommand{\CC}{\mathbb{C}}

\DeclarePairedDelimiter\abs{\lvert}{\rvert}
\DeclarePairedDelimiter\floor{\lfloor}{\rfloor}
\DeclarePairedDelimiter\ceil{\lceil}{\rceil}
\DeclarePairedDelimiter\lr{(}{)}
\DeclarePairedDelimiter\set{\{}{\}}
\DeclarePairedDelimiter\norm{\|}{\|}

\renewcommand{\labelenumi}{(\alph{enumi})}

\newcommand{\smallindent}{
    \geometry{left=1cm}% левое поле
    \geometry{right=1cm}% правое поле
    \geometry{top=1.5cm}% верхнее поле
    \geometry{bottom=1cm}% нижнее поле
}

\newcommand{\header}[3]{
    \pagestyle{fancy} % All pages have headers and footers
    \fancyhead{} % Blank out the default header
    \fancyfoot{} % Blank out the default footer
    \fancyhead[L]{#1}
    \fancyhead[C]{#2}
    \fancyhead[R]{#3}
}

\newcommand{\dividedinto}{
    \,\,\,\vdots\,\,\,
}

\newcommand{\littletaller}{\mathchoice{\vphantom{\big|}}{}{}{}}

\newcommand\restr[2]{{
    \left.\kern-\nulldelimiterspace % automatically resize the bar with \right
    #1 % the function
    \littletaller % pretend it's a little taller at normal size
    \right|_{#2} % this is the delimiter
}}

\DeclareGraphicsExtensions{.pdf,.png,.jpg}

\newenvironment{enumerate_boxed}[1][enumi]{\begin{enumerate}[label*=\protect\fbox{\arabic{#1}}]}{\end{enumerate}}



\header{ЦРОД $\bullet$ Математика}{\textit{Битва}}{4 ноября 2022}


%----------------------------------------------------------------------------------------

\begin{document}
    \large


    \begin{center}
        \textbf{$\mathbb{M}$@т$e$м@т$i4e\mathbb{C}$кий boy!}
    \end{center}

    \begin{enumerate}[label*=\arabic{enumi}.]

        \item Однажды утром Гера очень громко сообщил всем, что нашел наименьшее, состоящее из чётных цифр, число,
        делящееся на 99.
        Какое число у него получилось?

        \item В треугольнике $ABC$ сторона $AC$ длиннее стороны $AB$.
        Вписанная окружность с центром $I$ касается сторон $BC, CA$ и $AB$ в точках $D$, $E$ и $F$ соответственно.
        Прямая $AI$ пересекает прямые $DE$ и $DF$ в точках $X$ и $Y$ соответственно. $Z$ --- основание перпендикуляра, опущенного из $A$ на $BC$.
        Докажите, что $D$ --- центр вписанной окружности треугольника $XYZ$.
%\item Найдите все тройки натуральных попарно взаимнопростых чисел $a,b,c$ такие, что $(a+b+c)(\frac{1}{a} + \frac{1}{b} + \frac{1}{c})$

        \item На доске в аудитории у 9 классов Глеб Мангукович написал и забыл стереть уравнение
        $x^2 + ax + b = 0$, где коэффициенты $a$ и $b$ целые и не равны 0.
        Его ученики, проходя мимо доски, стирали старое уравнение и писали новое такого же вида так, чтобы корнями нового уравнения были коэффициенты старого.
        В какой-то момент новое составленное уравнение совпало с тем, что было написано на доске изначально Глебом Мангуковичем.
        Какое уравнение изначально было написано на доске?

        \item Перед матбоем составителям задач очень захотелось доказать для положительных чисел $a,b,c$ неравенство
        \[\left(\frac{a}{b+c} + \frac{1}{2}\right)\left(\frac{b}{c+a} + \frac{1}{2}\right)\left(\frac{c}{a+b} + \frac{1}{2}\right) \geqslant 1\]
        Они справились всего за 10 минут!
        Докажите и вы это неравенство.

        \item Существует ли многочлен $P$ с целыми коэффициентами, у которого \[P(-10) = 145, P(9) = 164,  P(0) = -25\]


        \item У Антона есть 8 монет, про которые он знает только, что 7 из них настоящие и весят одинаково, а одна фальшивая и отличается от настоящей по весу, неизвестно в какую сторону.
        У Тимофея Дмитриевича есть чашечные весы - они показывают, какая чашка тяжелее, но не показывают, насколько.
        За каждое взвешивание Антон платит Тимофею Дмитриевичу (до взвешивания) одну монету из имеющихся у него.
        Но Тимофей Дмитриевич не так прост, как кажется.
        Если уплачена настоящая монета, он сообщит Антону верный результат взвешивания, а если фальшивая, то случайный.
        Антон хочет определить 5 настоящих монет и не отдать ни одну из этих монет Тимофею Дмитриевичу.
        Может ли Антон гарантированно этого добиться?

        \item В выпуклом четырехугольнике $ABCD$ $AD = CD$,  $AB = AC$ и $ \angle ADC = \angle BAC$.
        На сторонах $AB$ и $AD$ соответственно отмечены их середины --- $N$ и $M$.
        Докажите, что $\Delta MNC$ --- равнобедренный.

        \item На паре Анастасия Константиновна написала на доске число 2.
        Миша и Кирилл играют в такую игру.
        Миша записывает на доску число, делящееся на 2, затем Кирилл выписывает число, делящееся на 3, затем Миша --- число, делящееся на 4 и т.д. При этом новое число нужно получить из предыдущего либо дописав одну цифру в конец, либо стерев последнюю цифру предыдущего числа, либо переставив цифры предыдущего числа (оставлять число без изменения нельзя).
        Проигрывает тот, кто не сможет сделать ход.
        Кто выиграет при правильной игре?

    \end{enumerate}
\end{document}