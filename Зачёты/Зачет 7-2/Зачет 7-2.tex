\documentclass{article}
\usepackage[12pt]{extsizes}
\usepackage[T2A]{fontenc}
\usepackage[utf8]{inputenc}
\usepackage[english, russian]{babel}

\usepackage{amssymb}
\usepackage{amsfonts}
\usepackage{amsmath}
\usepackage{enumitem}
\usepackage{graphics}

\usepackage{lipsum}



\usepackage{geometry} % Меняем поля страницы
\geometry{left=1cm}% левое поле
\geometry{right=1cm}% правое поле
\geometry{top=1.5cm}% верхнее поле
\geometry{bottom=1cm}% нижнее поле


\usepackage{fancyhdr} % Headers and footers
\pagestyle{fancy} % All pages have headers and footers
\fancyhead{} % Blank out the default header
\fancyfoot{} % Blank out the default footer
\fancyhead[L]{ЦРОД $\bullet$ Математика}
\fancyhead[C]{\textit{Зачёт}}
\fancyhead[R]{Май 2022}% Custom header text


%----------------------------------------------------------------------------------------

%\begin{document}\normalsize
\begin{document}\large


\begin{center}
\textbf{Группа 7-2}
\end{center}

\begin{center}
	\textbf{Игры}
\end{center}

\begin{enumerate}[label*=\protect\fbox{\arabic{enumi}}]
	
	\item На доске написаны числа от 1 до 10. 2 игрока по очереди вычеркивают по одному числу. Как надо делать ходы, чтобы выиграть в такой игре?
	
	\item Первый называет целое число, затем второй называет ещё одно. Если (a) сумма (b) произведение чисел чётно, выигрывает первый, если нечётно - второй.
	
	\item На столе лежит (a) 25 (b) 24 спичек. Играющие по очереди могут взять от одной до четырёх спичек. Кто не может сделать ход (спичек не осталось), проигрывает. У какого игрока есть выигрышная стратегия?
	
	\item Шоколадка представляет собой прямоугольник $3 \times 5$, разделённый углублениями на 15 квадратиков. Двое по очереди разламывают её на части по углублениям: за один ход можно разломить любой из кусков (больший одного квадратика) на два. Кто не может сделать хода (все куски уже разломаны), проигрывает.
	
\end{enumerate}

\begin{center}
	\textbf{Булева логика}
\end{center}

\begin{enumerate}[label*=\protect\fbox{\arabic{enumi}}]
	
	\item Верны ли утверждения:
	\begin{itemize}
		\item "$2 \times 2 = 5$ или $2 + 2 = 4$";
		
		\item "На поток приедет К.А. Сухов и лжецы иногда говорят правду";
		
		\item "Это утверждение ложно";
	\end{itemize}
	
	\item Cоставьте утверждения, подходящие под формулу:
	
		$A =$ «На полдник выдали сырки»
		
		$B =$ «На полдник выдали печенье»
		
		$C =$ «На полдник не выдали сок или чай» 
		
		$(A\vee B)\wedge \overline{C}$
		
	
	\item Постройте таблицу истинности для выражения $a \oplus b$
	
	\item Постройте отрицание к утверждению: "Я рыцарь или ты лжец".
	
\end{enumerate}


\begin{center}
	\textbf{Рыцари и лжецы}
\end{center}

\begin{enumerate}[label*=\protect\fbox{\arabic{enumi}}]
	
	\item Однажды островитянин Данил сказал: "Вчера мой друг-островитянин сказал, что он лжец". Кем является сам Данил?
	
	\item Однажды встретились два островитянина Саша и Максим. Саша сказал Максиму: "По крайней мере один из нас — лжец". Можно ли только по этой фразе определить, кто кем является?
	
	\item Собрались вместе два рыцаря и два лжеца и посмотрели друг на друга. Кто из них мог сказать фразу: 1) "Cреди нас все рыцари". 2) "Среди вас есть ровно один рыцарь". 3) "Среди вас есть ровно два рыцаря"? Для каждой фразы укажите всех, кто мог ее сказать, и объясните.
	
	\item Один островитянин говорит другому: “Я лжец или ты рыцарь”. Кто из островитян кто?
	
\end{enumerate}


\begin{center}
	\textbf{Турниры}
\end{center}

\begin{enumerate}[label*=\protect\fbox{\arabic{enumi}}]
	
	\item В однокруговом шахматном турнире (каждый играет с каждым ровно 1 раз) участвовало 20 человек. Сколько всего было сыграно партий?
	
	\item 20 команд сыграли турнир по олимпийской системе (тот, кто проиграл - выбывает). Сколько всего было сыграно матчей?
	
	\item В однокруговом шахматном турнире было сыграно 105 партий. Сколько всего участников на этом турнире?
	
	\item Трое друзей играли в шашки. Один из них сыграл 32 игр, а другой – 18 игр. Мог ли третий участник сыграть a) 36; b) 37; c) 58 игр?
	
\end{enumerate}


\begin{center}
	\textbf{Треугольники}
\end{center}

\begin{enumerate}[label*=\protect\fbox{\arabic{enumi}}]
	
	\item Чему равна сумма углов пятиугольника?
	
	\item Внешние углы при вершинах $A$ и $B$ треугольника $ABC$ равны $134^\circ$ и $99^\circ$ соответственно. Чему равна величина внешнего угла при вершине $C$ этого треугольника?
	
	\item В равнобедренном треугольнике один из углов равен $40^\circ$. Чему может быть равна величина наибольшего угла треугольника?
	
	\item В равнобедренном треугольнике один из углов в два раза больше другого. Чему может быть равна величина наименьшего угла этого треугольника?
	
\end{enumerate}

\begin{center}
\textbf{Параллелограммы}
\end{center}

\begin{enumerate}[label*=\protect\fbox{\arabic{enumi}}]

\item  В прямоугольном треугольнике медиана, проведённая к гипотенузе, в два раза меньше гипотенузы.

\item Длина медианы, проведённой к стороне треугольника, меньше полусуммы длин двух других сторон.

\item В равнобедренной трапеции (у которой боковые стороны равны) равны углы, прилежащие к основанию, а также равны длины диагоналей.

\item \textit{Параллелограмм Вариньона}. Докажите, что середины сторон четырёхугольника являются вершинами параллелограмма.

\end{enumerate}

\end{document}