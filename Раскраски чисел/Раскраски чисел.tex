\documentclass{article}

\usepackage[12pt]{extsizes}
\usepackage[T2A]{fontenc}
\usepackage[utf8]{inputenc}
\usepackage[english, russian]{babel}

\usepackage{mathrsfs}
\usepackage[dvipsnames]{xcolor}

\usepackage{amsmath}
\usepackage{amssymb}
\usepackage{amsthm}
\usepackage{indentfirst}
\usepackage{amsfonts}
\usepackage{enumitem}
\usepackage{graphics}
\usepackage{tikz}
\usepackage{tabu}
\usepackage{diagbox}
\usepackage{hyperref}
\usepackage{mathtools}
\usepackage{ucs}
\usepackage{lipsum}
\usepackage{geometry} % Меняем поля страницы
\usepackage{fancyhdr} % Headers and footers
\newcommand{\range}{\mathrm{range}}
\newcommand{\dom}{\mathrm{dom}}
\newcommand{\N}{\mathbb{N}}
\newcommand{\R}{\mathbb{R}}
\newcommand{\E}{\mathbb{E}}
\newcommand{\D}{\mathbb{D}}
\newcommand{\M}{\mathcal{M}}
\newcommand{\Prime}{\mathbb{P}}
\newcommand{\A}{\mathbb{A}}
\newcommand{\Q}{\mathbb{Q}}
\newcommand{\Z}{\mathbb{Z}}
\newcommand{\F}{\mathbb{F}}
\newcommand{\CC}{\mathbb{C}}

\DeclarePairedDelimiter\abs{\lvert}{\rvert}
\DeclarePairedDelimiter\floor{\lfloor}{\rfloor}
\DeclarePairedDelimiter\ceil{\lceil}{\rceil}
\DeclarePairedDelimiter\lr{(}{)}
\DeclarePairedDelimiter\set{\{}{\}}
\DeclarePairedDelimiter\norm{\|}{\|}

\renewcommand{\labelenumi}{(\alph{enumi})}

\newcommand{\smallindent}{
    \geometry{left=1cm}% левое поле
    \geometry{right=1cm}% правое поле
    \geometry{top=1.5cm}% верхнее поле
    \geometry{bottom=1cm}% нижнее поле
}

\newcommand{\header}[3]{
    \pagestyle{fancy} % All pages have headers and footers
    \fancyhead{} % Blank out the default header
    \fancyfoot{} % Blank out the default footer
    \fancyhead[L]{#1}
    \fancyhead[C]{#2}
    \fancyhead[R]{#3}
}

\newcommand{\dividedinto}{
    \,\,\,\vdots\,\,\,
}

\newcommand{\littletaller}{\mathchoice{\vphantom{\big|}}{}{}{}}

\newcommand\restr[2]{{
    \left.\kern-\nulldelimiterspace % automatically resize the bar with \right
    #1 % the function
    \littletaller % pretend it's a little taller at normal size
    \right|_{#2} % this is the delimiter
}}

\DeclareGraphicsExtensions{.pdf,.png,.jpg}

\newenvironment{enumerate_boxed}[1][enumi]{\begin{enumerate}[label*=\protect\fbox{\arabic{#1}}]}{\end{enumerate}}



\smallindent

\header{Математика}{\textit{Разное}}{18 декабря 2022}

%----------------------------------------------------------------------------------------

\begin{document}
    \large

    \begin{center}
        \textbf{Рисование по точкам}
    \end{center}


    \begin{enumerate_boxed}

        \item Можно ли раскрасить в (a) 2 (b) 3 цвета все натуральные числа так, чтобы любые два числа $a, b$, такие что $a = b + 1$ или $a = 2b$, были разноцветные.

        \item Плоскость раскрашена в два цвета, причем каждый цвет использован.
        \begin{enumerate}
            \item Докажите, что найдутся две точки одного цвета, расстояние между которыми не больше 2023 нм.

            \item Докажите, что найдутся две точки одного цвета, расстояние между которыми равно 20232022 м.

            \item Докажите, что найдутся две точки разных цветов, расстояние между которыми равно 1 а.е.
        \end{enumerate}

        \item Все целые числа на прямой раскрашены в 2 цвета.
        Докажите, что найдутся 3 одноцветных числа $a, b, c$, такие, что $b = \dfrac{a + c}{2}$.

        \item На прямой проведено 100 отрезков разной длины.
        Их концами являются 200 попарно различных точек, каждую из них покрасили в синий цвет.
        Могут ли середины всех проведенных отрезков быть синими?

        \item Можно ли раскрасить все натуральные числа, в три цвета (каждое число – в один цвет, все три цвета должны использоваться) так, чтобы цвет сумма любых двух чисел разного цвета отличалась от цвета каждого из слагаемых?

        \item Можно ли раскрасить все натуральные числа, большие 1, в три цвета (каждое число – в один цвет, все три цвета должны использоваться) так, чтобы цвет произведения любых двух чисел разного цвета отличался от цвета каждого из сомножителей?

        \item Бесконечная клетчатая доска раскрашена в три цвета (каждая клеточка — в один из цветов).
        Докажите, что найдутся четыре клеточки одного цвета, расположенные в вершинах прямоугольника со сторонами, параллельными стороне одной клеточки.

    \end{enumerate_boxed}
\end{document}