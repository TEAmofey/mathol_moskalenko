\documentclass{article}

\usepackage[12pt]{extsizes}
\usepackage[T2A]{fontenc}
\usepackage[utf8]{inputenc}
\usepackage[english, russian]{babel}

\usepackage{mathrsfs}
\usepackage[dvipsnames]{xcolor}

\usepackage{amsmath}
\usepackage{amssymb}
\usepackage{amsthm}
\usepackage{indentfirst}
\usepackage{amsfonts}
\usepackage{enumitem}
\usepackage{graphics}
\usepackage{tikz}
\usepackage{tabu}
\usepackage{diagbox}
\usepackage{hyperref}
\usepackage{mathtools}
\usepackage{ucs}
\usepackage{lipsum}
\usepackage{geometry} % Меняем поля страницы
\usepackage{fancyhdr} % Headers and footers
\newcommand{\range}{\mathrm{range}}
\newcommand{\dom}{\mathrm{dom}}
\newcommand{\N}{\mathbb{N}}
\newcommand{\R}{\mathbb{R}}
\newcommand{\E}{\mathbb{E}}
\newcommand{\D}{\mathbb{D}}
\newcommand{\M}{\mathcal{M}}
\newcommand{\Prime}{\mathbb{P}}
\newcommand{\A}{\mathbb{A}}
\newcommand{\Q}{\mathbb{Q}}
\newcommand{\Z}{\mathbb{Z}}
\newcommand{\F}{\mathbb{F}}
\newcommand{\CC}{\mathbb{C}}

\DeclarePairedDelimiter\abs{\lvert}{\rvert}
\DeclarePairedDelimiter\floor{\lfloor}{\rfloor}
\DeclarePairedDelimiter\ceil{\lceil}{\rceil}
\DeclarePairedDelimiter\lr{(}{)}
\DeclarePairedDelimiter\set{\{}{\}}
\DeclarePairedDelimiter\norm{\|}{\|}

\renewcommand{\labelenumi}{(\alph{enumi})}

\newcommand{\smallindent}{
    \geometry{left=1cm}% левое поле
    \geometry{right=1cm}% правое поле
    \geometry{top=1.5cm}% верхнее поле
    \geometry{bottom=1cm}% нижнее поле
}

\newcommand{\header}[3]{
    \pagestyle{fancy} % All pages have headers and footers
    \fancyhead{} % Blank out the default header
    \fancyfoot{} % Blank out the default footer
    \fancyhead[L]{#1}
    \fancyhead[C]{#2}
    \fancyhead[R]{#3}
}

\newcommand{\dividedinto}{
    \,\,\,\vdots\,\,\,
}

\newcommand{\littletaller}{\mathchoice{\vphantom{\big|}}{}{}{}}

\newcommand\restr[2]{{
    \left.\kern-\nulldelimiterspace % automatically resize the bar with \right
    #1 % the function
    \littletaller % pretend it's a little taller at normal size
    \right|_{#2} % this is the delimiter
}}

\DeclareGraphicsExtensions{.pdf,.png,.jpg}

\newenvironment{enumerate_boxed}[1][enumi]{\begin{enumerate}[label*=\protect\fbox{\arabic{#1}}]}{\end{enumerate}}


\usepackage[framemethod=TikZ]{mdframed}

\newcommand{\definebox}[3]{%
    \newcounter{#1}
    \newenvironment{#1}[1][]{%
        \stepcounter{#1}%
        \mdfsetup{%
            frametitle={%
            \tikz[baseline=(current bounding box.east),outer sep=0pt]
            \node[anchor=east,rectangle,fill=white]
            {\strut #2~\csname the#1\endcsname\ifstrempty{##1}{}{##1}};}}%
        \mdfsetup{innertopmargin=1pt,linecolor=#3,%
            linewidth=3pt,topline=true,
            frametitleaboveskip=\dimexpr-\ht\strutbox\relax,}%
        \begin{mdframed}[]
            \relax%
            }{
        \end{mdframed}}%
}

\definebox{theorem_boxed}{Теорема}{ForestGreen!24}
\definebox{definition_boxed}{Определение}{blue!24}
\definebox{task_boxed}{Задача}{orange!24}
\definebox{paradox_boxed}{Парадокс}{red!24}

\theoremstyle{plain}
\newtheorem{theorem}{Теорема}
\newtheorem{task}{Задача}
\newtheorem{lemma}{Лемма}
\newtheorem{statement}{Утверждение}
\newtheorem{corollary}{Следствие}

\theoremstyle{remark}
\newtheorem{remark}{Замечание}
\newtheorem{example}{Пример}

\smallindent

\header{Математика}{\textit{Топология}}{}

%----------------------------------------------------------------------------------------


\begin{document}
    \large

    \setcounter{task_boxed}{0}


    \section{Топология}
    \begin{definition_boxed}
        \textit{\textbf{Топологическое пространство}}~--- это пара $\lr*{X, \Omega}$, где $\Omega \subset 2^X$ и выполнено 3 свойства:

        1) $\varnothing, X \in \Omega$,

        2) $A,B \in X \Rightarrow A \cap B \in X$,

        3) $A_i \in X, i \in I \Rightarrow \bigcup\limits_{i \in I} A_i \in X$.\\
        Элементы множества $\Omega$ называются \textbf{\textit{открытыми}} множествами.\\
        Если $A$~--- \textit{открыто}, то $X \setminus A$~--- \textbf{\textit{замкнуто}}.
    \end{definition_boxed}

    \begin{task_boxed}
        Переформулируйте аксиомы для замкнутых множеств.
    \end{task_boxed}

    \begin{example}
        Топология называется \textit{тривиальной} или \textit{антидискретной}, если $\Omega = {\varnothing, X}$.
    \end{example}

    \begin{task_boxed}
        Докажите, что \textit{тривиальная} топология~--- топология.
    \end{task_boxed}

    \begin{example}
        Топология называется \textit{дискретной}, если $\Omega = 2^X$.
    \end{example}

    \begin{task_boxed}
        Докажите, что \textit{дискретная} топология~--- топология.
    \end{task_boxed}

    \begin{task_boxed}
        Пусть $X$ есть луч $[0, +\infty)$, а $\Omega$ состоит из $\varnothing$, $X$ и всевозможных лучей $(a, +\infty)$, где $a > 0$.\\
        Докажите, что $\Omega$ — топология на $X$.\\
        Такая топология называется топология \textit{стрелки}
    \end{task_boxed}

    \begin{task_boxed}
        Пусть $X$ есть плоскость.
        Является ли топологической структурой набор множеств, состоящих из $\varnothing$, $X$ и открытых кругов с центром в начале координат и всевозможными радиусами?
    \end{task_boxed}

    \begin{task_boxed}
        Пусть $X$ состоит из четырёх элементов: $X=\{a, b, c, d\}$.
        Выясните, какие из следующих трёх наборов его подмножеств являются топологическими структурами в $X$ (т.е. удовлетворяют аксиомам топологической структуры):

        1) $\varnothing$, $X$, $\{a\}$, $\{b\}$, $\{a, c\}$, $\{a, b, c\}$, $\{a, b\}$;

        2) $\varnothing$, $X$, $\{a\}$, $\{b\}$, $\{a, b\}$, $\{b, d\}$;

        3) $\varnothing$, $X$, $\{a, c, d\}$, $\{b, c, d\}$.
    \end{task_boxed}

    \begin{task_boxed}
        \textbf{Свойства замкнутых множеств.}
        Докажите что:

        1) Пересечение любого набора замкнутых множеств замкнуто;

        2) Объединение любого конечного набора замкнутых множеств замкнуто;

        3) Пустое множество и всё пространство (т.е. всё множество — носитель топологической структуры) замкнуты.
    \end{task_boxed}


    \begin{definition_boxed}
        \textit{\textbf{База}} топологии~--- некоторый набор открытых множеств, такой, что всякое непустое открытое множество представимо в виде объединения множеств из этого набора.
    \end{definition_boxed}

    \begin{example}
        Всевозможные интервалы составляют базу стандартной топологии на $\R$.
    \end{example}

    \begin{task_boxed}
        Докажите эквивалентные определения базы:

        1) Совокупность $\Sigma$ открытых множеств является базой топологии $\Omega$, когда для всякого множества $U \in \Omega$ и всякой точки $x \in U$ существует такое множество $V \in \Sigma$, что $x \in V \subseteq U$.

        2) Совокупность $\Sigma$ подмножеств множества $X$ является базой некоторой топологии в $X$, когда $X$ есть объединение множеств из $\Sigma$ и пересечение любых двух множеств из $\Sigma$ представляется в виде объединения множеств из $\Sigma$.
    \end{task_boxed}

    \begin{task_boxed}
        Рассмотрим следующие три набора подмножеств плоскости \(\mathbb{R}^2\):

        1) набор \(\Sigma_2\), состоящий из всевозможных открытых кругов (т.е. кругов, в которые не включаются ограничивающие их окружности);

        2) набор \(\Sigma_\infty\), состоящий из всевозможных открытых квадратов (квадратов без граничных точек — сторон и вершин), стороны которых параллельны координатным осям (они задаются неравенствами вида \(\max\{|x - a|, |y - b|\} < r\));

        3) набор \(\Sigma_1\), состоящий из всевозможных открытых квадратов, стороны которых параллельны биссектрисам координатных углов (они задаются неравенствами вида \(|x - a| + |y - b| < r\)).

        Докажите, что каждый из наборов \(\Sigma_2\), \(\Sigma_\infty\) и \(\Sigma_1\) служит базой некоторой топологической структуры в \(\mathbb{R}^2\), и структуры, определяемые этими базами, совпадают.
    \end{task_boxed}

    \begin{task_boxed}[*]
        Докажите, что всевозможные бесконечные арифметические прогрессии, состоящие из натуральных чисел, образуют базу некоторой топологии в $\N$.\\
        С помощью этой топологии докажите, что множество простых чисел бесконечно.\\
        Воспользуйтесь тем, что в противном случае множество $\set*{1}$ было бы открытым (?!).
    \end{task_boxed}

    \begin{definition_boxed}
        Если $\Omega_1$ и $\Omega_2$ — топологические структуры в множестве $X$ и $\Omega_1 \subset \Omega_2$, то говорят, что структура $\Omega_2$ \textbf{\textit{тоньше}}, чем $\Omega_1$, а $\Omega_1$ — \textbf{\textit{грубее}}, чем $\Omega_2$.
    \end{definition_boxed}

    \begin{example}
        Дискретная топология самая тонкая, а антидискретная самая грубая.
    \end{example}

    \newpage


    \section{Метрическое пространство}
    \begin{definition_boxed}
        \textit{\textbf{Метрическое пространство}}~--- это пара $\lr*{X, d}$, где $d : X \times X \rightarrow R_+$ и выполнено 3 свойства:

        1) $d(x, y) = 0 \Leftrightarrow x = y,$

        2) $d(x, y) = d(y, x), \forall x, y \in X,$

        3) Неравенство треугольника $\forall x, y, z$
        \[
            d(x, y) + d(y, z) \geqslant d(x, z).
        \]
        $d$ называется \textbf{\textit{метрикой}} или \textbf{\textit{расстоянием}}
    \end{definition_boxed}

    \begin{example}
        $\lr*{\R^n, \sqrt[p]{\sum\limits_{i=1}^{n}\lr*{x_i - y_i})}}$~--- \textit{Евклидово расстояние}
    \end{example}

    \begin{task_boxed}
        Докажите, что \textit{Евклидово расстояние}~--- метрика
    \end{task_boxed}

    \begin{example}
        $\lr*{X, d}, d =
        \begin{cases*}
            1, x \neq y \\
            0, x = y
        \end{cases*}$~---
        \textit{метрика лентяя}, \textit{дискретная метрика}
    \end{example}

    \begin{task_boxed}
        Докажите, что \textit{дискретная метрика}~--- метрика
    \end{task_boxed}

    \begin{definition_boxed}
        $\norm*{x}_p = p ^ {-\nu_p(x)}$~--- \textit{$p$-адическая норма}
    \end{definition_boxed}

    \begin{example}
        $\lr*{\Q, d}, d(r, s) = \norm*{r - s}_p$~---
        \textit{$p$-адическая метрика}
    \end{example}

    \begin{task_boxed}
        Докажите, что \textit{$p$-адическая метрика}~--- метрика
    \end{task_boxed}

    \begin{definition_boxed}
        $\lr*{X, d}$~--- метрическое пространство.\\
        \textbf{\textit{Открытый шар}}~--- $B_r(x_0) = \set*{y \in X \mid d(y, x_0) < x}.$\\
        \textbf{\textit{Замкнутый шар}}~--- $\overline{B_r(x_0)} = \set*{y \in X \mid d(y, x_0) \leqslant x}.$
    \end{definition_boxed}

    \begin{task_boxed}
        Как устроены шары в \textit{метрике лентяя}?
    \end{task_boxed}

    \begin{task_boxed}
        Как устроены шары в \textit{$p$-адической метрике}?
    \end{task_boxed}

    \begin{definition_boxed}
        $\lr*{X, d}$~--- метрическое пространство.\\
        Топология $\Omega_d$ \textbf{\textit{индуцированная}} метрикой определяется так:\\
        $A \in \Omega_d$, если $A$ представляется как объединение открытых шаров в $X$.
    \end{definition_boxed}

    \begin{task_boxed}
        Проверьте корректность определения \textit{индуцированной} топологии.
    \end{task_boxed}

    \begin{example}
        $\R$ со стандартной метрикой.\\
        Открытые шары = открытые интервалы.\\
        Примеры замкнутых множеств: $[0,1], \set*{2, 3, 9}$;
    \end{example}

    \begin{task_boxed}
        Докажите, что $A = \set*{\dfrac{1}{n}}_{n \in \Z_{>0}}$ \textit{замкнутым} не является, а
        $A \cup \set*{0}$~--- \textit{замкнуто}
    \end{task_boxed}


    \begin{task_boxed}
        $\lr*{X, d}$~--- метрическое пространство.\\
        $U \subset X$~--- \textit{открыто} $\Leftrightarrow \forall x \in U \exists \varepsilon: B_{\varepsilon}(x) \subset U$
    \end{task_boxed}


\end{document}