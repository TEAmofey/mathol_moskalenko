\documentclass{article}
\usepackage[12pt]{extsizes}
\usepackage[T2A]{fontenc}
\usepackage[utf8]{inputenc}
\usepackage[english, russian]{babel}

\usepackage{amssymb}
\usepackage{amsfonts}
\usepackage{amsmath}
\usepackage{enumitem}
\usepackage{graphics}
\usepackage{graphicx}

\usepackage{lipsum}
\DeclareGraphicsExtensions{.pdf,.png,.jpg}



\usepackage{geometry} % Меняем поля страницы
\geometry{left=1cm}% левое поле
\geometry{right=1cm}% правое поле
\geometry{top=1.5cm}% верхнее поле
\geometry{bottom=1cm}% нижнее поле


\usepackage{fancyhdr} % Headers and footers
\pagestyle{fancy} % All pages have headers and footers
\fancyhead{} % Blank out the default header
\fancyfoot{} % Blank out the default footer
\fancyhead[L]{Математика}
\fancyhead[C]{\textit{Алгебра}}
\fancyhead[R]{8 февраля 2023}% Custom header text


%----------------------------------------------------------------------------------------

%\begin{document}\normalsize
\begin{document}\large
	

\begin{center}
\textbf{Рациональность}
\end{center}

\begin{enumerate}[label*=\protect\fbox{\arabic{enumi}}]

\item Числа $x, y$ и $z$ таковы,что все три числа $x+ y z, y+ z x$ и $z+ x y$ рациональны, а $x^2 +  y^2 = 1$. Докажите, что число  $x y z^2$ также рационально.

\item Один из корней уравнения  $x^2 +  a x +  b = 0$ равен $1 + \sqrt{3}$. Найдите $a$ и $b$, если известно, что они рациональны.


\item Олег нарисовал пустую таблицу $50 \times 50$ и написал сверху от каждого столбца и слева от каждой строки по числу. Оказалось, что все 100 написанных чисел различны, причём 50 из них рациональные, а остальные 50 – иррациональные. Затем в каждую клетку таблицы он записал произведение чисел, написанных около её строки и её столбца (”таблица умножения”). Какое наибольшее количество произведений в этой таблице могли оказаться рациональными числами?

\item Десять попарно различных ненулевых чисел таковы, что для каждых двух из них либо сумма этих чисел, либо их произведение – рациональное число. Докажите, что квадраты всех чисел рациональны.

\item Докажите, что если выражение  $\dfrac{x}{x^2 + x + 1}$ принимает рациональное значение, то и выражение  $\dfrac{x^2}{x^4 + x^2 + 1}$  также рационально.

\item Числовое множество $M$, содержащее 2023 различных положительных числа, таково, что для любых трех различных элементов  $a,  b, c$ из $M$ число  $a^2 +  bc$ рационально. Докажите, что можно выбрать такое натуральное $n$, что для любого $a$ из $M$ число  $a\sqrt{n}$ рационально.

\item Пусть $A$ и $B$ – два прямоугольника. Из прямоугольников, равных $A$, сложили прямоугольник, подобный $B$. Докажите, что из прямоугольников, равных $B$, можно сложить прямоугольник, подобный $A$.

\end{enumerate}
\end{document}