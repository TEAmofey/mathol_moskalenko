\documentclass{article}

\usepackage[12pt]{extsizes}
\usepackage[T2A]{fontenc}
\usepackage[utf8]{inputenc}
\usepackage[english, russian]{babel}

\usepackage{mathrsfs}
\usepackage[dvipsnames]{xcolor}

\usepackage{amsmath}
\usepackage{amssymb}
\usepackage{amsthm}
\usepackage{indentfirst}
\usepackage{amsfonts}
\usepackage{enumitem}
\usepackage{graphics}
\usepackage{tikz}
\usepackage{tabu}
\usepackage{diagbox}
\usepackage{hyperref}
\usepackage{mathtools}
\usepackage{ucs}
\usepackage{lipsum}
\usepackage{geometry} % Меняем поля страницы
\usepackage{fancyhdr} % Headers and footers
\newcommand{\range}{\mathrm{range}}
\newcommand{\dom}{\mathrm{dom}}
\newcommand{\N}{\mathbb{N}}
\newcommand{\R}{\mathbb{R}}
\newcommand{\E}{\mathbb{E}}
\newcommand{\D}{\mathbb{D}}
\newcommand{\M}{\mathcal{M}}
\newcommand{\Prime}{\mathbb{P}}
\newcommand{\A}{\mathbb{A}}
\newcommand{\Q}{\mathbb{Q}}
\newcommand{\Z}{\mathbb{Z}}
\newcommand{\F}{\mathbb{F}}
\newcommand{\CC}{\mathbb{C}}

\DeclarePairedDelimiter\abs{\lvert}{\rvert}
\DeclarePairedDelimiter\floor{\lfloor}{\rfloor}
\DeclarePairedDelimiter\ceil{\lceil}{\rceil}
\DeclarePairedDelimiter\lr{(}{)}
\DeclarePairedDelimiter\set{\{}{\}}
\DeclarePairedDelimiter\norm{\|}{\|}

\renewcommand{\labelenumi}{(\alph{enumi})}

\newcommand{\smallindent}{
    \geometry{left=1cm}% левое поле
    \geometry{right=1cm}% правое поле
    \geometry{top=1.5cm}% верхнее поле
    \geometry{bottom=1cm}% нижнее поле
}

\newcommand{\header}[3]{
    \pagestyle{fancy} % All pages have headers and footers
    \fancyhead{} % Blank out the default header
    \fancyfoot{} % Blank out the default footer
    \fancyhead[L]{#1}
    \fancyhead[C]{#2}
    \fancyhead[R]{#3}
}

\newcommand{\dividedinto}{
    \,\,\,\vdots\,\,\,
}

\newcommand{\littletaller}{\mathchoice{\vphantom{\big|}}{}{}{}}

\newcommand\restr[2]{{
    \left.\kern-\nulldelimiterspace % automatically resize the bar with \right
    #1 % the function
    \littletaller % pretend it's a little taller at normal size
    \right|_{#2} % this is the delimiter
}}

\DeclareGraphicsExtensions{.pdf,.png,.jpg}

\newenvironment{enumerate_boxed}[1][enumi]{\begin{enumerate}[label*=\protect\fbox{\arabic{#1}}]}{\end{enumerate}}



\smallindent

\header{Математика}{\textit{Комбинаторика}}{2024}

%----------------------------------------------------------------------------------------

\begin{document}
    \large


    \begin{center}
        \textbf{Группы перестановок}
    \end{center}

    \begin{enumerate_boxed}

        \item Буквы слова <<Алиса>> занимают пять позиций, пронумерованных
        слева направо числами $1, 2, \dotsc , 5.$
        Напишите, какое слово получится после
        перестановки
        \[\left(
        \begin{array}{c c c c c}
            1 & 2 & 3 & 4 & 5 \\
            3 & 2 & 4 & 5 & 1
        \end{array}
        \right).\]

        \item Запишите перестановку, переводящую слово КОНУС в слово СУКНО.

        \item Найдите $F \circ F$, если $F = \left(
        \begin{array}{c c c}
            1 & 2 & 3 \\
            2 & 3 & 1
        \end{array}
        \right).$

        \item Найдите обратную к перестановке
        \[\left(
        \begin{array}{c c c c c}
            1 & 2 & 3 & 4 & 5 \\
            3 & 2 & 4 & 5 & 1
        \end{array}
        \right).\]

        \item Приведите пример двух перестановок $F$ и $G$, для которых $F \circ G \neq G \circ F$.

        \item Для перестановок трёх элементов найдите композицию \[(13) \circ (23) \circ (12)\]

        \item Приведите пример перестановки 4 элементов, которая не представляется в виде произведения двух или менее транспозиций.

        \item Приведите пример перестановки 10 элементов, которая не представляется в виде произведения 8 или менее транспозиций.

        \item Покажите, что обратная перестановка имеет ту же самую чётность, что и исходная.

        \item Определите, чётна или нечётна перестановка
        \[\left(
        \begin{array}{c c c c c}
            1 & 2 & 3 & 4 & 5 \\
            3 & 2 & 4 & 5 & 1
        \end{array}
        \right).\]

        \item Чему равно $F^{1001}$, где $F =
        \left(
        \begin{array}{c c c c c}
            1 & 2 & 3 & 4 & 5 \\
            3 & 2 & 4 & 5 & 1
        \end{array}
        \right).$

        \item Запишите перестановку $F =
        \left(
        \begin{array}{c c c c c}
            1 & 2 & 3 & 4 & 5 \\
            3 & 2 & 4 & 5 & 1
        \end{array}
        \right)$ в виде произведения нескольких циклов

        \item Найдите чётность цикла из $k$ элементов.

        \item Докажите, что всякая чётная перестановка раскладывается в произведение циклов длины 3.

        \item Докажите, что цикл длины $n$ нельзя представить как произведение $n - 2$ или менее транспозиций.

        \item Как по циклическому типу перестановки понять её порядок (как порядок элемента в группе)?

        \item Каков максимально возможный порядок перестановки множества из 10 элементов?

        \item Сколько перестановок вершин квадрата получаются из его движений?
        Тот же вопрос для правильного пятиугольника.
        Группы таких перестановок называются Диэдральные группы.

        \item Докажите, что любую перестановку объектов, стоящих в клетках прямоугольной таблицы, можно представить в виде композиции трёх перестановок: первая и третья переставляют элементы внутри строк (каждый объект остаётся в той же строке, где был, но может сменить столбец), а вторая — внутри столбцов.

    \end{enumerate_boxed}
\end{document}