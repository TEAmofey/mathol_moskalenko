\documentclass{article}

\usepackage[12pt]{extsizes}
\usepackage[T2A]{fontenc}
\usepackage[utf8]{inputenc}
\usepackage[english, russian]{babel}

\usepackage{mathrsfs}
\usepackage[dvipsnames]{xcolor}

\usepackage{amsmath}
\usepackage{amssymb}
\usepackage{amsthm}
\usepackage{indentfirst}
\usepackage{amsfonts}
\usepackage{enumitem}
\usepackage{graphics}
\usepackage{tikz}
\usepackage{tabu}
\usepackage{diagbox}
\usepackage{hyperref}
\usepackage{mathtools}
\usepackage{ucs}
\usepackage{lipsum}
\usepackage{geometry} % Меняем поля страницы
\usepackage{fancyhdr} % Headers and footers
\newcommand{\range}{\mathrm{range}}
\newcommand{\dom}{\mathrm{dom}}
\newcommand{\N}{\mathbb{N}}
\newcommand{\R}{\mathbb{R}}
\newcommand{\E}{\mathbb{E}}
\newcommand{\D}{\mathbb{D}}
\newcommand{\M}{\mathcal{M}}
\newcommand{\Prime}{\mathbb{P}}
\newcommand{\A}{\mathbb{A}}
\newcommand{\Q}{\mathbb{Q}}
\newcommand{\Z}{\mathbb{Z}}
\newcommand{\F}{\mathbb{F}}
\newcommand{\CC}{\mathbb{C}}

\DeclarePairedDelimiter\abs{\lvert}{\rvert}
\DeclarePairedDelimiter\floor{\lfloor}{\rfloor}
\DeclarePairedDelimiter\ceil{\lceil}{\rceil}
\DeclarePairedDelimiter\lr{(}{)}
\DeclarePairedDelimiter\set{\{}{\}}
\DeclarePairedDelimiter\norm{\|}{\|}

\renewcommand{\labelenumi}{(\alph{enumi})}

\newcommand{\smallindent}{
    \geometry{left=1cm}% левое поле
    \geometry{right=1cm}% правое поле
    \geometry{top=1.5cm}% верхнее поле
    \geometry{bottom=1cm}% нижнее поле
}

\newcommand{\header}[3]{
    \pagestyle{fancy} % All pages have headers and footers
    \fancyhead{} % Blank out the default header
    \fancyfoot{} % Blank out the default footer
    \fancyhead[L]{#1}
    \fancyhead[C]{#2}
    \fancyhead[R]{#3}
}

\newcommand{\dividedinto}{
    \,\,\,\vdots\,\,\,
}

\newcommand{\littletaller}{\mathchoice{\vphantom{\big|}}{}{}{}}

\newcommand\restr[2]{{
    \left.\kern-\nulldelimiterspace % automatically resize the bar with \right
    #1 % the function
    \littletaller % pretend it's a little taller at normal size
    \right|_{#2} % this is the delimiter
}}

\DeclareGraphicsExtensions{.pdf,.png,.jpg}

\newenvironment{enumerate_boxed}[1][enumi]{\begin{enumerate}[label*=\protect\fbox{\arabic{#1}}]}{\end{enumerate}}



\smallindent

\header{Математика}{\textit{Алгебра}}{23 апреля 2024}

%----------------------------------------------------------------------------------------

\begin{document}
    \large

    \begin{center}
        \textbf{Иррациональность}
    \end{center}

    \textbf{Определение.}
    Число называется рациональным, если его можно представить в виде $\frac{m}{n}$, где $m \in \Z, n \in \N$.
    Действительные числа, не представимые в таком виде, называются иррациональными.
    \begin{enumerate_boxed}

        \item Докажите, что числа $\sqrt{2}$, $\sqrt[3]{2}$, $\sqrt[5]{2^3}$ – иррациональны.

        \item Может ли:
        \begin{enumerate}
            \item Сумма двух иррациональных чисел быть рациональной?
            \item Произведение двух иррациональных чисел быть рациональным?
            \item Произведение иррационального с рациональным быть рациональным?
            \item Иррациональное число в рациональной степени быть рациональным?
        \end{enumerate}

        \item Докажите, что $\sqrt{n}$, где $n$ – натурально, является либо целым числом, либо иррациональным.

        \item Пусть $x$ – такое число, что $10^x = 2.$ Докажите, что $x$ – иррационально.

        \item Пусть $a, b, c$ – рациональные числа, $\sqrt{a} + \sqrt{b} = c$.
        Докажите, что $\sqrt{a}, \sqrt{b}$ – рациональные числа.

        \item Пусть $m, n$ – целые числа, такие, что $\sqrt{m} + \sqrt[3]{n} \in \Z$.
        Верно ли, что оба слагаемых – целые числа?

        \item Иррациональны ли числа:
        \begin{enumerate}
            \item $\sqrt{7 + 4\sqrt{3}} + \sqrt{7 - 4\sqrt{3}}$
            \item $\sqrt{17 - 4\sqrt{9 + 4\sqrt{5}}} + \sqrt{5}$
            \item $\sqrt{5\sqrt{2} - 1} + \left(\sqrt{2} - 3\right)\sqrt{\sqrt{2} + 1}$
            \item $\sqrt{2} + \sqrt{3} + \sqrt{5}$
        \end{enumerate}

        \item Последовательность задана соотношением $x_{n + 1} = 1 - \abs{1 - 2x_{n}}$, $0 < x_0 < 1$.
        Докажите, что эта последовательность периодична тогда и только тогда, когда $x_0$– рационально.

    \end{enumerate_boxed}
\end{document}