\documentclass{article}

\usepackage[12pt]{extsizes}
\usepackage[T2A]{fontenc}
\usepackage[utf8]{inputenc}
\usepackage[english, russian]{babel}

\usepackage{mathrsfs}
\usepackage[dvipsnames]{xcolor}

\usepackage{amsmath}
\usepackage{amssymb}
\usepackage{amsthm}
\usepackage{indentfirst}
\usepackage{amsfonts}
\usepackage{enumitem}
\usepackage{graphics}
\usepackage{tikz}
\usepackage{tabu}
\usepackage{diagbox}
\usepackage{hyperref}
\usepackage{mathtools}
\usepackage{ucs}
\usepackage{lipsum}
\usepackage{geometry} % Меняем поля страницы
\usepackage{fancyhdr} % Headers and footers
\newcommand{\range}{\mathrm{range}}
\newcommand{\dom}{\mathrm{dom}}
\newcommand{\N}{\mathbb{N}}
\newcommand{\R}{\mathbb{R}}
\newcommand{\E}{\mathbb{E}}
\newcommand{\D}{\mathbb{D}}
\newcommand{\M}{\mathcal{M}}
\newcommand{\Prime}{\mathbb{P}}
\newcommand{\A}{\mathbb{A}}
\newcommand{\Q}{\mathbb{Q}}
\newcommand{\Z}{\mathbb{Z}}
\newcommand{\F}{\mathbb{F}}
\newcommand{\CC}{\mathbb{C}}

\DeclarePairedDelimiter\abs{\lvert}{\rvert}
\DeclarePairedDelimiter\floor{\lfloor}{\rfloor}
\DeclarePairedDelimiter\ceil{\lceil}{\rceil}
\DeclarePairedDelimiter\lr{(}{)}
\DeclarePairedDelimiter\set{\{}{\}}
\DeclarePairedDelimiter\norm{\|}{\|}

\renewcommand{\labelenumi}{(\alph{enumi})}

\newcommand{\smallindent}{
    \geometry{left=1cm}% левое поле
    \geometry{right=1cm}% правое поле
    \geometry{top=1.5cm}% верхнее поле
    \geometry{bottom=1cm}% нижнее поле
}

\newcommand{\header}[3]{
    \pagestyle{fancy} % All pages have headers and footers
    \fancyhead{} % Blank out the default header
    \fancyfoot{} % Blank out the default footer
    \fancyhead[L]{#1}
    \fancyhead[C]{#2}
    \fancyhead[R]{#3}
}

\newcommand{\dividedinto}{
    \,\,\,\vdots\,\,\,
}

\newcommand{\littletaller}{\mathchoice{\vphantom{\big|}}{}{}{}}

\newcommand\restr[2]{{
    \left.\kern-\nulldelimiterspace % automatically resize the bar with \right
    #1 % the function
    \littletaller % pretend it's a little taller at normal size
    \right|_{#2} % this is the delimiter
}}

\DeclareGraphicsExtensions{.pdf,.png,.jpg}

\newenvironment{enumerate_boxed}[1][enumi]{\begin{enumerate}[label*=\protect\fbox{\arabic{#1}}]}{\end{enumerate}}



\smallindent

\header{Математика}{\textit{Комбинаторика}}{Август 2022}

%----------------------------------------------------------------------------------------

\begin{document}
    \large

    \begin{center}
        \textbf{Двудольные графы}
    \end{center}
    
    \begin{enumerate_boxed}

        \item Докажите, что в двудольном графе сумма степеней вершин одного цвета равна сумме степеней вершин другого цвета.

        \item а) Какое наибольшее число рёбер может быть в двудольном графе с $k$ белыми и $m$ чёрными вершинами?

        б) Какое наибольшее количество рёбер может быть в двудольном графе с $2n$ вершинами?
        А с $2n + 1$?

        \item Докажите, что если в двудольном графе степени всех вершин одинаковы, то вершин каждого цвета поровну.

        \item 10 школьников решали 10 задач.
        Выяснилось, что каждый школьник решил ровно две задачи, и каждую задачу решило два школьника.
        Докажите, что можно так организовать рассказ решений, чтобы каждый школьник рассказал ровно одну из решенных им задач, и при этом все задачи были бы рассказаны.

        \item \textbf{Теорема Кёнига}.
        Граф является двудольным тогда и только тогда, когда он не содержит цикла нечётной длины.

        \item В каждой строке и в каждом столбце таблицы $8 \times 8$ стоит ровно две фишки.
        Докажите, что их можно покрасить в чёрный и белый цвет так, чтобы в каждом столбце и в каждой строке стояли разноцветные фишки.

        \item \textit{Лемма}.
        Пусть $G$ — двудольный граф с черными и белыми вершинами.

        а) Если в $G$ есть замкнутый цикл, проходящий через каждую вершину ровно по одному разу, то вершин каждого цвета — поровну.

        б) Если в $G$ есть путь, проходящий через каждую вершину ровно по одному разу, то что число белых вершин отличается от числа черных вершин не более чем на $1$.

        \item Замок в форме треугольника со стороной 50 метров разбит на 100 треугольных залов со сторонами 5 метров.
        В каждой стенке между залами есть дверь.
        Какое наибольшее число залов сможет обойти турист, не заходя ни в какой зал дважды?

        \item а) В клетки доски $8 \times 8$ записали числа $1, 2, \dotsc , 64$ в неизвестном порядке.
        Разрешается узнать сумму чисел в любой паре клеток с общей стороной.
        Всегда ли можно узнать расположение всех чисел?

        б) Тот же вопрос для доски $9 \times 9$.

        \item а) Найдется ли правильный треугольник с вершинами в узлах квадратной сетки?

        б) Вершины графа – это узлы клетчатой бумаги, ребра – отрезки фиксированной длины $L$.
        Докажите, что получившийся граф – двудольный.

    \end{enumerate_boxed}
\end{document}