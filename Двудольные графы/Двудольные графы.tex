\documentclass{article}
\usepackage[12pt]{extsizes}
\usepackage[T2A]{fontenc}
\usepackage[utf8]{inputenc}
\usepackage[english, russian]{babel}

\usepackage{amssymb}
\usepackage{amsfonts}
\usepackage{amsmath}
\usepackage{enumitem}
\usepackage{graphics}
\usepackage{graphicx}

\usepackage{lipsum}

\newtheorem{theorem}{Теорема}
\newtheorem{task}{Задача}
\newtheorem{lemma}{Лемма}
\newtheorem{definition}{Определение}
\newtheorem{example}{Пример}
\newtheorem{statement}{Утверждение}
\newtheorem{corollary}{Следствие}


\usepackage{geometry} % Меняем поля страницы
\geometry{left=1cm}% левое поле
\geometry{right=1cm}% правое поле
\geometry{top=1.5cm}% верхнее поле
\geometry{bottom=1cm}% нижнее поле


\usepackage{fancyhdr} % Headers and footers
\pagestyle{fancy} % All pages have headers and footers
\fancyhead{} % Blank out the default header
\fancyfoot{} % Blank out the default footer
\fancyhead[L]{Математика}
\fancyhead[C]{\textit{Комбинаторика}}
\fancyhead[R]{Август 2022}% Custom header text


%----------------------------------------------------------------------------------------

%\begin{document}\normalsize
\begin{document}\large
	
\begin{center}
	\textbf{Двудольные графы}
\end{center}


\begin{enumerate}[label*=\protect\fbox{\arabic{enumi}}]
	
\item Докажите, что в двудольном графе сумма степеней вершин одного цвета равна сумме степеней вершин другого цвета.

\item а) Какое наибольшее число рёбер может быть в двудольном графе с $k$ белыми и $m$ чёрными вершинами?

б) Какое наибольшее количество рёбер может быть в двудольном графе с $2n$ вершинами? А с $2n + 1$?

\item Докажите, что если в двудольном графе степени всех вершин одинаковы, то вершин каждого цвета поровну.

\item 10 школьников решали 10 задач. Выяснилось, что каждый школьник решил ровно две задачи, и каждую задачу решило два школьника. Докажите, что можно так организовать рассказ решений, чтобы каждый школьник рассказал ровно одну из решенных им задач, и при этом все задачи были бы рассказаны.

\item \textbf{Теорема Кёнига}. Граф является двудольным тогда и только тогда, когда он не содержит цикла нечётной длины.

\item В каждой строке и в каждом столбце таблицы $8 \times 8$ стоит ровно две фишки. Докажите, что их можно покрасить в чёрный и белый цвет так, чтобы в каждом столбце и в каждой строке стояли разноцветные фишки.

\item \textit{Лемма}. Пусть $G$ — двудольный граф с черными и белыми вершинами.

а) Если в $G$ есть замкнутый цикл, проходящий через каждую вершину ровно по одному разу, то вершин каждого цвета — поровну.

б) Если в $G$ есть путь, проходящий через каждую вершину ровно по одному разу, то что число белых вершин отличается от числа черных вершин не более чем на $1$.

\item Замок в форме треугольника со стороной 50 метров разбит на 100 треугольных залов со сторонами 5 метров. В каждой стенке между залами есть дверь. Какое наибольшее число залов сможет обойти турист, не заходя ни в какой зал дважды?

\item а) В клетки доски $8 \times 8$ записали числа $1, 2, . . . , 64$ в неизвестном порядке. Разрешается узнать сумму чисел в любой паре клеток с общей стороной. Всегда ли можно узнать расположение всех чисел?

б) Тот же вопрос для доски $9 \times 9$.

\item 
а) Найдется ли правильный треугольник с вершинами в узлах квадратной сетки?

б) Вершины графа – это узлы клетчатой бумаги, ребра – отрезки фиксированной длины $L$. Докажите, что получившийся граф – двудольный. 


\end{enumerate}
\end{document}