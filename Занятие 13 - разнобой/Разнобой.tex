\documentclass{article}

\usepackage[12pt]{extsizes}
\usepackage[T2A]{fontenc}
\usepackage[utf8]{inputenc}
\usepackage[english, russian]{babel}

\usepackage{mathrsfs}
\usepackage[dvipsnames]{xcolor}

\usepackage{amsmath}
\usepackage{amssymb}
\usepackage{amsthm}
\usepackage{indentfirst}
\usepackage{amsfonts}
\usepackage{enumitem}
\usepackage{graphics}
\usepackage{tikz}
\usepackage{tabu}
\usepackage{diagbox}
\usepackage{hyperref}
\usepackage{mathtools}
\usepackage{ucs}
\usepackage{lipsum}
\usepackage{geometry} % Меняем поля страницы
\usepackage{fancyhdr} % Headers and footers
\newcommand{\range}{\mathrm{range}}
\newcommand{\dom}{\mathrm{dom}}
\newcommand{\N}{\mathbb{N}}
\newcommand{\R}{\mathbb{R}}
\newcommand{\E}{\mathbb{E}}
\newcommand{\D}{\mathbb{D}}
\newcommand{\M}{\mathcal{M}}
\newcommand{\Prime}{\mathbb{P}}
\newcommand{\A}{\mathbb{A}}
\newcommand{\Q}{\mathbb{Q}}
\newcommand{\Z}{\mathbb{Z}}
\newcommand{\F}{\mathbb{F}}
\newcommand{\CC}{\mathbb{C}}

\DeclarePairedDelimiter\abs{\lvert}{\rvert}
\DeclarePairedDelimiter\floor{\lfloor}{\rfloor}
\DeclarePairedDelimiter\ceil{\lceil}{\rceil}
\DeclarePairedDelimiter\lr{(}{)}
\DeclarePairedDelimiter\set{\{}{\}}
\DeclarePairedDelimiter\norm{\|}{\|}

\renewcommand{\labelenumi}{(\alph{enumi})}

\newcommand{\smallindent}{
    \geometry{left=1cm}% левое поле
    \geometry{right=1cm}% правое поле
    \geometry{top=1.5cm}% верхнее поле
    \geometry{bottom=1cm}% нижнее поле
}

\newcommand{\header}[3]{
    \pagestyle{fancy} % All pages have headers and footers
    \fancyhead{} % Blank out the default header
    \fancyfoot{} % Blank out the default footer
    \fancyhead[L]{#1}
    \fancyhead[C]{#2}
    \fancyhead[R]{#3}
}

\newcommand{\dividedinto}{
    \,\,\,\vdots\,\,\,
}

\newcommand{\littletaller}{\mathchoice{\vphantom{\big|}}{}{}{}}

\newcommand\restr[2]{{
    \left.\kern-\nulldelimiterspace % automatically resize the bar with \right
    #1 % the function
    \littletaller % pretend it's a little taller at normal size
    \right|_{#2} % this is the delimiter
}}

\DeclareGraphicsExtensions{.pdf,.png,.jpg}

\newenvironment{enumerate_boxed}[1][enumi]{\begin{enumerate}[label*=\protect\fbox{\arabic{#1}}]}{\end{enumerate}}



\smallindent

\header{Математика}{\textit{Олимпиадная подготовка}}{20 февраля 2023}

%----------------------------------------------------------------------------------------

%\begin{document}\normalsize
\begin{document}
    \large

    \begin{center}
        \textbf{Разнобой}
    \end{center}

    \begin{enumerate_boxed}
        \item Последовательность из пяти цифр $a_1, a_2, a_3, a_4, a_5$ будем называть «горой», если $a_1 < a_2 < a_3 > a_4 > a_5$, и «ямой», если $a_1 > a_2 > a_3 < a_4 < a_5$.
        Чего больше: «гор» или «ям»?

        \item Кубик Рубика $3 \times 3 \times 3$ надо распилить на единичные кубики.
        После распила части можно перекладывать и прикладывать так, чтобы можно было пилить несколько частей одновременно.
        Какое наименьшее число распилов нам понадобится?

        \item Докажите, что хотя бы одно из следующих чисел
        \begin{center}
            \texttt{ИКС, МИКС, ПЯТИКЛАССНИК и ПЛЯСКИТИМАТИНАТАТАМИ}
        \end{center}
        составное (разным буквами соответствуют разные цифры, одинаковым — одинаковые, первые цифры чисел не равны 0).

        \item 100 гирек веса 1, 2, $\dotsc$, 100г разложили на две чаши весов так, что есть равновесие.
        Докажите, что можно убрать по 2 гирьки с каждой чаши так что равновесие не нарушится.

        \item $n$ гирек веса 1, 2, $\dotsc$, $n$г разложили на две чаши весов так, что есть равновесие.
        Верно ли, что для любого $n > 3$ можно убрать по 2 гирьки с каждой чаши так что равновесие не нарушится?

        \item На окружности сидят 239 птиц, причем в одной точке могут сидеть несколько птиц.
        Две птицы видят друг друга, если соединяющая их дуга не больше $10^\circ$.
        Найдите наименьшее возможное количество пар птиц, видящих друг друга.

        \item В клетчатом квадрате $N\times N$ отмечены центры $3N$ клеток.
        Докажите, что среди попарных расстояний между отмеченными точками какие-то два отличаются в два раза.

    \end{enumerate_boxed}
\end{document}