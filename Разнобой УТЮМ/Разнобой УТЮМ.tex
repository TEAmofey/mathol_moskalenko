\documentclass{article}

\usepackage[12pt]{extsizes}
\usepackage[T2A]{fontenc}
\usepackage[utf8]{inputenc}
\usepackage[english, russian]{babel}

\usepackage{mathrsfs}
\usepackage[dvipsnames]{xcolor}

\usepackage{amsmath}
\usepackage{amssymb}
\usepackage{amsthm}
\usepackage{indentfirst}
\usepackage{amsfonts}
\usepackage{enumitem}
\usepackage{graphics}
\usepackage{tikz}
\usepackage{tabu}
\usepackage{diagbox}
\usepackage{hyperref}
\usepackage{mathtools}
\usepackage{ucs}
\usepackage{lipsum}
\usepackage{geometry} % Меняем поля страницы
\usepackage{fancyhdr} % Headers and footers
\newcommand{\range}{\mathrm{range}}
\newcommand{\dom}{\mathrm{dom}}
\newcommand{\N}{\mathbb{N}}
\newcommand{\R}{\mathbb{R}}
\newcommand{\E}{\mathbb{E}}
\newcommand{\D}{\mathbb{D}}
\newcommand{\M}{\mathcal{M}}
\newcommand{\Prime}{\mathbb{P}}
\newcommand{\A}{\mathbb{A}}
\newcommand{\Q}{\mathbb{Q}}
\newcommand{\Z}{\mathbb{Z}}
\newcommand{\F}{\mathbb{F}}
\newcommand{\CC}{\mathbb{C}}

\DeclarePairedDelimiter\abs{\lvert}{\rvert}
\DeclarePairedDelimiter\floor{\lfloor}{\rfloor}
\DeclarePairedDelimiter\ceil{\lceil}{\rceil}
\DeclarePairedDelimiter\lr{(}{)}
\DeclarePairedDelimiter\set{\{}{\}}
\DeclarePairedDelimiter\norm{\|}{\|}

\renewcommand{\labelenumi}{(\alph{enumi})}

\newcommand{\smallindent}{
    \geometry{left=1cm}% левое поле
    \geometry{right=1cm}% правое поле
    \geometry{top=1.5cm}% верхнее поле
    \geometry{bottom=1cm}% нижнее поле
}

\newcommand{\header}[3]{
    \pagestyle{fancy} % All pages have headers and footers
    \fancyhead{} % Blank out the default header
    \fancyfoot{} % Blank out the default footer
    \fancyhead[L]{#1}
    \fancyhead[C]{#2}
    \fancyhead[R]{#3}
}

\newcommand{\dividedinto}{
    \,\,\,\vdots\,\,\,
}

\newcommand{\littletaller}{\mathchoice{\vphantom{\big|}}{}{}{}}

\newcommand\restr[2]{{
    \left.\kern-\nulldelimiterspace % automatically resize the bar with \right
    #1 % the function
    \littletaller % pretend it's a little taller at normal size
    \right|_{#2} % this is the delimiter
}}

\DeclareGraphicsExtensions{.pdf,.png,.jpg}

\newenvironment{enumerate_boxed}[1][enumi]{\begin{enumerate}[label*=\protect\fbox{\arabic{#1}}]}{\end{enumerate}}



\smallindent

\header{Математика}{\textit{Разное}}{1 декабря 2022}

%----------------------------------------------------------------------------------------

\begin{document}
    \large

    \begin{center}
        \textbf{Разнобой УТЮМа 1}
    \end{center}


    \begin{enumerate_boxed}

        \item Можно ли заполнить клетки таблицы $2020 \times 2020$ натуральными числами от $1$ до $4 080 400$ так, чтобы сумма чисел в каждой строке, начиная со второй, была на 1 больше, чем сумма чисел во всех предыдущих строках?

        \item В остроугольном треугольнике $ABC$ проведена высота $AH$ и отмечены середины $A_1$, $B_1$ и $C_1$ сторон $BC$, $CA$ и $AB$ соответственно.
        Точка $K$ симметрична точке $B_1$ относительно прямой $BC$.
        Докажите, что прямая $C_{1}K$ делит отрезок $HA_1$ пополам.

        \item Каждую клетку доски $2022 \times 2022$ красят в чёрный или белый цвет.
        В некоторые клетки ставят хромых ферзей.
        Хромой ферзь с клетки $A$ бьёт клетку $B$, если клетки $A$ и $B$ находятся на одной линии (горизонтали, вертикали или диагонали) и все клетки этой линии от $A$ до $B$ включительно покрашены в один цвет.
        При каком наибольшем $k$ можно покрасить доску и расставить на ней $k$ хромых ферзей так, чтобы они не били друг друга?

        \item На столе стоит несколько гирь суммарного веса $s$.
        Назовём гирю раздвоителем, если после её удаления все остальные гири можно разбить на две группы, суммарный вес каждой из которых не больше $\frac{s}{2}$.
        Докажите, что вес самого большого раздвоителя больше суммы весов всех нераздвоителей.

        \item Высоты $AA_1$, $BB_1$ и $CC_1$ остроугольного треугольника $ABC$ пересекаются в точке $H$.
        Прямые $AA_1$ и $B_{1}C_1$ пересекаются в точке $X$.
        Перпендикуляр к $AC$, проведённый через точку $X$, пересекает сторону $AB$ в точке $Y$.
        Докажите, что прямая $YA_1$ делит отрезок $BH$ пополам.

        \item Нечётная раскраска графа – это такая раскраска множества его вершин в несколько цветов, что любые две соседние вершины покрашены в разный цвет и при этом для каждой вершины можно указать цвет, в который покрашено нечётное число её соседей.
        Барон Мюнхгаузен нарисовал граф и создал нечётную раскраску его вершин в $1022$ цвета.
        «Вы можете мне не поверить, друзья, — говорит барон, — но на этом графе не существует нечётных раскрасок с меньшим числом цветов.
        Однако после того как я добавил всего одну вершину и соединил её с некоторыми вершинами этого графа, для нечётной раскраски мне понадобилось всего три цвета».
        Не обманывает ли нас барон?

        \item Докажите, что нечётное число $p > 1$ — простое тогда и только тогда, когда среди любых $\dfrac{p+1}{2}$ различных натуральных чисел можно найти два числа, сумма которых хотя бы в $p$ раз больше их наибольшего общего делителя.


        \item Даны различные ненулевые цифры $a, b, c, d$.
        Известно, что ни одно из чисел $\overline{abcd}$, $\overline{bcda}, \overline{cdab}, \overline{dabc}$ не имеет простых делителей, меньших 10.
        Чему может быть равна сумма этих четырёх четырёхзначных чисел?

        \item На доске написано несколько различных неотрицательных чисел.
        Оказалось, что про- изведение любых двух выписанных чисел также есть на этой доске.
        Какое наибольшее количество чисел может быть написано?

        \item Два равных отрезка $AB$ и $CD$ пересекаются в точке $P$.
        Точка $M$~--- середина отрезка $BD$.
        Оказалось, что точка $M$ равноудалена от точек $A$ и $C$.
        Докажите, что $AP = CP$.

        \item Дан клетчатый квадрат $101\times 101$.
        Внутри него выбирается квадрат $100\times 100$.
        Внутри этого квадрата выбирается квадрат $99 \times 99$, и так далее, пока не будет выбран квадрат $1\times 1$.
        Оказалось, что выбранный квадрат $1\times1$ совпадает с центральной клеткой исходного квадрата $101 \times 101$.
        Сколько существует таких последовательностей квадратов?
        Ответ не должен содержать знака многоточия.

        \item В стране из 1000 городов некоторые города соединены дорогами, по которым можно двигаться в обе стороны.
        Известно, что в этой стране нет циклического маршрута.
        При каком наибольшем $k$ всегда можно выбрать $k$ городов так, чтобы каждый выбранный город был соединен не более чем с двумя из остальных выбранных?

        \item Серёжа придумал два положительных не целых числа $a$ и $b$.
        Затем он подсчитал четыре выражения: $a+b, a-b, a\cdot b, \dfrac{a}{b}$.
        Докажите, что хотя бы одно из получившихся чисел не целое.

        \item Даны $36$ различных чисел (не обязательно целых).
        Докажите, что их можно расставить в клетках таблицы $6 \times 6$ так, чтобы для любых двух чисел, стоящих в соседних по стороне ячейках, их разность была не равна $1$.

        \item На какое наибольшее количество нулей может оканчиваться произведение че- тырёхзначного числа, не содержащего в своей записи нулей, на его сумму цифр?

        \item На плоскости отмечено $10$ точек.
        Докажите, что существует не более $90$ равно- бедренных прямоугольных треугольников с вершинами в этих точках.

        \item В остроугольном треугольнике $ABC$ проведены высоты $CF$ и $BE$.
        На отрезке $BE$ нашлась такая точка $P$, что $BP = AC$.
        На продолжении отрезка $CF$ за точку $F$ нашлась такая точка $Q$, что $CQ = AB$.
        Докажите, что $AP \perp AQ$.

    \end{enumerate_boxed}
\end{document}