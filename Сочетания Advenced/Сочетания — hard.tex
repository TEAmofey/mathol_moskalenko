\documentclass{article}
\usepackage[12pt]{extsizes}
\usepackage[T2A]{fontenc}
\usepackage[utf8]{inputenc}
\usepackage[english, russian]{babel}

\usepackage{amssymb}
\usepackage{amsfonts}
\usepackage{amsmath}
\usepackage{enumitem}
\usepackage{graphics}
\usepackage{graphicx}

\usepackage{lipsum}

\newtheorem{theorem}{Теорема}
\newtheorem{task}{Задача}
\newtheorem{lemma}{Лемма}
\newtheorem{definition}{Определение}
\newtheorem{example}{Пример}
\newtheorem{statement}{Утверждение}
\newtheorem{corollary}{Следствие}


\usepackage{geometry} % Меняем поля страницы
\geometry{left=1.5cm}% левое поле
\geometry{right=1cm}% правое поле
\geometry{top=1.5cm}% верхнее поле
\geometry{bottom=1.5cm}% нижнее поле


\usepackage{fancyhdr} % Headers and footers
\pagestyle{fancy} % All pages have headers and footers
\fancyhead{} % Blank out the default header
\fancyfoot{} % Blank out the default footer
\fancyhead[L]{ЦРОД \textbullet Математика}
\fancyhead[C]{\textit{Комбинаторика}}
\fancyhead[R]{ЛФМШ 2022}% Custom header text


%----------------------------------------------------------------------------------------

%\begin{document}\normalsize
\begin{document}\large
	
\begin{center}
	\textbf{Сочетания}
\end{center}

\begin{definition}
	Количество способов выбрать $k$ предметов из $n$ имеющихся называется числом сочетаний из $n$ элементов по $k$ и обозначается $C^k_n = \binom{n}{k}$ .
\end{definition}

\begin{enumerate}[label*=\protect\fbox{\arabic{enumi}}]
\item Докажите, что $C^k_n = \frac{n!}{k!(n - k)!}$

\item У семиклассника Пети есть $7$ детективов, а у восьмиклассника Васи – $8$ книг по математике. Сколькими способами они могут обменять три книги одного на три книги другого?

\item $15$ человек нужно разбить на баскетбольную, волейбольную и футбольную команды по пять человек. Сколькими способами это можно сделать?

\item \textbf{(a)} Что можно выбрать большим числом способов: двух преподавателей из $15$ для работы
в группе "Олимпиадная математика - $4$" или $13$ преподавателей, которые не решаются работать с группой "Олимпиадная математика - $4$"? \textbf{(b)} Докажите тождество алгебраически, комбинаторно и через треугольник Паскаля. $C^k_n = C^{n-k}_n$

\item \textbf{(a)} В группе $15$ человек. Сколькими способами из них можно выбрать шестерых, которые пойдут на лекцию, если Даниил категорически отказывается туда идти, так как ему нужно дорешать задачи? \textbf{(b)} Для проведения матбоя нужна команда из шести человек, в которой Саша будет капитаном. Сколькими способами можно собрать такую команду из группы в $15$ человек? \textbf{(c)} Докажите тождество алгебраически, комбинаторно и через треугольник Паскаля $C^k_n = C^{k-1}_{n - 1} + C^k_{n- 1}$

\item \textbf{(a)} В классе $15$ человек. Учитель по физкультуре Андрей Леопольдович выбирает команду в футбол на кубок ЦРОДа (из 6 человек). Капитаном он выберет самого высокого из них (в классе все дети разного роста). Но Андрей Леопольдович решил сначала выбрать капитана, а потом всю оставшуюся команду. Помогите ему посчитать количество возможных команд. \textbf{(b)} Докажите тождество комбинаторно и через треугольник Паскаля $C^{k + 1}_{n + 1} = C^{k}_{k} + C^{k}_{k + 1} + \dotso + C^k_{n}$

\item \textbf{(a)} У акулы было $100$ зубов. Сколько различных улыбок могло у неё остаться после встречи
с катером? \textbf{(b)} Докажите, что $C^0_n + C^1_n + ... + C^n_n = 2^n$

\item \textbf{(a)} Сколькими способами из $15$ человек можно выбрать команду из шести человек, возглавляемую капитаном? \textbf{(b)} Сколькими способами можно выбрать из $n$ человек $k$ человек в парламент, возглавляемого президентом парламента? \textbf{(c)} Докажите тождество двумя способами – комбинаторно и алгебраически: $k \cdot C^k_n = n \cdot C^{k-1}_{n-1}$

\item Сколькими способами можно выбрать положительные числа $x_1, \dotso , x_k$ такие, что $x_1 + \dotso + x_k = n$.

\item Сколькими способами можно выбрать неотрицательные числа $x_1, \dotso , x_k$ такие, что $x_1 + \dotso + x_k = n$.

\item В коробке лежат $n$ синих и $n$ красных шариков (все шарики разные). Сформулируйте
вопрос, позволяющий доказать, что $C^0_n \cdot C^k_n +C^1_n \cdot C^{k-1}_n+\dotso+C^i_n \cdot C^{k-i}_n +\dotso + C^k_n \cdot C^0_n =C^k_{2n}$

\item Найдите, значение выражения $C^0_n \cdot C^k_m +C^1_n \cdot C^{k-1}_m+\dotso+C^i_n \cdot C^{k-i}_m +\dotso + C^k_n \cdot C^0_m$

\item Найдите, значение выражения $(C^0_n)^2 +(C^1_n)^2 +\dotso+(C^i_n)^2 +\dotso + (C^n_n)^2$

\item Найдите, значение выражения $C^0_n +C^1_n \cdot 2 +\dotso+C^i_n \cdot 2^i +\dotso + C^n_n \cdot 2^n$

\item Для натурального числа $n$ оказалось, что каждое из чисел $C^1_n$, ... ,$C^{k - 1}_n$
делится на $n$, а число $C^k_n$ — нет. Докажите, что $k$ — простое число.

\end{enumerate}
\end{document}