\documentclass{article}

\usepackage[12pt]{extsizes}
\usepackage[T2A]{fontenc}
\usepackage[utf8]{inputenc}
\usepackage[english, russian]{babel}

\usepackage{mathrsfs}
\usepackage[dvipsnames]{xcolor}

\usepackage{amsmath}
\usepackage{amssymb}
\usepackage{amsthm}
\usepackage{indentfirst}
\usepackage{amsfonts}
\usepackage{enumitem}
\usepackage{graphics}
\usepackage{tikz}
\usepackage{tabu}
\usepackage{diagbox}
\usepackage{hyperref}
\usepackage{mathtools}
\usepackage{ucs}
\usepackage{lipsum}
\usepackage{geometry} % Меняем поля страницы
\usepackage{fancyhdr} % Headers and footers
\newcommand{\range}{\mathrm{range}}
\newcommand{\dom}{\mathrm{dom}}
\newcommand{\N}{\mathbb{N}}
\newcommand{\R}{\mathbb{R}}
\newcommand{\E}{\mathbb{E}}
\newcommand{\D}{\mathbb{D}}
\newcommand{\M}{\mathcal{M}}
\newcommand{\Prime}{\mathbb{P}}
\newcommand{\A}{\mathbb{A}}
\newcommand{\Q}{\mathbb{Q}}
\newcommand{\Z}{\mathbb{Z}}
\newcommand{\F}{\mathbb{F}}
\newcommand{\CC}{\mathbb{C}}

\DeclarePairedDelimiter\abs{\lvert}{\rvert}
\DeclarePairedDelimiter\floor{\lfloor}{\rfloor}
\DeclarePairedDelimiter\ceil{\lceil}{\rceil}
\DeclarePairedDelimiter\lr{(}{)}
\DeclarePairedDelimiter\set{\{}{\}}
\DeclarePairedDelimiter\norm{\|}{\|}

\renewcommand{\labelenumi}{(\alph{enumi})}

\newcommand{\smallindent}{
    \geometry{left=1cm}% левое поле
    \geometry{right=1cm}% правое поле
    \geometry{top=1.5cm}% верхнее поле
    \geometry{bottom=1cm}% нижнее поле
}

\newcommand{\header}[3]{
    \pagestyle{fancy} % All pages have headers and footers
    \fancyhead{} % Blank out the default header
    \fancyfoot{} % Blank out the default footer
    \fancyhead[L]{#1}
    \fancyhead[C]{#2}
    \fancyhead[R]{#3}
}

\newcommand{\dividedinto}{
    \,\,\,\vdots\,\,\,
}

\newcommand{\littletaller}{\mathchoice{\vphantom{\big|}}{}{}{}}

\newcommand\restr[2]{{
    \left.\kern-\nulldelimiterspace % automatically resize the bar with \right
    #1 % the function
    \littletaller % pretend it's a little taller at normal size
    \right|_{#2} % this is the delimiter
}}

\DeclareGraphicsExtensions{.pdf,.png,.jpg}

\newenvironment{enumerate_boxed}[1][enumi]{\begin{enumerate}[label*=\protect\fbox{\arabic{#1}}]}{\end{enumerate}}



\smallindent

\header{ЦРОД $\bullet$ Математика}{\textit{Логика}}{Май 2022}

%----------------------------------------------------------------------------------------

\begin{document}
    \large

    \begin{center}
        \textbf{Рыцари и лжецы}
    \end{center}
%На далеком острове живут благородные рыцари и хитрые лжецы. Каждый рыцарь всегда говорит: только правду, а каждый лжец всегда лжет. К сожалению, выглядят рыцари и лжецы совершенно одинаково, и никогда на первый взгляд не определишь, кто перед тобой. Друг про друга все островитяне знают, кто кем является — рыцарем или лжецом.
    \begin{enumerate_boxed}
        \item Однажды островитянин Данил сказал: <<Вчера мой друг-островитянин сказал, что он лжец>>.
        Кем является сам Данил?

        \item Однажды встретились два островитянина Саша и Максим.
        Саша сказал Максиму: <<По крайней мере один из нас — лжец>>.
        Можно ли только по этой фразе определить, кто кем является?

        \item Собрались вместе два рыцаря и два лжеца и посмотрели друг на друга.
        Кто из них мог сказать фразу: 1) <<Cреди нас все рыцари>>.
        2) <<Среди вас есть ровно один рыцарь>>.
        3) <<Среди вас есть ровно два рыцаря>>?
        Для каждой фразы укажите всех, кто мог ее сказать, и объясните.

        \item Один островитянин говорит другому: <<Я лжец или ты рыцарь>>.
        Кто из островитян кто?

        \item В круг встали 12 островитян.
        Каждый из них заявил, что следующие трое после него — лжецы.
        Сколько всего рыцарей среди них?

        \item Вождь спросил у четырех жителей острова: <<Сколько рыцарей среди вас?>>.
        Первый ответил: <<Все мы лжецы>>, второй: <<Среди нас ровно один лжец>>, третий: <<Среди нас ровно два лжеца>>, а четвертый: <<Я ни разу не солгал и сейчас не лгу>>.
        Кем является четвертый житель?

        \item  В течение одного вечера в столовую поужинать зашли 100 жителей острова, и каждый из них (кроме первого) записал на специальном листе бумаги, кто вошел в столовую перед ним — рыцарь или лжец.
        Если верить всем записям, то в дом входили только лжецы.
        Сколько на самом деле лжецов входили в этот дом?

        \item За круглым столом сидят 30 человек – рыцари и лжецы.
        Известно, что у каждого из них за этим же столом есть ровно один друг, причём у рыцаря этот друг – лжец, а у лжеца этот друг – рыцарь (дружба всегда взаимна).
        На вопрос <<Сидит ли рядом с вами ваш друг?>> сидевшие через одного ответили <<Да>>.
        Сколько из остальных могли также ответить <<Да>>?

        \item Все жители острова прошли социальный опрос.
        Некоторые из них заявили, что на острове четное число рыцарей, а остальные — что на острове нечетное число лжецов.
        Может ли число жителей острова быть равно 2023?
        Известно, что хотя бы один рыцарь и хотя бы один лжец на острове есть.

        \item 20 островитян приехали на турнир по настольным играм.
        В первый день турнира все собравшиеся сели за круглый стол, и перед началом каждый заявил: <<Оба моих соседа лжецы>>.
        Во второй день один островитянин заболел, и за круглый стол сели только 19 игроков.
        На этот раз каждый сказал: <<Раса обоих моих соседей отличается от моей>>.
        Кто заболел: рыцарь или лжец?

        \item На острове живёт 10 человек Каждый из жителей задумал какое-то целое число.
        Затем первый сказал: <<Моё число больше 1>>, второй сказал: <<Моё число больше 2>>, \ldots, десятый сказал: <<Моё число больше 10>>.
        После этого все десять, выступая в некотором порядке, сказали: <<Моё число меньше 1>>, <<Моё число меньше 2>>, \ldots, <<Моё число меньше 10>>.
        Какое максимальное число рыцарей могло быть среди этих 10 человек?

        \item По кругу сидят 2021 человек, каждый из которых либо рыцарь, который всегда говорит правду, либо лжец, который всегда лжет.
        Каждый из них сказал: «Если моего соседа справа спросить, кем является мой сосед слева, то он ответит — лжецом.». Сколько лжецов за этим столом?

    \end{enumerate_boxed}

\end{document}