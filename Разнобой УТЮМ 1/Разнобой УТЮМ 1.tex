\documentclass{article}
\usepackage[12pt]{extsizes}
\usepackage[T2A]{fontenc}
\usepackage[utf8]{inputenc}
\usepackage[english, russian]{babel}

\usepackage{amssymb}
\usepackage{amsfonts}
\usepackage{amsmath}
\usepackage{enumitem}
\usepackage{graphics}
\usepackage{graphicx}

\usepackage{lipsum}

\newtheorem{theorem}{Теорема}
\newtheorem{task}{Задача}
\newtheorem{lemma}{Лемма}
\newtheorem{definition}{Определение}
\newtheorem{example}{Пример}
\newtheorem{statement}{Утверждение}
\newtheorem{corollary}{Следствие}


\usepackage{geometry} % Меняем поля страницы
\geometry{left=1cm}% левое поле
\geometry{right=1cm}% правое поле
\geometry{top=1.5cm}% верхнее поле
\geometry{bottom=1cm}% нижнее поле


\usepackage{fancyhdr} % Headers and footers
\pagestyle{fancy} % All pages have headers and footers
\fancyhead{} % Blank out the default header
\fancyfoot{} % Blank out the default footer
\fancyhead[L]{Математика}
\fancyhead[C]{\textit{Разное}}
\fancyhead[R]{24 ноября}% Custom header text


%----------------------------------------------------------------------------------------

%\begin{document}\normalsize
\begin{document}\large
	
\begin{center}
	\textbf{Разнобой УТЮМа 1}
\end{center}


\begin{enumerate}[label*=\protect\fbox{\arabic{enumi}}]
	
\item Можно ли заполнить клетки таблицы $2020 \times 2020$ натуральными числами от $1$ до $4 080 400$ так, чтобы сумма чисел в каждой строке, начиная со второй, была на 1 больше, чем сумма чисел во всех предыдущих строках?

\item В остроугольном треугольнике $ABC$ проведена высота $AH$ и отмечены середины $A_1$, $B_1$ и $C_1$ сторон $BC$, $CA$ и $AB$ соответственно. Точка $K$ симметрична точке $B_1$ относительно прямой $BC$. Докажите, что прямая $C_1K$ делит отрезок $HA_1$ пополам.

\item Каждую клетку доски $2022 \times 2022$ красят в чёрный или белый цвет. В некоторые клетки ставят хромых ферзей. Хромой ферзь с клетки $A$ бьёт клетку $B$, если клетки $A$ и $B$ находятся на одной линии (горизонтали, вертикали или диагонали) и все клетки этой линии от $A$ до $B$ включительно покрашены в один цвет. При каком наибольшем $k$ можно покрасить доску и расставить на ней $k$ хромых ферзей так, чтобы они не били друг друга?

\item На столе стоит несколько гирь суммарного веса $s$. Назовём гирю раздвоителем, если после её удаления все остальные гири можно разбить на две группы, суммарный вес каждой из которых не больше $\frac{s}{2}$. Докажите, что вес самого большого раздвоителя больше суммы весов всех нераздвоителей.

\item Высоты $AA_1$, $BB_1$ и $CC_1$ остроугольного треугольника $ABC$ пересекаются в точке $H$. Прямые $AA_1$ и $B_1C_1$ пересекаются в точке $X$. Перпендикуляр к $AC$, проведённый через точку $X$, пересекает сторону $AB$ в точке $Y$. Докажите, что прямая $YA_1$ делит отрезок $BH$ пополам.

\item Нечётная раскраска графа – это такая раскраска множества его вершин в несколько цветов, что любые две соседние вершины покрашены в разный цвет и при этом для каждой вершины можно указать цвет, в который покрашено нечётное число её соседей. Барон Мюнхгаузен нарисовал граф и создал нечётную раскраску его вершин в $1022$ цвета. «Вы можете мне не поверить, друзья, — говорит барон, — но на этом графе не существует нечётных раскрасок с меньшим числом цветов. Однако после того как я добавил всего одну вершину и соединил её с некоторыми вершинами этого графа, для нечётной раскраски мне понадобилось всего три цвета». Не обманывает ли нас барон?

\item Докажите, что нечётное число $p > 1$ — простое тогда и только тогда, когда среди любых $\dfrac{p+1}{2}$ различных натуральных чисел можно найти два числа, сумма которых хотя бы в $p$ раз больше их наибольшего общего делителя.


\end{enumerate}
\end{document}