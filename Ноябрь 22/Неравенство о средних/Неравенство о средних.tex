\documentclass{article}
\usepackage[12pt]{extsizes}
\usepackage[T2A]{fontenc}
\usepackage[utf8]{inputenc}
\usepackage[english, russian]{babel}

\usepackage{amssymb}
\usepackage{amsfonts}
\usepackage{amsmath}
\usepackage{enumitem}
\usepackage{graphics}

\usepackage{lipsum}



\usepackage{geometry} % Меняем поля страницы
\geometry{left=1cm}% левое поле
\geometry{right=1cm}% правое поле
\geometry{top=1.5cm}% верхнее поле
\geometry{bottom=1cm}% нижнее поле


\usepackage{fancyhdr} % Headers and footers
\pagestyle{fancy} % All pages have headers and footers
\fancyhead{} % Blank out the default header
\fancyfoot{} % Blank out the default footer
\fancyhead[L]{ЦРОД $\bullet$ Математика}
\fancyhead[C]{\textit{Алгебра}}
\fancyhead[R]{1 ноября 2022}% Custom header text


%----------------------------------------------------------------------------------------

%\begin{document}\normalsize
\begin{document}\large
	
	
\begin{center}
	\textbf{Неравенство о средних}
\end{center}
	

\textit{Неравенство о средних} — это неравенство между \textit{средним квадратическим, средним арифметическим, средним геометрическим и средним гармоническим}: $$\sqrt{\frac{x_1^2+x_2^2+\dots+x_n^2}{n}}\ge\frac{x_1+x_2+\dots+x_n}{n}\ge\sqrt[n]{x_1x_2\dots x_n}\ge \dfrac{n}{\frac{1}{x_1}+\frac{1}{x_2}\dots+\frac{1}{x_n}}$$для любых \textbf{положительных} чисел $x_1, x_2, \dots , x_n$, причём равенство достигается тогда и только
тогда, когда $x_1 =x_2 =\dotsc=x_n$.
Частный случай, 
$$\sqrt{\dfrac{a^2 + b^2}{2}} \ge \dfrac{a+b}{2} \ge \sqrt{ab}  \ge \dfrac{2}{\frac{1}{a}+\frac{1}{b}}$$
	
\begin{enumerate}[label*=\protect\fbox{\arabic{enumi}}]

\item $1+x\ge 2\sqrt{x}$ при $x \geqslant 0$ через неравенство о средних.

\item $\dfrac{x}{y} + \dfrac{y}{x} \geqslant 2$  при $x, y > 0$ через неравенство о средних.

\item $\dfrac{1}{x}+\dfrac{1}{y}\geqslant\dfrac{4}{x+y}$ при $x, y > 0$ через неравенство о средних.

\item $\displaystyle \frac{a}{b} + \frac{b}{c} + \frac{c}{a} \geq 3$ при $a, b, c > 0$ .

\item $2(x^2 + y^2) \geqslant (x + y)^2$  при любых $x$ и $y$ через неравенство о средних.

\item $\dfrac{a+3b}{4}\geqslant \sqrt[4]{ab^3}$, при $a, b \geqslant 0$

\item $\displaystyle \frac{a^6 + b^9}{4}\geqslant 3a^2b^3-16$, при $b \geqslant 0$.

\item  $2x + \dfrac{3}{8} \geqslant \sqrt[4]{x}$, при $x \geqslant 0$.

\item $(2+x)(2+y)(2+z) \geqslant 27$, если $xyz = 1$ и $x, y, z > 0$.

\item $\dfrac{a}{b + c + d} + \dfrac{b}{a + c + d} + \dfrac{c}{a + b + d} + \dfrac{d}{a + b + c} \geqslant \dfrac{4}{3}$, при положительных $a, b, c, d$.

\item $\sqrt{a}+\sqrt{b}+\sqrt{c}\geqslant ab+bc+ac$, если $a+b+c = 3$.

\item Докажите, что для любого натурального $n$ выполнено неравенство  \\$(n - 1)^{n+1}(n + 1)^{n-1}< n^{2n}$.
		
	\end{enumerate}

\newpage

\begin{center}
	\textbf{Транснеравенство}
\end{center}

\textbf{Транснеравенство.} \textit{Пусть $a_1 \geqslant a_2 \geqslant a_3 \geqslant \ldots \geqslant a_n$ и $b_1 \geqslant b_2 \geqslant b_3 \geqslant \ldots \geqslant b_n$. И пусть числа $c_1, c_2, \ldots, c_n$ --- некоторая перестановка чисел $b_1, b_2, \ldots, b_n$. Тогда $$a_1b_1 + a_2b_2 + \ldots + a_nb_n \geqslant a_1c_1 + a_2c_2 + \ldots + a_nc_n \geqslant a_1b_n + a_2b_{n-1} + \ldots + a_nb_1.$$}

Во всех предложенных задачах подразумевается, что рассматриваемые числа положительны.\\[5pt]


\begin{enumerate}[label*=\protect\fbox{\arabic{enumi}}]
	
	\item  Докажите, что $$a^4 + b^4 + c^4 \geqslant a^3b + b^3c + c^3a.$$
	
	\item Докажите, что $$\frac{a^3}{b} + \frac{b^3}{c} + \frac{c^3}{a} \geqslant a^2 + b^2 + c^2.$$
	
	\item Докажите, что $$\frac{a_1}{a_2} + \frac{a_2}{a_3} + \ldots \frac{a_{n-1}}{a_n} + \frac{a_n}{a_1} \geqslant n.$$
	
	\item Докажите, что $$\frac{x}{x+y} + \frac{y}{y+z} + \frac{z}{z+x} \leqslant \frac{x}{y+z} + \frac{y}{z+x} + \frac{z}{x+y}.$$
	
	\item Докажите неравенство $$a + b + c \geqslant \frac{a(b+1)}{a+1} + \frac{b(c+1)}{b+1} + \frac{c(a+1)}{c+1}.$$
	
	\item Докажите неравенство $$\sqrt{ab} + \sqrt{ac} + \sqrt{ad} + \sqrt{bc} + \sqrt{bd} + \sqrt{cd} \leqslant \frac{3}{2}(a + b + c + d).$$
	
	\item Докажите неравенство $$\frac{a}{b(b+c)} + \frac{b}{c(c+a)} + \frac{c}{a(a+b)} \geqslant \frac{1}{a+b} + \frac{1}{b+c} + \frac{1}{c+a}.$$
	
	%\item Докажите, что: 
	%\footnotesize
	%$$\frac{((a-2)^2 + 2bc)((b-2)^2 + 2ca)}{(c-2)^2 + 2ab} + \frac{((b-2)^2 + 2ca)((c-2)^2 + 2ab)}{(a-2)^2 + 2bc} + \frac{((c-2)^2 + 2ab)((a-2)^2 + 2bc)}{(b-2)^2 + 2ca} \ge 8.$$
	
	\normalsize
	\item\textbf{Неравенство Чебышева.} Пусть $a_1 \geqslant a_2 \geqslant \ldots \geqslant a_n$ и $b_1 \geqslant b_2 \geqslant \ldots \geqslant b_n$. Докажите, что $$\frac{a_1b_1 + a_2b_2 + \ldots a_nb_n}{n} \geqslant \frac{a_1 + a_2 + \ldots + a_n}{n} \cdot \frac{b_1 + b_2 + \ldots b_n}{n}.$$
	
\end{enumerate}

\newpage 

\begin{center}
	\textbf{Неравенство КБШ}
\end{center}

Теперь докажем \textbf{неравенство Коши-Буняковского-Шварца (КБШ)}: для двух произвольных наборов вещественных чисел $a_1, a_2, \ldots, a_n$ и $b_1, b_2, \ldots, b_n$ выполнено неравенство
$$
(a_1^2+a_2^2+\ldots+a_n^2)(b_1^2+b_2^2+\ldots+b_n^2) \geqslant (a_1b_1+a_2b_2+\ldots+a_nb_n)^2.
$$

Для этого рассмотрим следующий вспомогательный квадратный трехчлен: $(a_1^2+a_2^2+\ldots+a_n^2)x^2+2\cdot(a_1b_1+\ldots+a_nb_n)x+(b_1^2+b_2^2+\ldots+b_n^2)$.
\textbf{Контрольный вопрос.} Когда в неравенстве КБШ достигается равенство?

\begin{enumerate}[label*=\protect\fbox{\arabic{enumi}}]
	
\item Пусть $a_1$, $a_2$, \dots, $a_n$ и $b_1$, $b_2$, \dots, $b_n$ ---
положительные числа. Докажите неравенство
$$ \left(a_1 b_1 +\ldots + a_n b_n \right) \left( \frac{a_1}{b_1} + \ldots + \frac{a_n}{b_n} \right) \geqslant
\left( a_1+\ldots+a_n \right)^2 .$$

\item \textbf{Важнейшая форма КБШ.} Докажите через КБШ, выбрав два нужных набора, КБШ для дробей: при \textit{положительных} $a_1, a_2, \ldots, a_n$ и $b_1, b_2, \ldots, b_n$ выполнено неравенство
$$
\frac{a_1^2}{b_1}+\frac{a_2^2}{b_2}+\ldots+\frac{a_n^2}{b_n} \geqslant \frac{(a_1+a_2+\ldots+a_n)^2}{b_1+b_2+\ldots+b_n}.
$$

\item Суммы двух наборов положительных чисел $a_1, a_2, \ldots, a_n$ и $b_1, b_2, \ldots, b_n$ равны. Докажите неравенство
$$
\frac{a_1^2}{a_1+b_1}+\frac{a_2^2}{a_2+b_2}+\ldots+\frac{a_n^2}{a_n+b_n} \geqslant \frac{a_1+a_2+\ldots+a_n}{2}.
$$

\item Для положительных $a$, $b$, $c$, $d$ докажите неравенство
$$
\frac{a^2}{b(a+c)} + \frac{b^2}{c(b+d)} + 
\frac{c^2}{d(a+c)} + \frac{d^2}{a(d+b)} \geqslant 2.
$$

\item Докажите, что при всех положительных $a$, $b$, $c$, $d$ выполнено
$$
\frac{a}{b+2c+3d} + \frac{b}{c+2d+3a} + \frac{c}{d+2a+3b} +
\frac{d}{a+2b+3c} \geqslant \frac{2}{3}.
$$

%\textit{Указание.} КБШ для дробей хорошо применять, когда в числителях дробей стоят квадраты. А как же их там получить?

\item Для положительных $a$, $b$, $c$, удовлетворяющих условию $abc=1$, докажите неравенство
$$
\frac{1}{a^3(b+c)}+\frac{1}{b^3(c+a)}+\frac{1}{c^3(a+b)} \geqslant \frac32.
$$
%вроде не делается срарзу, но надо разделить первую дробь на $a^2$ и т.д.

\item Даны положительные числа  $a,b,c$, сумма которых не меньше двух.
Докажите неравенство
$$\frac{a}{b\sqrt [3] {c}+a}+\frac{b}{c\sqrt [3] {a}+b}+\frac{c}{a\sqrt [3] {b}+c}\leqslant 2.$$

\end{enumerate}

\end{document}