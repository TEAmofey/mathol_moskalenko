\documentclass{article}
\usepackage[12pt]{extsizes}
\usepackage[T2A]{fontenc}
\usepackage[utf8]{inputenc}
\usepackage[english, russian]{babel}

\usepackage{amssymb}
\usepackage{amsfonts}
\usepackage{amsmath}
\usepackage{enumitem}
\usepackage{graphics}

\usepackage{lipsum}



\usepackage{geometry} % Меняем поля страницы
\geometry{left=1cm}% левое поле
\geometry{right=1cm}% правое поле
\geometry{top=1.5cm}% верхнее поле
\geometry{bottom=1cm}% нижнее поле


\usepackage{fancyhdr} % Headers and footers
\pagestyle{fancy} % All pages have headers and footers
\fancyhead{} % Blank out the default header
\fancyfoot{} % Blank out the default footer
\fancyhead[L]{ЦРОД $\bullet$ Математика}
\fancyhead[C]{\textit{Комбинаторика}}
\fancyhead[R]{31 октября 2021}% Custom header text


%----------------------------------------------------------------------------------------

%\begin{document}\normalsize
\begin{document}\large
	
	
\begin{center}
	\textbf{Двойной подсчёт}
\end{center}



\begin{enumerate}[label*=\protect\fbox{\arabic{enumi}}]

\item Можно ли расставить числа в таблице $6 \times 9$ так, чтобы в каждом столбце была сумма по $10$, а в каждой строке — по $20$?

\item В прямоугольной таблице $8$ столбцов, сумма в каждом столбце — по $10$, а в каждой строке — по $20$. Сколько в таблице строк?

\item В конференции участвовали $19$ ученых. После конференции каждый из них отправил $2$ или $4$ письма участникам этой конференции. Могло ли получиться так, что каждый участник получил по $3$ письма, если письма на почте не теряют?

\item Даны шесть 4-элементных подмножеств множества из 8-ми элементов, причём каждый из этих элементов лежит ровно в $m$ множествах. Найдите $m$.

\item Дано $25$ чисел. Какие бы три из них мы ни выбрали, среди оставшихся найдётся такое четвёртое, что сумма этих четырёх чисел будет положительна. Верно ли, что сумма всех чисел положительна?

\item Несколько восьмиклассников и девятиклассников обменялись рукопожатиями. При этом каждый восьмиклассник пожал руку девяти девятиклассникам, а каждый девятиклассник — восьми восьмиклассникам.

Кого среди них было больше — восьмиклассников или девятиклассников?

\item Комитет провел 40 заседаний, на каждом было ровно 10 присутствующих. При этом каждые два члена комитета встретились не более чем на одном заседании. Докажите, что в комитете > 60 членов.

\item Игорь закрасил в квадрате $6\times6$ несколько клеток. После этого оказалось, что во всех квадратиках $2\times2$ одинаковое число закрашенных клеток и во всех полосках $1\times3$ одинаковое число закрашенных клеток. Докажите, что старательный Игорь закрасил все клетки.

\item Можно ли занумеровать рёбра куба числами $1, 2, ..., 11, 12$ так, чтобы для каждой вершины сумма номеров трёх выходящих из неё рёбер была одной и той же.

\item Футбольный мяч сшит из $32$ лоскутов: белых шестиугольников и черных пятиугольников. Каждый черный лоскут граничит с пятью белыми, а каждый белый — с тремя черными и тремя белыми. Сколько лоскутов белого цвета?

\item В городе от каждой площади отходит ровно $5$ улиц. Докажите, что число площадей четно, а число улиц делится на $5$. (Улицы соединяют площади.)

\item Взяли несколько одинаковых равносторонних треугольников. Вершины каждого из них пометили цифрами $1, 2$ и $3$. Затем их сложили в стопку. Могло ли оказаться, что сумма чисел, находящихся в каждом углу, равна $55$?

\item Дано $2023$ число. Известно, что сумма любых четырёх чисел положительна. Верно ли, что сумма всех чисел положительна?

\item Докажите, что никакой выпуклый многоугольник нельзя порезать на невыпуклые четырёхугольники.

\item Можно ли в таблицу $5 \times 5$ записать числа $1, 2, 3,
\dots, 25$ так, чтобы в каждой строке сумма нескольких
записанных чисел была равна сумме остальных чисел
этой строки?

\item По окружности отметили $40$ красных, $30$ синих и $20$ зеленых точек. На каждой дуге между соседними красной и синей точками поставили цифру $1$, на каждой дуге между соседними красной и зеленой – цифру $2$, а на каждой дуге между соседними синей и зеленой – цифру $3$. (На дугах между одноцветными точками поставили $0$.) Найдите максимальную возможную сумму поставленных чисел.

\item В парламенте несколько человек, они образовали несколько комитетов, при этом все комитеты имеют одинаковую численность. Для каждой пары парламентёров количество комитетов, в которые они оба входят, одинаковое, т.е. не зависит от того, какую пару парламентёров мы выбрали. Докажите, что все парламентёры входят в одно и то же число комитетов.

\item Дан набор, состоящий из таких $2021$ числа, что если каждое число в наборе заменить на сумму остальных, то получится тот же набор. 
Докажите, что произведение чисел в наборе равно $0$.

\item По кругу расставлены красные и синие числа. Каждое красное число равно сумме соседних чисел, а каждое синее~--- полусумме соседних чисел. Докажите, что сумма красных чисел равна нулю. 

\item В некоторых клетках прямоугольной таблицы нарисованы звездочки. Известно, что для любой отмеченной клетки количество звездочек в её столбце совпадает с количеством звездочек в её строке. Докажите, что число строк в таблице, в которых есть хоть одна звездочка, равно числу столбцов таблицы, в которых есть хоть одна звездочка.

\item В библиотеке на полках стоят книги, ровно $k$ полок пусты. Книги переставили так, что теперь пустых полок нет. Докажите, что найдётся хотя бы $k + 1$ книга, которая теперь стоит на полке с меньшим числом книг, чем стояла раньше.



\end{enumerate}
\end{document}