\documentclass{article}
\usepackage[12pt]{extsizes}
\usepackage[T2A]{fontenc}
\usepackage[utf8]{inputenc}
\usepackage[english, russian]{babel}

\usepackage{amssymb}
\usepackage{amsfonts}
\usepackage{amsmath}
\usepackage{enumitem}
\usepackage{graphics}
\usepackage{graphicx}

\usepackage{lipsum}

\newtheorem{theorem}{Теорема}
\newtheorem{task}{Задача}
\newtheorem{lemma}{Лемма}
\newtheorem{definition}{Определение}
\newtheorem{example}{Пример}
\newtheorem{statement}{Утверждение}
\newtheorem{corollary}{Следствие}


\usepackage{geometry} % Меняем поля страницы
\geometry{left=1cm}% левое поле
\geometry{right=1cm}% правое поле
\geometry{top=1.5cm}% верхнее поле
\geometry{bottom=1cm}% нижнее поле


\usepackage{fancyhdr} % Headers and footers
\pagestyle{fancy} % All pages have headers and footers
\fancyhead{} % Blank out the default header
\fancyfoot{} % Blank out the default footer
\fancyhead[L]{ЦРОД $\bullet$ Математика}
\fancyhead[C]{\textit{Теория чисел}}
\fancyhead[R]{31 октября 2022}% Custom header text


%----------------------------------------------------------------------------------------

%\begin{document}\normalsize
\begin{document}\large
	
	
	\begin{center}
		\textbf{Обратный остаток}
	\end{center}

Обратное по модулю целого $a$ --- это такое целое число $x$, что произведение $ax$ сравнимо с 1 по модулю $m$.

\textbf{Теорема.} Если $(a,m) = 1$, то у $a$ есть обратный остаток по модулю $m$.


\begin{enumerate}[label*=\protect\fbox{\arabic{enumi}}]
	
\item Дано простое число $p$ и его некоторый ненулевой остаток $a$.
	\begin{enumerate}
		\item Докажите, что в последовательности $0 \cdot a, 1 \cdot a, 2 \cdot a, \dots ,(p - 1) \cdot a$ все числа дают разные остатки по модулю $p$.
		
		\item Докажите, что существует и при том единственный обратный остаток $b$
		
		\item Какие остатки совпадают со своими обратными остатками?
	\end{enumerate}

%\item Какой остаток дает $x + y$ при делении на $17$, если

%\begin{enumerate}
%\item $x-16y\equiv 2$ (mod $17$);

%\item $3x \equiv 5+14y$ (mod $17$);

%\item $-10x \equiv 100+27y$ (mod $17$);

%\item $28x + 10\equiv -11y$ (mod $17$);

%\item $34x - 8\equiv 14(y + x)$ (mod $17$);

%\item $1000x \equiv -1085y - 90$ (mod $17$)?
%\end{enumerate}
% 2 7
% 3 9
% 4 10
% 5 8
% 6 11
\item 
Решите сравнения (то есть найдите все подходящие $x$ и докажите, что других
нет)
\begin{enumerate}
	\item $5x \equiv 2 \pmod 3$;
	
	\item $3x \equiv 2 \pmod {11}$;
	
	\item $6x \equiv 1 \pmod {13}$;
	
\end{enumerate}


\item Какой остаток даёт $x$ при делении $13$, если:
	\begin{enumerate}
	 	\item $3x \equiv 4 + x \pmod {13}$;
	 	
	 	\item $7x \equiv 8 + 3x \pmod {13}$;
	 	
	 	\item $10x + 2 \equiv -x \pmod {13}$. 
	\end{enumerate}
\item \textbf{(Теорема Вильсона.)} Докажите, что $(p-1)!\equiv -1\pmod p$, если и только если $p$ является простым числом.

\item Пусть $p$ -- простое число и $k \leqslant p$. Докажите, что $(p - k)!(k - 1)! \underset{p}{\equiv} (-1)^k$.

\item Для простого числа $p$ и остатка a определим его \textit{показатель} по модулю $p$ как наименьшее такое натуральное число $d$, что $a^d \equiv 1 \bmod p$. Рассмотрим произведение всех остатков по модулю $p$, которые имеют одинаковый показатель. Какой остаток от деления на $p$ даёт это произведение?


\item Пусть числа $p$ и $p + 2$ являются простыми числами-близнецами. Докажите, 
что справедливо $4((p - 1)! + 1) + p \equiv 0 \pmod{p^2+2p}$.

\item Даны натуральные числа $a, b$ и $c$ такие, что $ab+9b+81$ и $bc+9c+81$ делятся на $101$. Докажите, что тогда и $ca + 9a + 81$ тоже делится на $101$.


\item Пусть $p \geqslant 3$~--- простое число. Докажите, что если сумму $\frac{1}{1} + \frac{1}{2} + \ldots+\frac{1}{p-1}$ привести к общему знаменателю, то числитель получившейся дроби будет делиться на $p$.

\item  На доске написаны числа $\frac{100}{1}, \frac{99}{2}, \ldots, \frac{1}{100}$. Можно ли выбрать какие-то девять из них, произведение которых равняется единице?

\item Докажите, что для любого простого $p > 3$ существует бесконечно много $n$ таких, что $2^n +3^n +6^n -1$ делится на $p$.


\end{enumerate}
\end{document}