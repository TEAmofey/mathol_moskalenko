\documentclass{article}
\usepackage[12pt]{extsizes}
\usepackage[T2A]{fontenc}
\usepackage[utf8]{inputenc}
\usepackage[english, russian]{babel}

\usepackage{amssymb}
\usepackage{amsfonts}
\usepackage{amsmath}
\usepackage{enumitem}
\usepackage{graphics}

\usepackage{lipsum}



\usepackage{geometry} % Меняем поля страницы


\geometry{top=1.5cm}% верхнее поле
\geometry{bottom=1cm}% нижнее поле
\geometry{left=1cm}% левое поле
\geometry{right=1cm}% правое поле


\usepackage{fancyhdr} % Headers and footers
\pagestyle{fancy} % All pages have headers and footers
\fancyhead{} % Blank out the default header
\fancyfoot{} % Blank out the default footer
\fancyhead[L]{ЦРОД $\bullet$ Математика}
\fancyhead[C]{\textit{Битва}}
\fancyhead[R]{4 ноября 2022}% Custom header text


%----------------------------------------------------------------------------------------

%\begin{document}\normalsize
\begin{document}\large


\begin{center}
	\textbf{$\mathbb{M}$@т$e$м@т$i4e\mathbb{C}$кий boy!}
\end{center}

\begin{enumerate}


\item Можно ли в ряд расставить $200$ натуральных чисел от $1$ до $200$ так, чтобы любые два числа, которые оказались соседними, отличались друг от друга либо в два раза, либо на два. 
\item  У котиков лапки и они не могут доказать, что для любых чисел $x$ и $y$ таких, что $xy > 0$ справедливо неравенство
$$\frac{x}{x^4 + y^2} + \frac{y}{y^4 + x^2} \leqslant \frac{1}{xy}$$
Помогите им!

\item Лёша очень любит жарить котлеты. У него есть волшебная плита, которая из $x$ котлет делает $f(x) =  x^2 + 10x + 20$ котлет. Плита настолько волшебная, что может готовить нецелое и даже отрицательное число котлет. У плиты есть особый режим <<пять подряд>>, при котором из $x$ котлет плита готовит $f(f(f(f(f(x)))))$ котлет. Лёша вычислил, сколько котлет нужно положить в плиту, чтобы при особом режиме она приготовила 0 котлет. Что у него получилось?

\item Влад написал натуральное число, кратное 495. Ксюша вставила между его цифрами два нуля подряд. Докажите, что полученное число тоже делится на 495.


\item  Есть $55$ равносторонних треугольников. Из них сложили полоску. Двое игроков, Аня и Лёня, красят ребра треугольников по очереди в любой из двух цветов --- серо-буро-малиновый или желто-розовый. Проигрывает игрок, если после его хода образовался равносторонний треугольник одного цвета. Кто выиграет при правильной игре? 

\item В стране Бангладеш десять городов. Некоторые пары городов соединены дорогами.
Известно, что есть два города, не соединенные дорогами, и ни для какой тройки городов количество дорог между ними не равно двум. Найдите наибольшее возможное количество дорог в Бангладеше.

\item Точка $M$ --- середина стороны $AC$ треугольника $ABC$. На стороне $BC$ отмечена такая точка $N$, что $BN:NC = 2:1$. Оказалось, что  $\angle BMN$ прямой. Докажите, что $AB = AM$.

\item У Анастасии Константиновны на сковородке лежит 100 котлет. Котлета приготовлена замечательно, если она жарилась с обеих сторон одинаковое количество раз. В ожидании котлет Тимофей Дмитриевич прошёл мимо сковородки 100 раз. Ему было скучно и каждый раз, когда он проходил мимо, он переворачивал некоторые котлеты. Сначала с номерами $1,2,3,4,\dotsc$; потом с номерами $2,4,6,8,\dotsc$; потом $3,6,9,12,\dotsc$ и так далее. Сколько котлет в итоге будут приготовлены замечательно?






\end{enumerate}
\end{document}