\documentclass{article}
\usepackage[12pt]{extsizes}
\usepackage[T2A]{fontenc}
\usepackage[utf8]{inputenc}
\usepackage[english, russian]{babel}

\usepackage{amssymb}
\usepackage{amsfonts}
\usepackage{amsmath}
\usepackage{enumitem}
\usepackage{graphics}
\usepackage{graphicx}

\usepackage{lipsum}

\newtheorem{theorem}{Теорема}
\newtheorem{task}{Задача}
\newtheorem{lemma}{Лемма}
\newtheorem{definition}{Определение}
\newtheorem{example}{Пример}
\newtheorem{statement}{Утверждение}
\newtheorem{corollary}{Следствие}


\usepackage{geometry} % Меняем поля страницы
\geometry{left=1cm}% левое поле
\geometry{right=1cm}% правое поле
\geometry{top=1.5cm}% верхнее поле
\geometry{bottom=1cm}% нижнее поле


\usepackage{fancyhdr} % Headers and footers
\pagestyle{fancy} % All pages have headers and footers
\fancyhead{} % Blank out the default header
\fancyfoot{} % Blank out the default footer
\fancyhead[L]{ЦРОД $\bullet$ Математика}
\fancyhead[C]{\textit{Теория чисел}}
\fancyhead[R]{3 ноября 2022}% Custom header text


%----------------------------------------------------------------------------------------

%\begin{document}\normalsize
\begin{document}\large
	
\begin{center}
	\textbf{Малая теорема Ферма}
\end{center}

\begin{theorem}
	Для любого простого $p$ и целого $a$ верно сравнение $a^p \equiv a \pmod p$
\end{theorem}

\begin{enumerate}[label*=\protect\fbox{\arabic{enumi}}]
	
\item Докажите, что $7^{120} - 1$  делится на $143$.
	
\item Докажите, что $60^{111}+111^{60}$ делится на 61.

\item Пусть $p$ – простое число. Докажите, что  $(a + b)^p \equiv a^p + b^p$ (mod $p$) для любых целых $a$ и $b$.

\item Известно, что  $a^{12} + b^{12} + c^{12} + d^{12} + e^{12} + f^{12}$  делится на 13 ($a, b, c, d, e, f$ – целые числа). Докажите, что $abcdef$ делится на 4826809.

\item Докажите, что если $p$ --- простое число и $p > 2$, то $7^p - 5^p - 2$ делится на $6p$.

\item Пусть $p>5$ --- простое число. Докажите, что $\underbrace{11\dots 1}_{p-1}$ делится на $p$.

\item Пусть $n$ – натуральное число, не кратное 17. Докажите, что либо $n^8 + 1$,  либо $n^4 + 1$,  либо $n^2 + 1$,  либо $n + 1$, либо $n - 1$  делится на 17.

\item Докажи, что $a^{73} - a$  делится на $2\cdot3\cdot5\cdot7\cdot13\cdot19\cdot37\cdot73$.

\item Докажите, что ни при каком целом $k$ число $k^2 + k + 1$  не делится на 101.

\item Андрей берет натуральное число $a$ и сначала прибавляет к нему число $a^2$, потом --- $a^3$, потом --- $a^4$ и т.д. Докажите, что когда-нибудь его сумма поделится на простое число $p$.

\item Пусть $p = 4k+3$ --- простое, а $m^2 + n^2 \, \vdots \, p$. Докажите, что $m,n \, \vdots \, p$.

\item Найти все такие натуральные числа $p$, что $p$ и $p^6 + 6$ – простые

\item 
a) Докажите, что равенство $\frac{10^n-1}{m}=\overline{a_1a_2\dots a_n}$ равносильно тому, что десятичная запись дроби $1/m$ имеет вид $0,(a_1a_2\dots a_n)$.

b) Пусть $p>5$ --- простое число.  Докажите, что $1/p=0,(a_1a_2\dots a_n)$ (т.е. дробь не имеет предпериода).

c) Запишем $1/p=0,(a_1a_2\dots a_n)$. Докажите, что $p-1$ делится на $n$.

d) Запишем $1/p=0,(a_1a_2\dots a_n)$. Докажите, что дробь $0,(a_2a_3\dots a_na_1)$ --- тоже дробь со знаменателем $p$.

\item Найдите такое шестизначное число $x$, что $x$, $2x$, $3x$, $4x$, $5x$ и $6x$ записываются одинаковым набором цифр, но в различном порядке.

\end{enumerate}
\end{document}