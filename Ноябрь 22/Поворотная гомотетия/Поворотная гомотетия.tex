\documentclass{article}
\usepackage[12pt]{extsizes}
\usepackage[T2A]{fontenc}
\usepackage[utf8]{inputenc}
\usepackage[english, russian]{babel}

\usepackage{amssymb}
\usepackage{amsfonts}
\usepackage{amsmath}
\usepackage{enumitem}
\usepackage{graphics}
\usepackage{graphicx}

\usepackage{lipsum}

\newtheorem{theorem}{Теорема}
\newtheorem{task}{Задача}
\newtheorem{lemma}{Лемма}
\newtheorem{definition}{Определение}
\newtheorem{example}{Пример}
\newtheorem{statement}{Утверждение}
\newtheorem{corollary}{Следствие}


\usepackage{geometry} % Меняем поля страницы
\geometry{left=1cm}% левое поле
\geometry{right=1cm}% правое поле
\geometry{top=1.5cm}% верхнее поле
\geometry{bottom=1cm}% нижнее поле


\usepackage{fancyhdr} % Headers and footers
\pagestyle{fancy} % All pages have headers and footers
\fancyhead{} % Blank out the default header
\fancyfoot{} % Blank out the default footer
\fancyhead[L]{ЦРОД $\bullet$ Математика}
\fancyhead[C]{\textit{Геометрия}}
\fancyhead[R]{2 ноября 2022}% Custom header text


%----------------------------------------------------------------------------------------

%\begin{document}\normalsize
\begin{document}\large

\begin{center}
	\textbf{Гомотетия (поворотная)}
\end{center}

\textbf{Основные свойства поворотной гомотетии:}

\begin{enumerate}[label*=\protect\fbox{\arabic{enumi}}]
	
\item Если на плоскости даны непараллельные отрезки $AB$ и $A'B'$, то поворотная гомотетия, переводящая $A$ в $A'$, а $B$ в $B'$, определяется однозначно. Если обозначить точку пересечения прямых $AB$ и $A'B'$ за $X$, то центр искомой поворотной гомотетии лежит на втором пересечении окружностей, описанных около треугольников $AA'X$ и $BB'X$.

\item Если точка $O$ является центром поворотной гомотетии, переводящей отрезок $AB$ в $CD$, то она является и центром поворотной гомотетии, переводящей $AC$ в $BD$.

\item Пусть две окружности пересекаются в точках $P$ и $Q$. Пусть два велосипедиста $A$ и $B$ одновременно выезжают из точки $P$, один по первой окружности, а другой по второй, причем их угловые скорости на соответствующих окружностях совпадают. Тогда прямая $AB$ всегда будет проходить через точку $Q$.

\item Докажите, что середины отрезков $AB$ лежат на одной окружности.

\item Докажите, что существует точка, равноудалённая от точек $A$ и $B$ в каждый момент времени.

\end{enumerate}

\textbf{Задачи:}

\begin{enumerate}[label*=\protect\fbox{\arabic{enumi}}]

\item Дан квадрат $ABCD$. Точки $P$ и $Q$ лежат соответственно на сторонах $AB$ и $BC$, причём $BP =BQ$. Пусть $H$ — основание перпендикуляра, опущенного из точки $B$ на отрезок $PC$. Докажите, что $\angle DHQ = 90^\circ$.

\item На катетах прямоугольного треугольника $ABC$ с прямым углом $C$ вовне построили квадраты $ACKL$ и $BCMN$; $CE$ — высота треугольника. Докажите, что угол $LEM$ прямой.

\item Прямые, содержащие стороны $AB$ и $CD$ четырёхугольника $ABCD$, пересекаются в точке $O$. Точка $M$ — середина $AB$, $N$ — середина $CD$. Докажите, что центры описанных окружностей треугольников $BCO$, $ADO$ и $MNO$ лежат на одной прямой.

\item На стороне $AB$ треугольника $ABC$ выбрана точка $D$. Описанная окружность треугольника $BCD$ вторично пересекает окружность, проходящую через точки $A$ и $D$ и касающуюся прямой $CD$, в точке $K$. Точка $M$ — середина $BC$, $N$ — середина $AD$. Докажите, что точки $B$, $M$, $N$ и $K$ лежат на одной окружности.

\item Боковые стороны $AB$ и $CD$ трапеции $ABCD$ повернули относительно их середин на $90^\circ$ против часовой стрелки, получились отрезки $A_0B_0$ и $C_0D_0$. Докажите, что $B_0C_0=A_0D_0$.

\item Вписанная в неравнобедренный треугольник $ABC$ окружность касается его сторон $BC, CA, AB$ в точках $A_1, B_1, C_1$. На прямой $AB$ отмечена такая точка $X$, что $A_1X\perp B_1C_1$. Окружности, описанные около треугольников $ABC$ и $AB_1C_1$, пересекаются второй раз в точке $Z$. Докажите, что $\angle XZC_1=90^\circ$.

\item  Пусть $ABCDE$ — выпуклый пятиугольник такой, что
$\angle BAC=\angle CAD=\angle DAE$ и $\angle CBA=\angle DCA=\angle EDA$.
Диагонали $BD$ и $CE$ пересекаются в точке $P$. Докажите, что прямая $AP$ делит отрезок $CD$ пополам.

\item Имеется два правильных пятиугольника с одной общей вершиной. Вершины каждого пятиугольника нумеруются по часовой стрелке цифрами от 1 до 5, причём в общей вершине ставится цифра 1. Вершины с одинаковыми номерами соединены прямыми. Доказать, что полученные четыре прямые пересекаются в одной точке.

\item a) Окружности $\omega_1$, $\omega_2$ и $\omega_3$ проходят через точку $O$. Окружности $\omega_1$ и $\omega_2$ повторно пересекаются в точке $A_1$, окружности $\omega_2$ и $\omega_3$ — в точке $A_2$, окружности $\omega_3$ и $\omega_1$ — в точке $A_3$. На окружности $\omega_1$ выбрана произвольная точка $X_1$. Прямая $X_1A_1$ повторно пересекает окружность $\omega_2$ в точке $X_2$, прямая $X_2A_2$ повторно пересекает окружность $\omega_3$ в точке $X_3$, прямая $X_3A_3$ повторно пересекает окружность $\omega_1$ в точке $X'_1$. Докажите, что $X_1=X_1'$.

b) Докажите аналогичное утверждение для $n$ окружностей.

\item Окружность, проходящая через вершины $A$ и $B$ треугольника $ABC$, пересекает сторону $BC$ в точке $D$. Окружность, проходящая через вершины $B$ и $C$, пересекает сторону $AB$ в точке $E$ и первую окружность вторично в точке $F$. Оказалось, что точки $A, E, D, C$ лежат на окружности с центром $O$. Докажите, что угол $BFO$ — прямой.

\item Окружность с центром $O$ проходит через вершины $A$ и $C$ треугольника $ABC$ и пересекает стороны $AB$ и $BC$ повторно в точках $K$ и $N$ соответственно. Пусть $M$ — точка пересечения описанных окружностей треугольников $ABC$ и $KBN$ (отличная от $B$). Докажите, что $\angle OMB=90^\circ$.

\item $ABCD$  — вписанный четырёхугольник, $X$ — точка пересечения его диагоналей. Некоторая прямая, проходящая через точку $X$, пересекает окружность, описанную около $ABCD$, в точках $N_1$ и $N_2$, и окружности, описанные около треугольников $ABX$ и $CDX$, в точках $M_1$ и $M_2$. Докажите, что $M_1N_1=M_2N_2$.

\item Внутри треугольника $ABC$ взята такая точка $D$, что $BD=CD$, $\angle BDC=120^\circ$. Вне треугольника $ABC$ взята такая точка $E$, что $AE=CE$, $\angle AEC=60^\circ$ и точки $B$ и $E$ находятся в разных полуплоскостях относительно $AC$. Докажите, что $\angle AFD=90^\circ$, где $F$ — середина отрезка $BE$.

\item На сторонах $AB$ и $AC$ треугольника $ABC$ выбраны такие точки $K$ и $L$ соответственно, что $\angle KCA=\angle LBA=\alpha$. Из точки $A$ опущены перпендикуляры $AE$ и $AF$ на прямые $BL$ и $CK$ соответственно. Точка $D$ — середина стороны $BC$. Найдите углы треугольника $DEF$.

\item На сторонах треугольника $ABC$ во внутреннюю сторону построены такие треугольники $ABC_1$, $BCA_1$ и $CAB_1$, что $\angle AC_1B+\angle BA_1C+\angle CB_1A=360^\circ$ и $\frac{AC_1}{C_1B} \cdot \frac{BA_1}{A_1C}\cdot \frac{CB_1}{B_1A}=1$. Докажите, что $\angle A_1B_1C_1=\angle BAC_1+\angle BCA_1$.

\item На сторонах четырёхугольника $ABCD$ во внешнюю сторону построили правильные треугольники $ABK$, $BCL$, $CDM$ и $DAN$. $X$ и $Y$ — середины отрезков $BL$ и $AN$, $Z$ — центр треугольника $CMD$.

a) Докажите, что $XY\perp KZ$.

b) Найдите отношение $XY:KZ$.

\item $AB$  — хорда окружности, $M$ и $N$ — середины дуг на которые делят окружность точки $A$ и $B$. При повороте вокруг точки $A$ на некоторый угол точка $B$ переходит в $B'$, а точка $M$ — в $M'$. Докажите, что отрезки, соединяющие середину отрезка $BB'$ с точками $M'$ и $N$, перпендикулярны.

\end{enumerate}
\end{document}