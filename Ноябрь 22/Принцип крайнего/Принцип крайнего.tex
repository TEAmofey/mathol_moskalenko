\documentclass{article}
\usepackage[12pt]{extsizes}
\usepackage[T2A]{fontenc}
\usepackage[utf8]{inputenc}
\usepackage[english, russian]{babel}

\usepackage{amssymb}
\usepackage{amsfonts}
\usepackage{amsmath}
\usepackage{enumitem}
\usepackage{graphics}
\usepackage{graphicx}

\usepackage{lipsum}

\newtheorem{theorem}{Теорема}
\newtheorem{task}{Задача}
\newtheorem{lemma}{Лемма}
\newtheorem{definition}{Определение}
\newtheorem{example}{Пример}
\newtheorem{statement}{Утверждение}
\newtheorem{corollary}{Следствие}


\usepackage{geometry} % Меняем поля страницы
\geometry{left=1cm}% левое поле
\geometry{right=1cm}% правое поле
\geometry{top=1.5cm}% верхнее поле
\geometry{bottom=1cm}% нижнее поле


\usepackage{fancyhdr} % Headers and footers
\pagestyle{fancy} % All pages have headers and footers
\fancyhead{} % Blank out the default header
\fancyfoot{} % Blank out the default footer
\fancyhead[L]{ЦРОД $\bullet$ Математика}
\fancyhead[C]{\textit{Методы}}
\fancyhead[R]{1 ноября}% Custom header text


%----------------------------------------------------------------------------------------

%\begin{document}\normalsize
\begin{document}\large
	
\begin{center}
	\textbf{Принцип Крайнего}
\end{center}

\begin{enumerate}[label*=\protect\fbox{\arabic{enumi}}]
	
\item 25 астрономов на двадцати пяти разных планетах наблюдают друг за другом, причем каждый наблюдает за ближайшим к нему (среди расстояний между планетами нет одинаковых). Докажите, что а) есть две планеты, астрономы которых наблюдают друг за другом; б) хотя бы за одним астрономом никто не наблюдает.

\item Шахматная доска разбита на домино. Докажите, что какая-то пара домино образует квадратик $2\times 2$.

\item На шахматной доске стоит несколько ладей. Докажите, что какая-то из ладей бьет не более двух других (ладья бьет фигуру тогда и только тогда, когда они стоят на одной вертикали или горизонтали и между ними нет других фигур).

\item На плоскости задано некоторое множество точек $M$ такое, что каждая точка из $M$ является серединой отрезка, соединяющего какую-либо пару точек того же множества $M$. Докажите, что множество $M$ содержит бесконечно много точек.

\item Дана бесконечная шахматная доска. В клетках доски записаны натуральные числа так, что каждое число равно среднему арифметическому четырех соседних чисел – верхнего, нижнего, правого и левого. Докажите, что все числа на доске равны между собой.

\item Можно ли расположить на плоскости 1000 отрезков так, чтобы каждый отрезок обоими своими концами упирался строго внутрь других отрезков?

\item Маляр-хамелеон ходит по клетчатой доске как хромая ладья (на одну клетку по вертикали или горизонтали). Попав в очередную клетку, он либо перекрашивается в её цвет, либо перекрашивает клетку в свой цвет. Белого маляра-хамелеона кладут на чёрную доску размером $8\times 8$ клеток. Сможет ли он раскрасить её в шахматном порядке?

\item Кубик Рубика $3\times 3 \times 3$ надо распилить на единичные кубики. После распила части можно перекладывать и прикладывать так, чтобы можно было пилить несколько частей одновременно. Сколько распилов понадобиться?

\item На полях доски $8 \times 8$ расставлены числа $1, 2, \dots, 64$. Докажите, что найдется пара соседних по стороне клеток, числа в которых отличаются не менее чем на 5.

\item По кругу стоят 30 положительных чисел, каждое из которых равно разности двух следующих за ним по часовой стрелке. Сумма всех чисел равна 1. Найдите эти числа.

\item На небе бесконечное число звёзд. Астроном приписал каждой звезде пару натуральных чисел, выражающую яркость и размер. Докажите, что найдутся две звезды, первая из которых не меньше второй как по яркости, так и по размеру.

\item На каждой клетке шахматной доски вначале стоит по ладье. Каждым ходом можно снять с доски ладью, которая бьет нечетное число ладей. Какое наибольшее число ладей можно снять? (Ладьи бьют друг друга, если они стоят на одной вертикали или горизонтали и между ними нет других ладей).

\item На конгресс собрались учёные, среди которых есть друзья. Оказалось, что каждые два из них, имеющие на конгрессе равное число друзей, не имеют общих друзей. Доказать, что найдётся учёный, который имеет ровно одного друга из числа участников конгресса.

\item На вечеринку пришли 100 человек. Затем те, у кого не было знакомых среди пришедших, ушли. Затем те, у кого был ровно один знакомый среди оставшихся, тоже ушли. Затем аналогично поступали те, у кого было ровно $2, 3, 4, \dots, 99$ знакомых среди оставшихся к моменту их ухода. 
Какое наибольшее число людей могло остаться в конце?
	
	
\end{enumerate}
\end{document}