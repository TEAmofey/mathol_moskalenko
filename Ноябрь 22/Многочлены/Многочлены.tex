\documentclass{article}
\usepackage[12pt]{extsizes}
\usepackage[T2A]{fontenc}
\usepackage[utf8]{inputenc}
\usepackage[english, russian]{babel}

\usepackage{amssymb}
\usepackage{amsfonts}
\usepackage{amsmath}
\usepackage{enumitem}
\usepackage{graphics}
\usepackage{graphicx}

\usepackage{lipsum}
\DeclareGraphicsExtensions{.pdf,.png,.jpg}



\usepackage{geometry} % Меняем поля страницы
\geometry{left=1cm}% левое поле
\geometry{right=1cm}% правое поле
\geometry{top=1.5cm}% верхнее поле
\geometry{bottom=1cm}% нижнее поле


\usepackage{fancyhdr} % Headers and footers
\pagestyle{fancy} % All pages have headers and footers
\fancyhead{} % Blank out the default header
\fancyfoot{} % Blank out the default footer
\fancyhead[L]{ЦРОД $\bullet$ Математика}
\fancyhead[C]{\textit{Алгебра}}
\fancyhead[R]{3 ноября 2022}% Custom header text


%----------------------------------------------------------------------------------------

%\begin{document}\normalsize
\begin{document}\large
	

\begin{center}
\textbf{Многочлены третей и четвёртой степени}
\end{center}

Для решения уравнений третьей степени вида $x^3 + px + q = 0$ есть формула Кардано:
$$\boxed{x=\sqrt[3]{-\frac{q^2}{4} + \sqrt{\frac{q^2}{4} +\frac{p^3}{27}}} + \sqrt[3]{-\frac{q^2}{4} - \sqrt{\frac{q^2}{4} +\frac{p^3}{27}}}}$$


\begin{enumerate}[label*=\protect\fbox{\arabic{enumi}}]

\item Решите уравнение: $x^3 - 15 x - 126 = 0.$

\item Решите уравнение: $x^3 - 6x^2 - 6x - 2 = 0.$

\item Решите уравнение: $x^3 - 6x - 4 = 0.$

\item Решите уравнение: $x^3 - 6x - 40 = 0.$

\item Решите уравнение: $x^3 - 6x^2 - 6x - 2 = 0.$


\end{enumerate}

А для решения уравнений четвёртой степени вида $x^4 + ax^2+ bx + c = 0$ есть метод Феррари.
$$x^4 + (2\alpha - \beta^2)  x^2  -2\beta\gamma x + \alpha^2- \gamma^2 = 0$$
$$(x^2 + \alpha)^2 - (\beta x + \gamma)^2 = 0$$
$$(x^2 + \alpha - \beta x - \gamma)(x^2 + \alpha + \beta x + \gamma) = 0$$

Мы хотим найти $\alpha, \beta, \gamma$ такие, что

\begin{equation*}
	\begin{cases}
		2\alpha - \beta^2 = a,
		\\
		-2\beta\gamma = b,
		\\
		\alpha^2- \gamma^2 = c.
	\end{cases}
\end{equation*}

Тогда корни исходного уравнения четвёртой степени будут корни двух получившихся квадратных трёхчленов.

\begin{enumerate}[label*=\protect\fbox{\arabic{enumi}}]

\item Решите уравнение: $x^4 - 8x^2 + 16 = 0.$

\item Решите уравнение: $x^4 - 9x^2 + 2x + 15 = 0.$

\item Решите уравнение: $x^4 + 4x^3 - 4x^2 - 20x - 5 = 0.$

\end{enumerate}

Для многочленов пятой степени и выше общей формулы нахождения решений не существует. Пример:
$$x^5 - 4x + 2 = 0$$
\end{document}