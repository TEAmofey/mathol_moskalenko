\documentclass{article}
\usepackage[12pt]{extsizes}
\usepackage[T2A]{fontenc}
\usepackage[utf8]{inputenc}
\usepackage[english, russian]{babel}

\usepackage{amssymb}
\usepackage{amsfonts}
\usepackage{amsmath}
\usepackage{enumitem}
\usepackage{graphics}

\usepackage{lipsum}


\newtheorem{theorem}{Теорема}
\newtheorem{task}{Задача}
\newtheorem{lemma}{Лемма}
\newtheorem{definition}{Определение}
\newtheorem{example}{Пример}
\newtheorem{statement}{Утверждение}
\newtheorem{corollary}{Следствие}


\usepackage{geometry} % Меняем поля страницы
\geometry{left=1cm}% левое поле
\geometry{right=1cm}% правое поле
\geometry{top=1.5cm}% верхнее поле
\geometry{bottom=1cm}% нижнее поле


\usepackage{fancyhdr} % Headers and footers
\pagestyle{fancy} % All pages have headers and footers
\fancyhead{} % Blank out the default header
\fancyfoot{} % Blank out the default footer
\fancyhead[L]{ЦРОД $\bullet$ Математика}
\fancyhead[C]{\textit{Теория множеств}}
\fancyhead[R]{Ноябрь 2022}% Custom header text


%----------------------------------------------------------------------------------------

%\begin{document}\normalsize
\begin{document}\large
	
	
\begin{center}
	\textbf{Бесконечные множества}
\end{center}

\begin{definition}
	Множество - неупорядоченная совокупность элементов.
\end{definition}

\textbf{Пример 1:} $\{1,2,3\} = \{3,2,1\} = \{1,1,3,2,3,3\}$

\textbf{Пример 2:} $\{\} =\varnothing$

\textbf{Пример 3:} $\{a,\{b,c\},\{\{d\},e\},\varnothing\}$

\begin{definition}
	$a \in A$ - означает, что во множестве $A$ есть элемент $a$
\end{definition}

\begin{definition}
	$B \subseteq A$ - означает, что если $b \in B$, то $b \in A$.
\end{definition}

\textbf{Пример 4:} $1\in\{1,2,3\},\{1,2\}\subseteq\{1,2,3\}$

\textbf{Пример 5:} $1\notin\varnothing$

\textbf{Пример 6:} $\{b,c\}\in \{a,\{b,c\},\{\{d\},e\},\varnothing\}$

\begin{definition}
	Введём особые символы для часто использующихся множеств. 
	
	$\mathbb{N} = \{1,2,3,4,\dotsc \}$ - множество натуральных чисел 
	
	$\mathbb{N}_0 = \{0,1,2,3,4,\dotsc \}$ - множество натуральных чисел и ноль
	
	$\mathbb{Z} = \{\dotsc, -3, -2, -1, 0,1,2,3,\dotsc \}$ - множество целых чисел 
	
	$\mathbb{Q}$ - множество рациональных чисел
	
	$\mathbb{R}$ - множество вещественных чисел 
	
	$\mathbb{R\backslash Q}$ - множество иррациональных чисел 
	
	$\mathbb{A}$ - множество алгебраических чисел (числа, которые могут быть корнями многочленов с целыми коэффициентами)
	
\end{definition}

\textbf{Пример 7:} $\mathbb{N} \subset \mathbb{N}_0 \subset \mathbb{Z} \subset \mathbb{Q} \subset \mathbb{A} \subset \mathbb{R}$

\begin{definition}
 	$|A|$ - мощность множества $A$ - количество элементов в нем.
 	
 	$|A| = |B|$ - множества $A$ и $B$ равномощны --- между ними есть биекция 
\end{definition}


\textbf{Пример 8:} $|\{a,\{b,c\},\{\{d\},e\},\varnothing\}| = |\{0,1,2,3\}| = 4$

\begin{definition}
	$A \times B$ - множество упорядоченных пар $(a,b)$, где $a\in A, b\in B$
	
	$A^n = \underbrace{A\times A \times \dotsc \times A}_{n \text{ раз}}$
\end{definition}

\textbf{Пример 9:} $\{a,b\}\times \{0,1,2\} = \{(a,0),(a,1),(a,2),(b,0),(b,1),(b,2)\} $

\textbf{Пример 10:} $\{a,b\}^2 = \{(a,a),(a,b),(b,a),(b,b)\} $

\begin{definition}
	$A^* = \{\varnothing\} \cup A \cup A^2 \cup A^3 \dotsc$ - множество всех конечных подпоследовательностей
\end{definition}

\textbf{Пример 10:} $\{a,b\}^* = \{\varnothing, a,b,aa,ab,ba,bb,aaa,aab,\dotsc\} $

\begin{definition}
	$2^A$ - множество всех подмножеств множества $A$
\end{definition}

\textbf{Пример 11:} $2^{\{1,2,3\}} = \{\{\}, \{1\},\{2\},\{3\},\{1,2\},\{1,3\},\{2,3\},\{1,2,3\}\} $

\begin{definition}
	$|A| = |B|$ - если между множествами $A$ и $B$ есть взаимно однозначное соответствие (биекция)
\end{definition}

\begin{definition}
	Если $|A| = |\mathbb{N}|$, то $A$ называется счётным множеством (его элементы можно пересчитать)
\end{definition}

\begin{enumerate}[label*=\protect\fbox{\arabic{enumi}}]
	
\item Докажите, что $|$Чётных положительных$| = |\mathbb{N}| $

\item Докажите, что $|$Нечётных положительных$| = |\mathbb{N}| $

\item Докажите, что $|\mathbb{Z}| = |\mathbb{N}|$

\item $|\mathbb{Q}| \stackrel{?}{=} |\mathbb{N}|$

\item $|\mathbb{N}^2| \stackrel{?}{=} |\mathbb{N}|$

\item Докажите, что если $|A| = |B| = |\mathbb{N}|$, то $A\cup B = |\mathbb{N}|$

\item $|\mathbb{N}^*| \stackrel{?}{=} |\mathbb{N}|$

\item $|\mathbb{A}| \stackrel{?}{=} |\mathbb{N}|$

\item Докажите, что $|\mathbb{R}| = |2^{\mathbb{N}}|$

\item Докажите, что $|\mathbb{R}| \neq |\mathbb{N}|$

\item Докажите, что для любого множества $A$: $|2^A| \neq |A|$

\item Докажите, что множество точек интервала $(0,1)$ равномощно множеству точек прямой $\mathbb{R}$

\item Докажите, что $|\mathbb{R}^2| = |\mathbb{R}|$

\item Докажите, что $|\mathbb{R}^n| = |\mathbb{R}|$

\item Докажите, что $|\mathbb{R}^*| = |\mathbb{R}|$








\end{enumerate}
\end{document}