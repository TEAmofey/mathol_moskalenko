\documentclass{article}

\usepackage[12pt]{extsizes}
\usepackage[T2A]{fontenc}
\usepackage[utf8]{inputenc}
\usepackage[english, russian]{babel}

\usepackage{mathrsfs}
\usepackage[dvipsnames]{xcolor}

\usepackage{amsmath}
\usepackage{amssymb}
\usepackage{amsthm}
\usepackage{indentfirst}
\usepackage{amsfonts}
\usepackage{enumitem}
\usepackage{graphics}
\usepackage{tikz}
\usepackage{tabu}
\usepackage{diagbox}
\usepackage{hyperref}
\usepackage{mathtools}
\usepackage{ucs}
\usepackage{lipsum}
\usepackage{geometry} % Меняем поля страницы
\usepackage{fancyhdr} % Headers and footers
\newcommand{\range}{\mathrm{range}}
\newcommand{\dom}{\mathrm{dom}}
\newcommand{\N}{\mathbb{N}}
\newcommand{\R}{\mathbb{R}}
\newcommand{\E}{\mathbb{E}}
\newcommand{\D}{\mathbb{D}}
\newcommand{\M}{\mathcal{M}}
\newcommand{\Prime}{\mathbb{P}}
\newcommand{\A}{\mathbb{A}}
\newcommand{\Q}{\mathbb{Q}}
\newcommand{\Z}{\mathbb{Z}}
\newcommand{\F}{\mathbb{F}}
\newcommand{\CC}{\mathbb{C}}

\DeclarePairedDelimiter\abs{\lvert}{\rvert}
\DeclarePairedDelimiter\floor{\lfloor}{\rfloor}
\DeclarePairedDelimiter\ceil{\lceil}{\rceil}
\DeclarePairedDelimiter\lr{(}{)}
\DeclarePairedDelimiter\set{\{}{\}}
\DeclarePairedDelimiter\norm{\|}{\|}

\renewcommand{\labelenumi}{(\alph{enumi})}

\newcommand{\smallindent}{
    \geometry{left=1cm}% левое поле
    \geometry{right=1cm}% правое поле
    \geometry{top=1.5cm}% верхнее поле
    \geometry{bottom=1cm}% нижнее поле
}

\newcommand{\header}[3]{
    \pagestyle{fancy} % All pages have headers and footers
    \fancyhead{} % Blank out the default header
    \fancyfoot{} % Blank out the default footer
    \fancyhead[L]{#1}
    \fancyhead[C]{#2}
    \fancyhead[R]{#3}
}

\newcommand{\dividedinto}{
    \,\,\,\vdots\,\,\,
}

\newcommand{\littletaller}{\mathchoice{\vphantom{\big|}}{}{}{}}

\newcommand\restr[2]{{
    \left.\kern-\nulldelimiterspace % automatically resize the bar with \right
    #1 % the function
    \littletaller % pretend it's a little taller at normal size
    \right|_{#2} % this is the delimiter
}}

\DeclareGraphicsExtensions{.pdf,.png,.jpg}

\newenvironment{enumerate_boxed}[1][enumi]{\begin{enumerate}[label*=\protect\fbox{\arabic{#1}}]}{\end{enumerate}}



\smallindent

\header{ЦРОД $\bullet$ Математика}{\textit{Комбинаторика}}{Стратегия 2021}

%----------------------------------------------------------------------------------------

\begin{document}
    \large

    \begin{center}
        \textbf{Игры}
    \end{center}

    Существует \textbf{три основных стратегии} решения задач на теорию игр:
    \begin{itemize}
        \item Стратегия дополнения

        \item Симметричная стратегия

        \item Выигрышные и проигрышные позиции

        \item Передача хода
    \end{itemize}

    \begin{enumerate_boxed}

        \item На доске написаны числа от 1 до 10.
        2 игрока по очереди вычеркивают по одному числу.
        Как надо делать ходы, чтобы выиграть в такой игре?

        \item
        \begin{enumerate}
            \item На столе лежат две кучки спичек: в одной 10, в другой 7.
            Игроки ходят по очереди.
            За один ход можно взять любое число спичек $(1, 2, 3, \dots)$ из одной из кучек (по выбору игрока).
            Кто не может сделать ход (спичек не осталось), проигрывает.
            \item Что будет в этой игре, если изначально в одной кучке $m$ спичек, а в другой $n$
        \end{enumerate}

        Какой игрок имеет выигрышную стратегию?

        \item На столе лежит (a) 25 (b) 24 спичек.
        Играющие по очереди могут взять от одной до четырёх спичек.
        Кто не может сделать ход (спичек не осталось), проигрывает.
        У какого игрока есть выигрышная стратегия?

        \item Дана доска $9 \times 9$.
        Двое по очереди выставляют на нее королей так, чтоб они не били друг друга.
        Проигрывает тот, кто не может сделать ход.
        Кто выигрывает при правильной игре и как он должен играть?

        \item Двое по очереди ставят слонов на шахматную доску.
        Очередным ходом нужно побить хотя бы одну небитую клетку.
        Фигура бьет и ту клетку, на которой стоит.
        Проигрывает тот, кто не может сделать ход.
        Кто выигрывает при правильной игре?

        \item Имеется три кучки камней: в первой --- 10, во второй --- 15, в третьей --- 20.
        За ход разрешается разбить любую кучку на две меньшие.
        Проигрывает тот, кто не сможет сделать ход.
        Кто выиграет?

        \item Шоколадка представляет собой прямоугольник $3 \times 5$, разделённый углублениями на 15 квадратиков.
        Двое по очереди разламывают её на части по углублениям: за один ход можно разломить любой из кусков (больший одного квадратика) на два.
        Кто не может сделать хода (все куски уже разломаны), проигрывает.

        \item Петя и Вася выписывают 100-значное число, ставя цифры по очереди, начиная со старшего разряда.
        Петя ставит только нечетные цифры, а Вася — только четные Начинает Петя.
        Докажите, что какие бы цифры он не писал, Вася всегда сможет добиться, чтобы получившееся число делилось на 9.

        \item Первый называет целое число, затем второй называет ещё одно.
        Если (a) сумма (b) произведение чисел чётно, выигрывает первый, если нечётно - второй.

        \item По кругу расставлены 50 фишек.
        Дима и Саша по очереди убирают фишки, выбирая каждым своим ходом любые три, пока не останется всего две фишки.
        Если две оставшиеся фишки вначале не стояли рядом, выигрывает Дима, а в противном случае выигрывает Саша.
        Дима ходит первым.
        Кто выиграет при правильной игре?

        \item Петя и Вася играют на доске размером $7 \times 7$.
        Они по очереди ставят в клетки доски цифры от 1 до 7 так, чтобы ни в одной строке и ни в одном столбце не оказалось одинаковых цифр.
        Первым ходит Петя.
        Проигрывает тот, кто не сможет сделать ход.
        Кто из них сможет победить, как бы ни играл соперник?

        \item  На столе лежит 9 спичек.
        Играющие по очереди могут взять 1, 2 или 4 спички.
        Кто не может сделать ход (спичек не осталось), проигрывает.

        \item На шахматной доске стоит король.
        Двое по очереди ходят им.
        Проигрывает игрок, после хода которого король оказывается в клетке, в которой побывал ранее.
        Кто побеждает при правильной игре: начинающий или его соперник?

        \item В одной из клеток шахматной доски стоит \textit{односторонняя ладья}, которая может двигаться влево или вниз.
        Двое игроков ходят по очереди, сдвигая ладью влево или вниз на любое число клеток (но не менее одной); кто не может сделать ход, проигрывает.
        Кто выигрывает при правильной игре?

        \item На клетчатой доске размером $23 \times 23$ клетки стоят четыре фишки: в левом ниж- нем и в правом верхнем углах доски — по белой фишке, а в левом верхнем и в правом нижнем углах — по чёрной.
        Белые и чёрные фишки ходят по очереди, начинают белые.
        Каждым ходом одна из фишек сдвигается на любую соседнюю (по стороне) свободную клетку.
        Белые фишки стремятся попасть в две соседние по стороне клетки.
        Могут ли чёрные им помешать?

        \item Двое играют в двойные шахматы: все фигуры ходят как обычно, но каждый делает по два шахматных хода подряд.
        Докажите, что первый может как минимум сделать ничью.

        \item Дана бесконечная клетчатая плоскость.
        Учительница и класс из 30 учеников играют в игру, делая ходы по очереди~--- сначала учительница, затем по очереди все ученики, затем снова учительница, и т.д.
        За один ход можно покрасить единичный отрезок, являющийся границей между двумя соседними клетками.
        Дважды красить отрезки нельзя.
        Учительница побеждает, если после хода одного из 31 игроков найдется клетчатый прямоугольник $1 \times 2$ или $2 \times 1$ такой, что у него вся граница покрашена, а единичный отрезок внутри него не покрашен.
        Докажите, что учительница сможет победить.

    \end{enumerate_boxed}
\end{document}