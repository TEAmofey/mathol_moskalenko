\documentclass{article}

\usepackage[12pt]{extsizes}
\usepackage[T2A]{fontenc}
\usepackage[utf8]{inputenc}
\usepackage[english, russian]{babel}

\usepackage{mathrsfs}
\usepackage[dvipsnames]{xcolor}

\usepackage{amsmath}
\usepackage{amssymb}
\usepackage{amsthm}
\usepackage{indentfirst}
\usepackage{amsfonts}
\usepackage{enumitem}
\usepackage{graphics}
\usepackage{tikz}
\usepackage{tabu}
\usepackage{diagbox}
\usepackage{hyperref}
\usepackage{mathtools}
\usepackage{ucs}
\usepackage{lipsum}
\usepackage{geometry} % Меняем поля страницы
\usepackage{fancyhdr} % Headers and footers
\newcommand{\range}{\mathrm{range}}
\newcommand{\dom}{\mathrm{dom}}
\newcommand{\N}{\mathbb{N}}
\newcommand{\R}{\mathbb{R}}
\newcommand{\E}{\mathbb{E}}
\newcommand{\D}{\mathbb{D}}
\newcommand{\M}{\mathcal{M}}
\newcommand{\Prime}{\mathbb{P}}
\newcommand{\A}{\mathbb{A}}
\newcommand{\Q}{\mathbb{Q}}
\newcommand{\Z}{\mathbb{Z}}
\newcommand{\F}{\mathbb{F}}
\newcommand{\CC}{\mathbb{C}}

\DeclarePairedDelimiter\abs{\lvert}{\rvert}
\DeclarePairedDelimiter\floor{\lfloor}{\rfloor}
\DeclarePairedDelimiter\ceil{\lceil}{\rceil}
\DeclarePairedDelimiter\lr{(}{)}
\DeclarePairedDelimiter\set{\{}{\}}
\DeclarePairedDelimiter\norm{\|}{\|}

\renewcommand{\labelenumi}{(\alph{enumi})}

\newcommand{\smallindent}{
    \geometry{left=1cm}% левое поле
    \geometry{right=1cm}% правое поле
    \geometry{top=1.5cm}% верхнее поле
    \geometry{bottom=1cm}% нижнее поле
}

\newcommand{\header}[3]{
    \pagestyle{fancy} % All pages have headers and footers
    \fancyhead{} % Blank out the default header
    \fancyfoot{} % Blank out the default footer
    \fancyhead[L]{#1}
    \fancyhead[C]{#2}
    \fancyhead[R]{#3}
}

\newcommand{\dividedinto}{
    \,\,\,\vdots\,\,\,
}

\newcommand{\littletaller}{\mathchoice{\vphantom{\big|}}{}{}{}}

\newcommand\restr[2]{{
    \left.\kern-\nulldelimiterspace % automatically resize the bar with \right
    #1 % the function
    \littletaller % pretend it's a little taller at normal size
    \right|_{#2} % this is the delimiter
}}

\DeclareGraphicsExtensions{.pdf,.png,.jpg}

\newenvironment{enumerate_boxed}[1][enumi]{\begin{enumerate}[label*=\protect\fbox{\arabic{#1}}]}{\end{enumerate}}



\smallindent

\header{ЦРОД $\bullet$ Математика}{\textit{Алгебра}}{ЛФМШ 2022}

%----------------------------------------------------------------------------------------

\begin{document}
    \large

    \begin{center}
        \textbf{Теорема Виета}
    \end{center}

    \textbf {Теорема Виета.} Если числа $x_1$ и $x_2$ --- корни квадратного уравнения $x^2 + px + q = 0$, то
    \[\boxed{~x_1 + x_2=-p, \quad x_{1}x_2=q.\strut~}\]

    \begin{enumerate_boxed}

        \item Пусть $x_1$, $x_2$ --- корни уравнения $2x^2-9x+1=0$.
        Найдите значение выражения $\dfrac1{x_1}+\dfrac1{x_2}$.

        \item Пусть $x_1$ и $x_2$ --- корни квадратного уравнения $x^2+px+q=0$.
        Выразите через $p$ и $q$ величины:
        \begin{enumerate}
            \item $x_1^2+x_2^2$;
            \item $x_1^3+x_2^3.$
        \end{enumerate}

        \item Уравнение $x^2+8x-3=0$ имеет корни $x_1$ и $x_2$.
        Составьте уравнение, корнями которого являются числа $2x_1+3$ и $2x_2+3$.

        \item Пусть $x_1$ и $x_2$ --- корни квадратного уравнения $x^2+px+q=0$.
        Напишите уравнения, корнями которых являются следующие пары чисел:
        \begin{enumerate}
            \item $x_1^2, x_2^2$;
            \item  $\dfrac1{x_1}, \dfrac1{x_2}$;
            \item  $\dfrac{x_2}{x_1}, \dfrac{x_1}{x_2}$.
        \end{enumerate}

        \item Докажите, что если для коэффициентов уравнения $ax^2+bx+c=0$ выполняется равенство $a+b+c=0$, то $x_1=1$, а $x_2=\dfrac{c}{a}$.

        \item При каких $p$, $q$ уравнению $x^2 + px + q = 0$ удовлетворяют два различных числа $2p$ и $p + q$?

        \item Решить систему уравнений $a^2+b^2=5$, $\dfrac1a+\dfrac1b=\dfrac12$.

        \item Известно, что корни уравнения  $x^2 + px + q = 0$~--- целые числа, а $p$ и $q$~--- простые числа.
        Найдите $p$ и $q$.

        \item  Корни уравнения $x^2 + ax + 1 = b$ --- целые, отличные от нуля, числа.
        Докажите, что число
        $a^2 + b^2$ является составным.

        \item Про различные числа $x$, $y$, $z$ известно, что выполняются равенства $x^3 - 3x = y^3 - 3y = z^3 - 3z$.
        Чему может равняться значение выражения $xy + yz + zx$?

    \end{enumerate_boxed}
\end{document}