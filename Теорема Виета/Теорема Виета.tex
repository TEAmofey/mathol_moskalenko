\documentclass{article}
\usepackage[12pt]{extsizes}
\usepackage[T2A]{fontenc}
\usepackage[utf8]{inputenc}
\usepackage[english, russian]{babel}

\usepackage{amssymb}
\usepackage{amsfonts}
\usepackage{amsmath}
\usepackage{enumitem}
\usepackage{graphics}
\usepackage{graphicx}

\usepackage{lipsum}

\newtheorem{theorem}{Теорема}
\newtheorem{task}{Задача}
\newtheorem{lemma}{Лемма}
\newtheorem{definition}{Определение}
\newtheorem{example}{Пример}
\newtheorem{statement}{Утверждение}
\newtheorem{corollary}{Следствие}


\usepackage{geometry} % Меняем поля страницы
\geometry{left=1cm}% левое поле
\geometry{right=1cm}% правое поле
\geometry{top=1.5cm}% верхнее поле
\geometry{bottom=1cm}% нижнее поле


\usepackage{fancyhdr} % Headers and footers
\pagestyle{fancy} % All pages have headers and footers
\fancyhead{} % Blank out the default header
\fancyfoot{} % Blank out the default footer
\fancyhead[L]{ЦРОД \textbullet Математика}
\fancyhead[C]{\textit{Алгебра}}
\fancyhead[R]{ЛФМШ 2022}% Custom header text


%----------------------------------------------------------------------------------------

%\begin{document}\normalsize
\begin{document}\large
	
	
	\begin{center}
		\textbf{Теорема Виета}
	\end{center}

\textbf {Теорема Виета.} Если числа $x_1$ и $x_2$ --- корни квадратного уравнения $x^2 + px + q = 0$, то
$$\boxed{~x_1 + x_2=-p, \quad x_1x_2=q.\strut~}$$

\begin{enumerate}[label*=\protect\fbox{\arabic{enumi}}]
	
\item Пусть $x_1$, $x_2$ --- корни уравнения $2x^2-9x+1=0$. Найдите значение выражения $\dfrac1{x_1}+\dfrac1{x_2}$.

\item  Пусть $x_1$ и $x_2$ --- корни квадратного уравнения $x^2+px+q=0$. Выразите через $p$ и $q$ величины:  

а) $x_1^2+x_2^2$;  

б)~$x_1^3+x_2^3$.

\item Уравнение $x^2+8x-3=0$ имеет корни $x_1$ и $x_2$. Составьте уравнение, корнями которого являются числа $2x_1+3$ и $2x_2+3$.

\item Пусть $x_1$ и $x_2$ --- корни квадратного уравнения $x^2+px+q=0$. Напишите уравнения, корнями которых являются следующие пары чисел:

а) $x_1^2$, $x_2^2$; б) $\dfrac1{x_1}$, $\dfrac1{x_2}$;  в) $\dfrac{x_2}{x_1}$, $\dfrac{x_1}{x_2}$.

\item Докажите, что если для коэффициентов уравнения $ax^2+bx+c=0$ выполняется равенство $a+b+c=0$, то $x_1=1$, а $x_2=\dfrac{c}{a}$.

\item При каких $p$, $q$ уравнению $x^2 + px + q = 0$ удовлетворяют два различных числа $2p$ и $p + q$?

\item Решить систему уравнений $a^2+b^2=5$, $\dfrac1a+\dfrac1b=\dfrac12$.

\item Известно, что корни уравнения  $x^2 + px + q = 0$  --- целые числа, а $p$ и $q$ --- простые числа. Найдите $p$ и $q$.

\item  Корни уравнения $x^2 + ax + 1 = b$ --- целые, отличные от нуля, числа. Докажите, что число
$a^2 + b^2$ является составным.

\item Про различные числа $x$, $y$, $z$ известно, что выполняются равенства $x^3 - 3x = y^3 - 3y = z^3 - 3z$. Чему может равняться значение выражения $xy + yz + zx$?



\end{enumerate}
\end{document}