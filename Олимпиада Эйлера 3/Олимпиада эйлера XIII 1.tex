\documentclass{article}
\usepackage[12pt]{extsizes}
\usepackage[T2A]{fontenc}
\usepackage[utf8]{inputenc}
\usepackage[english, russian]{babel}

\usepackage{amssymb}
\usepackage{amsfonts}
\usepackage{amsmath}
\usepackage{enumitem}
\usepackage{graphics}
\usepackage{graphicx}

\usepackage{lipsum}

\newtheorem{theorem}{Теорема}
\newtheorem{task}{Задача}
\newtheorem{lemma}{Лемма}
\newtheorem{definition}{Определение}
\newtheorem{example}{Пример}
\newtheorem{statement}{Утверждение}
\newtheorem{corollary}{Следствие}


\usepackage{geometry} % Меняем поля страницы
%\geometry{left=1cm}% левое поле
%\geometry{right=1cm}% правое поле
\geometry{top=3cm}% верхнее поле
%\geometry{bottom=1cm}% нижнее поле


\usepackage{fancyhdr} % Headers and footers
\pagestyle{fancy} % All pages have headers and footers
\fancyhead{} % Blank out the default header
\fancyfoot{} % Blank out the default footer
\fancyhead[L]{\textit{\textbf{XIII Олимпиада Эйлера}}}
\fancyhead[C]{}
\fancyhead[R]{25 декабря}% Custom header text


%----------------------------------------------------------------------------------------

%\begin{document}\normalsize
\begin{document}\large
	
\begin{center}
	\LARGE\textbf{8 класс}
\end{center}
\begin{center}
	\large\textbf{Первый день}
\end{center}


\begin{enumerate}[label*=8.{\arabic{enumi}}]

\item Натуральное число, большее 1000000, даёт одинаковые остатки при делении на 40 и на 125. Какая цифра может стоять у этого числа в разряде сотен?
\item Числа $x$ и $y$, не равные 0, удовлетворяют неравенствам $x^2-x > y^2$ и $y^2-y > x^2$. Какой знак может иметь произведение $xy$ (укажите все возможности)?
\item В группе из 79 школьников у каждого не более 39 знакомых, причем у любого мальчика есть знакомая девочка, а у любой девочки~--- знакомый мальчик. Может ли оказаться, что все девочки из этой группы имеют в ней поровну знакомых мальчиков, а все мальчики~--- поровну знакомых девочек? Все знакомства~--- взаимные.
\item Петя и Вася играют в игру. Вася кладёт в ряд 150 монет: некоторые «орлом» вверх, некоторые — «решкой». Петя своим ходом может показать на любые три лежащие подряд монеты, после чего Вася обязан перевернуть какие-то две монеты из этих трёх по своему выбору. Петя хочет, чтобы как можно больше монет лежали «решкой» вверх, а Вася хочет ему помешать. При каком наибольшем $k$ Петя сможет независимо от действий Васи добиться того, чтобы хотя бы $k$ монет лежали «решкой» вверх?
\item $CL$~--- биссектриса треугольника $ABC$. $CLBK$~--- параллелограмм. Прямая $AK$ пересекает отрезок $CL$ в точке $P$. Оказалось, что точка $P$ равноудалена от диагоналей параллелограмма $CLBK$ Докажите, что $AK \geqslant CL$.
	

\end{enumerate}
\end{document}