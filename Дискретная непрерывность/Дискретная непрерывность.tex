\documentclass{article}

\usepackage[12pt]{extsizes}
\usepackage[T2A]{fontenc}
\usepackage[utf8]{inputenc}
\usepackage[english, russian]{babel}

\usepackage{mathrsfs}
\usepackage[dvipsnames]{xcolor}

\usepackage{amsmath}
\usepackage{amssymb}
\usepackage{amsthm}
\usepackage{indentfirst}
\usepackage{amsfonts}
\usepackage{enumitem}
\usepackage{graphics}
\usepackage{tikz}
\usepackage{tabu}
\usepackage{diagbox}
\usepackage{hyperref}
\usepackage{mathtools}
\usepackage{ucs}
\usepackage{lipsum}
\usepackage{geometry} % Меняем поля страницы
\usepackage{fancyhdr} % Headers and footers
\newcommand{\range}{\mathrm{range}}
\newcommand{\dom}{\mathrm{dom}}
\newcommand{\N}{\mathbb{N}}
\newcommand{\R}{\mathbb{R}}
\newcommand{\E}{\mathbb{E}}
\newcommand{\D}{\mathbb{D}}
\newcommand{\M}{\mathcal{M}}
\newcommand{\Prime}{\mathbb{P}}
\newcommand{\A}{\mathbb{A}}
\newcommand{\Q}{\mathbb{Q}}
\newcommand{\Z}{\mathbb{Z}}
\newcommand{\F}{\mathbb{F}}
\newcommand{\CC}{\mathbb{C}}

\DeclarePairedDelimiter\abs{\lvert}{\rvert}
\DeclarePairedDelimiter\floor{\lfloor}{\rfloor}
\DeclarePairedDelimiter\ceil{\lceil}{\rceil}
\DeclarePairedDelimiter\lr{(}{)}
\DeclarePairedDelimiter\set{\{}{\}}
\DeclarePairedDelimiter\norm{\|}{\|}

\renewcommand{\labelenumi}{(\alph{enumi})}

\newcommand{\smallindent}{
    \geometry{left=1cm}% левое поле
    \geometry{right=1cm}% правое поле
    \geometry{top=1.5cm}% верхнее поле
    \geometry{bottom=1cm}% нижнее поле
}

\newcommand{\header}[3]{
    \pagestyle{fancy} % All pages have headers and footers
    \fancyhead{} % Blank out the default header
    \fancyfoot{} % Blank out the default footer
    \fancyhead[L]{#1}
    \fancyhead[C]{#2}
    \fancyhead[R]{#3}
}

\newcommand{\dividedinto}{
    \,\,\,\vdots\,\,\,
}

\newcommand{\littletaller}{\mathchoice{\vphantom{\big|}}{}{}{}}

\newcommand\restr[2]{{
    \left.\kern-\nulldelimiterspace % automatically resize the bar with \right
    #1 % the function
    \littletaller % pretend it's a little taller at normal size
    \right|_{#2} % this is the delimiter
}}

\DeclareGraphicsExtensions{.pdf,.png,.jpg}

\newenvironment{enumerate_boxed}[1][enumi]{\begin{enumerate}[label*=\protect\fbox{\arabic{#1}}]}{\end{enumerate}}



\smallindent

\header{Математика}{\textit{Комбинаторика}}{19 января 2023}

%----------------------------------------------------------------------------------------

\begin{document}
    \large

    \begin{center}
        \textbf{Дискретная непрерывность}
    \end{center}

    \begin{enumerate_boxed}

        \setcounter{enumi}{-1}

        \item В ряд сидят 10 мальчиков и 10 девочек.


        \begin{enumerate}
            \item Докажите, что можно выбрать 8 школьников подряд, чтобы среди них мальчиков и девочек
            было поровну.

            \item Всегда ли из них можно выбрать 16 школьников подряд, среди которых мальчиков и девочек поровну?
        \end{enumerate}

        \item Шеренга новобранцев стояла лицом к сержанту.
        По команде «Налево!» некоторые из них повернулись налево, некоторые — направо, а остальные — кругом.
        Всегда ли сержант сможет встать в строй так, чтобы с обеих сторон от него оказалось поровну новобранцев, стоящих к нему лицом?

        \item По кругу сидят 25 мальчиков и 25 девочек.

        \begin{enumerate}
            \item Докажите, что можно выбрать 20 школьников подряд, чтобы среди них мальчиков и девочек
            было поровну.

            \item Докажите, что для каждого чётного $n$ от 2 до 50 можно выбрать $n$ школьников подряд, чтобы среди них мальчиков и девочек было поровну.
        \end{enumerate}

        \item

        \begin{enumerate}
            \item На плоскости есть $2n$ точек.
            Докажите, что можно провести прямую так, что с каждой стороны от нее будет находиться n точек.

            \item На плоскости есть $2n + 1$ точка, никакие три не лежат на одной прямой.
            Докажите, что через любую из них можно провести прямую так, что с каждой стороны от нее будет находиться n точек.
        \end{enumerate}

        \item Вася и Петя играют в игру.
        На доске написаны два числа: $\frac{1}{2022}$ и  $\frac{1}{2023}$.
        На каждом ходу Вася называет любое число $x$, а Петя увеличивает любое из чисел на доске на $x$.
        Вася выигрывает, если когда-то одно из чисел на доске станет равным 1.
        Сможет ли Вася выиграть, как бы ни действовал Петя?

        \item За круглым столом сидит 100 дедов, причем у любых двух соседей число волос в бороде отличается не больше чем на 100.
        Докажите, что найдется пара дедов, сидящих друг напротив друга, у которых число волос в бородах тоже отличается не больше, чем на 100.

        \item $2n$ радиусов разделили круг на $2n$ равных секторов: $n$ синих и $n$ красных, чередующихся в произвольном порядке.
        В синие сектора, начиная с некоторого, записывают против хода часовой стрелки числа от 1 до $n$ по порядку.
        В красные сектора, начиная с некоторого, записывают те же числа, но по ходу часовой стрелки.
        Докажите, что найдется полукруг, в котором записаны все числа от 1 до $n$.

    \end{enumerate_boxed}
\end{document}