\documentclass{article}
\usepackage[12pt]{extsizes}
\usepackage[T2A]{fontenc}
\usepackage[utf8]{inputenc}
\usepackage[english, russian]{babel}

\usepackage{amssymb}
\usepackage{amsfonts}
\usepackage{amsmath}
\usepackage{enumitem}
\usepackage{graphics}
\usepackage{graphicx}

\usepackage{lipsum}

\newtheorem{theorem}{Теорема}
\newtheorem{task}{Задача}
\newtheorem{lemma}{Лемма}
\newtheorem{definition}{Определение}
\newtheorem{example}{Пример}
\newtheorem{statement}{Утверждение}
\newtheorem{corollary}{Следствие}


\usepackage{geometry} % Меняем поля страницы
\geometry{left=1.5cm}% левое поле
\geometry{right=1cm}% правое поле
\geometry{top=1.5cm}% верхнее поле
\geometry{bottom=1.5cm}% нижнее поле


\usepackage{fancyhdr} % Headers and footers
\pagestyle{fancy} % All pages have headers and footers
\fancyhead{} % Blank out the default header
\fancyfoot{} % Blank out the default footer
\fancyhead[L]{Математика}
\fancyhead[C]{\textit{Комбинаторика}}
\fancyhead[R]{19 января}% Custom header text


%----------------------------------------------------------------------------------------

%\begin{document}\normalsize
\begin{document}\large
	
\begin{center}
	\textbf{Дискретная непрерывность}
\end{center}

\begin{enumerate}[label*=\protect\fbox{\arabic{enumi}}]
	
\setcounter{enumi}{-1}

\item В ряд сидят 10 мальчиков и 10 девочек.

(a) Докажите, что можно выбрать 8 школьников подряд, чтобы среди них мальчиков и девочек
было поровну.

(b) Всегда ли из них можно выбрать 16 школьников подряд, среди которых мальчиков и девочек поровну?

\item Шеренга новобранцев стояла лицом к сержанту. По команде «Налево!» некоторые из них повернулись налево, некоторые — направо, а остальные — кругом. Всегда ли сержант сможет встать в строй так, чтобы с обеих сторон от него оказалось поровну новобранцев, стоящих к нему лицом?

\item По кругу сидят 25 мальчиков и 25 девочек.

(a) Докажите, что можно выбрать 20 школьников подряд, чтобы среди них мальчиков и девочек
было поровну.

(b) Докажите, что для каждого чётного $n$ от 2 до 50 можно выбрать $n$ школьников подряд, чтобы среди них мальчиков и девочек было поровну.

\item 

(a) На плоскости есть $2n$ точек. Докажите, что можно провести прямую так, что с каждой стороны от нее будет находиться n точек.

(b) На плоскости есть $2n + 1$ точка, никакие три не лежат на одной прямой. Докажите, что через любую из них можно провести прямую так, что с каждой стороны от нее будет находиться n точек.

\item Вася и Петя играют в игру. На доске написаны два числа: $\dfrac{1}{2022}$ и  $\dfrac{1}{2023}$ . На каждом ходу Вася называет любое число $x$, а Петя увеличивает любое из чисел на доске на $x$. Вася выигрывает, если когда-то одно из чисел на доске станет равным 1. Сможет ли Вася выиграть, как бы ни действовал Петя?

\item За круглым столом сидит 100 дедов, причем у любых двух соседей число волос в бороде отличается не больше чем на 100. Докажите, что найдется пара дедов, сидящих друг напротив друга, у которых число волос в бородах тоже отличается не больше, чем на 100.

\item $2n$ радиусов разделили круг на $2n$ равных секторов: $n$ синих и $n$ красных, чередующихся в произвольном порядке. В синие сектора, начиная с некоторого, записывают против хода часовой стрелки числа от 1 до $n$ по порядку. В красные сектора, начиная с некоторого, записывают те же числа, но по ходу часовой стрелки. Докажите, что найдется полукруг, в котором записаны все числа от 1 до $n$.


\end{enumerate}
\end{document}