\documentclass{article}
\usepackage[12pt]{extsizes}
\usepackage[T2A]{fontenc}
\usepackage[utf8]{inputenc}
\usepackage[english, russian]{babel}

\usepackage{amssymb}
\usepackage{amsfonts}
\usepackage{amsmath}
\usepackage{enumitem}
\usepackage{graphics}
\usepackage{graphicx}

\usepackage{lipsum}

\newtheorem{theorem}{Теорема}
\newtheorem{task}{Задача}
\newtheorem{lemma}{Лемма}
\newtheorem{definition}{Определение}
\newtheorem{example}{Пример}
\newtheorem{statement}{Утверждение}
\newtheorem{corollary}{Следствие}


\usepackage{geometry} % Меняем поля страницы
\geometry{left=1cm}% левое поле
\geometry{right=1cm}% правое поле
\geometry{top=1.5cm}% верхнее поле
\geometry{bottom=1cm}% нижнее поле


\usepackage{fancyhdr} % Headers and footers
\pagestyle{fancy} % All pages have headers and footers
\fancyhead{} % Blank out the default header
\fancyfoot{} % Blank out the default footer
\fancyhead[L]{Математика}
\fancyhead[C]{\textit{Методы}}
\fancyhead[R]{22 декабря}% Custom header text


%----------------------------------------------------------------------------------------

%\begin{document}\normalsize
\begin{document}\large
	
\begin{center}
	\textbf{Оценка + пример -- оценка}
\end{center}

\begin{enumerate}[label*=\protect\fbox{\arabic{enumi}}]
	
\item На доске записаны числа $1, 2^1, 2^2, 2^3, 2^4, 2^5$. Разрешается стереть любые два числа и вместо них записать их разность – неотрицательное число. 
Может ли на доске в результате нескольких таких операций остаться только число 15?
	
\item Существует ли признак делимости на 27 аналогичный делимости на 3 и на 9?
	
\item Существует ли треугольник, у которого все высоты меньше 1 см, а площадь больше 1 м$м^2$?

\item Можно ли, используя цифры от 1 до 9 каждую по разу, записать пять чисел, каждое из которых (кроме первого) делится на предыдущее?

\item В выпуклом четырехугольнике $ABCD$ равны стороны $AB$ и $CD$ и углы $A$ и $C$. Обязательно ли этот четырехугольник параллелограмм?

\item Может ли работа фирмы за любые пять месяцев быть прибыльной, а за весь год – убыточной?

\item Существуют ли на плоскости три такие точки $A, B$ и $C$, что для любой точки $X$ длина хотя бы одного из отрезков $XA, XB$ и $XC$ иррациональна?

\item Существуют ли натуральные числа $m$ и $n$, для которых верно равенство:  $$(-2a^nb^n)^m + (3a^mb^m)^n = a^6b^6$$

\item На плоскости даны две параболы: $y=x^2$ и $y=x^2-1$. Пусть $U$ – множество всех точек плоскости, лежащих между параболами (включая точки на самих параболах). Существует ли отрезок длины более $2022$, целиком содержащийся в $U$?

\item В каждой клетке доски $8\times 8$ написали по одному натуральному числу. Оказалось, что при любом разрезании доски на доминошки суммы чисел во всех доминошках будут разные. Может ли оказаться, что наибольшее записанное на доске число не больше 32?

\item Каждая грань куба заклеивается двумя равными прямоугольными треугольниками с общей гипотенузой, один из которых белый, другой — чёрный. Можно ли эти треугольники расположить так, чтобы при каждой вершине куба сумма белых углов была равна сумме чёрных углов? 

\item Можно ли разбить какой-нибудь клетчатый квадрат на клетчатые квадратики так, чтобы не все квадратики были одинаковы, но квадратиков каждого размера было одно и то же количество.
	
\item Натуральные числа от 1 до 2022 как-то разбили на пары, числа в каждой из пар сложили, а полученные 1011 сумм перемножили. 
Мог ли результат оказаться квадратом натурального числа?
	
\end{enumerate}
\end{document}