 \documentclass{article}
\usepackage[12pt]{extsizes}
\usepackage[T2A]{fontenc}
\usepackage[utf8]{inputenc}
\usepackage[english, russian]{babel}

\usepackage{amssymb}
\usepackage{amsfonts}
\usepackage{amsmath}
\usepackage{enumitem}
\usepackage{graphics}

\usepackage{lipsum}



\usepackage{geometry} % Меняем поля страницы
\geometry{left=1cm}% левое поле
\geometry{right=1cm}% правое поле
\geometry{top=1.5cm}% верхнее поле
\geometry{bottom=1cm}% нижнее поле

\newtheorem{definition}{Опредление}
\usepackage{fancyhdr} % Headers and footers
\pagestyle{fancy} % All pages have headers and footers
\fancyhead{} % Blank out the default header
\fancyfoot{} % Blank out the default footer
\fancyhead[L]{Математика}
\fancyhead[C]{\textit{Комбинаторика}}
\fancyhead[R]{27 марта 2023}% Custom header text


%----------------------------------------------------------------------------------------

%\begin{document}\normalsize
\begin{document}\large
	
	
\begin{center}
	\textbf{Информация --- 2}
\end{center}

\begin{enumerate}[label*=\protect\fbox{\arabic{enumi}}]

\item Имеются 60 монет, среди которых ровно одна фальшивая (неизвестно какая). Все настоящие монеты одного веса, а фальшивая легче или тяжелее. На чашечных весах можно сравнивать по весу любые две группы монет. Нужно найти фальшивую монету и выяснить, легче она или тяжелее. Сколько для этого необходимо взвешиваний?

\item Алиса и Боб держат в руках некоторую карточку. Карточка имеет две стороны, на одной стороне написано некоторое целое неотрицательное число $n$, а на другой — $n + 1$ Алиса видит одну из двух сторон карточки, а Боб — другую. Происходит следующий диалог. Алиса сообщает Бобу, что она не знает, что написано на другой стороне. После этого Боб сообщает то же самое Алисе (что он не знает, что написано на другой стороне). Этот диалог повторяется 10 раз — всего они делают 10 пар сообщений. После этого Алиса говорит, что теперь она знает число Боба. Какая
у них карточка?

\item Есть $2n$ монеток попарно различного веса. За одно действие можно сравнить любые две монетки. Какое минимальное количество взвешиваний потребуется, чтобы найти самую тяжёлую и самую лёгкую монетки?

\item Есть $2^n$ монеток попарно различного веса. За одно действие можно сравнить любые две монетки. Какое минимальное количество взвешиваний потребуется, чтобы найти самую тяжёлую и вторую по тяжести монетки?

\item В финале телешоу требуется угадать число от 1 до 144 с помощью вопросов, на которые ведущий отвечает <<Да>> или <<Нет>>. Однако за каждый ответ <<Да>> нужно платить по 1000 рублей из выигранных денег, а за ответ <<Нет>> --- 2000. Какой наименьшей суммы участник может лишиться, чтобы гарантированно угадать число и выиграть суперприз?

\item Фокусник и ассистент показывают фокус. Пока фокусника нет, зритель выкладывает в ряд 6 монет, после чего ассистент закрывает непрозрачной тканью $k$ монет. Наконец входит фокусник, который должен угадать, какой стороной вверх лежат закрытые монеты. При каком наибольшем $k$ ассистент и фокусник могут договориться, чтобы фокус удался?

\item Есть 15 монет, одна из которых фальшивая. Все настоящие монеты весят одинаково, а фальшивая весит иначе, но неизвестно, тяжелее она или легче. Какое наименьшее количество взвешиваний на двухчашечных весах без гирь необходимо, чтобы гарантированно найти фальшивую монету и сказать, тяжелее она или легче?

\item Та же задача, но теперь не надо говорить тяжелее фальшивая монета или легче.

\item Та же задача, но теперь про одну из монет вам известно, что она настоящая, и не надо говорить тяжелее фальшивая монета или легче. Изменится ли ответ, если отобрать у вас гарантированно настоящую монету?

\item Алиса и ее младший брат Боб играют в игру. Боб загадывает число от 1 до 1000, а Алиса пытается его угадать. Алиса называет Бобу число, а Боб говорит, верно ли, что оно больше загаданного. Алиса знает, что Боб, чтобы запутать Алису, может соврать один раз за игру (а может и не соврать). За какое наименьшее количество вопросов Алиса может гарантированно угадать загаданное Бобом число?

\end{enumerate}

\end{document}