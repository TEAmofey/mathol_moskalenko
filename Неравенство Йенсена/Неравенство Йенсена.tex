\documentclass{article}

\usepackage[12pt]{extsizes}
\usepackage[T2A]{fontenc}
\usepackage[utf8]{inputenc}
\usepackage[english, russian]{babel}

\usepackage{mathrsfs}
\usepackage[dvipsnames]{xcolor}

\usepackage{amsmath}
\usepackage{amssymb}
\usepackage{amsthm}
\usepackage{indentfirst}
\usepackage{amsfonts}
\usepackage{enumitem}
\usepackage{graphics}
\usepackage{tikz}
\usepackage{tabu}
\usepackage{diagbox}
\usepackage{hyperref}
\usepackage{mathtools}
\usepackage{ucs}
\usepackage{lipsum}
\usepackage{geometry} % Меняем поля страницы
\usepackage{fancyhdr} % Headers and footers
\newcommand{\range}{\mathrm{range}}
\newcommand{\dom}{\mathrm{dom}}
\newcommand{\N}{\mathbb{N}}
\newcommand{\R}{\mathbb{R}}
\newcommand{\E}{\mathbb{E}}
\newcommand{\D}{\mathbb{D}}
\newcommand{\M}{\mathcal{M}}
\newcommand{\Prime}{\mathbb{P}}
\newcommand{\A}{\mathbb{A}}
\newcommand{\Q}{\mathbb{Q}}
\newcommand{\Z}{\mathbb{Z}}
\newcommand{\F}{\mathbb{F}}
\newcommand{\CC}{\mathbb{C}}

\DeclarePairedDelimiter\abs{\lvert}{\rvert}
\DeclarePairedDelimiter\floor{\lfloor}{\rfloor}
\DeclarePairedDelimiter\ceil{\lceil}{\rceil}
\DeclarePairedDelimiter\lr{(}{)}
\DeclarePairedDelimiter\set{\{}{\}}
\DeclarePairedDelimiter\norm{\|}{\|}

\renewcommand{\labelenumi}{(\alph{enumi})}

\newcommand{\smallindent}{
    \geometry{left=1cm}% левое поле
    \geometry{right=1cm}% правое поле
    \geometry{top=1.5cm}% верхнее поле
    \geometry{bottom=1cm}% нижнее поле
}

\newcommand{\header}[3]{
    \pagestyle{fancy} % All pages have headers and footers
    \fancyhead{} % Blank out the default header
    \fancyfoot{} % Blank out the default footer
    \fancyhead[L]{#1}
    \fancyhead[C]{#2}
    \fancyhead[R]{#3}
}

\newcommand{\dividedinto}{
    \,\,\,\vdots\,\,\,
}

\newcommand{\littletaller}{\mathchoice{\vphantom{\big|}}{}{}{}}

\newcommand\restr[2]{{
    \left.\kern-\nulldelimiterspace % automatically resize the bar with \right
    #1 % the function
    \littletaller % pretend it's a little taller at normal size
    \right|_{#2} % this is the delimiter
}}

\DeclareGraphicsExtensions{.pdf,.png,.jpg}

\newenvironment{enumerate_boxed}[1][enumi]{\begin{enumerate}[label*=\protect\fbox{\arabic{#1}}]}{\end{enumerate}}



\smallindent

\header{Математика}{\textit{Алгебра}}{21 апреля 2024}

%----------------------------------------------------------------------------------------

%\begin{document}\normalsize
\begin{document}
    \large

    \begin{center}
        \textbf{Неравенство Йенсена}
    \end{center}

    \textbf{Определение:}
    Функция $f : [a, b] \rightarrow \mathbb{R}$ называется \emph{выпуклой} на $[a, b]$, если для любых $x, y \in [a, b]$ и любых $\alpha, \beta > 0$ таких, что $\alpha + \beta = 1$, выполняется неравенство
    \[
        f(\alpha x + \beta y) \leq \alpha f(x) + \beta f(y).
    \]
    Функция $f$ называется \emph{вогнутой} на $[a, b]$, если при тех же условиях выполняется аналогичное неравенство с противоположным знаком $\geqslant$.

    Геометрически, функция $f$ выпукла (вогнута), если любая точка любой хорды кривой $y = f(x)$ лежит над (под) этой кривой или на ней.

    \textbf{Факт:} Если для всех $x \in (a, b)$ выполняется условие $f''(x) > 0$, то функция $f$ выпукла на отрезке $[a, b]$.
    Если для всех $x \in (a, b)$ выполняется условие $f''(x) < 0$, то функция $f$ вогнута на отрезке $[a, b]$.
    \begin{enumerate_boxed}

        \item Функция $f$ выпукла на $[a, b]$.
        Пусть числа $x_1, \ldots, x_n$ принадлежат отрезку $[a, b]$, а числа $\alpha_1, \ldots, \alpha_n$ неотрицательны и в сумме дают 1.
        Докажите, что
        \[
            f(\alpha_1 x_1 + \cdots + \alpha_n x_n) \leqslant \alpha_1 f(x_1) + \cdots + \alpha_n f(x_n).
        \]
        Для вогнутой функции выполнено аналогичное неравенство со знаком $\geqslant$.

        \item Пусть $x_1, \ldots, x_n \in [0, \pi]$.
        Докажите, что
        \[
            \sin x_1 + \cdots + \sin x_n \leq n \cdot \sin \frac{x_1 + \cdots + x_n}{n}.
        \]

        \item Пусть $x_1, \ldots, x_n \geqslant 0$, $\alpha > 1$.
        Докажите, что
        \[
            \left(\frac{x_1^\alpha + x_2^\alpha + \dotsc + x_n^\alpha}{n}\right)^{1/\alpha} \geqslant \frac{x_1 + x_2 + \dotsc + x_n}{n}.
        \]

        \item Пусть $x_1, \ldots, x_n > 0$.
        Докажите, что
        \[
            \left(x_1 + x_2 + \dotsc + x_n\right)\left(\frac{1}{x_1} + \frac{1}{x_2} + \dotsc + \frac{1}{x_n}\right) \geqslant n^2.
        \]

        \item Числа $x_1, \ldots, x_n$ неотрицательны и в сумме дают 1.
        Докажите, что
        \[
            \frac{x_1}{\sqrt{1 - x_1}} + \cdots + \frac{x_n}{\sqrt{1 - x_n}} > \sqrt{\frac{n}{{n - 1}}}.
        \]

        \item Докажите, что для положительных $x_i$ и $y_i$ верно
        \[
            \sqrt{\left(x_1 + x_2 + \dotsc + x_n\right)^2 + \left(y_1 + y_2 + \dotsc + y_n\right)^2} \leq \sqrt{x_1^2 + y_1^2} + \sqrt{x_2^2 + y_2^2} + \dotsc + \sqrt{x_n^2 + y_n^2}.
        \]

        \item \textbf{Неравенство Гёльдера.} Пусть $p, q > 0$ таковы, что $\frac{1}{p} + \frac{1}{q} = 1$.
        Докажите, что для положительных $a_i$ и $b_i$ выполнено:

        \[
            (a_1 b_1 + a_2 b_2 + \dotsc + a_n b_n) \leqslant \left(a_1^p + a_2^p + \dotsc + a_n^p\right)^{1/p} \left(b_1^q + b_2^q+ \dotsc +b_n^q\right)^{1/q}.
        \]

    \end{enumerate_boxed}
\end{document}