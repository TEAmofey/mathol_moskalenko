\documentclass{article}
\usepackage[12pt]{extsizes}
\usepackage[T2A]{fontenc}
\usepackage[utf8]{inputenc}
\usepackage[english, russian]{babel}

\usepackage{amssymb}
\usepackage{amsfonts}
\usepackage{amsmath}
\usepackage{enumitem}
\usepackage{graphics}
\usepackage{graphicx}

\usepackage{lipsum}

\newtheorem{theorem}{Теорема}
\newtheorem{task}{Задача}
\newtheorem{lemma}{Лемма}
\newtheorem{definition}{Определение}
\newtheorem{example}{Пример}
\newtheorem{statement}{Утверждение}
\newtheorem{corollary}{Следствие}


\usepackage{geometry} % Меняем поля страницы
\geometry{left=1cm}% левое поле
\geometry{right=1cm}% правое поле
\geometry{top=1.5cm}% верхнее поле
\geometry{bottom=1cm}% нижнее поле


\usepackage{fancyhdr} % Headers and footers
\pagestyle{fancy} % All pages have headers and footers
\fancyhead{} % Blank out the default header
\fancyfoot{} % Blank out the default footer
\fancyhead[L]{Математика}
\fancyhead[C]{\textit{Комбинаторика}}
\fancyhead[R]{Тимофей Дмитриевич}% Custom header text


%----------------------------------------------------------------------------------------

%\begin{document}\normalsize
\begin{document}\large
	
\begin{center}
	\textbf{Графы}
\end{center}


\begin{enumerate}[label*=\protect\fbox{\arabic{enumi}}]
	
\item В государстве $100$ городов, и из каждого из них выходит $4$ дороги. Сколько всего дорог в государстве?

\item В деревне $15$ телефонов. Можно ли их соединить проводами так, чтобы каждый телефон был соединен ровно с пятью другими?

\item В классе $30$ человек. Может ли быть так, что $9$ из них имеют по $3$ друга (в этом классе), $11$ — по $4$ друга, а $10$ — по $5$ друзей (считается, что все дружбы взаимные)?

\item Может ли в государстве, в котором из каждого города выходит $3$ дороги, быть ровно $100$ дорог?

\item Каждый из $102$ учеников одной школы знаком не менее чем с $68$ другими. Докажите, что среди них найдутся четверо, имеющие одинаковое число знакомых.

\item В некоторой стране 15 городов, каждый из которых соединен дорогами не менее, чем с 7 другими. Докажите, что из любого города можно добраться до любого другого (возможно, проезжая через другие города).

\item Докажите, что любое ребро графа или является мостом, или лежит в каком- то цикле.

\item Нарисуйте все деревья с 5 вершинами и объясните, почему других нет.

\item Верно ли, что существует граф на $1001$ вершине, $1000$ вершин которого — висячие?

\item Волейбольная сетка имеет вид прямоугольника размером $50 \times 100$ клеток. Какое наибольшее число верёвочек можно перерезать так, чтобы сетка не распалась на куски?

\item В стране $100$ городов, некоторые из которых соединены авиалиниями. Известно, что от каждого города можно долететь до любого другого (возможно, с пересадками). Докажите, что можно побывать во всех городах, совершив не более $196$ перелётов.

\item Докажите, что в любом связном графе можно удалить некоторую вершину вместе со всеми выходящими из нее ребрами, чтобы он остался связным.

\item Докажите, что граф с $n$ вершинами, степень каждой из которых не менее $\dfrac{n - 1}{2}$, связен.

\item В Тридевятом царстве лишь один вид транспорта – ковер-самолет. Из столицы выходит 21 ковролиния, из города Дальний – одна, а из всех остальных городов – по 20. Докажите, что из столицы можно долететь в Дальний (возможно, с пересадками).

\item Как соединить 50 городов наименьшим числом авиалиний так, чтобы из каждого города можно было попасть в любой, сделав не более двух пересадок?

\item В стране из каждого города выходит 100 дорог и от каждого города можно добраться до любого другого. Одну дорогу закрыли на ремонт. Докажите, что и теперь от каждого города можно добраться до любого другого.

\item Связный граф на 10 вершинах таков, что при выкидывании любых двух  вершин вместе со всеми выходящими из них рёбрами он остаётся связным. Какое наименьшее количество рёбер может быть у этого графа?

\item Из полного 100-вершинного графа выкинули 98 рёбер. Доказать, что он остался связным.

\item В некоторой стране каждые два города соединены либо авиалинией, либо железной дорогой. Докажите, что

а) можно выбрать вид транспорта так, чтобы от каждого города можно было добраться до любого другого, пользуясь только этим видом транспорта;

б) из некоторого города, выбрав один из видов транспорта, можно добраться до любого другого города не более чем с одной пересадкой (пользоваться можно только выбранным видом транспорта);

в) каждый город обладает свойством из пункта б);

г) можно выбрать вид транспорта так, чтобы пользуясь только им, можно было добраться из каждого города до любого другого не более чем с двумя пересадками.

\item Между некоторыми из $2n$ городов установлено воздушное сообщение, причём каждый город связан (беспересадочными рейсами) не менее чем с $n$ другими. Докажите, что если отменить любые $n - 1$ рейсов, то всё равно из любого города можно добраться в любой другой на самолётах (с пересадками).

\item На клетчатой доске $11 \times 11$ отмечено 22 клетки так, что на каждой вертикали и на каждой горизонтали отмечено ровно две клетки. Два расположения отмеченных клеток эквивалентны, если, меняя любое число раз вертикали между собой и горизонтали между собой, мы из одного расположения можем получить другое. Сколько существует неэквивалентных расположений отмеченных клеток?

\item Степени всех вершин графа не меньше $n$, причем в нем нет циклов длины $3, 4$ и $5$. Докажите, что в нём существует $n^2 - n$ вершин, никакие две из которых не соединены ребром.
 
\end{enumerate}
\end{document}