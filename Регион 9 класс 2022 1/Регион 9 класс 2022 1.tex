\documentclass{article}
\usepackage[12pt]{extsizes}
\usepackage[T2A]{fontenc}
\usepackage[utf8]{inputenc}
\usepackage[english, russian]{babel}

\usepackage{amssymb}
\usepackage{amsfonts}
\usepackage{amsmath}
\usepackage{enumitem}
\usepackage{graphics}
\usepackage{graphicx}

\usepackage{lipsum}

\newtheorem{theorem}{Теорема}
\newtheorem{task}{Задача}
\newtheorem{lemma}{Лемма}
\newtheorem{definition}{Определение}
\newtheorem{example}{Пример}
\newtheorem{statement}{Утверждение}
\newtheorem{corollary}{Следствие}


\usepackage{geometry} % Меняем поля страницы
%\geometry{left=1cm}% левое поле
%\geometry{right=1cm}% правое поле
\geometry{top=3cm}% верхнее поле
%\geometry{bottom=1cm}% нижнее поле


\usepackage{fancyhdr} % Headers and footers
\pagestyle{fancy} % All pages have headers and footers
\fancyhead{} % Blank out the default header
\fancyfoot{} % Blank out the default footer
\fancyhead[L]{\textit{\textbf{Региональный этап ВСоШ по Математике}}}
\fancyhead[C]{}
\fancyhead[R]{31 декабря 2023}% Custom header text


%----------------------------------------------------------------------------------------

%\begin{document}\normalsize
\begin{document}\large
	
\begin{center}
	\LARGE\textbf{9 класс}
\end{center}
\begin{center}
	\large\textbf{Первый день}
\end{center}


\begin{enumerate}[label*=9.{\arabic{enumi}}]
\setcounter{enumi}{0}

%22.9.1
\item Петя написал на доске десять натуральных чисел, среди которых нет двух равных. Известно, что из этих десяти чисел можно выбрать три числа, делящихся на 5. Также известно, что из написанных десяти чисел можно выбрать четыре числа, делящихся на 4. Может ли сумма всех написанных на доске чисел быть меньше 75?

%22.9.2
\item На доске девять раз (друг под другом) написали некоторое натуральное число $N$. Петя к каждому из 9 чисел приписал слева или справа одну ненулевую цифру; при этом все приписанные цифры различны. Какое наибольшее количество простых чисел могло оказаться среди 9 полученных чисел?

%22.9.3
\item  Дан квадратный трёхчлен $P(x)$, не обязательно с целыми коэффициентами. Известно, что при некоторых целых $a$ и $b$ разность $P(a) - P(b)$ является квадратом натурального числа. Докажите, что существует более миллиона таких пар целых чисел $(c, d)$, что разность $P(c) - P(d)$ также является квадратом натурального числа.

%22.9.4
\item  В компании некоторые пары людей дружат (если $A$ дружит с $B$, то и $B$ дружит с $A$). Оказалось, что среди каждых 100 человек в компании количество пар дружащих людей нечётно. Найдите наибольшее возможное количество человек в такой компании.

%20.9.5
\item Четырёхугольник $ABCD$ описан около окружности $\omega$. Докажите, что диаметр окружности $\omega$ не превосходит длины отрезка, соединяющего середины сторон $BC$ и $AD$.

\end{enumerate}
\end{document}