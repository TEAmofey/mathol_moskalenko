\documentclass{article}

\usepackage[12pt]{extsizes}
\usepackage[T2A]{fontenc}
\usepackage[utf8]{inputenc}
\usepackage[english, russian]{babel}

\usepackage{mathrsfs}
\usepackage[dvipsnames]{xcolor}

\usepackage{amsmath}
\usepackage{amssymb}
\usepackage{amsthm}
\usepackage{indentfirst}
\usepackage{amsfonts}
\usepackage{enumitem}
\usepackage{graphics}
\usepackage{tikz}
\usepackage{tabu}
\usepackage{diagbox}
\usepackage{hyperref}
\usepackage{mathtools}
\usepackage{ucs}
\usepackage{lipsum}
\usepackage{geometry} % Меняем поля страницы
\usepackage{fancyhdr} % Headers and footers
\newcommand{\range}{\mathrm{range}}
\newcommand{\dom}{\mathrm{dom}}
\newcommand{\N}{\mathbb{N}}
\newcommand{\R}{\mathbb{R}}
\newcommand{\E}{\mathbb{E}}
\newcommand{\D}{\mathbb{D}}
\newcommand{\M}{\mathcal{M}}
\newcommand{\Prime}{\mathbb{P}}
\newcommand{\A}{\mathbb{A}}
\newcommand{\Q}{\mathbb{Q}}
\newcommand{\Z}{\mathbb{Z}}
\newcommand{\F}{\mathbb{F}}
\newcommand{\CC}{\mathbb{C}}

\DeclarePairedDelimiter\abs{\lvert}{\rvert}
\DeclarePairedDelimiter\floor{\lfloor}{\rfloor}
\DeclarePairedDelimiter\ceil{\lceil}{\rceil}
\DeclarePairedDelimiter\lr{(}{)}
\DeclarePairedDelimiter\set{\{}{\}}
\DeclarePairedDelimiter\norm{\|}{\|}

\renewcommand{\labelenumi}{(\alph{enumi})}

\newcommand{\smallindent}{
    \geometry{left=1cm}% левое поле
    \geometry{right=1cm}% правое поле
    \geometry{top=1.5cm}% верхнее поле
    \geometry{bottom=1cm}% нижнее поле
}

\newcommand{\header}[3]{
    \pagestyle{fancy} % All pages have headers and footers
    \fancyhead{} % Blank out the default header
    \fancyfoot{} % Blank out the default footer
    \fancyhead[L]{#1}
    \fancyhead[C]{#2}
    \fancyhead[R]{#3}
}

\newcommand{\dividedinto}{
    \,\,\,\vdots\,\,\,
}

\newcommand{\littletaller}{\mathchoice{\vphantom{\big|}}{}{}{}}

\newcommand\restr[2]{{
    \left.\kern-\nulldelimiterspace % automatically resize the bar with \right
    #1 % the function
    \littletaller % pretend it's a little taller at normal size
    \right|_{#2} % this is the delimiter
}}

\DeclareGraphicsExtensions{.pdf,.png,.jpg}

\newenvironment{enumerate_boxed}[1][enumi]{\begin{enumerate}[label*=\protect\fbox{\arabic{#1}}]}{\end{enumerate}}



\smallindent

\header{Математика}{\textit{Теория чисел}}{18 cентября 2022}

%----------------------------------------------------------------------------------------

\begin{document}
    \large

    \begin{center}
        \textbf{Малая теорема Ферма}
    \end{center}

    \begin{enumerate_boxed}

        \item Докажите, что $7^{120} - 1$  делится на $143$.

        \item Докажите, что $60^{111}+111^{60}$ делится на 61.

        \item Пусть $p$ – простое число.
        Докажите, что  $(a + b)^p \equiv a^p + b^p$ (mod $p$) для любых целых $a$ и $b$.

        \item Пусть $p>5$ --- простое число.
        Докажите, что $\underbrace{11\dots 1}_{p-1}$ делится на $p$.

        \item Докажите, что если $p$ --- простое число и $p > 2$, то $7^p - 5^p - 2$ делится на $6p$.

        \item Андрей берет натуральное число $a$ и сначала прибавляет к нему число $a^2$, потом --- $a^3$, потом --- $a^4$ и т.д. Докажите, что когда-нибудь его сумма поделится на простое число $p$.

        \item Докажите, что ни при каком целом $k$ число $k^2 + k + 1$  не делится на 101.

        \item Известно, что  $a^{12} + b^{12} + c^{12} + d^{12} + e^{12} + f^{12}$  делится на 13 ($a, b, c, d, e, f$ – целые числа).
        Докажите, что $abcdef$ делится на 4826809.

        \item Пусть $p = 4k+3$ --- простое, а $m^2 + n^2 \dividedinto p$.
        Докажите, что $m,n \dividedinto p$.

        \item Пусть $n$ – натуральное число, не кратное 17.
        Докажите, что: либо $n^8 + 1$, либо $n^4 + 1$, либо $n^2 + 1$, либо $n + 1$, либо $n - 1$ делится на 17.

        \item Докажи, что $a^{73} - a$ делится на $2\cdot3\cdot5\cdot7\cdot13\cdot19\cdot37\cdot73$.

        \item Найти все такие натуральные числа $p$, что $p$ и $p^6 + 6$ – простые

        \item
        a) Докажите, что равенство $\frac{10^n-1}{m}=\overline{a_{1}a_2\dots a_n}$ равносильно тому, что десятичная запись дроби $1/m$ имеет вид $0,(a_{1}a_2\dots a_n)$.

        b) Пусть $p>5$ --- простое число.
        Докажите, что $1/p=0,(a_{1}a_2\dots a_n)$ (т.е. дробь не имеет предпериода).

        c) Запишем $1/p=0,(a_{1}a_2\dots a_n)$.
        Докажите, что $p-1$ делится на $n$.

        d) Запишем $1/p=0,(a_{1}a_2\dots a_n)$.
        Докажите, что дробь $0,(a_{2}a_3\dots a_{n}a_1)$ --- тоже дробь со знаменателем $p$.

        \item Найдите такое шестизначное число $x$, что $x$, $2x$, $3x$, $4x$, $5x$ и $6x$ записываются одинаковым набором цифр, но в различном порядке.

    \end{enumerate_boxed}
\end{document}