\documentclass{article}
\usepackage[12pt]{extsizes}
\usepackage[T2A]{fontenc}
\usepackage[utf8]{inputenc}
\usepackage[english, russian]{babel}

\usepackage{amssymb}
\usepackage{amsfonts}
\usepackage{amsmath}
\usepackage{enumitem}
\usepackage{graphics}

\usepackage{lipsum}



\usepackage{geometry} % Меняем поля страницы
\geometry{left=1cm}% левое поле
\geometry{right=1cm}% правое поле
\geometry{top=1.5cm}% верхнее поле
\geometry{bottom=1cm}% нижнее поле


\usepackage{fancyhdr} % Headers and footers
\pagestyle{fancy} % All pages have headers and footers
\fancyhead{} % Blank out the default header
\fancyfoot{} % Blank out the default footer
\fancyhead[L]{ЦРОД $\bullet$ Математика}
\fancyhead[C]{\textit{Принцип Дирихле}}
\fancyhead[R]{Спешл для Оли}% Custom header text


%----------------------------------------------------------------------------------------

%\begin{document}\normalsize
\begin{document}\large
	
	
	\begin{center}
		\textbf{Принцип Дирихле}
	\end{center}
	
	
	\textit{Принцип Дирихле} - Если кролики рассажены в клетки, причём число кроликов больше числа клеток, то хотя бы в одной из клеток находится более одного кролика.
	
	\begin{enumerate}[label*=\protect\fbox{\arabic{enumi}}]
		
		\item В ковре размером $4 \times 4$ метра моль проела 15 дырок. Всегда ли можно вырезать коврик размером $1 \times 1$, не содержащий внутри дырок? (Дырки считаются точечными).
		
		\item В классе учатся 38 человек. Докажите, что среди них найдутся четверо, родившихся в один месяц.
		
		\item Обязательно ли среди двадцати пяти монет достоинством 1, 2, 5 и 10 рублей найдётся семь монет одинакового достоинства?
		
		\item Какое наибольшее число королей можно поставить на шахматной доске так, чтобы никакие два из них не били друг друга?
		
		\item В клетках таблицы $3 \times 3$ расставлены числа –1, 0, 1. Докажите, что какие-то две из восьми сумм по всем строкам, всем столбцам и двум главным диагоналям будут равны.
		
		\item Внутри правильного шестиугольника со стороной 1 расположено 7 точек. Докажите, что среди них найдутся две точки на расстоянии не больше 1.
		
		\item \begin{enumerate} 
			\item[а)] В каждой вершине куба написано число 1 или число 0. На каждой грани куба написана сумма четырёх чисел, написанных в вершинах этой грани. Может ли оказаться, что все числа, написанные на гранях, различны? 
			\item[б)] Тот же вопрос, если в вершинах написаны числа 1 или –1.
		\end{enumerate}
		
		\item На плоскости нарисовано 12 прямых, проходящих через точку О. Докажите, что можно выбрать две из них так, что угол между ними будет меньше 17 градусов.
		
		\item В мешке 70 шаров, отличающихся только цветом: 20 красных, 20 синих, 20 жёлтых, остальные – чёрные и белые. 
		Какое наименьшее число шаров надо вынуть из мешка, не видя их, чтобы среди них было не менее 10 шаров одного цвета?
		
		\item Даны $n$ точек. Некоторые из них соединены отрезками. Докажите, что найдутся две точки, из которых выходит поровну отрезков.
		
		\item Дано 8 различных натуральных чисел, не больших 15. Докажите, что среди их положительных попарных разностей есть три одинаковых.
		
		\item 10 школьников на олимпиаде решили 35 задач, причем известно, что среди них есть школьники, решившие ровно одну задачу, школьники, решившие ровно две задачи и школьники, решившие ровно три задачи. Докажите, что есть школьник, решивший не менее пяти задач.
		
		\item В квадрат со стороной 1 метр бросили 51 точку. Докажите, что какие-то три из них можно накрыть квадратом со стороной 20 см.
		
		\item Докажите, что равносторонний треугольник нельзя покрыть двумя меньшими равносторонними треугольниками.
		
		\item На складе имеется по 200 сапог 41, 42 и 43 размеров, причём среди этих 600 сапог 300 левых и 300 правых.
		Докажите, что из них можно составить не менее 100 годных пар обуви.
		
		\item Какое наибольшее число полей на доске $8 \times 8$ можно закрасить в чёрный цвет так, чтобы в каждом уголке из трёх полей было по крайней мере одно незакрашенное поле? 
		
		\item Докажите, что среди любых шести человек есть либо трое попарно знакомых, либо трое попарно незнакомых.
		
		\item В дискуссии приняли участие 15 депутатов. Каждый из них в своем выступлении раскритиковал ровно k из оставшихся 14 депутатов. При каком наименьшем k можно утверждать, что найдутся два депутата, которые раскритиковали друг друга?
		
		\item Докажите, что в любой компании найдутся два человека, имеющие одинаковое число друзей (из этой компании).
		
		\item Пятеро молодых рабочих получили на всех зарплату - 1500 рублей. Каждый из них хочет купить себе магнитофон ценой 320 рублей. Докажите, что кому-то из них придется подождать с покупкой до следующей зарплаты.
		
		\item В мешке лежат шарики двух разных цветов: черного и белого. Какое наименьшее число шариков нужно вынуть из мешка вслепую так, чтобы среди них заведомо оказались два шарика одного цвета?
		
		\item В лесу растет миллион елок. Известно, что на каждой из них не более 600000 иголок. Докажите, что в лесу найдутся две елки с одинаковым числом иголок
		
		\item В поход пошли 20 туристов. Самому старшему из них 35 лет, а самому младшему 20 лет. Верно ли, что среди туристов есть одногодки?
		
		\item Можно ли разложить 44 шарика на 9 кучек так, чтобы количество шариков в разных кучках было различным?
		
		\item Доказать, что если 21 человек собрали 200 орехов, то есть два человека, собравшие поровну орехов.
		
		\item Можно ли увезти из каменоломни 50 камней, веса которых равны 370, 372, \dots,  468 кг, на семи трёхтонках?
		
		\item Из чисел 1, 2, \dots , 49, 50 выбрали 26 чисел. Обязательно ли среди них найдутся два числа, отличающиеся друг от друга на 1?
		
		\item Докажите, что никакая прямая не может пересечь все три стороны треугольника (в точках, отличных от вершин).
		
		\item В классе 33 ученика, всем вместе 430 лет. Докажите, что если выбрать 20 самых старших из них, то им вместе будет не меньше, чем 260 лет. (Возраст любого ученика – целое число.)
		
		\item По кругу стоят мальчики и девочки (есть и те, и другие), всего 20 детей. Известно, что у каждого мальчика сосед по часовой стрелке – ребёнок в синей футболке, а у каждой девочки сосед против часовой стрелки – ребёнок в красной футболке. Можно ли однозначно установить, сколько в круге мальчиков?
		
		\item Квадрат разрезали 18 прямыми, из которых девять параллельны одной стороне квадрата, а девять – другой, на 100 прямоугольников. Оказалось, что ровно девять из них – квадраты. Докажите, что среди этих квадратов найдутся два равных между собой.
		
		
		
		
	\end{enumerate}
\end{document}