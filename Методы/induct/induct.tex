\documentclass{article}
\usepackage[12pt]{extsizes}
\usepackage[T2A]{fontenc}
\usepackage[utf8]{inputenc}
\usepackage[english, russian]{babel}

\usepackage{amssymb}
\usepackage{amsfonts}
\usepackage{amsmath}
\usepackage{enumitem}
\usepackage{graphics}

\usepackage{lipsum}



\usepackage{geometry} % Меняем поля страницы
\geometry{left=1cm}% левое поле
\geometry{right=1cm}% правое поле
\geometry{top=1.5cm}% верхнее поле
\geometry{bottom=1cm}% нижнее поле


\usepackage{fancyhdr} % Headers and footers
\pagestyle{fancy} % All pages have headers and footers
\fancyhead{} % Blank out the default header
\fancyfoot{} % Blank out the default footer
\fancyhead[L]{ЦРОД $\bullet$ Математика}
\fancyhead[C]{\textit{Метод математической индукции}}
\fancyhead[R]{10.11 --- 21.11.2020}% Custom header text


%----------------------------------------------------------------------------------------

%\begin{document}\normalsize
\begin{document}\large
	
	
	\begin{center}
		\textbf{Индукция}
	\end{center}
	
	Не забывайте про \textbf{3} составляющие индукции:
	\begin{enumerate}
		\item[1)] \textit{База} - показываем, что утверждение верно для простых частных случаев (Например при $n = 1$);
		
		\item[2)] \textit{Предположение} - предполагаем, что утверждение доказано для первых $k$ случаев;
		
		\item[3)] \textit{Переход} - используя предположение, доказываем утверждение для случая $n = k + 1$.
		
	\end{enumerate}
	
	Потеряете хотя бы одну из них $-$ индукция перестанет работать!
	
	
	\begin{enumerate}[label*=\protect\fbox{\arabic{enumi}}]
		
		\item Докажите, что $1 + 3 + 5 + \dots + (2n - 1) = n^2$
		
		\item Докажите по индукции, что
		\begin{enumerate}
			\item[a)] $1 + 2 + \dots + n =\dfrac{n (n+1)}{2}$
			
			\item[b)] $1^2 + 2^2 + \dots + n^2 =\dfrac{n (n+1)(2n + 1)}{6}$
			
			\item[*c)] $1^3 + 2^3 + \dots + n^3 =(1 + 2 + \dots + n) ^ 2$
		\end{enumerate}
	
		\item Докажите, что любую сумму начиная с 8 копеек можно уплатить монетами 3 и 5 копеек
	
		\item Докажите, что $\underbrace{111\dots111}_{3^n}\;\vdots\; 3^n$ для любого натурального $n$
		
		\item Для натуральных $n$ докажите, что $2^n > n$
		
		\item Для натуральных $n > 2$ докажите, что $2^n > n^2$
		
		\item Для натуральных $n > 2$ докажите, что $n! > 2^n$
		
		\item Несколько (a) прямых (b) оркужностей (c) прямых и оркужностей делят плоскость на части. Доказать, что можно раскрасить эти части в белый и чёрный цвет так, чтобы соседние части были разного цвета
		
		\item На плоскости проведены n прямых \textit{общего положения} (никакие 2 из них не параллельны, и никакие 3 не пересекаются в одной точке). На сколько частей делят плоскость 
		
		\item \textit{Неравенство Бернулли}: $(1+a)^n>1+an$, $n \in \mathbb{N}$, $a > - 1$
		
		\item Известно, что  $x + \dfrac{1}{x}$  – целое число. Докажите, что  $x^n + \dfrac{1}{x^n}$  – также целое при любом натуральном n.
		
		\item Докажите, что $n^n > (n + 1)^{n - 1}$, $n \in \mathbb{N}$
		
		
		
	\end{enumerate}
\end{document}