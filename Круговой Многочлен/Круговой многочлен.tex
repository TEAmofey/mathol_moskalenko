\documentclass{article}

\usepackage[12pt]{extsizes}
\usepackage[T2A]{fontenc}
\usepackage[utf8]{inputenc}
\usepackage[english, russian]{babel}

\usepackage{mathrsfs}
\usepackage[dvipsnames]{xcolor}

\usepackage{amsmath}
\usepackage{amssymb}
\usepackage{amsthm}
\usepackage{indentfirst}
\usepackage{amsfonts}
\usepackage{enumitem}
\usepackage{graphics}
\usepackage{tikz}
\usepackage{tabu}
\usepackage{diagbox}
\usepackage{hyperref}
\usepackage{mathtools}
\usepackage{ucs}
\usepackage{lipsum}
\usepackage{geometry} % Меняем поля страницы
\usepackage{fancyhdr} % Headers and footers
\newcommand{\range}{\mathrm{range}}
\newcommand{\dom}{\mathrm{dom}}
\newcommand{\N}{\mathbb{N}}
\newcommand{\R}{\mathbb{R}}
\newcommand{\E}{\mathbb{E}}
\newcommand{\D}{\mathbb{D}}
\newcommand{\M}{\mathcal{M}}
\newcommand{\Prime}{\mathbb{P}}
\newcommand{\A}{\mathbb{A}}
\newcommand{\Q}{\mathbb{Q}}
\newcommand{\Z}{\mathbb{Z}}
\newcommand{\F}{\mathbb{F}}
\newcommand{\CC}{\mathbb{C}}

\DeclarePairedDelimiter\abs{\lvert}{\rvert}
\DeclarePairedDelimiter\floor{\lfloor}{\rfloor}
\DeclarePairedDelimiter\ceil{\lceil}{\rceil}
\DeclarePairedDelimiter\lr{(}{)}
\DeclarePairedDelimiter\set{\{}{\}}
\DeclarePairedDelimiter\norm{\|}{\|}

\renewcommand{\labelenumi}{(\alph{enumi})}

\newcommand{\smallindent}{
    \geometry{left=1cm}% левое поле
    \geometry{right=1cm}% правое поле
    \geometry{top=1.5cm}% верхнее поле
    \geometry{bottom=1cm}% нижнее поле
}

\newcommand{\header}[3]{
    \pagestyle{fancy} % All pages have headers and footers
    \fancyhead{} % Blank out the default header
    \fancyfoot{} % Blank out the default footer
    \fancyhead[L]{#1}
    \fancyhead[C]{#2}
    \fancyhead[R]{#3}
}

\newcommand{\dividedinto}{
    \,\,\,\vdots\,\,\,
}

\newcommand{\littletaller}{\mathchoice{\vphantom{\big|}}{}{}{}}

\newcommand\restr[2]{{
    \left.\kern-\nulldelimiterspace % automatically resize the bar with \right
    #1 % the function
    \littletaller % pretend it's a little taller at normal size
    \right|_{#2} % this is the delimiter
}}

\DeclareGraphicsExtensions{.pdf,.png,.jpg}

\newenvironment{enumerate_boxed}[1][enumi]{\begin{enumerate}[label*=\protect\fbox{\arabic{#1}}]}{\end{enumerate}}


\usepackage[framemethod=TikZ]{mdframed}

\newcommand{\definebox}[3]{%
    \newcounter{#1}
    \newenvironment{#1}[1][]{%
        \stepcounter{#1}%
        \mdfsetup{%
            frametitle={%
            \tikz[baseline=(current bounding box.east),outer sep=0pt]
            \node[anchor=east,rectangle,fill=white]
            {\strut #2~\csname the#1\endcsname\ifstrempty{##1}{}{##1}};}}%
        \mdfsetup{innertopmargin=1pt,linecolor=#3,%
            linewidth=3pt,topline=true,
            frametitleaboveskip=\dimexpr-\ht\strutbox\relax,}%
        \begin{mdframed}[]
            \relax%
            }{
        \end{mdframed}}%
}

\definebox{theorem_boxed}{Теорема}{ForestGreen!24}
\definebox{definition_boxed}{Определение}{blue!24}
\definebox{task_boxed}{Задача}{orange!24}
\definebox{paradox_boxed}{Парадокс}{red!24}

\theoremstyle{plain}
\newtheorem{theorem}{Теорема}
\newtheorem{task}{Задача}
\newtheorem{lemma}{Лемма}
\newtheorem{definition}{Определение}
\newtheorem{statement}{Утверждение}
\newtheorem{corollary}{Следствие}

\theoremstyle{remark}
\newtheorem{remark}{Замечание}
\newtheorem{example}{Пример}

\smallindent

\header{Математика}{\textit{Алгебра}}{20 июля 2024}

%----------------------------------------------------------------------------------------

\begin{document}
    \large

    \begin{center}
        \textbf{Круговой многочлен}
    \end{center}

    \textbf{Определение:} Комплексное число $z$ называется \textbf{\textit{примитивным корнем}} степени $n$
    из $1$, если $z^n = 1$, но $z^k \neq 1$ при $1 \leqslant k < n$.

    \begin{enumerate_boxed}
        \item Докажите, что
        \begin{enumerate}
            \item Любой корень степени $n$ из 1 является степенью примитивного корня;
            \item Число $\varepsilon_1 = \cos\frac{2\pi}{n} + i\sin\frac{2\pi}{n}$ --- примитивный корень степени $n$ из 1;
            \item Все примитивные корни степени $n$ из 1 имеют вид $\varepsilon_d = \cos\frac{2\pi d}{n} + i\sin\frac{2\pi d}{n}$, где НОД$(d,n) = 1$;
            \item Если $\varepsilon$ --- примитивный корень степени $n$ из 1, то все примитивные корни степени $n$ из 1 имеют вид $\varepsilon^d$, где НОД$(d,n) = 1$
        \end{enumerate}
    \end{enumerate_boxed}

    \textbf{Определение:} \textit{Круговой многочлен} (или многочлен деления круга, или циклотомический многочлен) --- это многочлен $\Phi_n(x) = \prod (x - \varepsilon_k)$, где $\varepsilon_k$ --- все примитивные корни степени $n$ из 1.
    Ясно, что $\Phi_n(x) = \prod\limits_{d,(d,n) = 1} (x-\varepsilon^d)$, где $\varepsilon$ --- любой примитивный корень степени $n$ из 1.

    \begin{enumerate_boxed}
        \setcounter{enumi}{1}
        \item
        \begin{enumerate}
            \item  Найдите явно $\Phi_n(x)$ для $n = 2,3,5$.

            \item  Чему равна степень $\Phi_n(x)$?

            \item  Докажите, что $x^n - 1 = \prod\limits_{d|n} \Phi_d(x)$.

            \item  Найдите явно $\Phi_{81}(x); \Phi_n(x)$, если $n = p^k$, $p$ --- простое.

            \item  Докажите, что $\Phi_n(x) \in \Z[x]$.
        \end{enumerate}


        \textbf{Замечание.} Во всех примерах коэффициенты $\Phi_n(x)$ принадлежат множеству \\ $\{-1,0,1\}$.
        Однако это не всегда так!
        Наименьшее $n$, при котором это не так --- $n = 105$.
        Вообще, любое целое число встречается среди коэффициентов.

        \item Пусть $p$ --- простое число.
        Докажите, что:
        \begin{enumerate}
            \item  Если $p|n$, то $\Phi_n(x^p) = \Phi_{np}(x)$;
            \item  Если $p$ не делит $n$, то $\Phi_n(x^p) = \Phi_{np}(x)\Phi_n(x)$;
            \item  Если $n$ --- нечётно, то $\Phi_n(-x) = \Phi_{2n}(x)$;

        \end{enumerate}

        \textit{Подсказка: воспользоваться задачей 2(b).}

        \item Докажите, что если $(n,a) = 1$, то $\Phi_n(x^a) = \prod\limits_{d|a}\Phi_{nd}(x)$.

        \item Даны натуральные $n,k > 1$.
        Докажите, что $\Phi_n(k) \geq 2$.

        \item Докажите, что в следующей бесконечной последовательности нет простых чисел:
        \[10001, 100010001, 1000100010001, \ldots \]

        \item Докажите, что число $2^{2^n} + 2^{2^{n-1}} + 1$ раскладывается в произведение по крайней мере $n$ простых чисел (не обязательно различных).

        \item Докажите, что $\Phi_n(x)$ --- возвратный многочлен.
        (Многочлен $P(x) = a_{n}x^n + a_{n-1}x^{n-1} + \\ + \ldots + a_{1}x + a_0$ называется возвратным, если $a_k = a_{n-k}, 0 \leq k \leq n.)$

        \item Найдите $\Phi_n(1)$.

        \item Докажите, что для каждого многочлена $f(x) \in \Z[x]$ найдётся такой ненулевой многочлен $g(x) \in \Z[x]$, что $g(x^{10})$ делится на $f(x)$.
        % сказать про свободный член, \Phi(1),

    \end{enumerate_boxed}

    \textbf{Теорема 1.} Если $\Phi_n(a) \dividedinto p$, то:
    \begin{enumerate}[label*=$\bullet$]
        \item или показатель $a$ по модулю $p$ равен в точности $n$ (в частности, $p-1 \dividedinto n$);

        \item или $n \dividedinto p$.
    \end{enumerate}

    \begin{enumerate_boxed}
        \setcounter{enumi}{10}
        \item Что означает эта теорема при $n = 4$?

        \item Верно ли, что если выполнен один из пунктов заключения теоремы, то $\Phi_n(a) \dividedinto p$?

        \item Докажите, что все простые делители числа $m^2+m+1$ или равны 3, или имеют вид $6k+1$.

        % \item Решите в натуральных числах уравнение: $\frac{x^7-1}{x-1} = y^5 - 1$.

        \item Докажите частный случай теоремы Дирихле: для любого натурального $n$ существует бесконечно много простых чисел вида $nk+1$.

        \textbf{Докажем теорему 1.}
        \item Рассмотрим сначала случай $p = 2$.
        Чему равно (a) $\Phi_n(0)?$ (b) $\Phi_n(1)$? (c) Докажите, что $\Phi_n(a)$ чётно только при $n = 2^k$. % и при этом $\Phi_n(a)$ почти никогда не делится на 4.

%        \textit{Вспомним лемму об уточнении показателя в частном случае.} Пусть $p$ --- нечётное простое число, $a$ --- целое, не равное 1.
%        Тогда если $ord_p(a-1) > 0$, то $ord_p(a^n-1) =\\ = ord_p(a-1) + ord_p(n)$, где $ord_p(a)$ --- это наибольшее такое целое неотрицательное $k$, что $p^k | a$.
%
%        Если же $p = 2$, то это верно при условии $ord_2(a-1) \geq 2$.

        \item
        \begin{enumerate}
            \item Докажите, что $x^n-1 = \Phi_n(x)Q(x)$, где $Q(x) \dividedinto x^d - 1$ для всех $d | n$, $d < n$.
            \item Пусть в условии теоремы 1 показатель $a$ по модулю $p$ равен $d < n$.
            Докажите, что тогда $\frac{n}{d} \dividedinto p$.
            Завершите доказательство теоремы 1.
        \end{enumerate}

        \textbf{Теорема 2.} Если $\Phi_n(a) \dividedinto p$, $\Phi_m(a) \dividedinto p$, то $\frac{m}{n} = p^l$.
        В частности, при $m \neq n$ верно, что $\Nod(\Phi_n(a),\Phi_m(a)) = p^s$.

        \item \textbf{Докажем теорему 2.}
        \begin{enumerate}
            \item Докажите её для $p = 2$.
            \item Докажите усиление \textbf{16(b)} при $p \neq 2$: пусть показатель $a$ по модулю $p$ равен $d$ и $\Phi_n(a) \dividedinto p$.
            Тогда $\frac{n}{d} = p^k$.
        \end{enumerate}
        Завершите доказательство теоремы 2.

        \item Докажите, что при $a > 1$ число $a^{10} + a^5 + 1$ не является степенью простого числа.

        \item Докажите, что число $2^{2^n} + 2^{2^{n-1}} + 1$ раскладывается в произведение по крайней мере $n$ \textit{различных} простых чисел.

        \item Даны попарно различные простые числа $p_1,p_2,\ldots,p_n$.
        Докажите, что число $2^{p_{1}p_2\ldots p_n} + 1$ имеет хотя бы $2^{2^{n-1}}$ делителей.


    \end{enumerate_boxed}

\end{document}