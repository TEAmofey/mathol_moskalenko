\documentclass{article}
\usepackage[12pt]{extsizes}
\usepackage[T2A]{fontenc}
\usepackage[utf8]{inputenc}
\usepackage[english, russian]{babel}

\usepackage{amssymb}
\usepackage{amsfonts}
\usepackage{amsmath}
\usepackage{enumitem}
\usepackage{graphics}

\usepackage{lipsum}



\usepackage{geometry} % Меняем поля страницы
\geometry{left=1cm}% левое поле
\geometry{right=1cm}% правое поле
\geometry{top=1.5cm}% верхнее поле
\geometry{bottom=1cm}% нижнее поле


\newtheorem{definition}{Опредление}
\usepackage{fancyhdr} % Headers and footers
\pagestyle{fancy} % All pages have headers and footers
\fancyhead{} % Blank out the default header
\fancyfoot{} % Blank out the default footer
\fancyhead[L]{Математика}
\fancyhead[C]{\textit{Геометрия}}
\fancyhead[R]{}% Custom header text


%----------------------------------------------------------------------------------------

%\begin{document}\normalsize
\begin{document}\large



\begin{center}
\textbf{Треугольник}
\end{center}

\begin{enumerate}[label*=\protect\fbox{\arabic{enumi}}]

\item Формулы площади треугольника
\begin{enumerate} 
	\item $S = \dfrac{1}{2}ah_a$
	\item $S = \dfrac{1}{2}ab \sin(\gamma)$
	\item $S = pr$
	\item $S = \dfrac{abc}{4R}$
	\item \textbf{Формула Герона} $S = \sqrt{p(p-a)(p-b)(p-c)}$
	\item \textbf{Прямоугольный треугольник} $S = \dfrac{1}{2}aи$
	\item \textbf{Равносторонний треугольник} $S = \dfrac{a^2\sqrt{3}}{4}$
\end{enumerate}

\item \textbf{Теорема косинусов}  $$c^2 = a^2 + b^2 - 2ab\cos(\gamma)$$

\item \textbf{Теорема синусов}  $$\frac{a}{\sin(\alpha)} = \frac{b}{\sin(\beta)} = \frac{c}{\sin(\gamma)} = 2R$$

\item \textbf{Формула медианы} $$m_c = \frac{\sqrt{2a^2 + 2b^2 - c^2}}{2}$$

\item \textbf{Формула биссектрисы} 
$$\ell_c = \frac{2ab\cos{\frac{\gamma}{2}}}{a + b} = \sqrt{ab - xy}$$ 

\item \textbf{Формула высоты} 
$$h_c = \frac{2S}{c} = a\cdot\sin(\beta) = b\cdot\sin(\alpha)$$ 

	\textbf{Формула высоты в прямоугольном треугольнике}
$$h_c = \frac{ab}{c}$$ 

\item \textbf{Обобщённая формула чевианы (Теорема Стюарта)}
$$s_c = \sqrt{a^2 \cdot \frac{x}{x + y} + b^2 \cdot \frac{y}{x + y} - xy}$$

\item \textbf{Формула Эйлера}
$$OI^2 = R^2 - 2rR$$


\end{enumerate}

\begin{center}
\textbf{Четырёхугольник}
\end{center}

\begin{enumerate}[label*=\protect\fbox{\arabic{enumi}}]

\item Формулы площади четырёхугольника
\begin{enumerate} 
	\item $S = \frac{1}{2}ef\sin(\alpha)$
	\item \textbf{Параллелограмм} $S = ah_a$
	\item \textbf{Параллелограмм} $S = ab\sin(\alpha)$
	\item \textbf{Описанный} $S = pr$
	\item \textbf{Вписанный (Формула Брахмагупты)} $$S = \sqrt{(p-a)(p-b)(p-c)(p-d)}$$
	\item \textbf{Обобщённая Формула Брахмагупты} $$S = \sqrt{(p-a)(p-b)(p-c)(p-d) - abcd\cos^2(\theta)}$$
\end{enumerate}

\item \textbf{Формула Бретшнайдера}  $$e^2f^2 = a^2c^2 + b^2d^2 - 2abcd\cos(\alpha + \gamma)$$

\item \textbf{Теорема Птолемея для вписанного четырёхугольника}   $$ef = ac + bd$$



\end{enumerate}

\end{document}