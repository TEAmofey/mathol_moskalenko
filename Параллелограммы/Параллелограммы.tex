\documentclass{article}
\usepackage[12pt]{extsizes}
\usepackage[T2A]{fontenc}
\usepackage[utf8]{inputenc}
\usepackage[english, russian]{babel}

\usepackage{amssymb}
\usepackage{amsfonts}
\usepackage{amsmath}
\usepackage{enumitem}
\usepackage{graphics}

\usepackage{lipsum}



\usepackage{geometry} % Меняем поля страницы
\geometry{left=1cm}% левое поле
\geometry{right=1cm}% правое поле
\geometry{top=1.5cm}% верхнее поле
\geometry{bottom=1cm}% нижнее поле


\usepackage{fancyhdr} % Headers and footers
\pagestyle{fancy} % All pages have headers and footers
\fancyhead{} % Blank out the default header
\fancyfoot{} % Blank out the default footer
\fancyhead[L]{Математика}
\fancyhead[C]{\textit{Геометрия}}
\fancyhead[R]{Август 2022}% Custom header text


%----------------------------------------------------------------------------------------

%\begin{document}\normalsize
\begin{document}\large


\begin{center}
\textbf{Параллелограммы}
\end{center}

\begin{enumerate}[label*=\protect\fbox{\arabic{enumi}}]

\item Докажите, что в прямоугольном треугольнике медиана, проведённая к гипотенузе, в два раза меньше гипотенузы.

\item Докажите, что длина медианы, проведённой к стороне треугольника, меньше полусуммы длин двух других сторон.

\item Докажите, что в равнобедренной трапеции (у которой боковые стороны равны) равны углы, прилежащие к основанию, а также равны длины диагоналей.

\item
\begin{enumerate}
	\item \textit{Параллелограмм Вариньона}. Докажите, что середины сторон четырёхугольника являются вершинами параллелограмма.

 	\item Докажите, что прямые, соединяющие середины противоположных сторон четырёхугольника, пересекаются в середине отрезка, соединяющего середины диагоналей.
\end{enumerate}
%\item На плоскости даны $4$ точки, не являющиеся вершинами параллелограмма, никакие $3$ из которых не лежат на одной прямой. Сколько существует параллелограммов, у которых ровно $3$ из этих точек являются вершинами?

\item На сторонах $AB, BC, CD$ и $DA$ параллелограмма $ABCD$ выбраны точки $K, L, M$ и $N$ соответственно так, что $AK=CM$ и $AN=CL$ (причём никакая из точек $K, L, M$ и $N$ не является серединой стороны параллелограмма $ABCD$). Сколько существует параллелограммов с вершинами в восьми отмеченных точках (включая параллелограмм $ABCD$)?

\item Дан параллелограмм $ABCD$ с длинами сторон 12 и 8. Биссектрисы его углов при пересечении образуют четырёхугольник. Чему равны длины диагоналей этого четырёхугольника?

\item У выпуклого шестиугольника две пары противоположных стороны попарно равны и параллельны. Докажите, что оставшиеся стороны также равны и параллельны.

\item На стороне $AC$ треугольника $ABC$ взята точка $D$ так, что $AD:DC=1:2$. Докажите, что у треугольников $ADB$ и $CDB$ есть по равной медиане.

\item В треугольнике $ABC$ точка $M$ — середина стороны $AC$. На стороне $AB$ выбрана точка $X$ такая, что $\angle MXB=\angle ABC$. Длина стороны $BC$ равна $a$. Найдите длину отрезка $MX$.

\item Средняя линия трапеции. Докажите, что отрезок, соединяющий середины противоположных непараллельных сторон трапеции, параллелен основаниям и по длине равен полусумме длин оснований.

\item Дана трапеция $ABCD$ с основаниями $AD$ и $BC$, в которой $AB=BD$. Пусть $M$ — середина стороны $DC$. Докажите, что $\angle MBC=\angle BCA$.

\item Квадрат вписан в равнобедренный прямоугольный треугольник, причём одна вершина квадрата расположена на гипотенузе, противоположная ей вершина совпадает с вершиной прямого угла треугольника, а остальные лежат на катетах. Найдите сторону квадрата, если катет треугольника равен $a$.

\item Биссектрисы $AA_1$ и $CC_1$ прямоугольного треугольника $ABC$ ($\angle B=90^\circ$) пересекаются в точке $I$. Прямая, проходящая через точку $C_1$ и перпендикулярная прямой $AA_1$, пересекает прямую, проходящую через $A_1$ и перпендикулярную $CC_1$, в точке $K$. Докажите, что середина отрезка $KI$ лежит на отрезке $AC$.


\end{enumerate}
\end{document}