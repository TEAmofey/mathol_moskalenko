\documentclass{article}

\usepackage[12pt]{extsizes}
\usepackage[T2A]{fontenc}
\usepackage[utf8]{inputenc}
\usepackage[english, russian]{babel}

\usepackage{mathrsfs}
\usepackage[dvipsnames]{xcolor}

\usepackage{amsmath}
\usepackage{amssymb}
\usepackage{amsthm}
\usepackage{indentfirst}
\usepackage{amsfonts}
\usepackage{enumitem}
\usepackage{graphics}
\usepackage{tikz}
\usepackage{tabu}
\usepackage{diagbox}
\usepackage{hyperref}
\usepackage{mathtools}
\usepackage{ucs}
\usepackage{lipsum}
\usepackage{geometry} % Меняем поля страницы
\usepackage{fancyhdr} % Headers and footers
\newcommand{\range}{\mathrm{range}}
\newcommand{\dom}{\mathrm{dom}}
\newcommand{\N}{\mathbb{N}}
\newcommand{\R}{\mathbb{R}}
\newcommand{\E}{\mathbb{E}}
\newcommand{\D}{\mathbb{D}}
\newcommand{\M}{\mathcal{M}}
\newcommand{\Prime}{\mathbb{P}}
\newcommand{\A}{\mathbb{A}}
\newcommand{\Q}{\mathbb{Q}}
\newcommand{\Z}{\mathbb{Z}}
\newcommand{\F}{\mathbb{F}}
\newcommand{\CC}{\mathbb{C}}

\DeclarePairedDelimiter\abs{\lvert}{\rvert}
\DeclarePairedDelimiter\floor{\lfloor}{\rfloor}
\DeclarePairedDelimiter\ceil{\lceil}{\rceil}
\DeclarePairedDelimiter\lr{(}{)}
\DeclarePairedDelimiter\set{\{}{\}}
\DeclarePairedDelimiter\norm{\|}{\|}

\renewcommand{\labelenumi}{(\alph{enumi})}

\newcommand{\smallindent}{
    \geometry{left=1cm}% левое поле
    \geometry{right=1cm}% правое поле
    \geometry{top=1.5cm}% верхнее поле
    \geometry{bottom=1cm}% нижнее поле
}

\newcommand{\header}[3]{
    \pagestyle{fancy} % All pages have headers and footers
    \fancyhead{} % Blank out the default header
    \fancyfoot{} % Blank out the default footer
    \fancyhead[L]{#1}
    \fancyhead[C]{#2}
    \fancyhead[R]{#3}
}

\newcommand{\dividedinto}{
    \,\,\,\vdots\,\,\,
}

\newcommand{\littletaller}{\mathchoice{\vphantom{\big|}}{}{}{}}

\newcommand\restr[2]{{
    \left.\kern-\nulldelimiterspace % automatically resize the bar with \right
    #1 % the function
    \littletaller % pretend it's a little taller at normal size
    \right|_{#2} % this is the delimiter
}}

\DeclareGraphicsExtensions{.pdf,.png,.jpg}

\newenvironment{enumerate_boxed}[1][enumi]{\begin{enumerate}[label*=\protect\fbox{\arabic{#1}}]}{\end{enumerate}}



\smallindent

\header{Математика}{\textit{Геометрия}}{18 августа 2022}

%----------------------------------------------------------------------------------------

\begin{document}
    \large

    \begin{center}
        \textbf{Параллелограммы}
    \end{center}

    \begin{enumerate_boxed}

        \item Докажите, что в прямоугольном треугольнике медиана, проведённая к гипотенузе, в два раза меньше гипотенузы.

        \item Докажите, что длина медианы, проведённой к стороне треугольника, меньше полусуммы длин двух других сторон.

        \item Докажите, что в равнобедренной трапеции (у которой боковые стороны равны) равны углы, прилежащие к основанию, а также равны длины диагоналей.

        \item
        \begin{enumerate}
            \item \textit{Параллелограмм Вариньона}.
            Докажите, что середины сторон четырёхугольника являются вершинами параллелограмма.

            \item Докажите, что прямые, соединяющие середины противоположных сторон четырёхугольника, пересекаются в середине отрезка, соединяющего середины диагоналей.
        \end{enumerate}
%\item На плоскости даны $4$ точки, не являющиеся вершинами параллелограмма, никакие $3$ из которых не лежат на одной прямой. Сколько существует параллелограммов, у которых ровно $3$ из этих точек являются вершинами?

        \item На сторонах $AB, BC, CD$ и $DA$ параллелограмма $ABCD$ выбраны точки $K, L, M$ и $N$ соответственно так, что $AK=CM$ и $AN=CL$ (причём никакая из точек $K, L, M$ и $N$ не является серединой стороны параллелограмма $ABCD$). Сколько существует параллелограммов с вершинами в восьми отмеченных точках (включая параллелограмм $ABCD$)?

        \item Дан параллелограмм $ABCD$ с длинами сторон 12 и 8.
        Биссектрисы его углов при пересечении образуют четырёхугольник.
        Чему равны длины диагоналей этого четырёхугольника?

        \item У выпуклого шестиугольника две пары противоположных стороны попарно равны и параллельны.
        Докажите, что оставшиеся стороны также равны и параллельны.

        \item На стороне $AC$ треугольника $ABC$ взята точка $D$ так, что $AD:DC=1:2$.
        Докажите, что у треугольников $ADB$ и $CDB$ есть по равной медиане.

        \item В треугольнике $ABC$ точка $M$ — середина стороны $AC$.
        На стороне $AB$ выбрана точка $X$ такая, что $\angle MXB=\angle ABC$.
        Длина стороны $BC$ равна $a$.
        Найдите длину отрезка $MX$.

        \item Средняя линия трапеции.
        Докажите, что отрезок, соединяющий середины противоположных непараллельных сторон трапеции, параллелен основаниям и по длине равен полусумме длин оснований.

        \item Дана трапеция $ABCD$ с основаниями $AD$ и $BC$, в которой $AB=BD$.
        Пусть $M$ — середина стороны $DC$.
        Докажите, что $\angle MBC=\angle BCA$.

        \item Квадрат вписан в равнобедренный прямоугольный треугольник, причём одна вершина квадрата расположена на гипотенузе, противоположная ей вершина совпадает с вершиной прямого угла треугольника, а остальные лежат на катетах.
        Найдите сторону квадрата, если катет треугольника равен $a$.

        \item Биссектрисы $AA_1$ и $CC_1$ прямоугольного треугольника $ABC$ ($\angle B=90^\circ$) пересекаются в точке $I$.
        Прямая, проходящая через точку $C_1$ и перпендикулярная прямой $AA_1$, пересекает прямую, проходящую через $A_1$ и перпендикулярную $CC_1$, в точке $K$.
        Докажите, что середина отрезка $KI$ лежит на отрезке $AC$.
        
    \end{enumerate_boxed}
\end{document}