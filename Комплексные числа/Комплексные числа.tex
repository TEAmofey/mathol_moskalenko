\documentclass{article}

\usepackage[12pt]{extsizes}
\usepackage[T2A]{fontenc}
\usepackage[utf8]{inputenc}
\usepackage[english, russian]{babel}

\usepackage{mathrsfs}
\usepackage[dvipsnames]{xcolor}

\usepackage{amsmath}
\usepackage{amssymb}
\usepackage{amsthm}
\usepackage{indentfirst}
\usepackage{amsfonts}
\usepackage{enumitem}
\usepackage{graphics}
\usepackage{tikz}
\usepackage{tabu}
\usepackage{diagbox}
\usepackage{hyperref}
\usepackage{mathtools}
\usepackage{ucs}
\usepackage{lipsum}
\usepackage{geometry} % Меняем поля страницы
\usepackage{fancyhdr} % Headers and footers
\newcommand{\range}{\mathrm{range}}
\newcommand{\dom}{\mathrm{dom}}
\newcommand{\N}{\mathbb{N}}
\newcommand{\R}{\mathbb{R}}
\newcommand{\E}{\mathbb{E}}
\newcommand{\D}{\mathbb{D}}
\newcommand{\M}{\mathcal{M}}
\newcommand{\Prime}{\mathbb{P}}
\newcommand{\A}{\mathbb{A}}
\newcommand{\Q}{\mathbb{Q}}
\newcommand{\Z}{\mathbb{Z}}
\newcommand{\F}{\mathbb{F}}
\newcommand{\CC}{\mathbb{C}}

\DeclarePairedDelimiter\abs{\lvert}{\rvert}
\DeclarePairedDelimiter\floor{\lfloor}{\rfloor}
\DeclarePairedDelimiter\ceil{\lceil}{\rceil}
\DeclarePairedDelimiter\lr{(}{)}
\DeclarePairedDelimiter\set{\{}{\}}
\DeclarePairedDelimiter\norm{\|}{\|}

\renewcommand{\labelenumi}{(\alph{enumi})}

\newcommand{\smallindent}{
    \geometry{left=1cm}% левое поле
    \geometry{right=1cm}% правое поле
    \geometry{top=1.5cm}% верхнее поле
    \geometry{bottom=1cm}% нижнее поле
}

\newcommand{\header}[3]{
    \pagestyle{fancy} % All pages have headers and footers
    \fancyhead{} % Blank out the default header
    \fancyfoot{} % Blank out the default footer
    \fancyhead[L]{#1}
    \fancyhead[C]{#2}
    \fancyhead[R]{#3}
}

\newcommand{\dividedinto}{
    \,\,\,\vdots\,\,\,
}

\newcommand{\littletaller}{\mathchoice{\vphantom{\big|}}{}{}{}}

\newcommand\restr[2]{{
    \left.\kern-\nulldelimiterspace % automatically resize the bar with \right
    #1 % the function
    \littletaller % pretend it's a little taller at normal size
    \right|_{#2} % this is the delimiter
}}

\DeclareGraphicsExtensions{.pdf,.png,.jpg}

\newenvironment{enumerate_boxed}[1][enumi]{\begin{enumerate}[label*=\protect\fbox{\arabic{#1}}]}{\end{enumerate}}


\usepackage[framemethod=TikZ]{mdframed}

\newcommand{\definebox}[3]{%
    \newcounter{#1}
    \newenvironment{#1}[1][]{%
        \stepcounter{#1}%
        \mdfsetup{%
            frametitle={%
            \tikz[baseline=(current bounding box.east),outer sep=0pt]
            \node[anchor=east,rectangle,fill=white]
            {\strut #2~\csname the#1\endcsname\ifstrempty{##1}{}{##1}};}}%
        \mdfsetup{innertopmargin=1pt,linecolor=#3,%
            linewidth=3pt,topline=true,
            frametitleaboveskip=\dimexpr-\ht\strutbox\relax,}%
        \begin{mdframed}[]
            \relax%
            }{
        \end{mdframed}}%
}

\definebox{theorem_boxed}{Теорема}{ForestGreen!24}
\definebox{definition_boxed}{Определение}{blue!24}
\definebox{task_boxed}{Задача}{orange!24}
\definebox{paradox_boxed}{Парадокс}{red!24}

\theoremstyle{plain}
\newtheorem{theorem}{Теорема}
\newtheorem{task}{Задача}
\newtheorem{lemma}{Лемма}
\newtheorem{definition}{Определение}
\newtheorem{statement}{Утверждение}
\newtheorem{corollary}{Следствие}

\theoremstyle{remark}
\newtheorem{remark}{Замечание}
\newtheorem{example}{Пример}

\smallindent

\header{Математика}{Комплексные числа}{}


%----------------------------------------------------------------------------------------

%\begin{document}\normalsize
\begin{document}\large


\setcounter{task_boxed}{0}

\begin{definition_boxed}
	\textbf{\textit{Комплексное число}} — число вида $z = a+bi \in \CC$, где $i$ — комплексная единица: $i^2 = -1$, а $a, b \in \R$ такая форма записи называется \textbf{\textit{алгебраической}}.\\
	\textbf{\textit{Вещественной частью}} $z$ называется число $\text{Re}(z) = a$\\
	\textbf{\textit{Мнимой частью}} $z$ называется число $\text{Im}(z) = b$
\end{definition_boxed}

\begin{example}
$1 + 2i \in \CC$
\end{example}


\begin{definition_boxed}
	Сложение и вычитание комплексных чисел определяется так:
	\[(a + bi) \pm (c + di) = (a \pm c) + (b \pm d)i\]
\end{definition_boxed}

\begin{definition_boxed}
	Умножение комплексных чисел определяется так:
	\[(a + bi) \cdot (c + di) = (ac - bd) + (ad + bc)i\]
\end{definition_boxed}

\begin{definition_boxed}
	\textbf{\textit{Сопряженное}} комплексное число к $z = a + bi$ это $\overline{z} = a - bi$
\end{definition_boxed}

\begin{example}
	$z + \overline{z} \in \R$
\end{example}

\begin{example}
	$z \cdot \overline{z} \in \R$
\end{example}

\begin{task_boxed}
	Пусть $x_0$, $x_1$, $x_2\in \mathbb{C}$, $f(x)$ --- многочлен с вещественными коэффициентами.
	Докажите, что
	\begin{enumerate}
		\item $\overline{x_1+x_2}=\overline{x_1}+\overline{x_2}$,
		\item $\overline{x_1\cdot x_2}=\overline{x_1}\cdot \overline{x_2}$,
		\item $\overline{x_1/x_2}=\overline{x_1}/\overline{x_2}$,
		\item $\overline{f(x_0)}=f(\overline{x_0})$.
		\item Докажите, что если $x_0$ --- корень многочлена $f(x)$.
		Докажите, что и $\overline{x_0}$ корень $f(x)$.
	\end{enumerate}
\end{task_boxed}

\begin{task_boxed}
	Докажите, что если есть комплексное число $t$, такое, что $t + z \in \R$, $t \cdot z \in \R$, и $\text{Im}(z) \neq 0$. Тогда $t = \overline{z}$
\end{task_boxed}

\begin{definition_boxed}
	\textbf{\textit{Модуль}} комплексного числа к $z = a + bi$ является $|z| = \sqrt{a^2 + b^2}$
\end{definition_boxed}

\begin{example}
	$z \cdot \overline{z} = |z|^2$
\end{example}

\begin{task_boxed}
	Докажите, что $|z_1|\cdot|z_2| = |z_1\cdot z_2|$
\end{task_boxed}

\begin{definition_boxed}
	Обратное к комплексному числу $z = a + bi$ является $\dfrac{1}{z} = \dfrac{\overline{z}}{|z|^2}$
\end{definition_boxed}

\begin{task_boxed}
	Выведите формулу деления комплексного числа $a + bi$ на $c + di$
\end{task_boxed}

\begin{task_boxed}
	Найдите $\dfrac{1+4i}{2-3i} + 3i - 4$
\end{task_boxed}

\begin{theorem_boxed}[ Эйлера]
	\[e^{i\theta} = \cos(\theta) + i\sin(\theta)\]
\end{theorem_boxed}

\begin{definition_boxed}
	\textbf{\textit{Тригонометрической формой}} записи комплексного числа $z = a + bi$ является запись $z = r(\cos(\varphi) + i \sin(\varphi)) = re^{i\varphi}$\\
	$\arg(z) = \varphi$~--- \textbf{\textit{аргумент}} комплексного числа $z$
\end{definition_boxed}

\begin{task_boxed}
Найдите тригонометрическую форму для $z=10+10\sqrt{3}i$
\end{task_boxed}

\begin{task_boxed}
	Найти множество точек комплексной плоскости, удовлетворяющих: $|z-i| = 1$
\end{task_boxed}

\begin{task_boxed}
	Найти на плоскости множество решений, удовлетворяющих данному условию: $1 < |z + 3 + i| < 3$
\end{task_boxed}

\begin{task_boxed}
	Докажите, что $\arg(z_1 \cdot z_2) = \arg(z_1) + \arg(z_2)$
\end{task_boxed}

\begin{task_boxed}
	Найдите формулу умножения комплексных чисел в тригонометрической форме
\end{task_boxed}

\begin{task_boxed}
	Найдите формулу деления комплексных чисел в тригонометрической форме
\end{task_boxed}

\begin{task_boxed}
	Докажите, что \textbf{возведение в целую степень} комплексного числа $z = r(\cos(\varphi) + i \sin(\varphi)) = re^{i\varphi}$ работает так:

	\[z^n = r^n e^{in\varphi} = r^n(\cos(n\varphi) + i \sin(n\varphi))\]
\end{task_boxed}

\begin{task_boxed}
Найти $(1 + \sqrt{3}i)^{9}$
\end{task_boxed}

\begin{task_boxed}
Найти $(1 + i)^{1000}$
\end{task_boxed}

\begin{definition_boxed}
	\textbf{Взятие корня} $n$-той степени из комплексного числа $z = r(\cos(\varphi) + i \sin(\varphi)) = re^{i\varphi}$\\
	работает так:
	\[\sqrt[n]{z} = \sqrt[n]{r} e^{i\frac{\varphi + 2\pi k}{n}} = r^n\lr*{\cos\lr*{\frac{\varphi + 2\pi k}{n}} + i \sin\lr*{\frac{\varphi + 2\pi k}{n}}} \forall k \in \{1,2,\dotsc, n\}\]
\end{definition_boxed}

\begin{task_boxed}
	Извлеките корень из числа $z=3+4i$
\end{task_boxed}

\begin{task_boxed}
	Решите уравнение $z^3=-1$
\end{task_boxed}

\begin{task_boxed}
	Вычислите корни третьей степени из комплексного числа $2+2i$
\end{task_boxed}

\begin{task_boxed}
	Решите в комплексных числах следующие квадратные уравнения:

	а)  $z^2 + z + 1 = 0$

	б)  $z^2 + 4z + 29 = 0$

\end{task_boxed}

\begin{task_boxed}
	Известно, что  $z + z^{-1} = 2\cos{\alpha}$.\\
	Докажите, что  $z^n + z^{-n} = 2\cos{(\alpha \cdot n)}$.
\end{task_boxed}

\begin{task_boxed}
	Пусть $P(x^n)$ делится на  $x - 1$.
	Докажите, что $P(x^n)$ делится на  $x^n - 1$.
\end{task_boxed}

\begin{task_boxed}
	Вычислите

	а) $C_{100}^0 - C_{100}^2 + C_{100}^4 \dots + C_{100}^{100} $

	б) $C_{100}^1 - C_{100}^3 + C_{100}^5 \dots - C_{100}^{99} $
\end{task_boxed}
\begin{task_boxed}
Докажите, что многочлен  $P(x) = (\cos{\varphi} + x\sin{\varphi})^n - \cos{(\varphi\cdot n)} - x\sin{(\varphi\cdot n)}$  делится на  $x^2 + 1$.
\end{task_boxed}

%\begin{task_boxed}
%	На доске написаны три функции:  $$f_1(x) = x + 1/x,   f_2(x) = x^2,   f_3(x) = (x - 1)^2.$$\\
%	Можно складывать, вычитать и перемножать эти функции, умножать их на произвольное число, прибавлять к ним произвольное число, а также проделывать эти операции с полученными выражениями.
%	
%	a) Получите таким образом функцию $1/x$.\\
%	Докажите, что если стереть с доски
%
%	b) функцию  $f_1$
%	
%	c) функцию $f_3$\\
%	то получить $1/x$ невозможно.
%\end{task_boxed}

\begin{task_boxed}
	Докажите, что для произвольных комплексных чисел $z$ и $w$ выполняется равенство  $|z + w|^2 + | z - w|^2 = 2(|z|^2 + |w|^2)$.
	Какой геометрический смысл оно имеет?
\end{task_boxed}

\begin{task_boxed}
	 Докажите данное равенство: $\sin{\dfrac{2\pi}{n}} + \sin{\dfrac{4\pi}{n}} \dots +\sin{\dfrac{2(n-1)\pi}{n}} = 0$
\end{task_boxed}

\begin{task_boxed}
При каких вещественных $p$ и $q$ двучлен  $x^4 + 1$  делится на  $x^2 + px + q$?%разложить левую часть на два многочлена
\end{task_boxed}





\end{document}