\documentclass{article}
\usepackage[12pt]{extsizes}
\usepackage[T2A]{fontenc}
\usepackage[utf8]{inputenc}
\usepackage[english, russian]{babel}

\usepackage{amssymb}
\usepackage{amsfonts}
\usepackage{amsmath}
\usepackage{enumitem}
\usepackage{graphics}

\usepackage{lipsum}



\usepackage{geometry} % Меняем поля страницы
\geometry{left=1cm}% левое поле
\geometry{right=1cm}% правое поле
\geometry{top=1.5cm}% верхнее поле
\geometry{bottom=1cm}% нижнее поле


\usepackage{fancyhdr} % Headers and footers
\pagestyle{fancy} % All pages have headers and footers
\fancyhead{} % Blank out the default header
\fancyfoot{} % Blank out the default footer
\fancyhead[L]{ЦРОД $\bullet$ Математика}
\fancyhead[C]{\textit{Геометрия}}
\fancyhead[R]{Май 2022}% Custom header text


%----------------------------------------------------------------------------------------

%\begin{document}\normalsize
\begin{document}\large


\begin{center}
\textbf{Геома для Стажёров}
\end{center}


\begin{enumerate}[label*=\protect\fbox{\arabic{enumi}}]

\item Внутри параллелограмма $ABCD$ выбрана точка $P$ так, что $\angle AP B+\angle CP D = 180^\circ$. Докажите, что $\angle PBC = \angle PDC$.

\item Внутри выпуклого четырёхугольника $ABCD$ расположены четыре окружности одного радиуса так, что они имеют общую точку и каждая из них вписана в один из углов четырёхугольника. Докажите, что четырёхугольник $ABCD$ вписанный.
\item Окружность $\omega$ описана около остроугольного треугольника $ABC$. На стороне $AB$ выбрана точка $D$, а на стороне $BC$ — точка $E$ так, что $AC \parallel DE$. Точки $P$ и $Q$ на меньшей дуге $AC$ окружности $\omega$ таковы, что $DP \parallel EQ$. Лучи $QA$ и $PC$ пересекают прямую $DE$ в точках $X$ и $Y$ соответственно. Докажите, что $\angle XBY + \angle P BQ = 180^\circ$.
\item В окружности $\omega$ с центром в точке $O$ провели непересекающиеся хорды $AB$ и $CD$ так, что $\angle AOB = \angle COD = 120^\circ$. Касательная к $\omega$ в точке $A$ пересекает луч $CD$ в точке $X$, а касательная к $\omega$ в точке $B$ пересекает луч $DC$ в точке $Y$. Прямая $l$ проходит через центры окружностей, описанных около треугольников $DOX$ и $COY$ . Докажите, что $l$ касается $\omega$.
\item Дан остроугольный треугольник $ABC$, в котором $AB < AC$. Пусть $M$ и $N$ — середины сторон $AB$ и $AC$ соответственно, а $D$ — основание высоты, проведенной из $A$. На отрезке $MN$ нашлась точка $K$ такая, что $BK = CK$. Луч $KD$ пересекает окружность $\omega$, описанную около треугольника $ABC$, в точке $Q$. Докажите, что точки $C, N, K$ и $Q$ лежат на одной окружности.
\item Треугольник $ABC$, в котором $AB > AC$, вписан в окружность с центром в точке $O$. В нём проведены высоты $AA_0$ и $BB_0$, и $BB_0$ повторно пересекает описанную окружность в точке $N$. Пусть $M$ — середина отрезка $AB$. Докажите, что если $\angle OBN = \angle NBC$, то прямые $AA_0$, $ON$ и $MB_0$ пересекаются в одной точке.
\item На сторонах $AB$ и $AC$ треугольника $ABC$ выбраны точки $P$ и $Q$ соответственно так, что $PQ \parallel BC$. Отрезки $BQ$ и $CP$ пересекаются в точке $O$. Точка $A'$ симметрична точке $A$ относительно прямой $BC$. Отрезок $A'O$ пересекает окружность $\omega$, описанную около треугольника $APQ$, в точке $S$. Докажите, что окружность, описанная около треугольника $BSC$, касается окружности $\omega$.


\end{enumerate}
\end{document}