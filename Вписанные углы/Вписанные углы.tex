\documentclass{article}

\usepackage[12pt]{extsizes}
\usepackage[T2A]{fontenc}
\usepackage[utf8]{inputenc}
\usepackage[english, russian]{babel}

\usepackage{mathrsfs}
\usepackage[dvipsnames]{xcolor}

\usepackage{amsmath}
\usepackage{amssymb}
\usepackage{amsthm}
\usepackage{indentfirst}
\usepackage{amsfonts}
\usepackage{enumitem}
\usepackage{graphics}
\usepackage{tikz}
\usepackage{tabu}
\usepackage{diagbox}
\usepackage{hyperref}
\usepackage{mathtools}
\usepackage{ucs}
\usepackage{lipsum}
\usepackage{geometry} % Меняем поля страницы
\usepackage{fancyhdr} % Headers and footers
\newcommand{\range}{\mathrm{range}}
\newcommand{\dom}{\mathrm{dom}}
\newcommand{\N}{\mathbb{N}}
\newcommand{\R}{\mathbb{R}}
\newcommand{\E}{\mathbb{E}}
\newcommand{\D}{\mathbb{D}}
\newcommand{\M}{\mathcal{M}}
\newcommand{\Prime}{\mathbb{P}}
\newcommand{\A}{\mathbb{A}}
\newcommand{\Q}{\mathbb{Q}}
\newcommand{\Z}{\mathbb{Z}}
\newcommand{\F}{\mathbb{F}}
\newcommand{\CC}{\mathbb{C}}

\DeclarePairedDelimiter\abs{\lvert}{\rvert}
\DeclarePairedDelimiter\floor{\lfloor}{\rfloor}
\DeclarePairedDelimiter\ceil{\lceil}{\rceil}
\DeclarePairedDelimiter\lr{(}{)}
\DeclarePairedDelimiter\set{\{}{\}}
\DeclarePairedDelimiter\norm{\|}{\|}

\renewcommand{\labelenumi}{(\alph{enumi})}

\newcommand{\smallindent}{
    \geometry{left=1cm}% левое поле
    \geometry{right=1cm}% правое поле
    \geometry{top=1.5cm}% верхнее поле
    \geometry{bottom=1cm}% нижнее поле
}

\newcommand{\header}[3]{
    \pagestyle{fancy} % All pages have headers and footers
    \fancyhead{} % Blank out the default header
    \fancyfoot{} % Blank out the default footer
    \fancyhead[L]{#1}
    \fancyhead[C]{#2}
    \fancyhead[R]{#3}
}

\newcommand{\dividedinto}{
    \,\,\,\vdots\,\,\,
}

\newcommand{\littletaller}{\mathchoice{\vphantom{\big|}}{}{}{}}

\newcommand\restr[2]{{
    \left.\kern-\nulldelimiterspace % automatically resize the bar with \right
    #1 % the function
    \littletaller % pretend it's a little taller at normal size
    \right|_{#2} % this is the delimiter
}}

\DeclareGraphicsExtensions{.pdf,.png,.jpg}

\newenvironment{enumerate_boxed}[1][enumi]{\begin{enumerate}[label*=\protect\fbox{\arabic{#1}}]}{\end{enumerate}}



\smallindent

\header{ЦРОД $\bullet$ Математика}{\textit{Геометрия}}{Май 2022}

%----------------------------------------------------------------------------------------

%\begin{document}\normalsize
\begin{document}
    \large


    \begin{center}
        \textbf{Геома для Стажёров}
    \end{center}


    \begin{enumerate_boxed}

        \item Внутри параллелограмма $ABCD$ выбрана точка $P$ так, что $\angle AP B+\angle CP D = 180^\circ$.
        Докажите, что $\angle PBC = \angle PDC$.

        \item Внутри выпуклого четырёхугольника $ABCD$ расположены четыре окружности одного радиуса так, что они имеют общую точку и каждая из них вписана в один из углов четырёхугольника.
        Докажите, что четырёхугольник $ABCD$ вписанный.

        \item Окружность $\omega$ описана около остроугольного треугольника $ABC$.
        На стороне $AB$ выбрана точка $D$, а на стороне $BC$ — точка $E$ так, что $AC \parallel DE$.
        Точки $P$ и $Q$ на меньшей дуге $AC$ окружности $\omega$ таковы, что $DP \parallel EQ$.
        Лучи $QA$ и $PC$ пересекают прямую $DE$ в точках $X$ и $Y$ соответственно.
        Докажите, что $\angle XBY + \angle P BQ = 180^\circ$.

        \item В окружности $\omega$ с центром в точке $O$ провели непересекающиеся хорды $AB$ и $CD$ так, что $\angle AOB = \angle COD = 120^\circ$.
        Касательная к $\omega$ в точке $A$ пересекает луч $CD$ в точке $X$, а касательная к $\omega$ в точке $B$ пересекает луч $DC$ в точке $Y$.
        Прямая $l$ проходит через центры окружностей, описанных около треугольников $DOX$ и $COY$.
        Докажите, что $l$ касается $\omega$.

        \item Дан остроугольный треугольник $ABC$, в котором $AB < AC$.
        Пусть $M$ и $N$ — середины сторон $AB$ и $AC$ соответственно, а $D$ — основание высоты, проведенной из $A$.
        На отрезке $MN$ нашлась точка $K$ такая, что $BK = CK$.
        Луч $KD$ пересекает окружность $\omega$, описанную около треугольника $ABC$, в точке $Q$.
        Докажите, что точки $C, N, K$ и $Q$ лежат на одной окружности.

        \item Треугольник $ABC$, в котором $AB > AC$, вписан в окружность с центром в точке $O$.
        В нём проведены высоты $AA_0$ и $BB_0$, и $BB_0$ повторно пересекает описанную окружность в точке $N$.
        Пусть $M$ — середина отрезка $AB$.
        Докажите, что если $\angle OBN = \angle NBC$, то прямые $AA_0$, $ON$ и $MB_0$ пересекаются в одной точке.

        \item На сторонах $AB$ и $AC$ треугольника $ABC$ выбраны точки $P$ и $Q$ соответственно так, что $PQ \parallel BC$.
        Отрезки $BQ$ и $CP$ пересекаются в точке $O$.
        Точка $A'$ симметрична точке $A$ относительно прямой $BC$.
        Отрезок $A'O$ пересекает окружность $\omega$, описанную около треугольника $APQ$, в точке $S$.
        Докажите, что окружность, описанная около треугольника $BSC$, касается окружности $\omega$.


    \end{enumerate_boxed}
\end{document}