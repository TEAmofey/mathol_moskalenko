\documentclass{article}

\usepackage[12pt]{extsizes}
\usepackage[T2A]{fontenc}
\usepackage[utf8]{inputenc}
\usepackage[english, russian]{babel}

\usepackage{mathrsfs}
\usepackage[dvipsnames]{xcolor}

\usepackage{amsmath}
\usepackage{amssymb}
\usepackage{amsthm}
\usepackage{indentfirst}
\usepackage{amsfonts}
\usepackage{enumitem}
\usepackage{graphics}
\usepackage{tikz}
\usepackage{tabu}
\usepackage{diagbox}
\usepackage{hyperref}
\usepackage{mathtools}
\usepackage{ucs}
\usepackage{lipsum}
\usepackage{geometry} % Меняем поля страницы
\usepackage{fancyhdr} % Headers and footers
\newcommand{\range}{\mathrm{range}}
\newcommand{\dom}{\mathrm{dom}}
\newcommand{\N}{\mathbb{N}}
\newcommand{\R}{\mathbb{R}}
\newcommand{\E}{\mathbb{E}}
\newcommand{\D}{\mathbb{D}}
\newcommand{\M}{\mathcal{M}}
\newcommand{\Prime}{\mathbb{P}}
\newcommand{\A}{\mathbb{A}}
\newcommand{\Q}{\mathbb{Q}}
\newcommand{\Z}{\mathbb{Z}}
\newcommand{\F}{\mathbb{F}}
\newcommand{\CC}{\mathbb{C}}

\DeclarePairedDelimiter\abs{\lvert}{\rvert}
\DeclarePairedDelimiter\floor{\lfloor}{\rfloor}
\DeclarePairedDelimiter\ceil{\lceil}{\rceil}
\DeclarePairedDelimiter\lr{(}{)}
\DeclarePairedDelimiter\set{\{}{\}}
\DeclarePairedDelimiter\norm{\|}{\|}

\renewcommand{\labelenumi}{(\alph{enumi})}

\newcommand{\smallindent}{
    \geometry{left=1cm}% левое поле
    \geometry{right=1cm}% правое поле
    \geometry{top=1.5cm}% верхнее поле
    \geometry{bottom=1cm}% нижнее поле
}

\newcommand{\header}[3]{
    \pagestyle{fancy} % All pages have headers and footers
    \fancyhead{} % Blank out the default header
    \fancyfoot{} % Blank out the default footer
    \fancyhead[L]{#1}
    \fancyhead[C]{#2}
    \fancyhead[R]{#3}
}

\newcommand{\dividedinto}{
    \,\,\,\vdots\,\,\,
}

\newcommand{\littletaller}{\mathchoice{\vphantom{\big|}}{}{}{}}

\newcommand\restr[2]{{
    \left.\kern-\nulldelimiterspace % automatically resize the bar with \right
    #1 % the function
    \littletaller % pretend it's a little taller at normal size
    \right|_{#2} % this is the delimiter
}}

\DeclareGraphicsExtensions{.pdf,.png,.jpg}

\newenvironment{enumerate_boxed}[1][enumi]{\begin{enumerate}[label*=\protect\fbox{\arabic{#1}}]}{\end{enumerate}}



\smallindent

\header{Математика}{\textit{Геометрия}}{}

%----------------------------------------------------------------------------------------

\begin{document}
    \large

    \begin{center}
        \textbf{Вписанные углы}
    \end{center}

    \begin{enumerate_boxed}

        \item Даны два угла $\angle ABC = 90^\circ$ и $\angle ADC = 90^\circ$.
        Докажите, что $A, B, C, D$ лежат на одной окружности.

        \item Дан треугольник $ABC$. $I$ - центр вписанной окружности.
        Докажите (и запомните), что $\angle AIB = 90^\circ + \frac{\angle C}{2}$

        \item Дан треугольник $ABC$. $H$ - ортоцентр (точка пересечения высот).
        Докажите (и запомните), что $\angle AHB = 180^\circ - \angle C$

        \item Дан треугольник $ABC$. $BH_1$, $CH_2$~--- высоты треугольника.
        Докажите, что $C,B,H_1, H_2$ лежат на одной окружности.

        \item В условии предыдущей задачи пусть $H = BH_1 \cap CH_2$.
        Докажите, что $A,H,H_1, H_2$ лежат на одной окружности.

        \item Рассмотрим вписанный четырёхугольник $ABCD$.
        Пусть дуга $\breve{AB} = \alpha$, дуга $\breve{CD} = \beta$. $O$ - точка пересечения диагоналей.
        Докажите, что $\angle AOB = \frac{\alpha +  \beta}{2}$.

        \item Дана точка $O$ и окружность $\omega$, так что $O \notin \omega$.
        Через $O$ провели 2 прямые, которые пересекают $\omega$ в точках $A, B$ и $C, D$.
        Докажите, что $OA \cdot OB = OC \cdot OD$

        \item На гипотенузе $AB$ прямоугольного треугольника $ABC$ во внешнюю сторону треугольника построен квадрат с центром в точке $O$.
        Докажите, что $CO$ — биссектриса угла $ACB$.

        \item В остроугольном треугольнике $ABC$ на высоте, проведённой из вершины $C$, выбрана точка $X$.
        Пусть $A_1$ и $B_1$ — основания перпендикуляров из точки $X$ на стороны $AC$ и $BC$ соответственно.
        Докажите, что точки $A$, $B$, $B_1$, $A_1$ лежат на одной окружности.

        \item Пусть $AA_1$, $BB_1$, $CC_1$ — высоты остроугольного треугольника $ABC$.
        Докажите, что основания перпендикуляров из точки $A_1$ на прямые $AB$, $AC$, $BB_1$, $CC_1$ лежат на одной прямой.

        \item Дан выпуклый шестиугольник $ABCDEF$.
        Известно, что $\angle FAE = \angle BDC$, а четырёхугольники $ABDF$ и $ACDE$ являются вписанными.
        Докажите, что прямые $BF$ и $CE$ параллельны.

        \item Дан остроугольный треугольник $ABC$, в котором $AB < AC$.
        Пусть $M$ и $N$ — середины сторон $AB$ и $AC$ соответственно, а $D$ — основание высоты, проведённой из $A$.
        На отрезке $MN$ нашлась точка K такая, что $BK = CK$.
        Луч $KD$ пересекает окружность Ω, описанную около треугольника $ABC$, в точке $Q$.
        Докажите, что точки $C, N, K$ и $Q$ лежат на одной окружности.

        \item Даны две окружности, пересекающиеся в точках $X$ и $Y$.
        Прямая, проходящая через $X$, пересекает первую окружность в точке $A$, а вторую — в точке $C$.
        Другая прямая, проходящая через $Y$, первую окружность пересекает в точке $B$, а вторую — в точке $D$.
        Докажите, что $AB \parallel CD$.

        \item В окружность вписан шестиугольник.
        Найдите сумму углов при трёх его несоседних вершинах.

        \item Окружности с центрами $O_1$ и $O_2$ пересекаются в точках $A$ и $B$.
        Луч $O_{2}A$ пересекает первую окружность в точке $C$.
        Докажите, что точки $O_1$, $O_2$, $B$, $C$ лежат на одной окружности.

        \item Докажите, что в равнобедренной трапеции вершины боковой стороны, точка пересечения диагоналей и центр описанной окружности лежат на одной окружности.

        \item Дан выпуклый четырёхугольник $ABCD$ Рассмотрим точки пересечения биссектрис его углов $A$ и $B$, $B$ и $C$, $C$ и $D$, $D$ и $A$.
        Докажите, что эти четыре точки являются вершинами вписанного четырёхугольника.

        \item На хорде $AB$ окружности с центром в точке $O$ выбрана точка $C$.
        Описанная окружность треугольника $AOC$ пересекает исходную окружность в точке $D$.
        Докажите, что $BC=CD$.

        \item Про выпуклый четырёхугольник $ABCD$ известно, что $AB=BC=CD$.
        Диагонали четырёхугольника пересекаются в точке $M$, $K$ — точка пересечения биссектрис углов $A$ и $D$.
        Докажите, что точки $A$, $M$, $K$, $D$ лежат на одной окружности.

        \item Пусть дан треугольник $ABC$, и в точке $B$ построена касательная к описанной окружности треугольника $ABC$.
        Рассмотрим произвольную прямую, параллельную этой касательной, и отметим точки $D$ и $E$ пересечения с прямыми $AB$ и $BC$ соответственно.
        Докажите, что четыре точки $A$, $C$, $D$, $E$ лежат на одной окружности.

        \item Диагонали вписанного четырёхугольника $ABCD$ пересекаются в точке $O$.
        Докажите, что прямая, соединяющая середины дуг $AB$ и $CD$, параллельна биссектрисе угла $AOB$.

        \item Четырёхугольник $ABCD$ таков, что в него можно вписать и около него можно описать окружности.
        Диаметр описанной окружности совпадает с диагональю $AC$.
        Докажите, что модули разностей длин его противоположных сторон равны.

        \item Пусть $I$ — центр вписанной окружности остроугольного треугольника $ABC$, $M$ и $N$ — точки касания вписанной окружности сторон $AB$ и $BC$ соответственно.
        Через точку $I$ проведена прямая $\l$, параллельная стороне $AC$, и на неё опущены перпендикуляры $AP$ и $CQ$.
        Докажите, что точки $M$, $N$, $P$ и $Q$ лежат
        на одной окружности.

        \item Внутри параллелограмма $ABCD$ выбрана точка $P$ так, что $\angle AP B+\angle CP D = 180^\circ$.
        Докажите, что $\angle PBC = \angle PDC$.

        \item Внутри выпуклого четырёхугольника $ABCD$ расположены четыре окружности одного радиуса так, что они имеют общую точку и каждая из них вписана в один из углов четырёхугольника.
        Докажите, что четырёхугольник $ABCD$ вписанный.

        \item Окружность $\omega$ описана около остроугольного треугольника $ABC$.
        На стороне $AB$ выбрана точка $D$, а на стороне $BC$ — точка $E$ так, что $AC \parallel DE$.
        Точки $P$ и $Q$ на меньшей дуге $AC$ окружности $\omega$ таковы, что $DP \parallel EQ$.
        Лучи $QA$ и $PC$ пересекают прямую $DE$ в точках $X$ и $Y$ соответственно.
        Докажите, что $\angle XBY + \angle P BQ = 180^\circ$.

        \item В окружности $\omega$ с центром в точке $O$ провели непересекающиеся хорды $AB$ и $CD$ так, что $\angle AOB = \angle COD = 120^\circ$.
        Касательная к $\omega$ в точке $A$ пересекает луч $CD$ в точке $X$, а касательная к $\omega$ в точке $B$ пересекает луч $DC$ в точке $Y$.
        Прямая $l$ проходит через центры окружностей, описанных около треугольников $DOX$ и $COY$.
        Докажите, что $l$ касается $\omega$.

        \item Дан остроугольный треугольник $ABC$, в котором $AB < AC$.
        Пусть $M$ и $N$ — середины сторон $AB$ и $AC$ соответственно, а $D$ — основание высоты, проведенной из $A$.
        На отрезке $MN$ нашлась точка $K$ такая, что $BK = CK$.
        Луч $KD$ пересекает окружность $\omega$, описанную около треугольника $ABC$, в точке $Q$.
        Докажите, что точки $C, N, K$ и $Q$ лежат на одной окружности.

        \item Треугольник $ABC$, в котором $AB > AC$, вписан в окружность с центром в точке $O$.
        В нём проведены высоты $AA_0$ и $BB_0$, и $BB_0$ повторно пересекает описанную окружность в точке $N$.
        Пусть $M$ — середина отрезка $AB$.
        Докажите, что если $\angle OBN = \angle NBC$, то прямые $AA_0$, $ON$ и $MB_0$ пересекаются в одной точке.

        \item На сторонах $AB$ и $AC$ треугольника $ABC$ выбраны точки $P$ и $Q$ соответственно так, что $PQ \parallel BC$.
        Отрезки $BQ$ и $CP$ пересекаются в точке $O$.
        Точка $A'$ симметрична точке $A$ относительно прямой $BC$.
        Отрезок $A'O$ пересекает окружность $\omega$, описанную около треугольника $APQ$, в точке $S$.
        Докажите, что окружность, описанная около треугольника $BSC$, касается окружности $\omega$.

        \item Точка $H$ является ортоцентром остроугольного треугольника $ABC$ ($AB> AC$). Точка $E$ симметрична $C$ относительно высоты $AH$.
        Обозначим за $F$ точку пересечения прямых $EH$ и $AC$.
        Докажите, что центр описанной окружности треугольника $AEF$ лежит на прямой $AB$.

        \item В остроугольном треугольнике угол $A$ равен $60^\circ$.
        Докажите, что прямая, соединяющая центр описанной окружности с ортоцентром, отсекает от треугольника равносторонний треугольник.

        \item Пусть $H'$ — проекция ортоцентра на касательную в точке $A$ к описанной окружности треугольника $ABC$.
        Докажите, что середина стороны $BC$ равноудалена от точек $A$ и $H'$.

        \item Остроугольный треугольник $ABC$ ($AB < AC$) вписан в окружность $\Omega$.
        Пусть $M$ — точка пересечения его медиан, а $AH$ — высота этого треугольника.
        Луч $MH$ пересекает $\Omega$ в точке $A'$.
        Докажите, что окружность, описанная около треугольника $A'HB$,
        касается $AB$.

        \item Окружность $\omega$ касается сторон угла $BAC$ в точках $B$ и $C$.
        Прямая $l$ пересекает отрезки $AB$ и $AC$ в точках $K$ и $L$ соответственно.
        Окружность $\omega$ пересекает $l$ в точках $P$ и $Q$.
        Точки $S$ и $T$ выбраны на отрезке $BC$ так, что $KS \parallel AC$ и $LT \parallel AB$.
        Докажите, что точки $P, Q, S$ и $T$ лежат на одной окружности


    \end{enumerate_boxed}
\end{document}