\documentclass{article}

\usepackage[12pt]{extsizes}
\usepackage[T2A]{fontenc}
\usepackage[utf8]{inputenc}
\usepackage[english, russian]{babel}

\usepackage{mathrsfs}
\usepackage[dvipsnames]{xcolor}

\usepackage{amsmath}
\usepackage{amssymb}
\usepackage{amsthm}
\usepackage{indentfirst}
\usepackage{amsfonts}
\usepackage{enumitem}
\usepackage{graphics}
\usepackage{tikz}
\usepackage{tabu}
\usepackage{diagbox}
\usepackage{hyperref}
\usepackage{mathtools}
\usepackage{ucs}
\usepackage{lipsum}
\usepackage{geometry} % Меняем поля страницы
\usepackage{fancyhdr} % Headers and footers
\newcommand{\range}{\mathrm{range}}
\newcommand{\dom}{\mathrm{dom}}
\newcommand{\N}{\mathbb{N}}
\newcommand{\R}{\mathbb{R}}
\newcommand{\E}{\mathbb{E}}
\newcommand{\D}{\mathbb{D}}
\newcommand{\M}{\mathcal{M}}
\newcommand{\Prime}{\mathbb{P}}
\newcommand{\A}{\mathbb{A}}
\newcommand{\Q}{\mathbb{Q}}
\newcommand{\Z}{\mathbb{Z}}
\newcommand{\F}{\mathbb{F}}
\newcommand{\CC}{\mathbb{C}}

\DeclarePairedDelimiter\abs{\lvert}{\rvert}
\DeclarePairedDelimiter\floor{\lfloor}{\rfloor}
\DeclarePairedDelimiter\ceil{\lceil}{\rceil}
\DeclarePairedDelimiter\lr{(}{)}
\DeclarePairedDelimiter\set{\{}{\}}
\DeclarePairedDelimiter\norm{\|}{\|}

\renewcommand{\labelenumi}{(\alph{enumi})}

\newcommand{\smallindent}{
    \geometry{left=1cm}% левое поле
    \geometry{right=1cm}% правое поле
    \geometry{top=1.5cm}% верхнее поле
    \geometry{bottom=1cm}% нижнее поле
}

\newcommand{\header}[3]{
    \pagestyle{fancy} % All pages have headers and footers
    \fancyhead{} % Blank out the default header
    \fancyfoot{} % Blank out the default footer
    \fancyhead[L]{#1}
    \fancyhead[C]{#2}
    \fancyhead[R]{#3}
}

\newcommand{\dividedinto}{
    \,\,\,\vdots\,\,\,
}

\newcommand{\littletaller}{\mathchoice{\vphantom{\big|}}{}{}{}}

\newcommand\restr[2]{{
    \left.\kern-\nulldelimiterspace % automatically resize the bar with \right
    #1 % the function
    \littletaller % pretend it's a little taller at normal size
    \right|_{#2} % this is the delimiter
}}

\DeclareGraphicsExtensions{.pdf,.png,.jpg}

\newenvironment{enumerate_boxed}[1][enumi]{\begin{enumerate}[label*=\protect\fbox{\arabic{#1}}]}{\end{enumerate}}



\smallindent

\header{Математика}{\textit{Алгебра}}{7 апреля 2023}

%----------------------------------------------------------------------------------------

\begin{document}
    \large


    \begin{center}
        \textbf{Неравенство КБШ}
    \end{center}

    Теперь докажем \textbf{неравенство Коши-Буняковского-Шварца (КБШ)}: для двух произвольных наборов вещественных чисел $a_1, a_2, \ldots, a_n$ и $b_1, b_2, \ldots, b_n$ выполнено неравенство
    \[
        (a_1^2+a_2^2+\ldots+a_n^2)(b_1^2+b_2^2+\ldots+b_n^2) \geqslant (a_{1}b_1+a_{2}b_2+\ldots+a_{n}b_n)^2.
    \]

    Для этого рассмотрим следующий вспомогательный квадратный трехчлен: $(a_1^2+a_2^2+\ldots+a_n^2)x^2+2\cdot(a_{1}b_1+\ldots+a_{n}b_n)x+(b_1^2+b_2^2+\ldots+b_n^2)$.
    \textbf{Контрольный вопрос.} Когда в неравенстве КБШ достигается равенство?

    \begin{enumerate_boxed}

        \item Пусть $a_1$, $a_2$, \dots, $a_n$ и $b_1$, $b_2$, \dots, $b_n$ ---
        положительные числа.
        Докажите неравенство
        \[ \left(a_1 b_1 +\ldots + a_n b_n \right) \left( \frac{a_1}{b_1} + \ldots + \frac{a_n}{b_n} \right) \geqslant
        \left( a_1+\ldots+a_n \right)^2 .\]

        \item \textbf{Важнейшая форма КБШ.} Докажите через КБШ, выбрав два нужных набора, КБШ для дробей: при \textit{положительных} $a_1, a_2, \ldots, a_n$ и $b_1, b_2, \ldots, b_n$ выполнено неравенство
        \[
            \frac{a_1^2}{b_1}+\frac{a_2^2}{b_2}+\ldots+\frac{a_n^2}{b_n} \geqslant \frac{(a_1+a_2+\ldots+a_n)^2}{b_1+b_2+\ldots+b_n}.
        \]

        \item Суммы двух наборов положительных чисел $a_1, a_2, \ldots, a_n$ и $b_1, b_2, \ldots, b_n$ равны.
        Докажите неравенство
        \[
            \frac{a_1^2}{a_1+b_1}+\frac{a_2^2}{a_2+b_2}+\ldots+\frac{a_n^2}{a_n+b_n} \geqslant \frac{a_1+a_2+\ldots+a_n}{2}.
        \]

        \item Для положительных $a$, $b$, $c$, $d$ докажите неравенство
        \[
            \frac{a^2}{b(a+c)} + \frac{b^2}{c(b+d)} +
            \frac{c^2}{d(a+c)} + \frac{d^2}{a(d+b)} \geqslant 2.
        \]

        \item Докажите, что при всех положительных $a$, $b$, $c$, $d$ выполнено
        \[
            \frac{a}{b+2c+3d} + \frac{b}{c+2d+3a} + \frac{c}{d+2a+3b} +
            \frac{d}{a+2b+3c} \geqslant \frac{2}{3}.
        \]

        %\textit{Указание.} КБШ для дробей хорошо применять, когда в числителях дробей стоят квадраты. А как же их там получить?

        \item Для положительных $a$, $b$, $c$, удовлетворяющих условию $abc=1$, докажите неравенство
        \[
            \frac{1}{a^3(b+c)}+\frac{1}{b^3(c+a)}+\frac{1}{c^3(a+b)} \geqslant \frac{3}{2}.
        \]
        %вроде не делается срарзу, но надо разделить первую дробь на $a^2$ и т.д.

        \item Даны положительные числа  $a,b,c$, сумма которых не меньше двух.
        Докажите неравенство
        \[\frac{a}{b\sqrt [3] {c}+a}+\frac{b}{c\sqrt [3] {a}+b}+\frac{c}{a\sqrt [3] {b}+c}\leqslant 2.\]

    \end{enumerate_boxed}
\end{document}