\documentclass{article}
\usepackage[12pt]{extsizes}
\usepackage[T2A]{fontenc}
\usepackage[utf8]{inputenc}
\usepackage[english, russian]{babel}

\usepackage{amssymb}
\usepackage{amsfonts}
\usepackage{amsmath}
\usepackage{enumitem}
\usepackage{graphics}
\usepackage{graphicx}

\usepackage{lipsum}

\newtheorem{theorem}{Теорема}
\newtheorem{task}{Задача}
\newtheorem{lemma}{Лемма}
\newtheorem{definition}{Определение}
\newtheorem{example}{Пример}
\newtheorem{statement}{Утверждение}
\newtheorem{corollary}{Следствие}


\usepackage{geometry} % Меняем поля страницы
%\geometry{left=1cm}% левое поле
%\geometry{right=1cm}% правое поле
\geometry{top=3cm}% верхнее поле
%\geometry{bottom=1cm}% нижнее поле


\usepackage{fancyhdr} % Headers and footers
\pagestyle{fancy} % All pages have headers and footers
\fancyhead{} % Blank out the default header
\fancyfoot{} % Blank out the default footer
\fancyhead[L]{\textit{\textbf{Олимпиада Эйлера}}}
\fancyhead[C]{}
\fancyhead[R]{6 февраля 2023}% Custom header text


%----------------------------------------------------------------------------------------

%\begin{document}\normalsize
\begin{document}\large
	
\begin{center}
	\LARGE\textbf{8 класс}
\end{center}
\begin{center}
	\large\textbf{Первый день}
\end{center}


\begin{enumerate}[label*=8.{\arabic{enumi}}]
\setcounter{enumi}{0}
\item Петя написал на доске десять натуральных чисел, среди которых нет двух равных. Известно, что из этих десяти чисел можно выбрать три числа, делящихся на 5. Также известно, что из написанных десяти чисел можно выбрать четыре числа, деля- щихся на 4. Может ли сумма всех написанных на доске чисел быть меньше 75?

\item 10 бегунов стартуют одновременно: пятеро в синих майках с одного конца беговой дорожки, пятеро в красных майках — с другого. Их скорости постоянны и различны, причём скорость каждого бегуна больше 9 км/ч, но меньше 12 км/ч. Добежав до конца до- рожки, каждый бегун сразу бежит назад, а, вернувшись к месту своего старта, заканчивает бег. Тренер ставит в блокноте галочку каждый раз, когда встречаются (лицом к лицу или один догоняет другого) двое бегунов в разноцветных майках (больше двух бегу- нов в одной точке за время бега не встречались). Сколько галочек поставит тренер к моменту, когда закончит бег самый быстрый из бегунов?

\item На боковых сторонах $AB$ и $AC$ равнобедренного треугольника $ABC$ выбраны точки $P$ и $Q$ соответственно так, что $PQ \parallel BC$. На биссектрисах треугольников $ABC$ и $APQ$, исходящих из вершин $B$ и $Q$, выбраны точки $X$ и $Y$ соответственно так, что $XY \parallel BC$. Докажите, что $PX = CY$.

\item Дан квадратный трёхчлен $P(x)$, не обязательно с целыми коэффициентами. Известно, что при некоторых целых $a$ и $b$ разность $P(a) - P(b)$ является квадратом натурального числа. Докажите, что существует более миллиона таких пар целых чисел $(c, d)$, что разность $P(c) - P(d)$ также является квадратом натурального числа.

\item Между городами страны организованы двусторонние беспосадочные авиарейсы таким образом, что от каждого города до каждого другого можно добраться (возможно, с пересадками). Более того, для каждого города $A$ существует город $B$ такой, что любой из остальных городов соединён напрямую с $A$ или с $B$. Докажите, что от любого города можно добраться до любого другого не более, чем с двумя пересадками.

\end{enumerate}
\end{document}