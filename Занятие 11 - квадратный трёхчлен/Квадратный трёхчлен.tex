\documentclass{article}

\usepackage[12pt]{extsizes}
\usepackage[T2A]{fontenc}
\usepackage[utf8]{inputenc}
\usepackage[english, russian]{babel}

\usepackage{mathrsfs}
\usepackage[dvipsnames]{xcolor}

\usepackage{amsmath}
\usepackage{amssymb}
\usepackage{amsthm}
\usepackage{indentfirst}
\usepackage{amsfonts}
\usepackage{enumitem}
\usepackage{graphics}
\usepackage{tikz}
\usepackage{tabu}
\usepackage{diagbox}
\usepackage{hyperref}
\usepackage{mathtools}
\usepackage{ucs}
\usepackage{lipsum}
\usepackage{geometry} % Меняем поля страницы
\usepackage{fancyhdr} % Headers and footers
\newcommand{\range}{\mathrm{range}}
\newcommand{\dom}{\mathrm{dom}}
\newcommand{\N}{\mathbb{N}}
\newcommand{\R}{\mathbb{R}}
\newcommand{\E}{\mathbb{E}}
\newcommand{\D}{\mathbb{D}}
\newcommand{\M}{\mathcal{M}}
\newcommand{\Prime}{\mathbb{P}}
\newcommand{\A}{\mathbb{A}}
\newcommand{\Q}{\mathbb{Q}}
\newcommand{\Z}{\mathbb{Z}}
\newcommand{\F}{\mathbb{F}}
\newcommand{\CC}{\mathbb{C}}

\DeclarePairedDelimiter\abs{\lvert}{\rvert}
\DeclarePairedDelimiter\floor{\lfloor}{\rfloor}
\DeclarePairedDelimiter\ceil{\lceil}{\rceil}
\DeclarePairedDelimiter\lr{(}{)}
\DeclarePairedDelimiter\set{\{}{\}}
\DeclarePairedDelimiter\norm{\|}{\|}

\renewcommand{\labelenumi}{(\alph{enumi})}

\newcommand{\smallindent}{
    \geometry{left=1cm}% левое поле
    \geometry{right=1cm}% правое поле
    \geometry{top=1.5cm}% верхнее поле
    \geometry{bottom=1cm}% нижнее поле
}

\newcommand{\header}[3]{
    \pagestyle{fancy} % All pages have headers and footers
    \fancyhead{} % Blank out the default header
    \fancyfoot{} % Blank out the default footer
    \fancyhead[L]{#1}
    \fancyhead[C]{#2}
    \fancyhead[R]{#3}
}

\newcommand{\dividedinto}{
    \,\,\,\vdots\,\,\,
}

\newcommand{\littletaller}{\mathchoice{\vphantom{\big|}}{}{}{}}

\newcommand\restr[2]{{
    \left.\kern-\nulldelimiterspace % automatically resize the bar with \right
    #1 % the function
    \littletaller % pretend it's a little taller at normal size
    \right|_{#2} % this is the delimiter
}}

\DeclareGraphicsExtensions{.pdf,.png,.jpg}

\newenvironment{enumerate_boxed}[1][enumi]{\begin{enumerate}[label*=\protect\fbox{\arabic{#1}}]}{\end{enumerate}}



\smallindent

\header{Математика}{\textit{Олимпиадная подготовка}}{6 февраля 2023}

%----------------------------------------------------------------------------------------

\begin{document}
    \large

    \begin{center}
        \textbf{Квадратный трёхчлен}
    \end{center}

    \begin{enumerate}[label*=\protect\fbox{\arabic{enumi}}]

        \item $a$ и $b$ — положительные числа.
        Сумма минимального значения квадратного трехчлена $f(x) = ax^2 + x + b$ и минимального значения трехчлена $g(x) = bx^2 + x + a$ равна нулю.
        Докажите, что эти минимальные значения сами равны нулю.

        \item Какое наибольшее количество точек пересечения может быть у графиков трёх функций $y = ax^2 + bx + c, y = bx^2 + cx + a$ и $y = cx^2 + ax + b,$ где $a, b$ и $c$ — попарно различные целые числа?

        \item У квадратного трёхчлена разрешается заменить любой из его трёх коэффициентов на его дискриминант.
        Верно ли, что из любого квадратного трёхчлена, не имеющего корней, можно за несколько таких операций получить квадратный трёхчлен, у которого есть корни?

        \item Верно, что любой квадратный трехчлен можно представить в виде суммы двух других квадратных трехчленов, у каждого из которых дискриминанты равны 0?

        \item Сумма трёх неотрицательных чисел $a, b, c$ равна 3.
        Докажите, что

        \[
            \frac{a^2 + 2}{a^3 + 3} +
            \frac{b^2 + 2}{b^3 + 3}+
            \frac{c^2 + 2}{c^3 + 3} \leqslant \frac{9}{4}
        \]

    \end{enumerate}
\end{document}