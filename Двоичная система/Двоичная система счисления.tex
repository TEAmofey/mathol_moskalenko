\documentclass{article}
\usepackage[12pt]{extsizes}
\usepackage[T2A]{fontenc}
\usepackage[utf8]{inputenc}
\usepackage[english, russian]{babel}

\usepackage{amssymb}
\usepackage{amsfonts}
\usepackage{amsmath}
\usepackage{enumitem}
\usepackage{graphics}
\usepackage{graphicx}

\usepackage{lipsum}

\newtheorem{theorem}{Теорема}
\newtheorem{task}{Задача}
\newtheorem{lemma}{Лемма}
\newtheorem{definition}{Определение}
\newtheorem{example}{Пример}
\newtheorem{statement}{Утверждение}
\newtheorem{corollary}{Следствие}


\usepackage{geometry} % Меняем поля страницы
\geometry{left=1.5cm}% левое поле
\geometry{right=1cm}% правое поле
\geometry{top=1.5cm}% верхнее поле
\geometry{bottom=1.5cm}% нижнее поле


\usepackage{fancyhdr} % Headers and footers
\pagestyle{fancy} % All pages have headers and footers
\fancyhead{} % Blank out the default header
\fancyfoot{} % Blank out the default footer
\fancyhead[L]{ЦРОД \textbullet Математика}
\fancyhead[C]{\textit{Комбинаторика}}
\fancyhead[R]{ЛФМШ 2022}% Custom header text


%----------------------------------------------------------------------------------------

%\begin{document}\normalsize
\begin{document}\large
	
\begin{center}
	\textbf{Двоичная система счисления}
\end{center}

\begin{enumerate}[label*=\protect\fbox{\arabic{enumi}}]
\item Переведите в десятичную систему числа $100101_2$, $1010011_2$, $10000000000_2$

\item Переведите в двоичную систему счисления числа $42$, $1234$, $65536$

\item Дан мешок сахарного песка, чашечные весы, много невесомых пакетиков и гирька в $1$г. Можно ли за $10$ взвешиваний отмерить $1$ кг сахара? 

\item На столе лежат $100$ карт рубашкой вверх. Роберт может либо просто перевернуть любую карту, лежащую рубашкой вверх, либо перевернуть её и любые карты справа. Может ли он проделывать эти действия бесконечно долго?

\item В наборе имеются гири массой $1$г, $2$г, $4$г, $8$г, ... (все степени числа $2$), причём среди гирь могут быть одинаковые. На две чашки весов положили гири так, чтобы наступило равновесие. Известно, что на левой чашке все гири различны.  Докажите, что на правой чашке не меньше гирь, чем на левой.

\item Сформулируйте и докажите признак делимости на $3$ в двоичной системе счисления.

\item Преподаватель придумал $1000$ задач, и ему стало понятно, что среди них ровно один гроб. У преподавателя в распоряжении $10$ учеников. Он может \textbf{одновременно} выдать им по листику с каким-то набором задач (каждому свой). Если ученик натыкается на гроб, он не приходит на следующую пару. Как преподавателю за одну пару узнать, какая задача была гробиной?

\item Странный калькулятор умеет делать только 2 операции:

$\bullet$ из числа $x$ получить число $4x+1$;

$\bullet$ из числа $x$ получить число $\left[\frac{x}{2}\right]$.

Сколько чисел от $0$ до $2020$ можно получить с помощью одной или нескольких таких операций, если изначально на калькуляторе горит число $0$?

\item Назовем число коварным, если в его двоичной записи ровно $2$ единицы и оно нечетное. Может ли произведение $n$ различных коварных чисел быть равно произведению $m$ различных коварных чисел, если $m \neq n$?

\item Докажите, что среди чисел вида $[{n\sqrt{2}}]$ есть бесконечно много составных.

\item В шеренге стоят $2^n$ солдат, где $n$ — положительное целое число. При каждом перестроении в начало новой шеренги встают солдаты, до этого стоявшие на нечётных местах (в том же порядке), а за ними — солдаты, до этого стоявшие на чётных местах (в том же порядке). Докажите, что по прошествии $n$ перестроений солдаты окажутся в таком же порядке, с которого начали.

\end{enumerate}
\end{document}