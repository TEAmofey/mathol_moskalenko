\documentclass{article}
\usepackage[12pt]{extsizes}
\usepackage[T2A]{fontenc}
\usepackage[utf8]{inputenc}
\usepackage[english, russian]{babel}

\usepackage{amssymb}
\usepackage{amsfonts}
\usepackage{amsmath}
\usepackage{enumitem}
\usepackage{graphics}

\usepackage{lipsum}



\usepackage{geometry} % Меняем поля страницы
\geometry{left=1cm}% левое поле
\geometry{right=1cm}% правое поле
\geometry{top=1.5cm}% верхнее поле
\geometry{bottom=1cm}% нижнее поле


\newtheorem{definition}{Опредление}
\usepackage{fancyhdr} % Headers and footers
\pagestyle{fancy} % All pages have headers and footers
\fancyhead{} % Blank out the default header
\fancyfoot{} % Blank out the default footer
\fancyhead[L]{Математика}
\fancyhead[C]{\textit{Геометрия}}
\fancyhead[R]{16 ноября 2023}% Custom header text


%----------------------------------------------------------------------------------------

%\begin{document}\normalsize
\begin{document}\large



\begin{center}
\textbf{Симедиана}
\end{center}

\textbf{Определение:} Прямая, \textit{си}мметричная \textit{медиане} треугольника относительно {биссектрисы} проведенной из той же вершины, называется \textit{\textbf{симедианой}}.


\begin{enumerate}[label*=\protect\fbox{\arabic{enumi}}]

\item Дан треугольник $ ABC $, в котором $ AC = BC $, и точка $ P $ внутри такая, что $ \angle PAB = \angle PBC$. Обозначим середину $ AB $ через $ M $. Докажите, что $ \angle APM+\angle BPC=180^\circ.$

\item Три различные точки $ A, B, C  $ расположены на прямой в указанном порядке. Пусть окружность $\omega$ проходит через $ A $ и $ C $, и ее центр не лежит на $ AC $. Обозначим через $ P $ точку пересечения касательных к $\omega$ в точках $ A $ и $ C $. Пусть отрезок $ PB $ пересекает $\omega$ в точке $ Q $. Докажите, что основание биссектрисы угла $ \angle Q $ треугольника $ AQC $ не зависит от выбора $\omega$.

\item Две окружности пересекаются в точках $ A $ и $ B $ и касаются их общей касательной в точках $ P $ и $ Q $. Пусть $ S $~--- точка пересечения касательных в точках $ P $ и $ Q $ к описанной окружности треугольника $ AP Q $, а точка $ H $ симметрична $ B $ относительно $ P Q $. Докажите, что $ A, S $ и $ H $ лежат на одной прямой.

\item  Дан треугольник $ ABC $. Пусть $ X $~--- центр поворотной гомотетии, переводящей $ B $ в $ A $ и $ A $ в $ C $. Докажите, что $ AX $ содержит семидиану треугольника $ ABC $.

\item Дан треугольник $ ABC $. На стороне $ BC $ выбирается точка $ P $. Точки $ Q $ и $ R $ на $ AC $ и $ AB $ соответственно таковы, что $ PQ \parallel AB$  и  $PR \parallel AC$. Докажите, что описанная окружность треугольника $ AQR $ проходит через точку $ X $, не зависящую от выбора точки $ P $.

\begin{enumerate}
	\item Докажите, что симедиана прямоугольного треугольника, проведенная из вершины прямого угла, совпадает с высотой.
	\item Докажите, что если симедиана треугольника совпадает с высотой, то он либо равнобедренный, либо прямоугольный.
\end{enumerate}


\item 
\begin{enumerate}
	\item Докажите, что расстояния от любой точки на медиане $ AM $ треугольника $ ABC $ до сторон $ AB $ и $ AC $ обратно пропорциональны этим сторонам.
	\item Докажите, что расстояния от любой точки на симедиане $ AL $ треугольника $ ABC $ до сторон $ AB $ и $ AC $ прямо пропорциональны этим сторонам.
\end{enumerate}

\item Касательные к описанной окружности треугольника $ ABC $ в точках $ B $ и $ C $ пересекаются в точке $ P $. Пусть $ X $ и $ Y $ — проекции $ P $ на $ AB $ и $ AC $, а $ M $ — середина $ BC $. Докажите, что

\begin{enumerate}
	\item $ M $ — ортоцентр треугольника $ AXY $;
	\item $ AP $ — симедиана треугольника $ ABC $.
\end{enumerate}

\item Докажите, что три симедианы треугольника пересекаются в одной точке.
\item Симедиана треугольника $ ABC $, проведенная из вершины $ A $, пересекает отрезок $ BC $ в точке $ L $, а описанную окружность — в точке $ D \neq A $. Докажите, что
\begin{enumerate}
	\item $ \dfrac{BD}{DC} = \dfrac{BA}{AC}; $
	\item $ \dfrac{BL}{LC} = \left( \dfrac{BA}{AC}\right) ^2$.
\end{enumerate}

\item Четырехугольник $ ABCD $ вписан в окружность $\omega$. Касательные к $\omega$ в точках $ A $ и $ C $ пересекаются на прямой $ BD $. Докажите, что касательные к $\omega$ в точках $ B $ и $ D $ пересекаются на прямой $ AC $ или параллельны ей.
\item Касательная к описанной окружности треугольника $ ABC $ в точке $ A $ пересекает прямую $ BC $ в точке $ D $. Касательные к описанной окружности треугольника $ ACD $ в точках $ A $ и $ C $ пересекаются в точке $ K $. Докажите, что прямая $ DK $ делит отрезок $ AB $ пополам.
\item Дан вписанный четырехугольник $ ABCD $, в котором $AB^2 + CD^2 = AD^2 $. На стороне $ AD $ выбрана точка $ P $ такая, что $ \angle APB = \angle CPD $. Докажите, что точка пересечения диагоналей, середина $ BC $ и точка $ P $ лежат на одной прямой.

\item Пусть $ ABC $~--- остроугольный треугольник, $  M, N, P $ ~--- середины сторон $ BC, CA, AB  $ соответственно. Пусть серединные перпендикуляры к $ AB $ и $ AC $ пересекают $ AM $ в точках $ D $ и $ E $ соответственно. Прямые $ BD $ и $ CE $ пересекаются в точке $ F $ внутри треугольника $ ABC $. Докажите, что точки $ A, N, F,  $ и $ P $ лежат на одной окружности.
\item Треугольник $ ABC $ вписан в окружность $\omega$. Касательные к $\omega$ в точках $ B $ и $ C $ пересекаются в $ T $. Точка $ S $ на прямой $ BC $ такова, что $ AS \perp AT $. Точки $ B_1 $ и $ C_1 $ лежат на прямой $ ST $ так, что $ B_1T  = BT = C_1T$. Докажите, что треугольники $ ABC $ и $ AB_1C_1 $ подобны.
\item Пусть $ A $ ~--- одна из точек пересечения окружностей $ \omega_1 $ и $ \omega_2 $ с центрами $ O_1 $ и $ O_2 $. Общая касательная к $\omega_1$ и $\omega_2$ касается их в точках $ B $ и $ C $. Пусть $ O_3 $ ~--- центр описанной окружности треугольника $ ABC $. Обозначим через $ D $ такую точку, что $ A $ ~--- середина отрезка $ O_3D $. Пусть $ M $ ~--- середина $ O_1O_2 $. Докажите, что $ \angle O_1DM $ = $ \angle O_2DA $.
\end{enumerate}
\end{document}