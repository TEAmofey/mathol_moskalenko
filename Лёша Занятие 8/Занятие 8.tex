\documentclass{article}
\usepackage[12pt]{extsizes}
\usepackage[T2A]{fontenc}
\usepackage[utf8]{inputenc}
\usepackage[english, russian]{babel}

\usepackage{amssymb}
\usepackage{amsfonts}
\usepackage{amsmath}
\usepackage{enumitem}
\usepackage{graphics}
\usepackage{graphicx}

\usepackage{lipsum}

\newtheorem{theorem}{Теорема}
\newtheorem{task}{Задача}
\newtheorem{lemma}{Лемма}
\newtheorem{definition}{Определение}
\newtheorem{example}{Пример}
\newtheorem{statement}{Утверждение}
\newtheorem{corollary}{Следствие}


\usepackage{geometry} % Меняем поля страницы
\geometry{left=1cm}% левое поле
\geometry{right=1cm}% правое поле
\geometry{top=1.5cm}% верхнее поле
\geometry{bottom=1cm}% нижнее поле


\usepackage{fancyhdr} % Headers and footers
\pagestyle{fancy} % All pages have headers and footers
\fancyhead{} % Blank out the default header
\fancyfoot{} % Blank out the default footer
\fancyhead[L]{Математика}
\fancyhead[C]{\textit{Разнобой}}
\fancyhead[R]{13 ноября 2023}% Custom header text


%----------------------------------------------------------------------------------------

%\begin{document}\normalsize
\begin{document}\large
	
\begin{center}
	\textbf{Разнобой}
\end{center}


\begin{enumerate}[label*=\protect\fbox{\arabic{enumi}}]

\item Сколько треугольников изображено на рисунке ниже?

\begin{figure}[h]
	\centering
	\includegraphics[width=0.3\linewidth]{img.png}
\end{figure}

\item Можно ли разбить квадрат $8\times 8$ с отрезанным уголком на прямоугольники $1\times 3$?

\item Можно ли из квадрата $7\times 7$ вырезать 10 квадратов $2\times2$?

\item В кружке художественного свиста у каждого ровно один друг и ровно один враг. Докажите, что в кружке четное число людей.

\item Треугольник разбит на треугольнички (25 штук), как показано на рисунке. Жук может ходить по треугольнику, переходя между соседними (по стороне) треугольничками. Какое максимальное количество треугольничков может пройти жук, если в каждом он побывал не больше одного раза?

\begin{figure}[h]
	\centering
	\includegraphics[width=0.3\linewidth]{img.png}
\end{figure}


\end{enumerate}


\end{document}