\documentclass{article}
\usepackage[12pt]{extsizes}
\usepackage[T2A]{fontenc}
\usepackage[utf8]{inputenc}
\usepackage[english, russian]{babel}

\usepackage{amssymb}
\usepackage{amsfonts}
\usepackage{amsmath}
\usepackage{enumitem}
\usepackage{graphics}
\usepackage{graphicx}

\usepackage{lipsum}

\newtheorem{theorem}{Теорема}
\newtheorem{task}{Задача}
\newtheorem{lemma}{Лемма}
\newtheorem{definition}{Определение}
\newtheorem{exercise}{Упражнение}
\newtheorem{example}{Пример}
\newtheorem{statement}{Утверждение}
\newtheorem{corollary}{Следствие}


\usepackage{geometry} % Меняем поля страницы
\geometry{left=1cm}% левое поле
\geometry{right=1cm}% правое поле
\geometry{top=1.5cm}% верхнее поле
\geometry{bottom=1cm}% нижнее поле


\usepackage{fancyhdr} % Headers and footers
\pagestyle{fancy} % All pages have headers and footers
\fancyhead{} % Blank out the default header
\fancyfoot{} % Blank out the default footer
\fancyhead[L]{ЦРОД $\bullet$ Математика}
\fancyhead[C]{\textit{На первый взгляд дичь...}}
\fancyhead[R]{ЦРОД 2022}% Custom header text


%----------------------------------------------------------------------------------------

%\begin{document}\normalsize
\begin{document}\large
	
\begin{center}
	\textbf{Сопряжённые числа в $\mathbb{Q}[\sqrt{m}]$ и $\mathbb{Z}[\sqrt{m}]$}
\end{center}


\begin{definition}
Пусть $m$ --- натуральное число, не точный квадрат. 
Обозначим через $\mathbb{Q}[\sqrt{m}]$ множество чисел вида 
$a+b\sqrt{m}$, где $a$,~$b\in \mathbb{Q}$. Аналогично определим $\mathbb{Z}[\sqrt{m}]$.
\end{definition}

\begin{exercise}
Пусть $a_i$, $b_i\in \mathbb{Q}$. 
Докажите, что если $a_1+b_1\sqrt{m}=a_2+b_2\sqrt{m}$, то $a_1=a_2$ и $b_1=b_2$. 
\end{exercise}

\begin{definition}
Пусть $x\in\mathbb{Q}[\sqrt{m}]$, $x=a+b\sqrt{m}$. Сопряженным к $x$
называется число $\overline{x}=a-b\sqrt{m}$.
\end{definition}

\begin{exercise}
Почему Упражнение 1. должно идти раньше определения сопряжения?
\end{exercise}

\begin{exercise}
Пусть $x\in \mathbb{Q}[\sqrt{m}]$. 
\begin{enumerate}[label*=\alph*)]
\item Докажите, что $x+\overline{x}\in \mathbb{Q}$ и $x\cdot \overline{x}\in\mathbb{Q}$. 
Выведите из этого, что $x$ --- корень квадратного трёхчлена с рациональными коэффициентами.
\item Докажите, что если $x\notin \mathbb{Q}$, то существует единственное $y\in \mathbb{R}$, что $x+y\in \mathbb{Q}$, $x\cdot y\in \mathbb{Q}$.
\end{enumerate}
\end{exercise}

\begin{exercise}
Пусть $x_1$, $x_2\in \mathbb{Q}[\sqrt{m}]$. Понятно, что $x_1+x_2$ и ${x_1\cdot x_2}$ также принадлежат
$\mathbb{Q}[\sqrt{m}]$. Докажите, что если $x_2\neq 0$, то $x_1/x_2$ принадлежит $\mathbb{Q}[\sqrt{m}]$.
\end{exercise}

\begin{exercise}
Пусть $x_0$, $x_1$, $x_2\in \mathbb{Q}[\sqrt{m}]$, $f(x)$ --- многочлен с рациональными 
коэффициентами. Докажите, что 
\begin{enumerate}[label*=\alph*)]
	\item $\overline{x_1+x_2}=\overline{x_1}+\overline{x_2}$,
	\item $\overline{x_1\cdot x_2}=\overline{x_1}\cdot \overline{x_2}$,
	\item $\overline{x_1/x_2}=\overline{x_1}/\overline{x_2}$, 
	\item $\overline{f(x_0)}=f(\overline{x_0})$.
	\item Докажите, что если $x_0$ --- корень многочлена $f(x)$, то и $\overline{x_0}$ корень $f(x)$.
\end{enumerate}
\end{exercise}

\begin{enumerate}[label*=\protect\fbox{\arabic{enumi}}]
	
	\item Найдите первые 100 цифр после запятой в числе $(2+\sqrt{3})^{1000}$. 
	
	\item Докажите, что для рациональных чисел $x$, $y$, $z$ и $t$ не может выполняться равенство
	$$(x+y\sqrt{2})^4+(z+t\sqrt{2})^4=5+4\sqrt{2}.$$
	
	\item Докажите, что для натуральных $m$ и $n$ не может выполняться равенство
	$(5+3\sqrt{2})^m=(3+5\sqrt{2})^n$.
	
	\item Докажите, что $(\sqrt{2}-1)^{2022}=\sqrt{k}-\sqrt{k-1}$ для некоторого $k\in \mathbb{N}$. 
	
	\item Докажите, что уравнение $n^2-2m^2 = 1$ имеет 
	\begin{enumerate}
		\item 4 решения в целых числах
		\item бесконечно много решений в целых числах
\end{enumerate}
\end{enumerate}
\end{document}