\documentclass{article}

\usepackage[12pt]{extsizes}
\usepackage[T2A]{fontenc}
\usepackage[utf8]{inputenc}
\usepackage[english, russian]{babel}

\usepackage{mathrsfs}
\usepackage[dvipsnames]{xcolor}

\usepackage{amsmath}
\usepackage{amssymb}
\usepackage{amsthm}
\usepackage{indentfirst}
\usepackage{amsfonts}
\usepackage{enumitem}
\usepackage{graphics}
\usepackage{tikz}
\usepackage{tabu}
\usepackage{diagbox}
\usepackage{hyperref}
\usepackage{mathtools}
\usepackage{ucs}
\usepackage{lipsum}
\usepackage{geometry} % Меняем поля страницы
\usepackage{fancyhdr} % Headers and footers
\usepackage[framemethod=TikZ]{mdframed}

\newcommand{\definebox}[3]{%
    \newcounter{#1}
    \newenvironment{#1}[1][]{%
        \stepcounter{#1}%
        \mdfsetup{%
            frametitle={%
            \tikz[baseline=(current bounding box.east),outer sep=0pt]
            \node[anchor=east,rectangle,fill=white]
            {\strut #2~\csname the#1\endcsname\ifstrempty{##1}{}{##1}};}}%
        \mdfsetup{innertopmargin=1pt,linecolor=#3,%
            linewidth=3pt,topline=true,
            frametitleaboveskip=\dimexpr-\ht\strutbox\relax,}%
        \begin{mdframed}[]
            \relax%
            }{
        \end{mdframed}}%
}

\definebox{theorem_boxed}{Теорема}{ForestGreen!24}
\definebox{definition_boxed}{Определение}{blue!24}
\definebox{task_boxed}{Задача}{orange!24}
\definebox{paradox_boxed}{Парадокс}{red!24}

\theoremstyle{plain}
\newtheorem{theorem}{Теорема}
\newtheorem{task}{Задача}
\newtheorem{lemma}{Лемма}
\newtheorem{definition}{Определение}
\newtheorem{statement}{Утверждение}
\newtheorem{corollary}{Следствие}

\theoremstyle{remark}
\newtheorem{remark}{Замечание}
\newtheorem{example}{Пример}
\newcommand{\range}{\mathrm{range}}
\newcommand{\dom}{\mathrm{dom}}
\newcommand{\N}{\mathbb{N}}
\newcommand{\R}{\mathbb{R}}
\newcommand{\E}{\mathbb{E}}
\newcommand{\D}{\mathbb{D}}
\newcommand{\M}{\mathcal{M}}
\newcommand{\Prime}{\mathbb{P}}
\newcommand{\A}{\mathbb{A}}
\newcommand{\Q}{\mathbb{Q}}
\newcommand{\Z}{\mathbb{Z}}
\newcommand{\F}{\mathbb{F}}
\newcommand{\CC}{\mathbb{C}}

\DeclarePairedDelimiter\abs{\lvert}{\rvert}
\DeclarePairedDelimiter\floor{\lfloor}{\rfloor}
\DeclarePairedDelimiter\ceil{\lceil}{\rceil}
\DeclarePairedDelimiter\lr{(}{)}
\DeclarePairedDelimiter\set{\{}{\}}
\DeclarePairedDelimiter\norm{\|}{\|}

\renewcommand{\labelenumi}{(\alph{enumi})}

\newcommand{\smallindent}{
    \geometry{left=1cm}% левое поле
    \geometry{right=1cm}% правое поле
    \geometry{top=1.5cm}% верхнее поле
    \geometry{bottom=1cm}% нижнее поле
}

\newcommand{\header}[3]{
    \pagestyle{fancy} % All pages have headers and footers
    \fancyhead{} % Blank out the default header
    \fancyfoot{} % Blank out the default footer
    \fancyhead[L]{#1}
    \fancyhead[C]{#2}
    \fancyhead[R]{#3}
}

\newcommand{\dividedinto}{
    \,\,\,\vdots\,\,\,
}

\newcommand{\littletaller}{\mathchoice{\vphantom{\big|}}{}{}{}}

\newcommand\restr[2]{{
    \left.\kern-\nulldelimiterspace % automatically resize the bar with \right
    #1 % the function
    \littletaller % pretend it's a little taller at normal size
    \right|_{#2} % this is the delimiter
}}

\DeclareGraphicsExtensions{.pdf,.png,.jpg}

\newenvironment{enumerate_boxed}[1][enumi]{\begin{enumerate}[label*=\protect\fbox{\arabic{#1}}]}{\end{enumerate}}



\smallindent

\header{ЦРОД $\bullet$ Математика}{\textit{Алгебра}}{ЛФМШ 2022}

%----------------------------------------------------------------------------------------

\begin{document}
    \large

    \begin{center}
        \textbf{Сопряжённые числа в $\Q\left[\sqrt{m}\right]$ и $\Z\left[\sqrt{m}\right]$}
    \end{center}

    \begin{definition}
        Пусть $m$ --- натуральное число, не точный квадрат.
        Обозначим через $\Q\left[\sqrt{m}\right]$ множество чисел вида
        $a+b\sqrt{m}$, где $a$,~$b\in \Q$.
        Аналогично определим $\Z\left[\sqrt{m}\right]$.
    \end{definition}

    \begin{exercise}
        \label{ex:1}
        Пусть $a_i$, $b_i\in \Q$.
        Докажите, что если $a_1+b_1\sqrt{m}=a_2+b_2\sqrt{m}$, то $a_1=a_2$ и $b_1=b_2$.
    \end{exercise}

    \begin{definition}
        Пусть $x\in \Q\left[\sqrt{m}\right]$, $x=a+b\sqrt{m}$.
        Сопряженным к $x$ называется число $\overline{x}=a-b\sqrt{m}$.
    \end{definition}

    \begin{exercise}
        Почему Упражнение~\ref{ex:1} должно идти раньше определения сопряжения?
    \end{exercise}

    \begin{exercise}
        Пусть $x\in \Q\left[\sqrt{m}\right]$.
        \begin{enumerate}[label*=\alph*)]
            \item Докажите, что $x+\overline{x}\in \Q$ и $x\cdot \overline{x}\in\Q$.
            Выведите из этого, что $x$ --- корень квадратного трёхчлена с рациональными коэффициентами.
            \item Докажите, что если $x\notin \Q$, то существует единственное $y\in \R$, что $x+y\in \Q$, $x\cdot y\in \Q$.
        \end{enumerate}
    \end{exercise}

    \begin{exercise}
        Пусть $x_1$, $x_2\in \Q\left[\sqrt{m}\right]$.
        Понятно, что $x_1+x_2$ и ${x_1\cdot x_2}$ также принадлежат
        $\Q\left[\sqrt{m}\right]$.
        Докажите, что если $x_2\neq 0$, то $x_1/x_2$ принадлежит $\Q\left[\sqrt{m}\right]$.
    \end{exercise}

    \begin{exercise}
        Пусть $x_0$, $x_1$, $x_2\in \Q\left[\sqrt{m}\right]$, $f(x)$ --- многочлен с рациональными коэффициентами.
        Докажите, что
        \begin{enumerate}[label*=\alph*)]
            \item $\overline{x_1+x_2}=\overline{x_1}+\overline{x_2}$,
            \item $\overline{x_1\cdot x_2}=\overline{x_1}\cdot \overline{x_2}$,
            \item $\overline{x_1/x_2}=\overline{x_1}/\overline{x_2}$,
            \item $\overline{f(x_0)}=f(\overline{x_0})$.
            \item Докажите, что если $x_0$ --- корень многочлена $f(x)$, то и $\overline{x_0}$ корень $f(x)$.
        \end{enumerate}
    \end{exercise}

    \begin{enumerate}[label*=\protect\fbox{\arabic{enumi}}]

        \item Найдите первые 100 цифр после запятой в числе $(2+\sqrt{3})^{1000}$.

        \item Докажите, что для рациональных чисел $x$, $y$, $z$ и $t$ не может выполняться равенство
        \[\left(x+y\sqrt{2}\right)^4+\left(z+t\sqrt{2}\right)^4=5+4\sqrt{2}.\]

        \item Докажите, что для натуральных $m$ и $n$ не может выполняться равенство
        $(5+3\sqrt{2})^m=\left(3+5\sqrt{2}\right)^n$.

        \item Докажите, что $(\sqrt{2}-1)^{2022}=\sqrt{k}-\sqrt{k-1}$ для некоторого $k\in \mathbb{N}$.

        \item Докажите, что уравнение $n^2-2m^2 = 1$ имеет
        \begin{enumerate}
            \item 4 решения в целых числах
            \item бесконечно много решений в целых числах
        \end{enumerate}
    \end{enumerate}
\end{document}