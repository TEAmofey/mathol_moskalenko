\documentclass{article}

\usepackage[12pt]{extsizes}
\usepackage[T2A]{fontenc}
\usepackage[utf8]{inputenc}
\usepackage[english, russian]{babel}

\usepackage{mathrsfs}
\usepackage[dvipsnames]{xcolor}

\usepackage{amsmath}
\usepackage{amssymb}
\usepackage{amsthm}
\usepackage{indentfirst}
\usepackage{amsfonts}
\usepackage{enumitem}
\usepackage{graphics}
\usepackage{tikz}
\usepackage{tabu}
\usepackage{diagbox}
\usepackage{hyperref}
\usepackage{mathtools}
\usepackage{ucs}
\usepackage{lipsum}
\usepackage{geometry} % Меняем поля страницы
\usepackage{fancyhdr} % Headers and footers
\newcommand{\range}{\mathrm{range}}
\newcommand{\dom}{\mathrm{dom}}
\newcommand{\N}{\mathbb{N}}
\newcommand{\R}{\mathbb{R}}
\newcommand{\E}{\mathbb{E}}
\newcommand{\D}{\mathbb{D}}
\newcommand{\M}{\mathcal{M}}
\newcommand{\Prime}{\mathbb{P}}
\newcommand{\A}{\mathbb{A}}
\newcommand{\Q}{\mathbb{Q}}
\newcommand{\Z}{\mathbb{Z}}
\newcommand{\F}{\mathbb{F}}
\newcommand{\CC}{\mathbb{C}}

\DeclarePairedDelimiter\abs{\lvert}{\rvert}
\DeclarePairedDelimiter\floor{\lfloor}{\rfloor}
\DeclarePairedDelimiter\ceil{\lceil}{\rceil}
\DeclarePairedDelimiter\lr{(}{)}
\DeclarePairedDelimiter\set{\{}{\}}
\DeclarePairedDelimiter\norm{\|}{\|}

\renewcommand{\labelenumi}{(\alph{enumi})}

\newcommand{\smallindent}{
    \geometry{left=1cm}% левое поле
    \geometry{right=1cm}% правое поле
    \geometry{top=1.5cm}% верхнее поле
    \geometry{bottom=1cm}% нижнее поле
}

\newcommand{\header}[3]{
    \pagestyle{fancy} % All pages have headers and footers
    \fancyhead{} % Blank out the default header
    \fancyfoot{} % Blank out the default footer
    \fancyhead[L]{#1}
    \fancyhead[C]{#2}
    \fancyhead[R]{#3}
}

\newcommand{\dividedinto}{
    \,\,\,\vdots\,\,\,
}

\newcommand{\littletaller}{\mathchoice{\vphantom{\big|}}{}{}{}}

\newcommand\restr[2]{{
    \left.\kern-\nulldelimiterspace % automatically resize the bar with \right
    #1 % the function
    \littletaller % pretend it's a little taller at normal size
    \right|_{#2} % this is the delimiter
}}

\DeclareGraphicsExtensions{.pdf,.png,.jpg}

\newenvironment{enumerate_boxed}[1][enumi]{\begin{enumerate}[label*=\protect\fbox{\arabic{#1}}]}{\end{enumerate}}



\smallindent

\header{Математика}{\textit{Геометрия}}{23 февраля 2023}

%----------------------------------------------------------------------------------------

\begin{document}
    \large

    \begin{center}
        \textbf{Радикальные оси}
    \end{center}

    \begin{enumerate_boxed}

        \item Внутри треугольника $ABC$ отметили такую точку $M$, что $\angle ABM=\angle MAC, \angle CBM=\angle MCA$.
        Докажите, что точка $M$ лежит на медиане, проведённой из вершины $B$.

        \setcounter{enumi}{1}
        \item На гипотенузе $AB$ прямоугольного равнобедренного треугольника $ABC$ выбрана произвольная точка $M$.
        Докажите, что общая хорда окружностей с центром $C$ и радиусом $CA$ и с центром $M$ и радиусом $MC$ проходит через середину $AB$.

        \item Окружности $s_1$, $s_2$ и $s_3$ проходят через точку $A$.
        Известно, что прямая, содержащая общую хорду $s_1$ и $s_2$, проходит через центр $s_3$, а прямая, содержащая общую хорду $s_2$ и $s_3$ — через центр $s_1$.
        Докажите, что прямая, содержащая общую хорду $s_1$ и $s_3$, проходит через центр $s_2$.

        \item На сторонах $BC$, $AC$, $AB$ треугольника $ABC$ отмечены точки $A_1$ и $A_2$, $B_1$ и $B_2$, $C_1$ и $C_2$.
        Оказалось, что четырёхугольники $A_{1}A_{2}B_{2}B_1$, $A_{1}A_{2}C_{2}C_1$ и $B_{1}B_{2}C_{1}C_2$ вписанные.
        Докажите, что шесть отмеченных точек лежат на одной окружности.

        \item Дана неравнобокая трапеция $ABCD$ ($AD\parallel BC$). Окружность, проходящая через точки $B$ и $C$, пересекает боковые стороны трапеции в точках $M$ и $N$, а диагонали — в точках $X$ и $Y$.
        Докажите, что прямые $XY$, $MN$ и $AD$ пересекаются в одной точке.

        \item В равнобедренном треугольнике $ABC$ ($AB=BC$) проведена высота $BH$.
        Точка $F$ — основание перпендикуляра из точки $H$ на сторону $BC$.
        Докажите, что прямая, перпендикулярная $AF$ и проходящая через точку $B$, делит отрезок $HF$ пополам.

        \item Точки $B$ и $C$ — середины касательных, проведённых из точки $A$ к окружности $\omega$.
        На прямой $BC$ выбрали точки $X$ и $Y$.
        Из точек $X$ и $Y$ провели касательные к $\omega$, пересекающиеся в точке $F$, причём $\omega$ располагается внутри четырёхугольника $AXFY$.
        Докажите, что в четырёхугольник $AXFY$ можно вписать окружность.

        \item Вписанная окружность треугольника $ABC$ касается сторон $AB$, $AC$, $BC$ в точках $C_1$, $B_1$, $A_1$ соответственно. $P$ — произвольная точка плоскости.
        Серединный перпендикуляр к отрезку $PA_1$ пересекает прямую $BC$ в точке $A_2$.
        Аналогично определяются точки $B_2$ и $C_2$.
        Докажите, что точки $A_2$, $B_2$ и $C_2$ лежат на одной прямой.

        \item Вписанная окружность неравнобедренного треугольника $ABC$ касается сторон $AB$, $AC$, $BC$ в точках $C_1$, $B_1$, $A_1$ соответственно.
        Точка $K$ — середина отрезка $A_{1}C_1$.
        Докажите, что центр описанной окружности треугольника $BKB_1$ лежит на прямой $AC$.

        \newpage
        \textbf{Посложнее}

        \item Высоты $AA_1$, $BB_1$, $CC_1$ треугольника $ABC$ пересекаются в точке $H$.
        На стороне $BC$ выбрана произвольная точка $X$.
        Описанные окружности треугольников $XCB_1$ и $XBC_1$ повторно пересекаются в точке $Y$.
        Докажите, что четырёхугольник $XYHA_1$ вписанный.

        \item На сторонах треугольника $ABC$ взято по две точки так, что шесть отрезков, соединяющих вершину с точкой на противолежащей стороне, равны.
        Докажите, что середины этих отрезков лежат на одной окружности.


    \end{enumerate_boxed}
\end{document}