\documentclass{article}

\usepackage[12pt]{extsizes}
\usepackage[T2A]{fontenc}
\usepackage[utf8]{inputenc}
\usepackage[english, russian]{babel}

\usepackage{mathrsfs}
\usepackage[dvipsnames]{xcolor}

\usepackage{amsmath}
\usepackage{amssymb}
\usepackage{amsthm}
\usepackage{indentfirst}
\usepackage{amsfonts}
\usepackage{enumitem}
\usepackage{graphics}
\usepackage{tikz}
\usepackage{tabu}
\usepackage{diagbox}
\usepackage{hyperref}
\usepackage{mathtools}
\usepackage{ucs}
\usepackage{lipsum}
\usepackage{geometry} % Меняем поля страницы
\usepackage{fancyhdr} % Headers and footers
\newcommand{\range}{\mathrm{range}}
\newcommand{\dom}{\mathrm{dom}}
\newcommand{\N}{\mathbb{N}}
\newcommand{\R}{\mathbb{R}}
\newcommand{\E}{\mathbb{E}}
\newcommand{\D}{\mathbb{D}}
\newcommand{\M}{\mathcal{M}}
\newcommand{\Prime}{\mathbb{P}}
\newcommand{\A}{\mathbb{A}}
\newcommand{\Q}{\mathbb{Q}}
\newcommand{\Z}{\mathbb{Z}}
\newcommand{\F}{\mathbb{F}}
\newcommand{\CC}{\mathbb{C}}

\DeclarePairedDelimiter\abs{\lvert}{\rvert}
\DeclarePairedDelimiter\floor{\lfloor}{\rfloor}
\DeclarePairedDelimiter\ceil{\lceil}{\rceil}
\DeclarePairedDelimiter\lr{(}{)}
\DeclarePairedDelimiter\set{\{}{\}}
\DeclarePairedDelimiter\norm{\|}{\|}

\renewcommand{\labelenumi}{(\alph{enumi})}

\newcommand{\smallindent}{
    \geometry{left=1cm}% левое поле
    \geometry{right=1cm}% правое поле
    \geometry{top=1.5cm}% верхнее поле
    \geometry{bottom=1cm}% нижнее поле
}

\newcommand{\header}[3]{
    \pagestyle{fancy} % All pages have headers and footers
    \fancyhead{} % Blank out the default header
    \fancyfoot{} % Blank out the default footer
    \fancyhead[L]{#1}
    \fancyhead[C]{#2}
    \fancyhead[R]{#3}
}

\newcommand{\dividedinto}{
    \,\,\,\vdots\,\,\,
}

\newcommand{\littletaller}{\mathchoice{\vphantom{\big|}}{}{}{}}

\newcommand\restr[2]{{
    \left.\kern-\nulldelimiterspace % automatically resize the bar with \right
    #1 % the function
    \littletaller % pretend it's a little taller at normal size
    \right|_{#2} % this is the delimiter
}}

\DeclareGraphicsExtensions{.pdf,.png,.jpg}

\newenvironment{enumerate_boxed}[1][enumi]{\begin{enumerate}[label*=\protect\fbox{\arabic{#1}}]}{\end{enumerate}}



\smallindent

\header{Математика}{\textit{Олимпиадная подготовка}}{14 ноября 2023}

%----------------------------------------------------------------------------------------

\begin{document}
    \large

    \begin{center}
        \textbf{Теорема Виета}
    \end{center}

    \textbf{Теорема Виета:} Пусть многочлен $ a_{n}x^n + a_{n-1}x^{n-1} + \ldots + a_{1}x + a_0$ имеет корни $x_1, x_2, \ldots, x_n.$
    Тогда:
    \begin{eqnarray*}
        x_1 + x_2 + \ldots + x_n &=& -\frac{a_{n-1}}{a_n},\\
        x_{1}x_2 + x_{1}x_3 + \ldots + x_{n-1}x_n &=& \frac{a_{n-2}}{a_n}, \\
        &\vdots& \\
        \sum_{1\leq i_1 < i_2 < \ldots < i_k \leqslant n} x_{i_1}x_{i_2}\ldots x_{i_k} &=& (-1)^k\frac{a_{n-k}}{a_n}, \\
        &\vdots& \\
        x_{1}x_2\ldots x_n &=& (-1)^n\frac{a_0}{a_n}.
    \end{eqnarray*}

    \begin{enumerate_boxed}

        \item Пусть $x_1, x_2, x_3$ — корни уравнения $x^3 - 2x^2 + x + 1 = 0.$
        Составьте кубическое уравнение, корнями которого являются числа $\frac{1}{x_1^2}, \frac{1}{x_2^2}, \frac{1}{x_3^2}$.

        \item У многочлена с целыми коэффициентами $x^3 + px + q$ имеется три различных корня.
        Докажите, что сумма кубов этих корней есть целое число, кратное трём.

        \item Известно, что $a + b + c = d$, и что \[\frac{1}{a} + \frac{1}{b} + \frac{1}{c} = \frac{1}{d}.\]
        Докажите, что по меньшей мере одно из чисел $a, b, c$ равно $d$.

        \item Даны действительные числа $a_1 \leqslant a_2 \leqslant a_3$  и  $b_1 \leqslant b_2 \leqslant b_3,$
        такие что \begin{gather*}
                      a_1 + a_2 + a_3 = b_1 + b_2 + b_3,\\
                      a_{1}a_2 + a_{2}a_3 + a_{1}a_3 = b_{1}b_2 + b_{2}b_3 + b_{1}b_3.\\
        \end{gather*}
        Докажите, что если $a_1 \leq b_1$, то $a_3 \leq b_3$.

        \item На доске написано несколько приведённых многочленов 37-й степени, все коэффициенты которых неотрицательны.
        Разрешается выбрать любые два выписанных многочлена $f$ и $g$ и заменить их на такие два приведённых многочлена 37-й степени $f_1$ и $g_1$, что $f + g = f_1 + g_1$ или $fg = f_{1}g_1$.
        Докажите, что после применения любого конечного числа таких операций не может оказаться, что каждый многочлен на доске имеет 37 различных положительных корней.

        \item Натуральные числа $a, b, c, d, e, f$ таковы, что число $S = a + b + c + d + e + f$ делит
        числа $abc + def$ и  $ab + bc + ca - de - ef - df.$ Докажите, что $S$ составное.


    \end{enumerate_boxed}
\end{document}