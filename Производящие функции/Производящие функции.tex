\documentclass{article}
\usepackage[12pt]{extsizes}
\usepackage[T2A]{fontenc}
\usepackage[utf8]{inputenc}
\usepackage[english, russian]{babel}

\usepackage{amssymb}
\usepackage{amsfonts}
\usepackage{amsmath}
\usepackage{enumitem}
\usepackage{graphics}
\usepackage{graphicx}
\usepackage{epigraph}

\usepackage{lipsum}
\DeclareGraphicsExtensions{.pdf,.png,.jpg}



\usepackage{geometry} % Меняем поля страницы
\geometry{left=1cm}% левое поле
\geometry{right=1cm}% правое поле
\geometry{top=1.5cm}% верхнее поле
\geometry{bottom=1cm}% нижнее поле



\usepackage{fancyhdr} % Headers and footers
\pagestyle{fancy} % All pages have headers and footers
\fancyhead{} % Blank out the default header
\fancyfoot{} % Blank out the default footer
\fancyhead[L]{ЦРОД $\bullet$ Математика}
\fancyhead[C]{\textit{Алгебра}}
\fancyhead[R]{25 марта 2024}% Custom header text

%----------------------------------------------------------------------------------------

%\begin{document}\normalsize
\begin{document}\large
	

\begin{center}
\textbf{Производящие функции}
\end{center}

%\epigraph{\textit{Повторение --- мать учения}}{}

Производящей функцией последовательности $\{g_n\}_{n = 0}^{\infty}$ называется формальный степенной ряд.

$$\sum_{n = 0}^{\infty}g_nx^n = g_0 + g_1x + g_2x^2 + \dotsc$$
\begin{enumerate}[label*=\protect\fbox{\arabic{enumi}}]

\item Дан многочлен $f(x)$ степени не выше $n$. Докажите, что для любого вещественного $x_0$ существуют и единственные $A_1, \dotsc , A_n$ такие, что

$$\frac{f(x)}{(x - x_0)^n} = \frac{A_1}{x-x_0} + \frac{A_2}{(x-x_0)^2} + \dotsc \frac{A_n}{(x-x_0)^n}.$$

\item Дан многочлен $f(x)$ степени не выше $n$. Докажите, что для любых различных вещественных чисел $x_0, x_1, . . . , x_n$ существуют и единственные $A_0, A_1, \dotsc , A_n$ такие, что

$$\frac{f(x)}{(x - x_0)(x-x_1)\dotsc(x-x_n)} = \frac{A_0}{x-x_0} + \frac{A_1}{x-x_1} + \dotsc \frac{A_n}{x-x_n}.$$

\item Посчитайте производящую функцию последовательности $\{1\}_{n = 0}^{\infty}$.

\item Посчитайте производящую функцию последовательности $\{F_n\}_{n = 0}^{\infty}$ и выведите из неё формулу $n$-того числа Фибоначчи.

Производной степенного ряда
$$G(x) = \sum_{n = 0}^{\infty}g_nx^n = g_0 + g_1x + g_2x^2 + \dotsc$$
называется
$$G'(x) = \sum_{n = 0}^{\infty}ng_nx^{n-1} = g_1 + 2g_2x + 3g_3x^2 + \dotsc$$

Тогда заметим, что $g_n = n! \cdot G^{(n)}(0)$

\item Найдите степенной ряд функции $(1 - 4x)^{1/2}$.

\item Докажите, что $$4^n = \sum_{i = 0}^n \binom{2i}{i} \binom{2(n-i)}{n-i}$$

\item Найдите производящую функцию чисел Каталана.

\item Посчитайте количество подмножеств множества из 1000 элементов, количество элементов в которых делится на 5


\end{enumerate}

\end{document}