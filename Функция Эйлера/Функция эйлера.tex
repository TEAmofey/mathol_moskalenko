\documentclass{article}
\usepackage[12pt]{extsizes}
\usepackage[T2A]{fontenc}
\usepackage[utf8]{inputenc}
\usepackage[english, russian]{babel}

\usepackage{amssymb}
\usepackage{amsfonts}
\usepackage{amsmath}
\usepackage{enumitem}
\usepackage{graphics}
\usepackage{graphicx}

\usepackage{lipsum}

\newtheorem{theorem}{Теорема}
\newtheorem{task}{Задача}
\newtheorem{lemma}{Лемма}
\newtheorem{definition}{Определение}
\newtheorem{example}{Пример}
\newtheorem{statement}{Утверждение}
\newtheorem{corollary}{Следствие}


\usepackage{geometry} % Меняем поля страницы
\geometry{left=1cm}% левое поле
\geometry{right=1cm}% правое поле
\geometry{top=1.5cm}% верхнее поле
\geometry{bottom=1cm}% нижнее поле


\usepackage{fancyhdr} % Headers and footers
\pagestyle{fancy} % All pages have headers and footers
\fancyhead{} % Blank out the default header
\fancyfoot{} % Blank out the default footer
\fancyhead[L]{Математика}
\fancyhead[C]{\textit{Теория чисел}}
\fancyhead[R]{22 сентября 2022}% Custom header text


%----------------------------------------------------------------------------------------

%\begin{document}\normalsize
\begin{document}\large
	
\begin{center}
	\textbf{Функция Эйлера}
\end{center}

\begin{definition}
	\textbf{Функцией Эйлера} называется функция $\varphi$, такая, что $\varphi(n)$~--- это количество натуральных чисел от $1$ до $n$ взаимно простых с $n$.
\end{definition}

\begin{enumerate}[label*=\protect\fbox{\arabic{enumi}}]

\item Чему равно $\varphi(9)$; $\varphi(13)$; $\varphi(125)$; $\varphi(1)$?

\item Чему равно $\varphi(p)$; $\varphi(p^\alpha)$, где $p$~--- простое число?


Пусть числа $a$ и $b$ взаимно просты и в таблицу размером $a \times b$ выписаны все последовательные числа подряд начиная с 1.

\item Сколько в таблице чисел, взаимно простых с $b$?
\item Сколько в каждом столбце чисел, взаимно простых с $a$?
\item Докажите, что для взаимно простых $a, b$, выполняется: $\varphi(ab) = \varphi(a)\varphi(b)$. Это свойство называется \textbf{мультипликативностью}.
\item Докажите \textbf{формулу Эйлера}:
\begin{align*}
	\varphi(n) = p_1^{\alpha_1 - 1}p_2^{\alpha_2 - 1}\ldots p_k^{\alpha_k - 1}(p_1 - 1)(p_2 - 1)\ldots (p_k - 1) =\\
	= n\left(1 - \frac{1}{p_1}\right)\left(1 - \frac{1}{p_2}\right)\ldots\left(1 - \frac{1}{p_k}\right)
\end{align*}

\item Докажите, что при $n > 2$ $\varphi(n)$~--- четно.

\item Найдите сумму чисел взаимно простых с $n$, не превосходящих $n$.

\item При каких $m$ выполняется равенство $\varphi(m^k) = m^{k - 1} \varphi(m)$?

\item Найдите все такие $x$, что a) $\varphi(x) = \dfrac{x}{2}$; b) $\varphi(x) = \dfrac{x}{3}$; c) $\varphi(x) = \dfrac{x}{4}$; d) $\varphi(x) = \dfrac{x}{7}$; 

\item Рассмотрим ряд дробей: $\frac{1}{n}, \frac{2}{n}, \ldots, \frac{n}{n}$. Сократим каждую из дробей на НОД ее числителя и знаменателя. Сколько будет дробей со знаменателем $d$, где $d$~--- некоторый делитель числа $n$?

\item Докажите \textbf{тождество Эйлера-Гаусса}: $\varphi(d_1) + \varphi(d_2) + \ldots + \varphi(d_k) = n$, где $d_1, d_2, \ldots d_k$~--- все делители числа $n$.

\item Окружность разделена $n$ точками на $n$ равных частей. Сколько можно составить различных замкнутых ломаных из $n$ равных звеньев с вершинами в этих точках?

\end{enumerate}
\end{document}