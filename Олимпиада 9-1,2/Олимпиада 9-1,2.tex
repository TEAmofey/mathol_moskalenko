\documentclass{article}
\usepackage[12pt]{extsizes}
\usepackage[T2A]{fontenc}
\usepackage[utf8]{inputenc}
\usepackage[english, russian]{babel}

\usepackage{amssymb}
\usepackage{amsfonts}
\usepackage{amsmath}
\usepackage{enumitem}
\usepackage{graphics}
\usepackage{graphicx}

\usepackage{lipsum}

\newtheorem{theorem}{Теорема}
\newtheorem{task}{Задача}
\newtheorem{lemma}{Лемма}
\newtheorem{definition}{Определение}
\newtheorem{example}{Пример}
\newtheorem{statement}{Утверждение}
\newtheorem{corollary}{Следствие}


\usepackage{geometry} % Меняем поля страницы
%\geometry{left=1cm}% левое поле
%\geometry{right=1cm}% правое поле
\geometry{top=3cm}% верхнее поле
%\geometry{bottom=1cm}% нижнее поле


\usepackage{fancyhdr} % Headers and footers
\pagestyle{fancy} % All pages have headers and footers
\fancyhead{} % Blank out the default header
\fancyfoot{} % Blank out the default footer
\fancyhead[L]{\textit{\textbf{Олимпиада}}}
\fancyhead[C]{}
\fancyhead[R]{24 января}% Custom header text


%----------------------------------------------------------------------------------------

%\begin{document}\normalsize
\begin{document}\large
	
\begin{center}
	\LARGE\textbf{9-1}
\end{center}
\begin{center}
	\large\textbf{Второй тур}
\end{center}


\begin{enumerate}[label*=9.{\arabic{enumi}}]
\setcounter{enumi}{4}
\item Класс из 32 учеников на зимние каникулы поехал в лагерь в Комарово, где было 5 комнат. После смены выяснилось, что любые 2 ребёнка, жившие в одной комнате, стали ненавидеть друг друга. На весенние каникулы класс поехал в лагерь в Рощино, где было 6 комнат. Докажите, что ребят не удастся расселить так, чтобы дети, живущие в одной комнате пока что не испытывали ненависть друг к другу.

\item Дан приведённый квадратный трёхчлен $f(x)$. Оказалось, что трёхчлены $f(x) + 2022x - 2023$ и  $f(x) - 2023x + 2022$ не имеют корней. Докажите, что $2023\cdot f(x) - 2022$ тоже не имеет корней.

\item В выпуклом четырёхугольнике $ABCD$ верно тождество $AB + DC = AD + BC$. Оказалось, что радиусы вписанных окружностей треугольников $ABC$ и $ACD$ равны. Найдите угол между прямыми $АС$ и $BD$.

\item Тая, Таня и Маша пошли гулять. Маша искала интересные факты о натуральном числе $m$ и попросила девочек помочь. Тая заметила, что некоторое натуральное число $n>m$ можно представить в виде суммы 2023 целых неотрицательных степеней числа $m$. На что Таня ответила, что то же самое число $n$ можно представить и в виде суммы 2023 целых неотрицательных степеней числа $m + 1$. При каком наибольшем $m$ такое могло произойти?
\end{enumerate}
\end{document}