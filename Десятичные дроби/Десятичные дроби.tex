\documentclass{article}

\usepackage[12pt]{extsizes}
\usepackage[T2A]{fontenc}
\usepackage[utf8]{inputenc}
\usepackage[english, russian]{babel}

\usepackage{mathrsfs}
\usepackage[dvipsnames]{xcolor}

\usepackage{amsmath}
\usepackage{amssymb}
\usepackage{amsthm}
\usepackage{indentfirst}
\usepackage{amsfonts}
\usepackage{enumitem}
\usepackage{graphics}
\usepackage{tikz}
\usepackage{tabu}
\usepackage{diagbox}
\usepackage{hyperref}
\usepackage{mathtools}
\usepackage{ucs}
\usepackage{lipsum}
\usepackage{geometry} % Меняем поля страницы
\usepackage{fancyhdr} % Headers and footers
\newcommand{\range}{\mathrm{range}}
\newcommand{\dom}{\mathrm{dom}}
\newcommand{\N}{\mathbb{N}}
\newcommand{\R}{\mathbb{R}}
\newcommand{\E}{\mathbb{E}}
\newcommand{\D}{\mathbb{D}}
\newcommand{\M}{\mathcal{M}}
\newcommand{\Prime}{\mathbb{P}}
\newcommand{\A}{\mathbb{A}}
\newcommand{\Q}{\mathbb{Q}}
\newcommand{\Z}{\mathbb{Z}}
\newcommand{\F}{\mathbb{F}}
\newcommand{\CC}{\mathbb{C}}

\DeclarePairedDelimiter\abs{\lvert}{\rvert}
\DeclarePairedDelimiter\floor{\lfloor}{\rfloor}
\DeclarePairedDelimiter\ceil{\lceil}{\rceil}
\DeclarePairedDelimiter\lr{(}{)}
\DeclarePairedDelimiter\set{\{}{\}}
\DeclarePairedDelimiter\norm{\|}{\|}

\renewcommand{\labelenumi}{(\alph{enumi})}

\newcommand{\smallindent}{
    \geometry{left=1cm}% левое поле
    \geometry{right=1cm}% правое поле
    \geometry{top=1.5cm}% верхнее поле
    \geometry{bottom=1cm}% нижнее поле
}

\newcommand{\header}[3]{
    \pagestyle{fancy} % All pages have headers and footers
    \fancyhead{} % Blank out the default header
    \fancyfoot{} % Blank out the default footer
    \fancyhead[L]{#1}
    \fancyhead[C]{#2}
    \fancyhead[R]{#3}
}

\newcommand{\dividedinto}{
    \,\,\,\vdots\,\,\,
}

\newcommand{\littletaller}{\mathchoice{\vphantom{\big|}}{}{}{}}

\newcommand\restr[2]{{
    \left.\kern-\nulldelimiterspace % automatically resize the bar with \right
    #1 % the function
    \littletaller % pretend it's a little taller at normal size
    \right|_{#2} % this is the delimiter
}}

\DeclareGraphicsExtensions{.pdf,.png,.jpg}

\newenvironment{enumerate_boxed}[1][enumi]{\begin{enumerate}[label*=\protect\fbox{\arabic{#1}}]}{\end{enumerate}}



\smallindent

\header{ЦРОД $\bullet$ Математика}{\textit{Алгебра}}{3 ноября 2023}

%----------------------------------------------------------------------------------------

\begin{document}
    \large


    \begin{center}
        \textbf{Десятичные дроби}
    \end{center}

    Здесь и далее, число $b$~-- знаменатель дроби, a $a$~--числитель.

    \begin{enumerate_boxed}

        \item Докажите, что дробь является конечной тогда и только тогда, когда $b$ имеет вид $2^{n}5^m$.

    \end{enumerate_boxed}

    В дальнейшем считаем, что $b\neq 2^{n}5^m$.

    Вспомним \textbf{алгоритм деления столбиком}.\\
    При правильном взгляде на вещи он состоит в следующем.
    Полагаем $r_0=a$ и считаем рекуррентно $10\cdot r_{i-1}=bq_i+r_i$  (деление с остатком).
    При этом $q_i$ -- $i$-тая цифра после запятой в равенстве $\frac{a}{b}=0,q_{1}q_{2}q_3\dotsc$.

    \begin{enumerate_boxed}

        \setcounter{enumi}{1}
        \item
        \begin{enumerate}
            \item Докажите, что при делении в столбик получается периодическая дробь с периодом не более $b-1$;
            \item и даже сумма длин периода и предпериода не более $b-1$.
        \end{enumerate}

    \end{enumerate_boxed}

    {Еще одно понимание алгоритма деления столбиком}  состоит в следующем.
    Делим с остатком: $a\cdot 10^k=bQ_k+r_k$.
    Тогда $Q_k$ --- число, образованное первыми $k$ цифрами после запятой,
    $r_k$ --- то же самое, что ранее (тем самым $r_k$ оказывается остатком при делении $a\cdot 10^k$ на $b$).
    \begin{enumerate}[label*=\protect\fbox{\arabic{enumi}}]
        \setcounter{enumi}{2}
        \item
        \begin{enumerate}
            \item Докажите, что если $(b,10)=1$, то $(r_i,b)=1$.
            \item Докажите, что длина периода не превосходит $\varphi(b)$.
        \end{enumerate}

        \item Докажите, что если $(b,10)=1$, то зацикливание происходит без предпериода.
        При этом длина периода не зависит от $a$ и равна наименьшему $t$, для которого $10^t-1 \,\vdots\, b$, то есть показателю числа 10 по модулю $b$.

        \item Пусть наименьший период некоторой последовательности равен $\ell$, а $L$ --- некоторый другой период.
        Докажите, что $L \,\vdots\, \ell$.

        \item Докажите, что дробь $0,RTTT\dots$ ($R$ --- из $k$ цифр, $T$ --- из $t$ цифр) равна $\frac{R}{10^k}+\frac{T}{10^k(10^t-1)}$.

        \item Докажите, что если $(b,10)\neq 1$, то в десятичной записи $\frac{a}{b}$ обязательно есть предпериод.

        \item Пусть $a<b$, $(a,b)=1$, $b=2^x\cdot 5^y\cdot b'$, $\ell=\max\{x,y\}$.
        Докажите, что период дроби $\frac{a}{b}$ равен периоду дроби $\frac{1}{b'}$, а предпериод в точности равен $\ell$, и не может быть меньше.

        \item Каково наибольшее значение длины предпериода среди всех несократимых дробей со знаменателем не превосходящим 2024?

        \item Приведите пример дробей с предпериодами, при сложении которых предпериод исчезает, а период меньше, чем оба периода слагаемых.

        \item Докажите, что период суммы (разности) двух дробей является делителем НОКа периодов, а предпериод не превосходит максимума предпериодов.

        \item Пусть $p>5$ --- простое число.
        Известно, что длина наименьшего периода десятичной записи дроби $1/p$ равна $2n$.
        Докажите, что если этот период разбить на два $n$-значных куска, то сумма чисел в этих кусках равна $99\dots 9$ ($n$ девяток).
        Например, $1/7=0.(142857)$, $142+857=999$.

    \end{enumerate}
\end{document}